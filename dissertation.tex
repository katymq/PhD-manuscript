\documentclass[a4paper,twoside,openanay,11pt]{book}
% la classe book est naturellement en deux faces 
% ou \documentclass[twoside]{report}
% avec twoside pour que inner soit à gauche sur les page impaires et à droite sur les pages paires (voir ci-dessous)
%%%%%%%%%%%%%%%%%%%%%%%%%%%%%%%%%%%%
% chargement des packages 
% \usepackage[T1]{fontenc}
\usepackage{booktabs} %some rules toprule midrule etc
% \usepackage{mwe} %for chapt4 2 figures side by side
\usepackage{natbib}
\bibliographystyle{apa-good}
\usepackage{graphicx}
\usepackage{makecell} %cells in appendix
\graphicspath{{Figures/}}
\usepackage{etoc} %contents in each chapter
\usepackage{blindtext} 
% \usepackage[linktocpage]{hyperref} %hyperlinks table of contents on numbers
%%%%%%%%%%%%%%%%%%%%%%%%%%%%%%%%%%%%%
% \hypersetup{hidelinks}

% Hyperref with backref option
\usepackage[colorlinks=true,pagebackref=true,linktocpage,citecolor=blue]{hyperref}


\hypersetup{
  colorlinks   = true, %Colours links instead of ugly boxes
  urlcolor     = blue, %Colour for external hyperlinks
  linkcolor    = magenta, %Colour of internal links
  citecolor   = blue, %Colour of citations
  backref=true,  % Enable back-references
  pagebackref=true,  % Enable page number back-references
  hyperindex=true,  % Enable hyperlinks in index
}
\usepackage{geometry}
% ce package permet de régler très simplement les marges, la zone de texte effective, ...
% si on n'utilise pas ce package, il faut régler les marges avec les commandes de base mais cela est compliqué.

%
% Les dimensions de la page pour l'offset sont 175 x 245 mm. 
% Pour une impression sans réduction sur de l'A4 et avec un rognage à droite et en bas,
% l'offset demande une marge intérieure >= 15 mm et une marge supérieure >= 10 mm. 
%
% Remarque : la marge intérieure est à gauche sur les pages impaires et à droite sur les pages paires en double face,
%  elle toujours à gauche en simple face
% 
% Avec le package geometry, on règle la marge intérieure par inner, la marge supérieure par top et la zone utile par total. 
% Les autres dimensions sont automatiquement calculées en fonction du format du papier (A4 par défaut).  
% La clef includeheadfoot entraîne que les entêtes et pieds de page sont inclus dans la zone utile. 
% Sinon, il régler body qui fixe la zone de texte et tenir compte des dimensions des entêtes et pieds de page qui sont alors situés à l'extérieur de body.
% Voir la documentation de geometry pour plus d'info. 
%      

%\geometry{reset,inner=25mm,top=15mm,total={130mm,215mm},includeheadfoot,showframe} %version offset
\geometry{reset,centering,total={130mm,210mm},includeheadfoot}
\setlength{\headheight}{14pt}

% Ici, on a réglé la zone utile à 130 x 215 mm ce qui laisse 45 mm pour les marges intérieures et extérieures et 30 mm pour les marges en haut et en bas.
% Pour tenir compte de la reliure, j'ai mis 25 mm en intérieur, ce qui donnera 20 mm en extérieur après rognage droite. (mais on peut changer, c'est une question de goût)
% Pour le vertical, j'ai centré mes 215 mm dans les 245 mm soit une marge supérieure de 15 mm, ce qui donnera une marge inférieure de 15 mm après rognage en bas
%
% Remarque 1 : la zone utile peut aussi être élargie, la seule contrainte est de laisser les marges droite et inférieure suffisantes,
% étant entendu que les marges intérieure et supérieures sont au minimum 15 et 10 mm.  Pour rappel, on ne règle pas les marges extérieure 
% et inférieure car elles se calculent à patir des marges intérieure et supérieure et de total.     
% 
% TODO: Remarque 2 : la clef showframe permet de voir des lignes représentant les différentes dimensions. Utile dans la phase de mise au point.
% Ne pas oublier de supprimer cette clef dans la version finale.

% TODO: Remarque 3 : pour un résultat optimal, il faut convertir le ps en pdf (ou compiler en pdflatex)
% et demander à l'offset d'imprimer le pdf sans mise à l'échelle



\usepackage{todonotes}
\usepackage{etoolbox}
\usepackage{multirow}
\usepackage{diagbox}
\usepackage{hyperref}
\usepackage[utf8]{inputenc}
\usepackage[T1]{fontenc}
\usepackage{xcolor}
% \usepackage{fontspec}
\usepackage{caption}
\usepackage{subcaption}
\usepackage{tcolorbox}
\usepackage{bigfoot}
\usepackage{epigraph}
\usepackage[shortlabels]{enumitem}
\usepackage{hhline}
\usepackage{listings}
\usepackage{color}
\usepackage{amsmath}
\usepackage{amssymb}
\usepackage{array}
\usepackage{eurosym}
\usepackage{gensymb}
\usepackage{amsthm}
\usepackage[toc,page]{appendix}
\usepackage{pdfpages}
\usepackage{float}
\usepackage{amssymb}
\usepackage[makeroom]{cancel}

%------------------------------------
%------------------------------------
% Trying to improve the quality of the pdf
% \usepackage[libertine,cmintegrals,cmbraces,vvarbb]{newtxmath}
\usepackage[scaled=0.98]{zi4}
\usepackage{lmodern}
\usepackage[english]{babel}
%------------------------------------
%------------------------------------


%--------------------
% Unsupervised segmentation
\usepackage[export]{adjustbox}
\usepackage{tikz}
\usepackage{pgfplots}
\pgfplotsset{compat=1.17}
\usetikzlibrary{shapes, positioning, matrix, calc, pgfplots.statistics,
pgfplots.groupplots, pgfplots.colorbrewer, 3d, arrows, positioning, quotes,
backgrounds, arrows.meta, spy, fit, }

\newcommand{\tikzst}[2][line width=1pt,cyan,opacity=0.4]{%
  \tikz[baseline]{ \node[anchor=text,outer sep=0pt,inner sep=0pt, text=black] (A) {#2}; \draw[#1] ([xshift=-0.1ex]A.west) --  ([xshift=0.1ex]A.east);}%
}%because st from soul does not like Xref`'
\usepackage{subcaption}

\usepackage{array}
\newcolumntype{Z}[1]{>{\centering\let\newline\\\arraybackslash\hspace{0pt}}m{#1}}
%--------------------


% \usepackage{bm}
% % \usepackage{bbm}
% \usepackage{soul}

% --------------Algorithm
% \usepackage{algorithm}% http://ctan.org/pkg/algorithms
% \usepackage{algpseudocode}
% \usepackage{algorithm}
% \usepackage[noend]{algpseudocode}
\usepackage{algorithm}
\usepackage{algpseudocode}
%\usepackage[english, ruled,
%commentsnumbered,onelanguage]{algorithm2e}
% \usepackage[english, lined, boxed, linesnumbered,
% commentsnumbered,onelanguage]{algorithm2e}
% \SetKwInput{KwData}{Input}
% \algdef{SE}[DOWHILE]{Do}{doWhile}{\algorithmicdo}[1]{\algorithmicwhile\ #1}%
\algrenewcommand\algorithmicrequire{\textbf{Input:}}
\algrenewcommand\algorithmicensure{\textbf{Output:}}
% remove line number for specific lines with \nonl
\let\oldnl\nl% Store \nl in \oldnl
\newcommand{\nonl}{\renewcommand{\nl}{\let\nl\oldnl}}% Remove line number for one line
% -----------------


\DeclareMathOperator*{\argmax}{arg\,max}
\DeclareMathOperator*{\argmin}{arg\,min}
\setlength{\epigraphwidth}{0.51\textwidth}

\theoremstyle{definition}
\newtheorem{definition}{Definition}[section]
\newtheorem{example}{Example}[section]
\newtheorem{remark}{Remark}[section]
\newtheorem{remarks}{Remark}
\newtheorem{proposition}{Proposition}[section]
\newtheorem{theorem}{Theorem}[section]
\newtheorem{usecaseinner}{Use Case}
\newtheorem{lemma}{Lemma}[section]
\newtheorem{model}{Model}

\newtheorem{corollary}{Corollary}
% \newtheorem{proof}{Proof}
\newtheorem{scenario}{Scenario}

\newcommand{\katy}[1]{\todo[inline,color=blue]{#1 --- Katy TO-DO}}
\newcommand{\yohan}[1]{\todo[inline,color=yellow]{#1 --- Yohan Obs}}
\newcommand{\yohanobs}[1]{\todo[size=\tiny,color=orange]{#1  \hfill --- Yohan}}
\newcommand{\katyobs}[1]{\todo[size=\tiny,color=green]{#1  \hfill --- Katy}}
\newcommand{\nocontentsline}[3]{}
\newcommand{\tocless}[2]{\bgroup\let\addcontentsline=\nocontentsline#1{#2}\egroup}

\newenvironment{usecase}[1]{%
	\renewcommand\theusecaseinner{#1}%
	\usecaseinner
}{\endusecaseinner}
%\lstset{basicstyle=\linespread{0.4}}

\definecolor{dkgreen}{rgb}{0,0.6,0}
\definecolor{gray}{rgb}{0.5,0.5,0.5}
\definecolor{mauve}{rgb}{0.58,0,0.82}

\lstset{frame=tb,
	aboveskip=3mm,
	belowskip=3mm,
	showstringspaces=false,
	columns=flexible,
	basicstyle={\linespread{1}\small\ttfamily},
	numbers=none,
	numberstyle=\tiny\color{gray},
	keywordstyle=\color{blue},
	commentstyle=\color{dkgreen},
	stringstyle=\color{mauve},
	breaklines=true,
	breakatwhitespace=true,
	tabsize=3
}

\newtoggle{hidecomments} %==false by default
%\toggletrue{hidecomments} %uncomment this line to hide comments

\iftoggle{hidecomments}{ % ignores comments but still defines the corresponding commands to prevent compiling errors
	\newcommand{\reviewer}[1]{}
}{
	\newcommand{\reviewer}[1]{\todo[color=white!40, inline]{\footnotesize{Reviewer: #1}}}
}

\newcommand{\Csharp}{%
	{\settoheight{\dimen0}{C}C\kern-.05em \resizebox{!}{\dimen0}{\raisebox{\depth}{\#}}}}
%------------------------------------------------------------------
% CREATE ACRONYMS
%------------------------------------------------------------------
% \usepackage[acronym]{glossaries}
\usepackage[acronym,nonumberlist]{glossaries}

%todo: see below for acronyms/glossaries
%if glossary as well and not only acronyms, see: https://tex.stackexchange.com/questions/8946/how-to-combine-acronym-and-glossary
%to generate the list of acronyms:
%makeindex -s filename.ist -t filename.alg -o filename.acr filename.acn
%->
%makeindex -s latex_avec_réduction_pour_impression_recto_verso_et_rognage_minimum.ist -t latex_avec_réduction_pour_impression_recto_verso_et_rognage_minimum.alg -o latex_avec_réduction_pour_impression_recto_verso_et_rognage_minimum.acr latex_avec_réduction_pour_impression_recto_verso_et_rognage_minimum.acn

\makeglossaries
% \setglossarypreamble[acronym]{Various acronyms will be used throughout this dissertation to abbreviate frequent terms, some of which will even find usages across all sections. The expansion will be given at least on the first occurrence of each acronym in the text, but the following list of acronyms can be used as a reference if needed.}
%------------------------------------------------------------------
%------------------------------------------------------------------
% CREATE NOTATIONS
%------------------------------------------------------------------
\usepackage{nomencl}
\makenomenclature
% \renewcommand{\nomname}{List of Notations}


% \setnomenclaturepreamble{Various notations will be used throughout 
% this dissertation to abbreviate frequent terms, some of which will 
% even find usages across all sections. The expansion will be given 
% at least on the first occurrence of each notation in the text,
%  but the following list of notations can be used as a reference 
% if needed.}
%------------------------------------------------------------------
%------------------------------------------------------------------
\usepackage{titlesec}
\usepackage{titletoc}
\usepackage{fancyhdr}

% Define the page style for the front matter


\titleformat{\chapter}[display]
{\bfseries\huge}
{\filleft\Large\MakeUppercase{\chaptertitlename} \  \rlap{ \resizebox{!}{0.8cm}{\thechapter} \rule{5cm}{1.2cm}}}
{0ex}
{%\titlerule
	\vspace{2ex}%
	\filleft}
[\vspace{2ex}%
%\startcontents%
%\printcontents{l}{1}{\setcounter{tocdepth}{1}}%
\titlerule%\setcounter{tocdepth}{2}%
]

\titleformat{\section}
{\bfseries\Large}
{\thesection.}
{0.5em}
{}
[\vspace{0.5ex}
\titlerule]

\fancypagestyle{frontmatter}{
    \fancyhf{} % Clear header and footer
    \fancyhead[R]{\leftmark} % Set the chapter title in the upper-right corner
    \renewcommand{\headrulewidth}{0.4pt} % Add a horizontal rule under the header
}
% \includeonly{chapter4}

%-------------------------------------------------------------------------
%-------------------------------------------------------------------------
\begin{document}
% globaldefinitions.tex

\def\g{g_{\theta}}
\def\f{f_{\theta}}

\def\lat{h}
\def\obs{x}
% \def\Obs{X}
% \def\Lat{H}
\def\Obs{x}
\def\Lat{h}
\def\Lab{y}
\def\Latent{z}
% \def\Hid{h}
\def\latent{z}
\def\lab{y}
% \def\labl{y_{T}^{\mathcal{L}}}
% \def\labu{y_{T}^{\mathcal{U}}}
% \def\L{\mathcal{L}}
% \def\U{\mathcal{U}}
% \def\Du{\mathcal{D}^{\mathcal{U}}}
% \def\Dl{\mathcal{D}^{\mathcal{L}}}
\def\L{\mathcal{L}} %Loss

\def\lik{\mathbf{L}} %likelihood

\def\Du{\mathcal{D}^{\mathcal{O}}}
\def\Dl{\mathcal{D}^{\mathcal{H}}}

% \def\labl{ \mathbf{y}_{T}^{\mathbf{O}}}
% \def\labu{\mathbf{y}_{T}^{\mathbf{H}}}
% \def\U{\mathcal{H}}
% \def\LL{\mathcal{O}} %labelled
\def\labl{ y_{T}^{\;\mathbf{O}}}
\def\labu{y_{T}^{\;\mathbf{H}}}
\def\U{\mathbf{H}}
\def\LL{\mathbf{O}} %labelled

\def\triplet{{v}}

% \def\hl{{\mathbf h}_{T}^{\mathcal{L}}}
% \def\hu{{\mathbf h}_{T}^{\mathcal{U}}}

% \def\px{\psi^{x}_{\theta, t}}
% \def\pz{\psi^{z}_{\theta, t}}
% \def\py{\psi^{y}_{\theta, t}}
% \def\ph{\psi^{h}_{\theta, t}}
% \def\qz{\psi^{z}_{\phi,t}}
% \def\qh{\psi^{h}_{\phi,t}}
% \def\qy{\psi^{y}_{\phi,t}}

\def\px{\psi^{x}_{\theta}}
\def\pz{\psi^{z}_{\theta}}
\def\py{\psi^{y}_{\theta}}
\def\ph{\psi^{h}_{\theta}}
\def\qz{\psi^{z}_{\phi}}
\def\qh{\psi^{h}_{\phi}}
\def\qy{\psi^{y}_{\phi}}

\def\pxun{\psi^{x}_{\theta}}
\def\pxop{\psi^{x}_{\theta^*}}
\def\pzun{\psi^{z}_{\theta}}
\def\pyun{\psi^{y}_{\theta}}
\def\pyop{\psi^{y}_{\theta^*}}
\def\phun{\psi^{h}_{\theta}}
\def\qzun{\psi^{z}_{\phi}}
\def\qhun{\psi^{h}_{\phi}}
\def\qyun{\psi^{y}_{\phi}}
\def\param{w}
\def\ufr{\mathrm{ufr}}
\def\fr{\mathrm{fr}}
\def\pre{\mathrm{pre}}
\def\d{\mathrm{d}}

\def\mulatent{\mu_{\latent}^{q}}
\def\siglatent{\sigma_{\latent}^{q}}
% \def\mulatent{\mu_{\latent,t}^{q}}
% \def\siglatent{\sigma_{\latent,t}^{q}}
% \def\mulatent{\mu_{\phi}^{\latent}}
% \def\siglatent{\sigma_{\phi}^{\latent}}

\def\mulat{\mu_{\phi}^{\lat}}
\def\siglat{\sigma_{\phi}^{\lat}}

\def\mulatp{\mu_{\theta}^{\lat}}
\def\siglatp{\sigma_{\theta}^{\lat}}

\def\muobs{\mu_{\theta}^{\obs}}
\def\sigobs{\sigma_{\theta}^{\obs}}

% \def\mulatentp{\mu_{\latent,t}^{p}}
% \def\siglatentp{\sigma_{\latent,t}^{p}}
\def\mulatentp{\mu_{\theta}^{\latent}}
\def\siglatentp{\sigma_{\theta}^{\latent}}
\def\Siglatentp{\sigma_{\theta}^{\latent}}

\def\ropx{\rho_{\theta}^{\obs}}
% \def\ropy{\rho_{\lab, t}^{p}}
% \def\roqy{\rho_{\lab, t}^{q}}
\def\ropy{\rho_{\lab}^{p}}
\def\roqy{\rho_{\lab}^{q}}

\def\diag{\mathrm{diag}}

\def\hh{{\mathbf h'}}

%Proba 
\def\Var{\mathrm{Var}}
\def\Cov{\mathrm{Cov}}
\def\N{{\mathcal N}}
\def\Ber{{\mathcal Ber}}
\def\p{p_{\theta}}
\def\q{q_\phi}
\def\qs{q_{\phi^*}}
\def\dkl{\mathrm{D}_{\mathrm{KL}}}


\def\Q{\tilde{\mathcal{Q}}}
\def\Qsemi{\mathcal{Q}_{\mathrm{semi}}}
\def\Qsup{\mathcal{Q}_{\mathrm{sup}}}
\def\Qunsup{\mathcal{Q}_{\mathrm{unsup}}}
\def\Qgen{\mathcal{Q}_{\mathrm{gen}}}

\def\hh{\overline{h}}
\def\D{\mathcal{D}}




\def\ie{\emph{i.e.}}
\def\eg{\emph{e.g.}}
% \def\ie{\emph{i.e.}}
\def\iid{\emph{i.i.d. }}

% Natural, Complex etc 
\def\NN{\mathbb{N}} 
\def\CC{\mathbb{C}} 
\def\E{{\mathbb{E}}}
\def\T{{\mathcal R}_T}
\def\elbo{\mathcal{Q}}


 % Include global definitions

\includepdf[]{1ere.pdf}
\thispagestyle{empty}
\cleardoublepage
\frontmatter
% \cleardoublepage

% \pagestyle{front}
\thispagestyle{empty}
\chapter*{Acknowledgements}
\addcontentsline{toc}{chapter}{Acknowledgements}

I would like to express my deepest gratitude to everyone who contributed to the
completion of this PhD. 


First, I extend my heartfelt thanks to my jury for their
time, expertise, and valuable insights, which have greatly enriched this work. I
am especially grateful to my supervisor, Yohan Petetin, for their guidance,
support, and encouragement throughout this journey. Your mentorship has been
invaluable. I would also like to
express my deep appreciation to Hugo Ganglof, who generously shared his
knowledge and expertise. Your kindness and willingness to help throughout my PhD
were truly inspiring. 

A special thanks to the department director, Professor Wojciech
Pieczynski, whose support was crucial not only for the advancement of my
research but also for his genuine concern for my well-being.

To my lab partners, I feel fortunate to have met you all,
especially during this last year. It has been a pleasure to work alongside such
wonderful colleagues. I also want to thank Laura and Julie, for their constant 
assistance. Your help has been invaluable
in navigating the administrative aspects of this process.

This thesis would not have been possible without the help and support of many
people. To my family, for their unconditional love and support, especially my
mother Martha, my grandfather, Diana and Christopher, my father Felipe, and my uncle
Fernando, for being my source of strength and comfort.

To my friends from
Ecuador, particularly Josselyn, Mateo, and my high school friends, for keeping
me grounded and connected from afar. 
To my heart-family in France, especially
Any, Eli, Alex, Majo, Sofi, Quentin, François, and Albita, 
for their friendship and for
standing by me through this journey. And to Dani, for being my rock during
difficult times and my partner in this adventure. 
Lastly, I would like to thank my adoptive family in France for their warmth 
and hospitality, and everyone who,
in one way or another, has helped me along this journey. My deepest gratitude to
all of you.

\section*{Spanish}
Esta tesis doctoral no habría sido posible sin la ayuda de muchas personas.
Quiero expresar mis más sinceros agradeciemientos
a todos los que me han ayudado a lo largo de este viaje. 


En primer lugar, extiendo mi más sincero agradecimiento a mi jurado por su
tiempo, experiencia y valiosas aportaciones, que han enriquecido enormemente
este trabajo. Estoy especialmente agradecida a mi director de tesis, Yohan
Petetin, por su guía, apoyo y ánimo a lo largo de este viaje. Su mentoría ha
sido invaluable. 

A Hugo Ganglof, quien compartió generosamente su conocimiento y experiencia. Tu
amabilidad y disposición para ayudar durante todo mi doctorado fueron realmente
inspiradoras. 

Un agradecimiento especial al director del departamento, el
Profesor Wojciech Pieczynski, cuyo apoyo fue crucial no solo para el avance de
mi investigación, sino también por su genuino interés en mi bienestar. 


A mis compañeros de laboratorio, me siento afortunada de haberlos conocido,
especialmente durante este último año. 

A Laura y Julie por su
constante ayuda. Su asistencia ha sido invaluable para navegar los aspectos
administrativos de este proceso.


A mi familia, por su apoyo incondicional a la distancia, por su amor y 
saber que juntos podemos sobrellevar momentos difíciles. 
Cada momento que tuve
la oportunidad de pasar con ustedes durante el desarrollo 
de la tesis, fue un momento de felicidad y me ayudó a seguir adelante.

A mi madre Martha, por su amor, consejos y esfuerzos para que cumpliera lo
que siempre soñé. Gracias por ser mi ejemplo a seguir y por enseñarme
a ser fuerte y, sobretodo, por cuidarme en los momentos difíciles.
Ser mi paño de lágrimas y mi fuente de amor.

A mi abuelito Alberto, por  el amor que 
me ha dado a la distancia y sus preguntas sobre Francia y el viejo continente, 
que siempre me sacaban una sonrisa. 

A mis hermanos, por escucharme y hacerme reir cuando lo necesitaba.

Diana, por ser una hermana comprensiva, escucharme y darme consejos en 
los momentos en los cuales me sentía sola y desanimada.

Christopher, por su inociencia, por hacerme reír con sus ocurrencias, y
por darme bellas experiencias siempre que viajaba a casa.

A mi tío Fernando por cuidarme y apoyarme a lo largo de mi vida, siendo
como un segundo padre para mi.

A mi padre, por sus consejos y ánimos para que continuara con mi sueño, fue algo que
necesitaba en los momentos de duda y saber que usted confiaba en mi, me ayudó a seguir adelante.

A mi primer gordito, Edi por estar cerca de mi, por sus palabras
de aliento y por ser mi hermanito menor de corazón.


Mayerli, Brandon, Xavier, gracias por esas conversaciones que tuvimos,
el apoyo y por hacerme sentir como en casa.

A Laura, por su apoyo y por aceptarme como una hija más. 
A mi sobrina Mayte y a Erika.



A mis amigos de Ecuador, cada viaje que hice a casa fue una oportunidad
para saber que cuento con ustedes. Gracias por esas conversaciones
que me ayudaron a encontrarme de nuevo y recordar las experiencias
que vivimos juntos.
Gracias a  mis amigos de la U, en particular a Josselyn, por su apoyo
y por ser una amiga que siempre está ahí. A Mateo, por ser como un hermano mayor
y por escucharme y apoyarme en diferentes momentos de mi vida. 
A mis amigos del colegio, 
Bryan, Alex, Juanito, Steven (Monito), Angel y Geovanny (Chochito), 
por su apoyo y las salidas que siempre disfrutamos juntos.
A Edgar, por ser un amigo que siempre está ahí y el apoyo que me brindó
a la distancia.

A mi familia de corazón aquí en Francia: 

Any, mi mejor amiga, mi hermana, gracias por
cada momento que compartimos juntas, momentos de felicidad y de tristeza.
Gracias por estar siempre ahí y ayudarme a ver la luz y creer en mi, 
inlcuso cuando yo no lo hacía.

Eli, por tus consejos, por  estar ahí siempre para escuchar los momentos 
buenos y malos que me pasaron,  y por todos tus detalles.

Alex, por ser como un hermano mayor, por tus consejos y por escucharme, 
por compartir este camino doctoral juntos y por las risas y lágrimas que compartimos.

Majo, por ser una amiga sincera, comprensiva, por escucharme, por los consejos, y por llevarme al Crossfit.

Quentin y Francois, por ayudarme a sentirme como en casa e incursionarnos en la cultura francesa.
Viva el norte y la Bretagne!

Sofi, gracias por apoyarme y escucharme, por ser una amiga incondicional, por lo bueno 
y malo que hemos pasado juntas.

Albita,  por el apoyo y risas de nuestra experiencia doctoral, por ser una amiga
que a la distancia me ha demostrado mucho amor y apoyo.


Dani, gracias por compartir este experiencia doctoral, por los momentos de estrés, 
tristeza y felicidad que compartimos juntos. Gracias por ser mi soporte y por el amor 
que me has brindado. Que esta etapa sea el inicio de una aventura juntos.


A todos mis amigos de Francia, gracias!
Mona, Yannick, Nas, Belén, Vivi, Vale, Panchito, Mónica, Balthazar, Agustín, 
Louis (Lucho), Thimo, Geremy, David, Mica, GianKa, Cata. 
Con cada uno de ustedes he compartido momentos que 
guardo en mi corazón.


Por último, y no menos importante, a mi familia adoptiva de Francia
por su amor y por hacerme sentir como en casa. Gracias Isabelle, Olivier, 
Valentin, Camille, Chloe, por sus consejos y hospitalidad.


Hay muchas personas que están y estuvieron presentes a lo largo de este viaje, 
Luz Marina, mis tíos, mis primos, mis amigos de la infancia. 
Algunas que ya no forman parte de mi vida, pero que en algún momento
me ayudaron a seguir adelante y me brindaron su apoyo y amor. 
A todos ellos, gracias!
Siempre estarán en mi corazón y les deseo lo mejor en sus vidas.




\thispagestyle{empty}
\tableofcontents
\thispagestyle{empty}
% % !TEX root = latex_avec_réduction_pour_impression_recto_verso_et_rognage_minimum.tex

% % \chapter*{Nomenclaturexxx}
% \renewcommand{\nomgroup}[1]{%
%   \item[\bfseries
%   \ifthenelse{\equal{#1}{G}}{General}{%
%   \ifthenelse{\equal{#1}{P}}{Probability Theory}{}}%
% ]}


% % General
% \nomenclature[G]{$a$}{Light lower case letters represent scalars or functions}
% \nomenclature[G]{$A$}{Upper case letters represent matrices}
% % \nomenclature[G]{$I$}{Identity matrix of size $n$}
% \nomenclature[G]{$\mathbb{R}^n$}{Set of real vectors of dimension $n$}
% \nomenclature[G]{$\mathbb{R}^{n \times m}$}{Set of real matrices of dimension $n \times m$}
% % \nomenclature[G]{$\mathcal{T}$}{Toeplitz matrix}
% \nomenclature[G]{$\mathrm{diag}(\cdot)$}{Diagonal matrix}
% \nomenclature[G]{$\cdot^{\top}$}{Transpose operator of a matrix or a vector}

% % Probability
% \nomenclature[P]{$x$}{Random variable and its realization. As far as the context is clear, we use the same symbol for both.}
% \nomenclature[P]{$\mathcal{N}(\cdot, \cdot)$}{Normal (Gaussian)  density function (first argument is the mean, 
% second is the variance or covariance matrix if the random variable is multidimensional)}
% \nomenclature[P]{$\mathcal Ber (\cdot)$}{Bernoulli distribution (the argument is the probability of success)}
% \nomenclature[P]{$\sim$}{Indicates the law followed by a random variable}
% \nomenclature[P]{$p(x)$}{Probability  of a realization $x$ (discrete case) and 
% probability density function of $x$ (continuous case)}
% \nomenclature[P]{$x_t$}{Random variable or realization at time $t$}
% \nomenclature[P]{$x_{0:T}$}{Sequence of random variables or realizations from time $0$ to $T$}
% \nomenclature[P]{$\mathrm{D}_{\mathrm{KL}}(q \mid  p)$}{Kullback-Leibler divergence between distributions $q$ and $p$}

% \nomenclature[P]{$\mathbb{E} (x)$}{Expectation of a random variable $x$}
% \nomenclature[P]{$\mathbb{E}_{p(\cdot)} (\cdot)$}{Expectation under the distribution $p(\cdot)$}
% \nomenclature[P]{$\mathrm{Var} (x)$}{Variance of a random variable $x$}
% \nomenclature[P]{$\mathrm{Cov} (x, y)$}{Covariance between the random variables $x$, and $y$}

\renewcommand{\nomgroup}[1]{%
  \item[\bfseries
  \ifthenelse{\equal{#1}{G}}{General}{%
  \ifthenelse{\equal{#1}{P}}{Probability Theory}{}}%
]}

% General
\nomenclature[G, 01]{$a$}{Light lower case letters represent scalars or functions}
\nomenclature[G, 02]{$A$}{Upper case letters represent matrices}
\nomenclature[G, 03]{$\mathbb{R}^n$}{Set of real vectors of dimension $n$}
\nomenclature[G, 04]{$\mathbb{R}^{n \times m}$}{Set of real matrices of dimension $n \times m$}
\nomenclature[G, 05]{$\mathbb{C}$}{Set of complex numbers}
\nomenclature[G, 05]{$\Re$}{Real part of a complex number}
\nomenclature[G, 06]{$\mathrm{diag}(\cdot)$}{Diagonal matrix}
\nomenclature[G, 07]{$\cdot^{\top}$}{Transpose operator of a matrix or a vector}
% sigm
\nomenclature[G, 08]{$\rm sigm $}{Sigmoid function $ \rm sigm (x) = 1/(1 + \exp(-x))$}
% Probability
\nomenclature[P, 01]{$x$}{(Observed) random variable and its realization. As far as the context is clear, we use the same symbol for both.}
\nomenclature[P, 02]{$y$}{Random variable and its realization, which represents the label associated with $x$}
\nomenclature[P, 03]{$z$}{Latent random variable and its realization}
\nomenclature[P, 04]{$x_t$}{Random variable or realization at time $t$}
\nomenclature[P, 05]{$x_{t:k}$}{Random variable or realization from time $t$ to $k$}
\nomenclature[P, 06]{$x_{0:T}$}{Sequence of random variables or realizations from time $0$ to $T$}
\nomenclature[P, 07]{$\mathcal{N}(x; \mu, \sigma)$}{Gaussian density function with mean $\mu$ and covariance $\sigma$ taken at point $x$}
\nomenclature[P, 08]{$\mathcal{N}(\mu, \sigma)$}{Gaussian distribution with mean $\mu$ and covariance $\sigma$}
\nomenclature[P, 09]{$\mathcal Ber (p)$}{Bernoulli distribution of parameter $p$}
\nomenclature[P, 10]{$\sim$}{Indicates the law followed by a random variable}
\nomenclature[P, 11]{$p(x)$}{Probability of a realization $x$ (discrete case) and probability density function of $x$ (continuous case)}
\nomenclature[P, 12]{$\mathrm{D}_{\mathrm{KL}}(q \mid p)$}{Kullback-Leibler divergence between distributions $q$ and $p$}
\nomenclature[P, 13]{$\mathbb{E}(x)$}{Expectation of a random variable $x$}
\nomenclature[P, 14]{$\mathbb{E}_{p(\cdot)}(\cdot)$}{Expectation under the distribution $p(\cdot)$}
\nomenclature[P, 15]{$\mathrm{Var}(x)$}{Variance of a random variable $x$}
\nomenclature[P, 16]{$\mathrm{Cov}(x, y)$}{Covariance between the random variables $x$, and $y$}
\nomenclature[P, 17]{$[\mu_{\theta}^{x}, \sigma_{\theta}^{x}]$}{Mean and variance or covariance matrix of the random variable $x$ that depends on $\theta$}
\nomenclature[P, 18]{$\rho_{\theta}^{x}$}{Bernoulli parameter of the random variable $x$ depends on $\theta$}

\printnomenclature

% \printnomenclature
% \thispagestyle{empty}
% % !TEX root = latex_avec_réduction_pour_impression_recto_verso_et_rognage_minimum.tex

% \newglossaryentry{probabilisticmodels}{
%   name={Probabilistic Graphical Models},
%   description={\nopostdesc }, % no description
%   sort={aa} % to place it at the very beginning
% }
% \newglossaryentry{others}{
%   name={Others},
%   description={\nopostdesc},
%   sort={zz} % to place it at the end
% }
% Markov chains and Hidden Markov Models
% \newacronym[parent=probabilisticmodels]{dgms}{DGMs}{Deep Generative Models}
% \newacronym[parent=probabilisticmodels]{pmc}{PMC}{Pairwise Markov Chain}
% \newacronym[parent=probabilisticmodels]{hmc}{HMC}{Hidden Markov Chain}
% \newacronym[parent=probabilisticmodels]{tmc}{TMC}{Triplet Markov Chain}
% \newacronym{pmcs}{PMCs}{Pairwise Markov Chains}
% \newacronym{hmcs}{HMCs}{Hidden Markov Chains}
% \newacronym{tmcs}{TMCs}{Triplet Markov Chains}
% \newacronym[parent=probabilisticmodels]{dpmc}{DPMC}{Deep Pairwise Markov Chain}
% \newacronym[parent=probabilisticmodels]{dtmc}{DTMC}{Deep Triplet Markov Chain}
% \newacronym[parent=probabilisticmodels]{spmc}{SPMC}{Semi Pairwise Markov Chain}

\newacronym{dgms}{DGMs}{Deep Generative Models}
\newacronym{pmc}{PMC}{Pairwise Markov Chain}
\newacronym{hmc}{HMC}{Hidden Markov Chain}
\newacronym{tmc}{TMC}{Triplet Markov Chain}
\newacronym{dpmc}{DPMC}{Deep Pairwise Markov Chain}
\newacronym{dtmc}{DTMC}{Deep Triplet Markov Chain}
\newacronym{spmc}{SPMC}{Semi Pairwise Markov Chain}

\newacronym{gum}{GUM}{Generative Unified Model}
% Sequential Monte Carlo (SMC)
\newacronym{smc}{SMC}{Sequential Monte Carlo}
% \newacronym{hmm}{HMM}{Hidden Markov Model}
\newacronym{mc}{MC}{Markov Chain}
% \newacronym{mcmc}{MCMC}{Markov Chain Monte Carlo}
\newacronym{map}{MMAP}{Marginal Maximum A Posteriori}
% Variational Inference
\newacronym{vi}{VI}{Variational Inference}
\newacronym{vbi}{VBI}{Variational Bayesian Inference}
\newacronym{vae}{VAE}{Variational AutoEncoder}
\newacronym{cvae}{CVAE}{Conditional Variational AutoEncoder}
\newacronym{ae}{AE}{AutoEncoder}
\newacronym{kld}{KLD}{Kullback-Leibler Divergence}
\newacronym{elbo}{ELBO}{Evidence Lower Bound}
% Neural Networks
\newacronym{nns}{NN}{Neural Network}
\newacronym{gan}{GAN}{Generative Adversarial Network}
% \newacronym{dnn}{DNN}{Deep Neural Network}
\newacronym{dl}{DL}{Deep learning}
\newacronym{dnns}{DNN}{Deep Neural Network}
\newacronym{rnn}{RNN}{Recurrent Neural Network}
\newacronym{srnn}{SRNN}{Stochastic Recurrent Neural Network}
% \newacronym{storn}{STORN}{Stochastic Recurrent Neural Network}
\newacronym{svrnn}{SVRNN}{Semi-supervised Variational Recurrent Neural Network}
\newacronym{vsl}{VSL}{Variational Sequential Labeler}
\newacronym{vrnn}{VRNN}{Variational Recurrent Neural Network}
\newacronym{cnn}{CNN}{Convolutional Neural Network}
\newacronym{gru}{GRU}{Gated Recurrent Unit}
\newacronym{lstm}{LSTM}{Long Short-Term Memory}
\newacronym{mlp}{MLP}{Multi-Layer Perceptron}
% Others
\newacronym{gan}{GAN}{Generative Adversarial Network}
\newacronym{em}{EM}{Expectation-Maximization}
\newacronym{fb}{FB}{Forward Backward} 
\newacronym{ml}{ML}{Maximum Likelihood}
\
\newacronym{mle}{MLE}{Maximum Likelihood Estimation}
\newacronym{map}{MAP}{Maximum A Posteriori}
\newacronym{sem}{SEM}{Stochastic Expectation-Maximization}
\newacronym{semipmc}{SPMC}{Semi Pairwise Markov Chain}
\newacronym{ppmc}{PPMC}{Partially Pairwise Markov Chain}
\newacronym{dppmc}{DPPMC}{Deep Partially Pairwise Markov Chain}
\newacronym{pdf}{pdf}{probability density function}

\newacronym{gepro}{GEPROMED}{Groupe Européen de Recherche sur les Prothèses Appliquées à la Chirurgie
Vasculaire}

\newacronym{gmms}{GMMs}{Gaussian Mixture Models}
\newacronym{gans}{GANs}{Generative Adversarial Networks}
\newacronym{ml}{ML}{Maximum Likelihood}

% \addcontentsline{toc}{chapter}{List of Acronyms}
\printglossary[type=\acronymtype, title = List of Acronyms]

% \thispagestyle{empty}


% \chapter*{List of Notations}
\addcontentsline{toc}{chapter}{Nomenclature}
% !TEX root = latex_avec_réduction_pour_impression_recto_verso_et_rognage_minimum.tex

% % \chapter*{Nomenclaturexxx}
% \renewcommand{\nomgroup}[1]{%
%   \item[\bfseries
%   \ifthenelse{\equal{#1}{G}}{General}{%
%   \ifthenelse{\equal{#1}{P}}{Probability Theory}{}}%
% ]}


% % General
% \nomenclature[G]{$a$}{Light lower case letters represent scalars or functions}
% \nomenclature[G]{$A$}{Upper case letters represent matrices}
% % \nomenclature[G]{$I$}{Identity matrix of size $n$}
% \nomenclature[G]{$\mathbb{R}^n$}{Set of real vectors of dimension $n$}
% \nomenclature[G]{$\mathbb{R}^{n \times m}$}{Set of real matrices of dimension $n \times m$}
% % \nomenclature[G]{$\mathcal{T}$}{Toeplitz matrix}
% \nomenclature[G]{$\mathrm{diag}(\cdot)$}{Diagonal matrix}
% \nomenclature[G]{$\cdot^{\top}$}{Transpose operator of a matrix or a vector}

% % Probability
% \nomenclature[P]{$x$}{Random variable and its realization. As far as the context is clear, we use the same symbol for both.}
% \nomenclature[P]{$\mathcal{N}(\cdot, \cdot)$}{Normal (Gaussian)  density function (first argument is the mean, 
% second is the variance or covariance matrix if the random variable is multidimensional)}
% \nomenclature[P]{$\mathcal Ber (\cdot)$}{Bernoulli distribution (the argument is the probability of success)}
% \nomenclature[P]{$\sim$}{Indicates the law followed by a random variable}
% \nomenclature[P]{$p(x)$}{Probability  of a realization $x$ (discrete case) and 
% probability density function of $x$ (continuous case)}
% \nomenclature[P]{$x_t$}{Random variable or realization at time $t$}
% \nomenclature[P]{$x_{0:T}$}{Sequence of random variables or realizations from time $0$ to $T$}
% \nomenclature[P]{$\mathrm{D}_{\mathrm{KL}}(q \mid  p)$}{Kullback-Leibler divergence between distributions $q$ and $p$}

% \nomenclature[P]{$\mathbb{E} (x)$}{Expectation of a random variable $x$}
% \nomenclature[P]{$\mathbb{E}_{p(\cdot)} (\cdot)$}{Expectation under the distribution $p(\cdot)$}
% \nomenclature[P]{$\mathrm{Var} (x)$}{Variance of a random variable $x$}
% \nomenclature[P]{$\mathrm{Cov} (x, y)$}{Covariance between the random variables $x$, and $y$}

\renewcommand{\nomgroup}[1]{%
  \item[\bfseries
  \ifthenelse{\equal{#1}{G}}{General}{%
  \ifthenelse{\equal{#1}{P}}{Probability Theory}{}}%
]}

% General
\nomenclature[G, 01]{$a$}{Light lower case letters represent scalars or functions}
\nomenclature[G, 02]{$A$}{Upper case letters represent matrices}
\nomenclature[G, 03]{$\mathbb{R}^n$}{Set of real vectors of dimension $n$}
\nomenclature[G, 04]{$\mathbb{R}^{n \times m}$}{Set of real matrices of dimension $n \times m$}
\nomenclature[G, 05]{$\mathbb{C}$}{Set of complex numbers}
\nomenclature[G, 05]{$\Re$}{Real part of a complex number}
\nomenclature[G, 06]{$\mathrm{diag}(\cdot)$}{Diagonal matrix}
\nomenclature[G, 07]{$\cdot^{\top}$}{Transpose operator of a matrix or a vector}
% sigm
\nomenclature[G, 08]{$\rm sigm $}{Sigmoid function $ \rm sigm (x) = 1/(1 + \exp(-x))$}
% Probability
\nomenclature[P, 01]{$x$}{(Observed) random variable and its realization. As far as the context is clear, we use the same symbol for both.}
\nomenclature[P, 02]{$y$}{Random variable and its realization, which represents the label associated with $x$}
\nomenclature[P, 03]{$z$}{Latent random variable and its realization}
\nomenclature[P, 04]{$x_t$}{Random variable or realization at time $t$}
\nomenclature[P, 05]{$x_{t:k}$}{Random variable or realization from time $t$ to $k$}
\nomenclature[P, 06]{$x_{0:T}$}{Sequence of random variables or realizations from time $0$ to $T$}
\nomenclature[P, 07]{$\mathcal{N}(x; \mu, \sigma)$}{Gaussian density function with mean $\mu$ and covariance $\sigma$ taken at point $x$}
\nomenclature[P, 08]{$\mathcal{N}(\mu, \sigma)$}{Gaussian distribution with mean $\mu$ and covariance $\sigma$}
\nomenclature[P, 09]{$\mathcal Ber (p)$}{Bernoulli distribution of parameter $p$}
\nomenclature[P, 10]{$\sim$}{Indicates the law followed by a random variable}
\nomenclature[P, 11]{$p(x)$}{Probability of a realization $x$ (discrete case) and probability density function of $x$ (continuous case)}
\nomenclature[P, 12]{$\mathrm{D}_{\mathrm{KL}}(q \mid p)$}{Kullback-Leibler divergence between distributions $q$ and $p$}
\nomenclature[P, 13]{$\mathbb{E}(x)$}{Expectation of a random variable $x$}
\nomenclature[P, 14]{$\mathbb{E}_{p(\cdot)}(\cdot)$}{Expectation under the distribution $p(\cdot)$}
\nomenclature[P, 15]{$\mathrm{Var}(x)$}{Variance of a random variable $x$}
\nomenclature[P, 16]{$\mathrm{Cov}(x, y)$}{Covariance between the random variables $x$, and $y$}
\nomenclature[P, 17]{$[\mu_{\theta}^{x}, \sigma_{\theta}^{x}]$}{Mean and variance or covariance matrix of the random variable $x$ that depends on $\theta$}
\nomenclature[P, 18]{$\rho_{\theta}^{x}$}{Bernoulli parameter of the random variable $x$ depends on $\theta$}

\printnomenclature


% \chapter*{List of Acronyms}
\addcontentsline{toc}{chapter}{Acronyms}
% !TEX root = latex_avec_réduction_pour_impression_recto_verso_et_rognage_minimum.tex

% \newglossaryentry{probabilisticmodels}{
%   name={Probabilistic Graphical Models},
%   description={\nopostdesc }, % no description
%   sort={aa} % to place it at the very beginning
% }
% \newglossaryentry{others}{
%   name={Others},
%   description={\nopostdesc},
%   sort={zz} % to place it at the end
% }
% Markov chains and Hidden Markov Models
% \newacronym[parent=probabilisticmodels]{dgms}{DGMs}{Deep Generative Models}
% \newacronym[parent=probabilisticmodels]{pmc}{PMC}{Pairwise Markov Chain}
% \newacronym[parent=probabilisticmodels]{hmc}{HMC}{Hidden Markov Chain}
% \newacronym[parent=probabilisticmodels]{tmc}{TMC}{Triplet Markov Chain}
% \newacronym{pmcs}{PMCs}{Pairwise Markov Chains}
% \newacronym{hmcs}{HMCs}{Hidden Markov Chains}
% \newacronym{tmcs}{TMCs}{Triplet Markov Chains}
% \newacronym[parent=probabilisticmodels]{dpmc}{DPMC}{Deep Pairwise Markov Chain}
% \newacronym[parent=probabilisticmodels]{dtmc}{DTMC}{Deep Triplet Markov Chain}
% \newacronym[parent=probabilisticmodels]{spmc}{SPMC}{Semi Pairwise Markov Chain}

\newacronym{dgms}{DGMs}{Deep Generative Models}
\newacronym{pmc}{PMC}{Pairwise Markov Chain}
\newacronym{hmc}{HMC}{Hidden Markov Chain}
\newacronym{tmc}{TMC}{Triplet Markov Chain}
\newacronym{dpmc}{DPMC}{Deep Pairwise Markov Chain}
\newacronym{dtmc}{DTMC}{Deep Triplet Markov Chain}
\newacronym{spmc}{SPMC}{Semi Pairwise Markov Chain}

\newacronym{gum}{GUM}{Generative Unified Model}
% Sequential Monte Carlo (SMC)
\newacronym{smc}{SMC}{Sequential Monte Carlo}
% \newacronym{hmm}{HMM}{Hidden Markov Model}
\newacronym{mc}{MC}{Markov Chain}
% \newacronym{mcmc}{MCMC}{Markov Chain Monte Carlo}
\newacronym{map}{MMAP}{Marginal Maximum A Posteriori}
% Variational Inference
\newacronym{vi}{VI}{Variational Inference}
\newacronym{vbi}{VBI}{Variational Bayesian Inference}
\newacronym{vae}{VAE}{Variational AutoEncoder}
\newacronym{cvae}{CVAE}{Conditional Variational AutoEncoder}
\newacronym{ae}{AE}{AutoEncoder}
\newacronym{kld}{KLD}{Kullback-Leibler Divergence}
\newacronym{elbo}{ELBO}{Evidence Lower Bound}
% Neural Networks
\newacronym{nns}{NN}{Neural Network}
\newacronym{gan}{GAN}{Generative Adversarial Network}
% \newacronym{dnn}{DNN}{Deep Neural Network}
\newacronym{dl}{DL}{Deep learning}
\newacronym{dnns}{DNN}{Deep Neural Network}
\newacronym{rnn}{RNN}{Recurrent Neural Network}
\newacronym{srnn}{SRNN}{Stochastic Recurrent Neural Network}
% \newacronym{storn}{STORN}{Stochastic Recurrent Neural Network}
\newacronym{svrnn}{SVRNN}{Semi-supervised Variational Recurrent Neural Network}
\newacronym{vsl}{VSL}{Variational Sequential Labeler}
\newacronym{vrnn}{VRNN}{Variational Recurrent Neural Network}
\newacronym{cnn}{CNN}{Convolutional Neural Network}
\newacronym{gru}{GRU}{Gated Recurrent Unit}
\newacronym{lstm}{LSTM}{Long Short-Term Memory}
\newacronym{mlp}{MLP}{Multi-Layer Perceptron}
% Others
\newacronym{gan}{GAN}{Generative Adversarial Network}
\newacronym{em}{EM}{Expectation-Maximization}
\newacronym{fb}{FB}{Forward Backward} 
\newacronym{ml}{ML}{Maximum Likelihood}
\
\newacronym{mle}{MLE}{Maximum Likelihood Estimation}
\newacronym{map}{MAP}{Maximum A Posteriori}
\newacronym{sem}{SEM}{Stochastic Expectation-Maximization}
\newacronym{semipmc}{SPMC}{Semi Pairwise Markov Chain}
\newacronym{ppmc}{PPMC}{Partially Pairwise Markov Chain}
\newacronym{dppmc}{DPPMC}{Deep Partially Pairwise Markov Chain}
\newacronym{pdf}{pdf}{probability density function}

\newacronym{gepro}{GEPROMED}{Groupe Européen de Recherche sur les Prothèses Appliquées à la Chirurgie
Vasculaire}

\newacronym{gmms}{GMMs}{Gaussian Mixture Models}
\newacronym{gans}{GANs}{Generative Adversarial Networks}
\newacronym{ml}{ML}{Maximum Likelihood}

% \addcontentsline{toc}{chapter}{List of Acronyms}
\printglossary[type=\acronymtype, title = List of Acronyms]


%------------------------------------
%NCLUDED CHAPTERS:
\mainmatter
\chapter*{Introduction générale}
\markboth{INTRODUCTION}{INTRODUCTION} 
\addcontentsline{toc}{chapter}{Introduction générale}


% \section*{Context} 
% This thesis explores and models sequential data through the application of
% various probabilistic models with latent variables, complemented by deep neural
% networks. The motivation for this research is the development of dynamic models
% that adeptly capture the complex temporal dynamics inherent in sequential data.
% Designed to be versatile and adaptable, these models aim to be applicable across
% domains including classification, prediction, and data generation, and adaptable
% to diverse data types. 
% The research focuses on several key areas, each detailed
% in its respective chapter. Initially, the fundamental principles of deep
% learning, and Bayesian estimation are introduced. Sequential data modeling is then
% explored, emphasizing the Markov chain models, which set the stage for the
% generative models discussed in subsequent chapters. 
% In particular, the research delves into the sequential Bayesian classification 
% of data in supervised,
% semi-supervised, and unsupervised contexts. The integration of deep neural
% networks with well-established probabilistic models is a key strategic aspect of
% this research, leveraging the strengths of both approaches to address complex
% sequential data problems more effectively. This integration leverages the
% capabilities of deep neural networks to capture complex nonlinear relationships,
% significantly improving the applicability and performance of the models.


% In addition to our contributions, this thesis also
% proposes novel approaches to address specific challenges posed by the
% \gls*{gepro}. These proposed solutions reflect the practical and a possible
% impactful application of this research, demonstrating its potential
% contribution to the field of vascular surgery.
\vspace{.65cm}
\section*{Contexte}
Cette thèse vise à modéliser des données séquentielles à travers l'utilisation
de modèles probabilistes à variables latentes et paramétrés par des
architectures de type réseaux de neurones profonds. Notre objectif est de
développer des modèles dynamiques capables de capturer des dynamiques
temporelles complexes inhérentes aux données séquentielles tout en étant
applicables dans des domaines variés tels que la classification, la prédiction
et la génération de données pour n'importe quel type de données séquentielles.\\


Notre approche se concentre sur plusieurs problématiques liés à la modélisation
de ce type de données, chacune étant détaillé dans un chapitre de ce manuscrit.
Dans un premier temps, nous balayons les principes fondamentaux de
l'apprentissage profond et de l'estimation bayésienne. Par la suite, nous nous
focalisations sur la modélisation de données séquentielles par des modèles de
Markov cachés qui constitueront le socle commun des modèles génératifs
développés par la suite. Plus précisément, notre travail s'intéresse au problème
de la classification (bayésienne) séquentielle de séries temporelles dans
différents contextes : supervisé (les données observées sont étiquetées) ;
semi-supervisé (les données sont partiellement étiquetées) ; et enfin non
supervisés (aucune étiquette n'est disponible). Pour cela, la combinaison de
réseaux de neurones profonds avec des modèles probabilistes markoviens vise à
améliorer le pouvoir génératif des modélisations plus classiques mais pose de
nombreux défis du point de vue de l'inférence bayésienne : estimation d'un grand
nombre de paramètres, estimation de lois à postériori et interprétabilité de
certaines variables cachées (les labels). En plus de proposer une solution pour
chacun de ces problèmes, nous nous intéressons également à des approches
novatrices pour relever des défis spécifiques en imagerie médicale posés par le
Groupe Européen de Recherche sur les Prothèses Appliquées à la Chirurgie
Vasculaire (GEPROMED).\\


\vspace{.20cm}
\section*{Plan}
Notre manuscrit est organisé en 5 chapitres. 

Le chapitre~\ref{chap:main_concepts} 
consiste en une introduction technique dans lequel nous discutons du principe de
l'apprentissage profond et de l'estimation bayésienne. Nous y introduisons
également des modèles Markoviens pour le traitement des données temporelles.

Le chapitre~\ref{chap:pmc} 
propose s’intéresse aux chaînes de Markov génératives,
en se concentrant spécifiquement sur les chaînes de Markov couples (PMCs). 
Nous montrons que ce modèle propose un cadre unificateur pour les modèles de
Markov cachés ainsi que les récentes architectures de type « réseaux de neurones
récurrents stochastiques ». Nous proposons une paramétrisation de ces modèles
basée sur des réseaux de neurones profonds et nous détaillons des méthodes
d'estimation paramétriques basées sur l'adaptation de l'inférence variationnelle
au cas séquentiel. Nous mettons en évidence le pouvoir génératif de ces nouveaux
modèles, tant d'un point de vue expérimental que théorique.

Le chapitre~\ref{chap:semi_supervised_pmc_tmc}  
vise à utiliser les modèles précédemment développés pour le problème de la
classification séquentielle de données. Dans la mesure où le cas supervisé ne
présente pas de difficultés supplémentaires par rapport aux techniques mises en
place dans le Chapitre 2, nous nous intéressons au cas où les étiquettes/labels
associés aux données ne sont que partiellement accessibles. Cette contrainte
nous amène à revoir les méthodes d'inférence variationnelle précédemment
discutées et à étendre nos modèles de manière à pouvoir prendre en compte deux
types de variables cachées : les variables latentes du modèles et les labels non
accessibles que l'on cherche à retrouver. Pour cela, nous faisons appel aux
modèles de Markov triplet. Notre approche est validée par des simulations
numériques portant sur le problème de segmentation d'images binaires en contexte
semi-supervisé.




Le chapitre~\ref{chap:unsp_pmc_tmc} 
étend le problème au cas non supervisé. L'application directe des méthodes
précédentes peut conduire à l'apprentissage de modèles probabilistes dans
lesquels la variable étiquette/label n'est pas interprétable physiquement (ex :
classe blanc/noir associée à un pixel en niveau de gris), en particulier dans
des modèles reposant sur un grand nombre de paramètres. Pour résoudre ce
problème, nous proposons des méthodes d'estimation ad-hoc visant à prendre en
compte cette contrainte d'interprétabilité. Pour ce faire, nous commençons avec
des modèles de Markov couple visant à modéliser le couple observation/label,
puis nous réintroduisons dans un second temps une troisième variable latente
continue visant à complexifier la loi du couple précédent. Les apports de nos
modèles couple/triplet, paramétrisés par des architectures profondes, ainsi que
de nos algorithmes d'estimation paramétrique sont évalués sur différentes tâches
telles que la segmentation d'images biomédicales ou la reconnaissance
d'activités humaines.

Enfin, le chapitre~\ref{chap:medical_perspectives}  
 donne quelques perspectives sur les outils développés précédemment pour des
problématiques relatives aux données manipulées par le GEPROMED. Nous y
décrivons quelques problématiques liées aux images médicales acquises dans un
cadre préopératoire, présentons des résultats préliminaires et proposons une
feuille de route pour s'attaquer aux différents défis restant. 


Finalement, le manuscrit s'achève par un résumé des résultats ainsi qu'une discussion
sur les orientations futures pour l'exploration et l'application des résultats
obtenus.


\thispagestyle{empty}
\chapter*{General introduction}
\markboth{INTRODUCTION}{INTRODUCTION}
\addcontentsline{toc}{chapter}{General introduction}


\section*{Context} 

This thesis explores and models sequential data through the application of
various probabilistic models with latent variables, complemented by deep neural
networks. The motivation for this research is the development of dynamic models
that adeptly capture the complex temporal dynamics inherent in sequential data.
Designed to be versatile and adaptable, these models aim to be applicable across
domains including classification, prediction, and data generation, and adaptable
to diverse data types. 
The research focuses on several key areas, each detailed
in its respective chapter. Initially, the fundamental principles of deep
learning, and Bayesian estimation are introduced. Sequential data modeling is then
explored, emphasizing the Markov chain models, which set the stage for the
generative models discussed in subsequent chapters. 
In particular, the research delves into the sequential Bayesian classification 
of data in supervised,
semi-supervised, and unsupervised contexts. The integration of deep neural
networks with well-established probabilistic models is a key strategic aspect of
this research, leveraging the strengths of both approaches to address complex
sequential data problems more effectively. This integration leverages the
capabilities of deep neural networks to capture complex nonlinear relationships,
significantly improving the applicability and performance of the models.


In addition to our contributions, this thesis also
proposes novel approaches to address specific challenges posed by the
\gls*{gepro}. These proposed solutions reflect the practical and  possible
impactful application of this research, demonstrating its potential
contribution to the field of vascular surgery.

% \newpage
% \vspace{.30cm}
\section*{Neural networks}
% A \gls*{nns} 
Neural Networks (NNs)
are foundational in machine learning, used for tasks like
classification, regression, and clustering. Their strength lies in
representation learning, the ability to discern a data representation that
simplifies model building. Structurally, NNs are composed of layers of 
(artificial) neurons,
where each neuron generates an output that is a non-linear function of a linear
combination of its inputs. This architectural feature allows NNs to model
complex and non-linear relationships within data. The power of NNs is
fundamentally based on universal approximation 
theorems~\citep{cybenko1989approximation,hornik1991approximation,pinkus1999approximation,lu2017expressive, liang2016deep}, 
which affirms its ability to approximate any continuous multivariable function. This
theoretical basis is fundamental to their versatility and adaptability in a
variety of problem domains.


% \gls*{dnns}, 
Deep Neural Networks (DNNs),
characterized by their multiple hidden layers, further extend this
capability. 
Unlike traditional NNs, DNNs have a significantly larger number of layers, which
allows them to learn more complex data representations.
DNNs are particularly well suited for understanding intricate data
patterns, which has led to state-of-the-art performance in areas like speech
recognition~\citep{deng2013new, chan2016listen, abdel2013exploring, nassif2019speech},
image classification~\citep{huang2023comparative},
image recognition~\citep{fu2017look, traore2018deep, zheng2017learning}, and
natural language processing~\citep{li2018deep,collobert2008unified, goldberg2017neural}.
Their depth, that is, the number of hidden layers,
enables deeper  learning of data features at various levels of
abstraction, making (deep) NNs particularly well suited for our context,
sequential data modeling.

% \newpage
\vspace{.30cm}
\section*{Generative models}
Generative models are designed to capture the underlying distribution of data,
allowing them to generate new data points similar to those observed. These
models are fundamental in fields such as image and speech recognition and
natural language processing, where it is essential to understand and reproduce
the complexity of natural data. The range of generative models spans from
classical probabilistic models to the more recent deep generative models.

Classical generative models, such as~\gls*{gmms}  and
\gls*{hmc} models, have been fundamental in statistical modeling,
providing a solid foundation for understanding data distributions and
dependencies~\citep{harshvardhan2020comprehensive}. 
On the other hand, deep generative
models, such as~\gls*{gans} ~\citep{goodfellow2020generative},
and~\gls*{vae}~\citep{kingma2014}, 
represent a more recent paradigm that integrates the power of deep learning.
Both classical and deep
generative models continue to evolve, driven by advancements in computational
power and algorithmic innovations, further expanding their applications and
capabilities in various domains.

% \newpage
\vspace{.30cm}
\section*{VAE and Variational Inference}

VAEs integrate probabilistic approaches with neural
networks, allowing for the generation of complex data structures with
variability and flexibility. 
They use latent variables to model complex, high-dimensional data structures in
a way that classical models cannot efficiently capture. These models are
characterized by their parametric nature, where parameters are usually
determined by~\gls*{ml} estimation. However, in VAEs, these parameters are often
derived from deep neural networks, which adds a layer of complexity to the
learning process. 
Given the complexity of VAEs, the likelihood function of these
models is often intractable. This difficulty makes direct likelihood
maximization impractical or even impossible. To address this 
 problem, \gls*{vi}~\citep{jaakkola2000bayesian,Blei_2017} 
is employed. VI offers a powerful approach to approximate the
intractable likelihood, allowing VAEs to be trained and used efficiently in a
variety of applications~\citep{an2015variational, pu2016variational,
xu2017variational, chira2022image}.


Despite their ability to capture complex data patterns, VAEs often take a
fundamentally static perspective. They typically process each data point
independently, without considering the temporal or sequential dynamics
characteristic of many real-world datasets. This limitation is especially
evident in contexts involving time series, video, or text, where the inherent
sequence aspect of the data is crucial.

\newpage
% \vspace{.75cm}
\section*{Probabilistic models}
Popular probabilistic models such as~\gls*{hmc}~\citep{rabiner1989tutorial}, 
\gls*{pmc}~\citep{pieczynski2003pairwise, derrode2004signal}, 
and~\gls*{tmc}~\citep{wp-cras-chaines3,pieczynski2005triplet}
models, are capable of capturing temporal dependencies and latent factors in
sequential data. Each of these models provides a fundamental framework for
processing sequential data, offering unique advantages and posing distinct
challenges.

HMC models are widely used to model sequences with both hidden and observed variables. 
The applications of HMC models are diverse, 
including natural language processing
for tasks such as part-of-speech labeling; computer vision for image
segmentation;  bioinformatics for genetic sequence 
analysis~\citep{rabiner1989tutorial, gales2008application, 
yoon2009hidden, li2021new, kupiec1992robust, paul2015hidden}, etc. 
PMCs and TMCs extend the fundamental principles of HMCs. They aim to
relax some underlying assumptions of HMCs by extending the direct
dependencies between random variables or by incorporating an additional third
latent process. The assumptions inherent in each of these models are fundamental.
Not only do they shape the structure of the model, but they also define the
nature of the relationships between variables, thus simplifying the inference
and learning processes. The adaptability of these models to different types of
data and their ability to capture complex dependencies make them particularly
well suited for sequential data modeling.

The parameter estimation is usually
performed by maximizing the likelihood function with respect to the parameters. 
However, when dealing with sequential data, the likelihood function can be intractable.
Depending on the model structure, this estimator can be approximated by 
\gls*{vi} methods~\citep{jaakkola2000bayesian,Blei_2017} or by the
\gls*{em} algorithm~\citep{dempster1977maximum}.

%   \citep{girin2020dynamical}

% \newpage
\vspace{.30cm}
\section*{Sequential Bayesian classification}

Classification is a fundamental task in machine learning, and Bayesian
classification is a widely adopted approach to this challenge. In this approach,
the main goal is to estimate the posterior distribution of classes (labels)
given the observations. This task takes on additional complexity in the context
of sequential data, where the observations are a sequence of random variables,
and each observation is associated with a corresponding label. The estimation of
these labels from the observations depends on the posterior distribution, which
is usually unknown and can be estimated using a parametric model. This model
selection process involves choosing a suitable generative model and a learning
algorithm to estimate the model parameters. Markov chain models, such 
as~\gls*{hmc}, \gls*{pmc} and \gls*{tmc} models, 
are particularly suitable for modeling sequential data with both hidden
and observed variables. 

In the context of sequential classification, the learning process is influenced
by the availability of labels associated with the observations. 
In supervised scenarios, where labels
are fully observed, learning involves estimating model parameters using both
observations and labels. While in semi-supervised contexts, where only a subset of
labels is available, the challenge is to estimate the parameters from the
observations and this partial set of labels. Finally, in unsupervised learning,
no observed labels are available, then parameter estimation must be performed from
the observations alone.
Each of these learning contexts presents unique challenges and requires
specialized methodologies to address them effectively.



\section*{Colaboration with the GEPROMED}

GEPROMED\footnote{https://gepromed.com/en/aboutUs}
is a non-profit organization founded in 1993.
The organization emerged from the collaborative vision of Pr. Nabil Chakfé, a
vascular surgeon in Strasbourg, France, and Pr. Bernard Durand, an expert in the
mechanics of flexible materials. Their primary goal was to
investigate and understand the complications associated with vascular prostheses,
particularly focusing on the phenomena of tearing and rupture observed
post-implantation in patients.
GEPROMED is dedicated to promoting specialized learning methods, and
continuous quality improvement and ensuring patient safety in the field of
vascular surgery. A key area of focus for the organization is the advancement of
image processing techniques in vascular surgery. 
Previous research~\citep{gangloff2020probabilistic} has shown that deep 
learning methods, and probabilistic models
are very promising for addressing these challenges. 
For example, on medical image segmentation tasks,
probabilistic and deep learning methods have been shown 
good performance.


Medical image segmentation, a critical task in
this field, involves identifying regions of interest within images. These
identified regions are crucial for diagnosis, treatment planning and guidance of
surgical procedures. The overall goal is to develop automated approaches that
can be broadly applied to various types of medical images, thereby improving the
efficiency of diagnostics, and medical interventions.


% \newpage
\vspace{.30cm}
\section*{Contributions}

This thesis aims at proposing 
innovative methodologies that bridge the gap between
classical probabilistic models based on Markov Chains and deep neural networks,
specifically adapted to sequential data modeling. 
The results obtained have been presented at different national and international conferences
and published in peer-reviewed journals. Our contributions are detailed 
in the following sections and are based on the following publications:

\begin{itemize}
    \item \cite{morales2021variational}: Variational Bayesian inference for
    pairwise Markov models, In 2021 \textit{IEEE Statistical Signal Processing Workshop
    (SSP) (pp. 251-255). IEEE}.
    \item \citet*{gangloff2021general}:  A
    general parametrization framework for pairwise Markov models: An application
    to unsupervised image segmentation, In 2021 \textit{IEEE 31st International Workshop
    on Machine Learning for Signal Processing (MLSP) (pp. 1-6), IEEE}.
    \item \cite{morales2022pairwise}: Pairwise Markov Chains as Generative Models,
    \textit{Colloque GRETSI 2022}, (pp. 649–652).
    \item \citet*{gangloff2022chaines}: Chaînes de Markov cachées à bruit
    généralisé, \textit{Colloque GRETSI 2022},  (pp. 17–20).
    \item \citet*{gangloff2023deep}: Deep parameterizations of pairwise and
    triplet Markov models for unsupervised classification of sequential data,
    \textit{Computational Statistics \& Data Analysis, 180, 107663}.
    \item \cite{morales2023probabilistic}: 
    A Probabilistic Semi-Supervised Approach with Triplet Markov Chains, In 2023
    \textit{IEEE 33rd International Workshop
    on Machine Learning for Signal Processing (MLSP) (pp. 1-6). IEEE}
\end{itemize}

% In addition, 
This thesis also includes preliminary but promising results in the area
of low-resolution medical image segmentation. This area of research, while still
a work in progress, demonstrates the potential of our methodologies to make
significant advances in medical image analysis. Initial results are encouraging
and lay the groundwork for further exploration and refinement. 
These efforts are currently continuing with the goal of culminating 
in a future publication. 



% \newpage
\vspace{.30cm}
\section*{Outline}
This thesis is structured to introduce and explore  methodologies in
sequential data modeling, particularly through deep learning and Bayesian
estimation techniques. It provides a comprehensive examination of both
theoretical and practical aspects of generative models and their applications in
supervised,
semi-supervised and unsupervised classification tasks. 

The thesis comprises
five chapters. Chapter~\ref{chap:main_concepts} 
offers a technical introduction, discussing the
principles of deep learning, Bayesian estimation, and sequential data modeling
with Markov chains. This chapter sets the foundation by covering topics such as
maximum likelihood estimation with VI, and posterior distribution estimation.

Chapter~\ref{chap:pmc} 
delves into Generative Markov
Chains, specifically focusing on PMCs as a unified
model. It details parameter estimation methods, including general
parametrization and VI for PMCs, and presents experiments and
results that highlight the generative power of these models. 

Chapter~\ref{chap:semi_supervised_pmc_tmc}
extends the discussion to Generalized Hidden Markov Models for semi-supervised
classification. It introduces the problem of semi-supervised estimation in TMCs,
explores ELBO for semi-supervised learning, and describes the learning process. 
The chapter also includes experiments comparing deep TMCs
with existing models, and semi-supervised binary image
segmentation. 

Chapter~\ref{chap:unsp_pmc_tmc} 
addresses Markov Chains for unsupervised classification,
detailing Bayesian inference for PMCs and deep PMCs for unsupervised
classification. It further explores TMCs for unsupervised classification,
including VI, and deep TMCs. This
chapter also presents simulations and experiments on real datasets, such as
unsupervised segmentation of biomedical images and clustering for human activity
recognition. 

Chapter~\ref{chap:medical_perspectives}
shows a workflow adapted to data provided by the GEPROMED group,  and future
perspectives that can merge the models presented in the first chapters. 
Finally, the thesis concludes with a summary of findings, a
discussion on the implications of the research, and future directions for
exploration and application.


% % \katy{add medical images to the experiments} 
% % Ideas to present our beautiful results:

% % \textbf{Context: Time series}
% % We consider a sequence of observed random variables $\obs_T=(x_{0}, \dots, x_{T})$ 
% % with unknown distribution $p(\x_T)$.
% % Two related estimation problems are considered:
% % \begin{itemize}
% %     \item Prediction of the next observation $x_{T+1}$ given $\x_T$.
% %     \item This join distribution $p(\x_T)$ can be modeled by a family of distributions $\p(\x_T)$.
% % \end{itemize}
% %  It can be relevant to introduce a hidden sequence of random variables $\h_T=(h_{0}, \dots, h_{T})$
% %  which depends on $\x_T$. So, the generative distribution of $\x_T$ can be written as
% %  a marginalization of the joint distribution of $(\h_T,\x_T)$. $\p(\x_T)=\int \p(\h_T,\x_T) {\rm d}\h_T$. 

% %  -> framework of Bayesian estimation in a latent data model $\p(\x_T,\h_T)$.
 
% %  Problems:
% %  \begin{enumerate}
% %     \item Which family of parametric distributions $\p(\x_T)$ should we consider?
% %     \item For a given family of distributions $\p(\x_T)$, how to estimate the parameters $\theta$ from a given realization $\x_T$?
% %     \item For a given $theta$, how to compute or approximate quantities of interest such as the predictive distribution $\p(x_{T+1}|\x_T)$, any posterior/predictive distribution $\p(h_{t'}|\x_T)$, or the marginal likelihood $p(\x_T)$?
% %  \end{enumerate}



\thispagestyle{empty}
% !TEX root = latex_avec_réduction_pour_impression_recto_verso_et_rognage_minimum.tex
\chapter{Technical introduction}
\label{chap:main_concepts}

% \epigraph{
% ``It’s not an idea until you write it down.'' }{Ivan Sutherland}

% In this chapter  \\
\localtableofcontents

\pagebreak

% \section*{Introduction}
% \label{sec:introduction_ch1}

\section{Deep learning}
\label{sub:nn}

% \yohan{DONE it is important that in Chap 1, 
% you briefly explain that the gradient wrt
% the parameters of a DNN can be easily computed. FROM CHAPTER 4}

\subsection{Fundamental principle}
\label{sub:principle}
% \gls*{dnns}  
DNNs 
have significantly gained popularity in recent years
due to their remarkable performance in various tasks such as speech 
recognition~\citep{deng2013new, chan2016listen,
abdel2013exploring}, image recognition \citep{fu2017look, traore2018deep,
zheng2017learning}, natural language processing \citep{collobert2008unified,
goldberg2017neural}. %  including prediction and classification~\citep{li2018deep,huang2023comparative}. 
Mathematically, a DNN is a parameterized vector-valued function $\f(\obs)$, $\obs \in
\mathbb{R}^{d_\obs}$, constructed through the sequential and alternating
composition of linear and non-linear functions. 
If vector $\obs'$ represents the input to a specific hidden layer, the scalar
output of a neuron is computed as $\sigma(w\obs'+b)$, where
 $w\obs'$ is the dot product of a vector of
weights $w$ and $\obs'$. Here $b$
represents the bias, and $\sigma(\cdot)$ is the (non-linear) activation
function. Common activation functions include sigmoid $(\rm sigm(x) = \frac{1}{1 + e^{-x}})$,
 hyperbolic  tangent $(\tanh(x) = \frac{e^x - e^{-x}}{e^x + e^{-x}})$, 
 and  Rectified Linear Unit (ReLU=$\max(0,x)$).


The set of parameters $\theta$ of a DNN,
which includes all weights and biases, enables these networks to act as
universal approximators, theoretically capable of approximating 
any vector-valued function $f(\obs)$ under some assumptions
% DNNs can be seen as universal approximators in the sense that $\f(\obs)$ can
% theoretically approximate any vector-valued function $f(\obs)$, under some assumptions
~\citep{cybenko1989approximation,hornik1991approximation,
pinkus1999approximation,lu2017expressive, liang2016deep}.
The estimation of $\theta$ relies on the observation that the gradient of $\f$
w.r.t. $\theta$ can be exactly computed with the backpropagation algorithm,
a foundational technique for learning in neural networks, as described
by~\cite{rumelhart1985learning,hecht1992theory}.
This efficiency is because the algorithm takes advantage of the chain rule of
computation to decompose the global gradient computation into a series of
simpler local gradient computations along the layers of the network. 
% These local
% gradients propagate backward from the output layer to the input layer, hence the
% name ``backpropagation''.
For example, in  a classification problem of an observation $\obs$, 
the function  $\f(\obs)$ 
aims at approximating the conditional probability \( P(Y = y \mid x) \) for all
\( y \) in  the 
% the true class $\lab$, which belongs to the
set $\Omega = \{\omega_1, \ldots, \omega_{C}\}$, 
where $C$ is the number of classes.
Provided that we have access to a labeled training dataset
$\mathcal{D} = \{(\obs^i, \lab^i)\}_{i=1}^n$, it is possible to
minimize a loss function $\mathcal{L}(\mathcal{D})$, \eg~the cross-entropy loss,
 with a gradient descent approach~\citep{ruder2016overview}. 

% In supervised learning, DNNs are trained on labeled
% datasets, allowing the network to learn to predict the output $\lab$ 
% directly from the input data $\obs$. In
% semi-supervised learning, DNNs utilize a combination of a small amount of
% labeled data and a large amount of unlabeled data. This approach leverages the
% structure learned from the labeled data to make predictions about the unlabeled
% data, often leading to improved generalization over purely supervised methods.
% Unsupervised learning, on the other hand, does not use labeled data at all.
% Instead, DNNs are tasked with discovering the underlying patterns and
% distributions within the data, such as clustering or density estimation. Each of
% these learning strategies can exploit the representational power of DNNs to
% extract complex features and relationships within the data.


% \katyobs{Here a deleted remark about supervised, semi-supervised and unsupervised learning.}
% \begin{remark}
%     When the labels $\lab$ are available for all the observations $\obs$,
%     we have a supervised learning problem.
%     In the case when the labels  are not available for all the
%     observations, we have a semi-supervised learning problem characterized
%     by the availability of a dataset $\mathcal{D}_2$ contains both unlabeled data
%     $\{\tilde{\obs}^j\}_{j=1}^m$ and a subset of labeled data $\{(\obs^i, \lab^i)\}_{i=1}^n$
%     Finally, an unsupervised learning problem is
%     characterized by an unlabeled dataset $\mathcal{D}_3 =\{\obs^i\}_{i=1}^n$.   
%     We will discuss these learning problems in more detail in next chapters.  
% \end{remark}

\subsection{Deep neural networks architectures for sequential data}
\label{subsec:neural_networks}


While classic DNNs have demonstrated significant versatility and
power in various domains, their conventional architectures may not be optimal 
for processing sequential data, such as time series, audio signals, or textual
content. Different types of neural networks have been developed to address this
issue, such as~\gls*{rnn}~\citep{fausett2006fundamentals, 
medsker2001recurrent, mikolov2014learning}.
RNNs are  architected to process
sequential information, where dependencies exist across temporal intervals. 
This capability is achieved by incorporating recurrent connections within the
network, allowing information to be retained across time steps.
% RNNs are distinguished by their ability to encode temporal information in the
% network architecture, which is achieved through loops that act as memory gates.
The design allows an RNN to not only process the current input, but also to use
the context provided by previously received inputs. 
For instance, when predicting the next word in a sentence, the RNN considers the
sequence of words that preceded it to make a more accurate prediction.
% This architecture is particularly useful for tasks such as speech
% recognition~\citep{shewalkar2019performance}, language
% modeling~\citep{kombrink2011recurrent, xiao2020research}, and machine
% translation~\citep{sundermeyer2014translation}. 


In contrast to classic DNNs,
the parameters $\theta$ in an RNN are shared across different time steps,
rather than learning a separate set of parameters for each moment in time.
This sharing reduces the model's complexity and enables the RNN to generalize across
sequences of different
 lengths. At each time step $t$, the hidden
state $\lat_t \in \mathbb{R}^{d_\lat}$ of the RNN is updated based on the current
input $\obs_t$ and the
previous hidden state $\lat_{t-1}$. The RNN's output $o_t$  at time $t$ is
computed based on the hidden state $\lat_t$.
This model can be expressed as follows:
\begin{eqnarray} 
    \label{eq:rnn_v1} \lat_t&=&\f(\lat_{t-1},\obs_{t}) 
    \text{, } \text{for all } t\in\NN \text{,}\\
    \label{eq:rnn_v2} o_t&=&\g(\lat_{t})
    \text{, } \text{for all } t\in\NN \text{.} 
\end{eqnarray} 
Here $f_{\theta}$ and $\g$ are parameterized activation functions, \eg~neural networks.
Figure~\ref{fig:rnn_intro} illustrates the graphical representation of an RNN.
This architectural design enables the RNN to effectively handle data where 
current decisions depend on past information, such as time series data, speech, or text.
\begin{figure}[htb]
    \centering
    \includegraphics[width=0.45\textwidth]{Figures/Graphical_models/rnn_intro.pdf}
    \caption{Graphical representation of a Recurrent Neural Network.
    The recurrent connections between the nodes highlight the network's
    ability to process sequences of data by maintaining a `memory' of previous
    inputs through the hidden states.}
    \label{fig:rnn_intro}
\end{figure}

\begin{remark}
    The output $o_t$ of an RNN has a dual predictive capability. For instance,
    in a stock market analysis application, it could predict the label $\lab_t$
    categorizing market trends or forecast future stock prices $\obs_{t+1}$. 
    This versatility makes RNNs a tool of
    choice for various predictive modeling tasks.
    % The output  can be used to predict a label  of the input $\obs_t$, or to 
    % predict a next input in a sequence.
\end{remark}

Despite their advantages, RNNs are not without challenges.
They are particularly prone to issues of vanishing and exploding gradients, 
especially when dealing with longer sequences. 
To overcome these problems, architectures
such as~\gls*{lstm} and \gls*{gru} have been
developed. LSTMs and GRUs incorporate mechanisms that regulate the flow of
information, allowing the network to retain or forget information selectively.
This capability significantly improves their performance on tasks involving long
sequences or where the temporal gap between relevant information is large.
% , for example, incorporate gates that regulate information flow,
% thereby preserving relevant temporal information without the decay faced by
% traditional RNNs.
While these networks are beyond the scope of this thesis, interested readers can
refer to~\cite{sherstinsky2020fundamentals, LSTM, GRU} for more details.


\section{Bayesian estimation}
\label{sec:bayesian_estimation}


% Bayesian Estimation offers a  framework for dealing with uncertainty in
% model parameters. This approach integrates prior knowledge with new evidence,
% using Bayes' theorem to update the belief about the model parameters
% continuously. In a typical setting, Bayesian Estimation involves defining a
% prior distribution that encapsulates our beliefs about the parameters before
% observing any data. After data acquisition, the likelihood function, which
% represents the probability of observing the data given the parameters, is used
% to update these beliefs, resulting in the posterior distribution. 
% The utility of 
% Bayesian Estimation extends beyond simple parameter estimation; 
% it is also fundamental in the development of probabilistic models that can account for
% uncertainty and variability in complex systems. 



In the context of deep learning, we have seen how common DNNs, including RNNs, can be
used to approximate functions for various tasks. While these models are powerful,
they often do not directly account for the uncertainty inherent in real-world
data. Bayesian estimation extends the predictive power  by
incorporating a probabilistic framework capable of capturing not just the
observed data but also the underlying latent structures, such as the intrinsic
features of an image that are not immediately observable.

% Bayesian estimation provides a framework that models both the observed
% variables and latent structures, providing a more complete representation of
% data uncertainties.



In Bayesian Estimation, we deal with the observed random variable
 (r.v.) $\obs \in \mathbb{R}^{d_\obs}$ 
and the latent (unobserved or hidden) r.v. $\latent \in \mathbb{R}^{d_\latent}$, 
each playing a distinct role in the modeling process.
Throughout this thesis, we do not distinguish between random variables and their
realizations.
% The observed variable $\obs$ represents the data we have access to, while the
% latent variable $\latent$ captures the underlying structure of the data that is
% not directly observable.
% with focus on their joint distribution $p(\obs, \latent)$. 
Our interest, which will be explained in more detail later, lies in calculating
the posterior distribution
% For reasons that will be clearer later,
% our main interest is in computing the posterior distribution 
\begin{equation*}
    \label{eq:posterior}
    p(\Latent|\Obs) = \frac{p( \Latent, \Obs)}{p(\Obs)} \text{,}
\end{equation*}
which offers insights into the latent variables given
the observed data.
However, the direct computation of $p(\obs, \latent)$ 
is often impractical, whether due to the high-dimensional nature of the data,
which leads to computational complexity, or the unknown distributional
characteristics.
% To address this, we  introduce a parameterized model
Thus, we can start by parameterizing the joint distribution $p(\obs, \latent)$ 
with a set of parameters $\theta$,
leading to the model  $p_{\theta}(\obs, \latent)$.
% $p_{\theta}(\obs, \latent)$, where $\theta$ is the set of parameters.
% Parameterizing the joint distribution with a set of parameters
% % The joint distribution $p(\obs, \latent)$ is parameterized by a set of parameters
% $\theta$,  we obtain the model $\p(\obs, \latent)$.  
Once a class of distributions $\p$ has been chosen, the
objective is to estimate the parameter  $\theta$ from a realization $\obs$
in an unsupervised way, that is to say without observing $\latent$.
A common approach for parameter estimation is the Maximum-Likelihood (ML)
estimator,
$\hat{\theta}^{\mathrm{ML}} = \argmax_{\theta} \p(\obs) = \argmax_{\theta} \int \p(\latent, \obs) \d \latent$,
due to its statistical properties~\citep{Hube67,White-MLE}. 
However,  the ML estimator may not be tractable since $\p(\Obs)$ 
is not necessarily known
in a closed form. 
According to the
structure of $\p(\latent,\obs)$, the ML estimator can be approximated with a
gradient ascent method on the likelihood function, the
%  Expectation Maximization 
\gls*{em}
algorithm~\citep{dempster1977maximum} or a  
Variational Inference 
% \gls*{vi}
algorithm~\citep{jaakkola2000bayesian,Blei_2017}.
% % https://andrewcharlesjones.github.io/journal/em-and-vi.html

% The EM algorithm's approach to dealing with latent variables and its
% optimization process can be considered a particular case of the framework
% provided by VI due to their foundational similarities.
% We will focus on the latter in the following sections.
% % \katyobs{read the last paragraph and see if it is clear.}
% % When $\theta$ is estimated, the posterior distribution $\p(\latent|\obs)$ has to be
% % computed, or approximated.
% % Generally, this can be done by using a Monte-Carlo method such as normalized importance
% % sampling \citep{these-hesterberg}, however, in the context of high-dimensional
% % data, this approach can be computationally expensive.
% % Alternatively, VI can also be used to approximate the posterior
% % distribution with a simpler distribution, that is detailed below.


In summary, Bayesian estimation offers a probabilistic approach 
to modeling by considering
both observed and latent variables. This method provides a comprehensive
framework for understanding the underlying uncertainties in data. In practice,
computing the posterior distribution $\p(\latent|\obs)$ directly is often
infeasible due to high-dimensional data or unknown distribution characteristics.
VI provides a robust alternative by approximating the
true posterior with a simpler, and parameterized distribution
This approach  is discussed in the following subsection.


\subsection{Approximated Maximum Likelihood estimation with Variational Inference}
\label{subsec:vbi}

Given independent and identically distributed observations
$\{\obs^i\}_{i=1}^M$,  a
direct computation of $\argmax_{\theta} \p(\obs)$ becomes impractical except in
simpler cases, such as when $\p(\obs)$ is directly available (\eg~when $\latent$ is
discrete or in linear and Gaussian scenarios). 
VI offers a flexible and scalable alternative
for more complex models where such straightforward calculations are not feasible.
VI is introduced as a method for approximate inference in
models where the computation of the posterior distribution is complex or
intractable. Unlike Maximum Likelihood estimation, which focuses on finding
parameter values that maximize the likelihood of the observed data, VI 
approaches the problem by approximating the true posterior
distribution $\p(\latent|\obs)$ with a simpler, parameterized distribution 
$\q(\latent|\obs)$ (see \eg~\cite{Blei_2017} for a detailed introduction).



This method is the cornerstone of the Bayesian inference algorithms we propose in 
this thesis, for our (highly) parameterized models. 
Let us consider the general problem of computing or approximating a posterior
distribution $\p(\latent|\obs) \propto \p(\latent,\obs) $ known up to a constant
when $\obs$ is observed and $\Latent$ is latent. VI
relies on a parameterized
distribution $\q(\latent|\obs)$ that is optimized to fit the posterior
distribution $p(\latent|\obs)$ by minimizing the~\gls*{kld}  
%  Kullback-Leibler Divergence (KLD) 
 between $\q(\latent|\obs)$ and $\p(\latent|\obs)$, \ie 
\begin{align}
%\label{eq:DKL-1}
\dkl(\q,\p)& =\int \q(\latent|\obs) \log \left(\frac{\q(\latent|\obs)}{\p(\latent|\obs)} \right) \d \latent \geq 0 \text{,} \nonumber  \\
\label{eq:DKL-2}
&= \int \q(\latent|\obs) \log \left(\frac{\q(\latent|\obs)}{\p(\latent,\obs)} \right) \d \latent  + \log \left( \p(\obs) \right)
\end{align}
w.r.t. $\theta$.
The choice of the variational distribution $\q(\latent|\obs)$ is
critical, as the first term on the right-hand side of the above equation must be
computable or easily approximated, and subsequently optimized with respect to $\phi$. 
A popular choice of variational distribution is the mean-field approximation
\citep{bishop2006pattern} where the variational components of
$\latent=(\latent_1,\ldots,\latent_{d_\latent})$ are independent given $\obs$
and one set of parameters $\phi_i$ is associated to each component $\latent_i$,
\ie~$\q(\latent|\obs)=\prod_{i=1}^{d_x} q_{\phi_i}(\latent_i|\obs)$ and
$\phi=(\phi_1,\ldots,\phi_{d_\latent})$.


This approach also provides a parameter estimation method when some parameters
of the original model $\p$ are unknown. Indeed, 
we deduce from~\eqref{eq:DKL-2} that
\begin{equation}
\label{eq:elbo}
\log \p(\obs) \geq  - \int \q(\latent|\obs) \log 
\left(\frac{\q(\latent|\obs)}{\p(\latent,\obs)} \right) {\rm d}\latent 
= \elbo(\theta,\phi)\text{,}
\end{equation}
where equality holds when $\q(\latent|\obs)=\p(\latent|\obs)$.
%belongs to the same class of distributions as $\p$ 
%and $\phi=\theta$.




Computing the so-called \gls*{elbo}
% Evidence Lower Bound (ELBO) 
$\elbo(\theta,\phi)$ and next
maximizing it w.r.t. $(\theta,\phi)$ leads to a maximization of a lower bound of
the log-likelihood $\log \p(\obs)$. The resulting variational EM 
algorithm~\citep{variational-EM} is an alternative to the 
EM algorithm~\citep{dempster1977maximum}
when the original posterior $\p(\latent|\obs)$ is not available. 
% \katyobs{Add this remark????}
% \begin{remark}
% The EM algorithm can be considered a special case of VI, in which the
% variational distribution exactly matches the original posterior distribution.
% This means that EM assumes that the expectation on the posterior is computable
% and can be treated without approximations and, therefore, the divergence
% KL~\eqref{eq:DKL-2} becomes zero.
% \end{remark}
% \subsubsection*{Parameter Estimation}
\label{subsec:optimization_vae}
Our objective is to maximize the ELBO $\elbo(\theta,\phi)$  as defined in
Equation~\eqref{eq:elbo}, 
w.r.t. the parameters $(\theta, \phi)$.
% This optimization is typically performed using stochastic gradient ascent
% algorithms, or equivalently, by applying gradient descent on the negative
% ELBO. 
% To deal with scenarios where computing gradients
% directly is infeasible, we employ Monte Carlo estimators, which offer a
% practical solution to obtain unbiased estimates of the gradients using the
% statistical sampling technique.
% For continuous latent variables, we utilize the reparameterization
% trick~\citep{kingma2014} facilitates obtaining an unbiased estimator for the
% gradient of the ELBO. When $\Latent$ is discrete, the Gumbel-Softmax (G-S) trick
% is used~\citep{maddison2016concrete, jang2016categorical}. These techniques are
% detailed below.


To address scenarios whereThe, we
employ Monte Carlo estimators, which provide a practical solution for obtaining
unbiased gradient estimates using statistical sampling techniques. For
continuous latent variables, we use the reparameterization trick~\citep{kingma2014}, 
which facilitates obtaining an unbiased estimator for the gradient of the ELBO.
 When z is discrete, we
use the Gumbel-Softmax (G-S) trick~\citep{maddison2016concrete,
jang2016categorical}.
These techniques are detailed below

% Generally, the direct computation of the ELBO's gradient is intractable.
% A Monte Carlo estimator can be used for estimating the gradient with respect to
% $\theta$ and $\phi$.
% While unbiased gradient estimators with respect to $\theta$ are relatively
% straightforward to compute using samples from $\q(\Latent|\Obs)$, obtaining
% unbiased gradient estimators with respect to $\phi$ is more challenging, since
% the expectation is taken with respect to the variational distribution
% $\q(\Latent|\Obs)$, which itself depends on $\phi$.
% For continuous latent variables, we utilize the reparameterization trick~\citep{kingma2014} facilitates obtaining an unbiased estimator for the
% gradient of the ELBO. When $\Latent$ is discrete, the Gumbel-Softmax (G-S) trick is
% used~\citep{maddison2016concrete, jang2016categorical}.
% These techniques are detailed below.
% For discrete latent variables, an alternative approach,
% called the Gumbel-Softmax (G-S) trick~\citep{maddison2016concrete,
% jang2016categorical} is applied.

% Maximizing with respect to parameters indirectly maximizes the log-likelihood of
% our data. To deal with scenarios where computing
% gradients directly is infeasible, we employ Monte Carlo estimators, which offer
% a practical solution to obtain unbiased estimates of the gradients using the
% statistical sampling technique.

% For continuous latent variables, we utilize the reparameterization trick
% ~\citep{kingma2014},
% transforming the randomness in our model into a differentiable function with
% respect to , facilitating the use of gradient-based optimization methods. In
% cases where the latent variables are discrete, the Gumbel-Softmax (G-S) trick
% provides a differentiable approximation, enabling gradient computation in
% scenarios where traditional methods would falter.

% \katy{Here, I think that the orevious paragraph is confusing wrt the explanation
% of Monte Carlo estimation. I would suggest to remove the paragraph and to 
% add a simpler idea before we introduce a more detailed explanation of the
% reparameterization trick.}
\paragraph{Continuous latent variables: }
\label{subsec:reparameterization_trick}
The idea of the reparameterization trick 
is to rewrite the random variable $\Latent$
as a deterministic differentiable function of a random variable $\epsilon$,
%  function of a random variable $\epsilon$, 
that is independent of $\phi$~\citep{kingma2014}.
In other words, we want to rewrite the random variable $\Latent$ as
\begin{equation}
    \label{eq:reparameterization_trick}
    \Latent = g(\epsilon, \phi, \obs) \text{,}
\end{equation}
% $$\Latent = g(\epsilon, \phi, \obs),$$ 
where $\epsilon$ is independent of $\phi$ and $\obs$.
The expectations w.r.t $\q(\Latent|\Obs)$ can be then rewritten as
\begin{equation*}
    \label{eq:expectation_reparameterization}
    \E_{\q(\Latent|\Obs)}( f(z) ) = 
    \E_{p(\epsilon)} (g(\epsilon, \phi, \obs))\text{,}
\end{equation*}
and the gradients of the previous expectation w.r.t $\phi$, 
$$\nabla_{\phi} \E_{\q(\Latent|\Obs)}( f(z) ) =  \nabla_{\phi}
\E_{p(\epsilon)} f(g(\epsilon, \phi, \obs))\text{,}$$
% \begin{equation*}
%     \label{eq:grad_expectation_reparameterization}
%     \nabla_{\phi} \E_{\q(\Latent|\Obs)}( f(z) ) =  \nabla_{\phi}
%     \E_{p(\epsilon)} (f(z))\text{,}
% \end{equation*}
can be now estimated with a Monte Carlo estimator. 
We now obtain unbiased estimates of the gradient of the ELBO w.r.t $\phi$ and $\theta$.
The reparameterization trick is illustrated in 
Figure~\ref{fig:rep_trick_cont}
for the case of continuous latent variables.

\begin{figure}[htb]
\begin{subfigure}[b]{0.4\linewidth}
        \centering
        \includegraphics[width=0.9\textwidth]{Figures/Graphical_models/rep_trick_original.pdf}
        \caption{Original form.}
        \label{fig:rt_original}
\end{subfigure}
\begin{subfigure}[b]{0.5\linewidth}
        \centering
        \includegraphics[width=1.05\textwidth]{Figures/Graphical_models/rep_trick.pdf}
        \caption{Reparameterized form.}
        \label{fig:rt}
\end{subfigure}
\caption{Illustration of the reparameterization trick.
In the original form, we cannot compute the gradient of $f$ w.r.t $\phi$. While
in the reparameterized form, gradient of $f$ w.r.t $\phi$ is easily computed.
Diamonds indicate no stochasticity, while blue circles highlight its presence.
Figure based on~\citep{kingma2014}.}
\label{fig:rep_trick_cont}
\end{figure} 


\begin{example}
    \label{ex:gaussian_case}
    % \katy{Is it a good idea to add this example here? In the sense we are mixing
    % the concepts of Bayesian framework with deep learning. Maybe we can simplify the example
    % and add it in the next section}
    We present the Variational AutoEncoder model with a continuous latent variable, where 
    the joint distribution $\p(\Obs, \Latent)$ is factorized into 
    the prior distribution $\p(\Latent)$ and the conditional distribution 
    $\p(\Obs|\Latent)$, also called the probabilistic decoder.
    Here, the set of parameters $\theta$ could be the output of (deep)
    neural networks, which are estimated from a dataset 
    with the assumption that the data points are \iid. 
    The variational distribution (probabilistic encoder)
    $\q(\Latent|\Obs)$ is a multivariate Gaussian distribution with diagonal 
    covariance matrix,
    \begin{equation*}
        \q(\Latent|\Obs) = \mathcal{N}(\mu_{\phi}(\Obs), \diag(\sigma_{\phi}(\Obs))) \text{,}
    \end{equation*}
    where
    $\mu_{\phi}(\Obs)$ and $\sigma_{\phi}(\Obs)$ are the outputs of neural networks. 
    Next, a sample $z^{(m)}$ is drawn from $\q(\Latent|\Obs)$ with the 
    reparameterization trick,
    \begin{equation*}
        \latent^{(m)} = \mu_{\phi}(\Obs) + 
        \sigma_{\phi}(\Obs) \odot \epsilon^{(m)} \text{ for all } m = 1, \dots, M \text{,}
    \end{equation*}
    where $\epsilon^{(m)}$ is a sample from the standard Gaussian distribution
    and $\odot$ denotes the element-wise product.

    The ELBO~\eqref{eq:elbo} is then approximated with the 
    Monte Carlo estimator,
    \begin{equation*}
        \label{eq:elbo_vae_mc}
        \mathcal{Q}(\theta, \phi) \approx - \frac{1}{M} \sum_{m=1}^M \log 
        \frac{\q(\latent^{(m)}|\Obs)}
        { \p(\Obs|\latent^{(m)}) \p(\latent^{(m)}) }\text{.}
    \end{equation*}
    After this, the parameters $(\theta, \phi)$ are estimated by maximizing the 
    previous expression
    w.r.t  $(\theta, \phi)$ with a gradient ascent algorithm.
    Figure~\ref{fig:rt_example_gaussian} 
    illustrates the VAE model.
    
    \begin{figure}[htb]
        \centering
        % \includegraphics[width=0.5\textwidth]{Figures/Graphical_models/rep_trick_example_2.pdf}
        \includegraphics[width=0.95\textwidth]{Figures/Graphical_models/vae_gaussian.pdf}
        \caption{Illustration of a Gaussian-Variational AutoEncoder model.}
        \label{fig:rt_example_gaussian}
    \end{figure}
    
    \end{example}



\paragraph{Discrete latent variables:}
% \katy{Here, it has to be clear the use of the softmax and argmax functions
% in the training process.}
% We now focus on the case of discrete latent variables.
% The reparameterization trick, suitable for continuous variables, is not
% applicable here due to the discrete nature of the latent variable
% $\Latent$. Instead, 
% We can use the Gumbel-Max trick~\citep{gumbel1948statistical,
% maddison2014sampling} for sampling from a categorical distribution.
Let $\pi_c$ denote the probability of the class $c$, with the condition that
$\sum_{c=1}^C \pi_c = 1$. The Gumbel-Max trick~\citep{gumbel1948statistical,
maddison2014sampling} facilitates sampling from this
distribution by adding \iid  Gumbel (noise) samples to the
log-probabilities $\log \pi_c$.
The class corresponding to the highest resulting value is then selected as the
sample, \ie~$ k = \argmax_{c = 1, \dots, C} \big( \log \pi_c + G_c \big) \text{.}$
% \begin{equation*}
%     \label{eq:gumbel_max}
%     k = \argmax_{c = 1, \dots, C} \big( \log \pi_c + G_c \big) \text{.}
% \end{equation*}
Although the Gumbel-Max trick facilitates sampling, it does not inherently allow
for gradient-based optimization because the argmax operation is not
differentiable. 
% This poses a challenge in models such as Variational
% AutoEncoders (VAE), where the ability to backpropagate through the sampling
% process is crucial. 
To address this limitation, the Gumbel-Softmax
trick~\citep{maddison2016concrete, jang2016categorical} is used, which 
introduces a differentiable approximation to the
categorical distribution.
The G-S trick involves the softmax function, a continuous and
differentiable approximation of the $\argmax$ operation. It begins by
expressing the latent vector $\latent$ as a one-hot vector, 
\ie~$\latent \in \{0,1\}^C$, where~$C$ is the number of classes.
This  generates a $C$-dimensional vector $\latent^{G-S}$ 
within the range $[0,1]^C$, defined as
\begin{equation*}
    \label{eq:gumbel_softmax}
    \latent^{G-S}_{c}  = \frac{\exp((\log\pi_c + G_c)/\tau)}
    {\sum_{j}^{C}\exp((\log\pi_j + G_j)/\tau)} 
    \text{, for all } c = 1, \dots, C \text{,}
\end{equation*}
where $\tau$ is the temperature parameter, and $G_c$ is a Gumbel sample 
drawn from $\text{Gumbel}(0,1)$.
As the softmax temperature $\tau$ approaches  $0$, 
samples from the G-S distribution become one-hot and 
the G-S distribution becomes identical to the categorical distribution (more details 
in~\cite{maddison2016concrete}). 
The Gumbel-Max and Gumbel-Softmax tricks are illustrated in Figure~\ref{fig:rt_gumbel_summary}
with $C=3$. 


\begin{figure}[htb]
    \centering
    \includegraphics[width=0.6\textwidth]{Figures/Graphical_models/rep_trick_summary.pdf}
    \caption{Illustration of the Gumbel-Max and Gumbel-Softmax tricks with $C=3$.
    The blue circle represents the Gumbel samples drawn from  $\text{Gumbel}(0,1)$.
    The result of the Gumbel-Max trick is the index $c$ of the maximum value and 
    the result of the Gumbel-Softmax trick is a $C$-dimensional vector $\latent^{G-S}$ with values in $[0,1]^C$,
    which is a continuous, differentiable approximation of the $\argmax$.
    }
    \label{fig:rt_gumbel_summary}
\end{figure}


\begin{remark}
    \label{rem:gumbel_softmax}
    In machine learning models that require discrete decision-making 
    it is crucial to
    maintain consistency between the training and evaluation phases. The
    Straight Through Gumbel-Softmax~\citep{maddison2016concrete} technique addresses 
    this issue by using the
    G-S distribution for sampling during the forward step, followed
    immediately by an argmax operation to discretize the output into one-shot
    vectors. This ensures that the behavior of the model during training matches
    its evaluation. In the backward
    step, the original smooth probabilities from the G-S  distribution
    are used to compute gradients, thus maintaining differentiability and
    allowing efficient backpropagation. 
    This technique bridges the gap between
    the need for discrete outputs and the advantages of gradient-based
    optimization.
\end{remark}

% \begin{remark}
%     \katy{I think that this remark is not necessary and it is wrong}
%     Understanding the use of softmax and argmax functions in the context of forward
%     and backward passes during the neural network training process is important,
%     especially when dealing with discrete distributions and the inclusion of
%     stochastic elements like Gumbel noise.
%     For the forward pass, we select classes using argmax, and for the
%     backward pass, we rely on the softmax function to compute gradients while
%     preserving the essence of the choice made in the forward pass. The
%     Gumbel-Softmax trick bridges the gap between the discrete nature of the data and
%     the continuous operations required for gradient-based optimization, enabling the
%     effective training of models that involve categorical decisions.       
% \end{remark}



\subsection{Posterior distribution}
\label{sub:posterior_distribution}

In Bayesian Estimation, a fundamental objective is to compute the posterior
distribution $p(\latent|\obs)$, which provides insights into the hidden or latent
variable $\latent$ given the observed data $\obs$. 
However, as established earlier, direct
computation of this posterior is often infeasible due to the unknown or complex
nature of the joint distribution $p(\latent,\obs)$.
To address this, we can use approximation techniques such as
variational distributions~\citep{Blei_2017} and normalized importance 
sampling~\citep{doucet2001introduction}.
% \paragraph*{Variational Distribution: }
% \label{sub:variational_distribution}
As we previously discussed, the variational approach is a powerful tool for
approximating the posterior distribution 
$\p(\latent|\obs)$, which  involves defining a simpler, 
and parameterized variational distribution $\q(\latent|\obs)$. 
This distribution, often a tractable distribution such as a Gaussian, is
allows for efficient approximation and computation. 
% It is particularly effective
% in complex models where the true posterior is intractable.
The variational distribution depends on a set of parameters $\phi$ that can be
optimized by maximizing the ELBO $\elbo(\theta,\phi)$ in 
Equation~\eqref{eq:elbo} since the maximization
is w.r.t  $(\theta, \phi)$.
% \katy{here, we have to present a link between the previous section and this one. I mean 
% the parameters $\phi$ are obtained from the previous section and 
% now we are using them to approximate the posterior with the variational distribution, which
% as we said before, is a simpler distribution that allows to approximate the posterior.}
% \paragraph*{Normalized Importance Sampling: }
% \label{sub:normalized_importance_sampling}
% Another approach to approximate the posterior distribution is through normalized
% importance sampling. 
% This method involves drawing samples from an
% easier-to-sample distribution, known as the proposal distribution, and then
% weighting these samples to approximate the true posterior. The weights are
% calculated based on how likely each sample is under the true posterior relative
% to the proposal distribution. Normalized importance sampling is particularly
% useful when the variational approach is not feasible or when a non-parametric
% approximation of the posterior is desired. 


On the other hand, the normalized importance sampling technique involves selecting 
a proposal distribution that is easier to
sample from, calculating weights for these samples based on the ratio of the
posterior to the proposal distribution, and then normalizing these weights to
ensure they sum to one, thus transforming them into proper probabilities. 
We can use the variational distribution as the proposal
distribution because we generally choose a variational distribution from which
we can sample efficiently due to the optimization of the ELBO.  
% Normalized importance sampling is particularly useful when the variational
% approach is not feasible.
% This
% normalization process corrects for any bias introduced by the proposal
% distribution, allowing for accurate estimations of expected values under the
% posterior distribution.
% It provides a way to estimate
% expectations with respect to the true posterior without explicitly knowing its
% functional form. 
Both the variational distribution and normalized importance
sampling offer robust solutions for approximating the posterior distribution
% $p(\latent|\obs)$ 
in scenarios where direct computation is challenging. 
By leveraging these methods, we can gain valuable insights into the latent structures of
complex models, enhancing our understanding and predictive capabilities in
various applications of Bayesian Estimation.


% \yohan{we can use the variational distribution as proposal and we generally
% chose a variational distribution from which we can sample due to the
% optimization of the ELBO.}

\subsection{Discussion}
\label{sub:discussion}
% This discussion brings together the concepts introduced in the previous sections,
% highlighting the relation between Deep Learning and Bayesian Estimation
% methodologies and their applications in two significant areas: Generative
% Modeling and Unsupervised Bayesian Estimation/Classification.

This discussion synthesizes the concepts introduced in previous sections,
highlighting the relationship between Deep Learning and Bayesian Estimation
methodologies and their applications in significant areas like Generative
Modeling and Unsupervised Bayesian Estimation/Classification.

\paragraph*{Generative models:}
These models are mainly concerned with modeling the data distribution
$\p(\obs)$ or sampling new data points according to this distribution. 
The ability of generative models to learn complex distributions and 
generate new data is one of  their main advantages.
The connection with Bayesian estimation becomes evident when we consider that
$\p(\obs)$ can be expressed by the joint distribution and the posterior
distribution as $\p(\obs) = \frac{\p(\obs, \latent)}{\p(\latent|\obs)}$. 
Here, the latent variables $\latent$ play a crucial role in generative models
and are often used to capture the underlying structure of the data.
% Understanding or inferring these latent variables from the
% data involves computing the posterior distribution $\p(\latent|\obs)$, which
% is the core of Bayesian estimation.

% This relationship provides the basis for the entire generative process, which
% highlights the importance of the posterior in model formulation. 
On the other hand, a popular generative model is the Variational AutoEncoder,
which combines the principles of deep learning and Bayesian estimation to
generate new data samples (see Example~\ref{ex:gaussian_case}).
VAEs consist of two key components, an encoder and a decoder. 
The encoder, a neural network, maps the input data $\obs$
to a latent representation $\latent$, effectively approximating the posterior
$\p(\latent|\obs)$. Next, the decoder, another neural network, reconstructs the data
$\obs$ from the latent representation $\latent$, approximating $\p(\obs|\latent)$.
The integration of VAE into the broader context of deep learning highlights the
compatibility and complementarity of these frameworks. VAE provides a bridge the
representational capabilities of neural networks, and the probabilistic modeling
capabilities of Bayesian methods. This combination allows the creation of
powerful generative models that not only generate plausible and diverse data
samples, but also provide information about the underlying data distribution and
the latent structures present in it.


% exemplify the synergy between these probabilistic frameworks and neural networks.
% VAEs operate by encoding data into a latent space zz and reconstructing it back
% to xx, traversing through the latent variables that embody the posterior's
% intricacies. The encoder-decoder architecture inherent in VAEs not only enhances
% data generation capabilities but also provides a mechanism to probe the data's
% latent attributes, making it a robust tool for understanding complex data
% distributions.

\paragraph*{Unsupervised Bayesian classification:}
\label{sub:unsupervised_bayesian_estimation_classification}
In the context of unsupervised learning, Bayesian estimation methods are
adapted to provide insightful solutions. Here, the  latent variable $\latent$
can be redefined by a variable of interest $\lab$ ($\latent \leftarrow \lab$).
This adaptation enables the application of Bayesian inference techniques to
estimate $\lab$ from the observed data $\obs$,
leading to the computation of the posterior distribution $\p(\lab|\obs)$.
This approach overcomes the limitations of point estimation. Instead of
providing a single estimate of $\lab$, it provides access to the entire posterior
distribution of $\lab$. This comprehensive perspective is especially valuable in
unsupervised scenarios where direct observations of $\lab$ are not available. However,
the performance of this approach depends on the formulation of the model $\p(\lab, \obs)$. 
It is crucial that this model not only captures the relationship between $\obs$
and $\lab$, but also facilitates the interpretability of $\lab$ in an unsupervised context.
We can also consider a model $\p(\lab, \obs, \latent)$, where the latent variable $\latent$
is introduced to capture the relationship between $\obs$ and $\lab$.
This additional latent variable $\latent$ can help in capturing more complex, underlying
relationships within the data that might not be directly observable from $\obs$ alone.



\section{Sequential data modeling}
\label{sec:seq_data}

% \yohan{here you have to explain if you are interested in a generative model or
% directly classification. I will use the generative perspective, next introducing
% a latent variable to complexify the distribution of x and add a remark
% explaining that the model can also be used for classification if the latent
% variable becomes iinterpretable. Whence the interest of TMC (distinguishing
% latent and hidden variables)}
We now consider sequential data, building on the foundations presented in the
previous sections. Sequential data present unique challenges, particularly
 when it comes to
modeling temporal dependencies and extracting meaningful patterns over time.
This discussion leads us to focus on probabilistic models specifically designed
for sequential data, such as Hidden Markov Models (HMMs) and their extensions.
We denote a sequence of observations as 
$\obs_{0:T} = (\obs_0, \obs_1, \dots, \obs_T)$, 
where $T$ is the length of the sequence.
Similarly, we use $\latent_{0:T} = (\latent_0, \latent_1, \dots, \latent_T)$ to
denote a sequence of latent variables.

\subsection{Hidden Markov chains}
\label{sec:hmc}
% Our objective is to model the joint distribution $p(\obs_{0:T})$ with a
% parametric model $p_{\theta}(\obs_{0:T})$. A natural choice for sequential data
% is the family of Markov Chains (MCs), which incorporate the Markov property,
% \begin{equation}
%     \label{eq:markov_property}
%     \p(\obs_{0:T}) \overset{\text{MC}}{=}  \p(\obs_0) \prod_{t=1}^{T}  
%     \underbrace{\p(\obs_{t}|\obs_{t-1})}_{\p(\obs_{t}|\obs_{0:t-1})} 
%     \text{,} \quad \text{ for all } T \in \NN  \text{.}
% \end{equation}
% Here, $\obs_t$ depends only on $\obs_{t-1}$, and not on the complete previous
% observations $\obs_{0:t-1}$.
% We consider MCs that are homogeneous: 
% the transition probabilities $\p(\obs_{t}|\obs_{t-1})$ are the same for all $t$
% (see~\citep{bremaud2017discrete} for more details).

% MCs can be limited because of the previous Markovianity assumption.
% The introduction of a latent process $\{\lab_t\}_{t\in \NN}$ 
% can be relevant to overcome this limitation
% and to capture the dynamics of the system.
% After this, we are interested in the joint distribution of the sequence of observations
% and latent variables $\p(\obs_{0:T}, \lab_{0:T})$, for all $T \in \mathbb{N}$.
% Different models can be considered, depending on the dependencies between the 
% latent variables and the observations. 

HMCs are a class of probabilistic models
where the latent process is a Markov chain, 
and  the observations are conditionally independent
given the latent process and $\obs_t$ depends only on $\latent_t$.
The joint distribution of the sequence of observations and latent variables
is given by
\begin{equation}
    \label{eq:hmc_intro}
    \p(\latent_{0:T}, \obs_{0:T} ) \overset{\text{HMC}}{=} 
    \underbrace{\p(\latent_0)\prod_{t=1}^{T}  \p(\latent_{t}|\latent_{t-1})}_{\p(\latent_{0:T})}
    \;  \underbrace{\prod_{t=0}^{T}  \p(\obs_{t}|\latent_{t})}_{\p(\obs_{0:T}|\latent_{0:T})} 
    \text{,} \quad \text{ for all } T \in \NN  \text{.}
\end{equation}


% The Hidden Markov Chain model has found numerous applications in various domains,
% each with its (unique) interpretation of latent variables. 
% In speech recognition~\citep{gales2008application}, 
% the observations $\obs_{0:T}$
% consist of acoustic signals, with latent variables $\latent_{0:T}$ representing
% phonemes of the spoken content. In  bioinformatics~\citep{yoon2009hidden, li2021new},
% the observations could be sequences of
% DNA, RNA, or proteins, where $\latent_{0:T}$ correspond to coding regions or the
% secondary structures within these biological sequences. Furthermore, in natural
% language processing, as discussed by~\cite{kupiec1992robust, paul2015hidden},
% $\obs_{0:T}$ may depict a sentence, with $\latent_{0:T}$ capturing its grammatical structure or the
% syntactic patterns underlying the text. 
% In image segmentation~\citep{derrode2004signal}, $\latent_t$ could represent the
% class of a noisy observed pixel $\obs_t$.


When the parameters of the HMC are unknown, they can be estimated from a set of
observations that we have at our disposal.
The Maximum Likelihood (ML) approach for estimating the parameters of the HMC
has been widely theoretically studied in~\cite{douc2004asymptotic,Douc-ML-MIS}.
However, the distribution of the observations $\p(\obs_{0:T})$ is intractable 
in general. Therefore, the ML approach is not applicable. 
The distribution $\p(\obs_{0:T})$ can be approximated 
with~\gls*{smc}
 methods~\citep{livredoucetshort,chopin2020introduction}.
Nonetheless, the SMC methods are computationally expensive and differentiable
approximations to use gradient-based optimization methods could be a 
problem. This is due to the resampling steps of such 
algorithms~\citep{kantas2015particle}.
The EM algorithm~\citep{dempster1977maximum} is also an alternative approach for
estimating the parameters of the HMC~\eqref{eq:hmc_intro} 
(see Algorithm~\ref{algo:em_algorithm}).
% The E-step computes $ \mathcal{Q}(\theta|\theta^j)$ defined as
% the expectation of the complete log-likelihood
% $\log \p(\obs_{0:T}, \latent_{0:T})$ w.r.t the posterior distribution of the latent
% variables, where $\theta^j$ are the
% parameters at the $j$-th iteration. 
% % \begin{equation*}
% %     \mathcal{Q}(\theta|\theta^j)= \E_{p(\latent_{0:T}|\obs_{0:T},\theta^j)}
% %         \left[\log p(\obs_{0:T},\latent_{0:T}|\theta)\right] \text{.}
% % \end{equation*}
% Next, the M-step consists of maximizing the previous expression w.r.t $\theta$,
% and obtaining the new parameters $\theta^{j+1}  
% \leftarrow \argmax_{\theta} \mathcal{Q}(\theta|\theta^j)$. 
When the parameters of the HMC are estimated, the predictive
 distribution $\p(\obs_{T+1}|\obs_{0:T})$ can be sequentially computed or approximated.
SMC methods can be used to approximate this distribution.

% \begin{figure}[htb]
%     \begin{subfigure}[b]{0.48\linewidth}
%       \centering
%       \includegraphics[width=6cm]{Figures/Graphical_models/hmc.pdf}
%       \caption{HMM}
%       \label{fig:hmm}
%       \vspace{1.1cm}
%     \end{subfigure}
%     \hfill
%     \begin{subfigure}[b]{0.47\linewidth}
%       \centering
%       \includegraphics[width=6cm]{Figures/Graphical_models/rnn.pdf}
%       \caption{RNN}
%       \label{fig:rnn}
%       \vspace{1.1cm}
%     \end{subfigure}
  
%     \begin{subfigure}[b]{0.48\linewidth}
%       \centering
%       \includegraphics[width=6.0cm]{Figures/Graphical_models/gum.pdf}
%       \caption{GUM}
%       \label{fig:gum}
%     \end{subfigure}
%     \hfill
%     \begin{subfigure}[b]{0.48\linewidth}
%       \centering
%       \includegraphics[width=6.0cm]{Figures/Graphical_models/pmc.pdf}
%       \caption{PMC}
%       \label{fig:pmc}
%     \end{subfigure}
%     \caption{Conditional dependencies of the \gls*{hmc}, \gls*{rnn}, 
%     \gls*{gum}, and \gls*{pmc}. In the \gls*{rnn}, 
%     the hidden states $\latent_t$ are shown as diamonds to stress 
%     that they are no source of stochasticity. 
%     The \gls*{hmc}, \gls*{rnn}, and \gls*{gum} are particular cases of the \gls*{pmc}.}
% \end{figure}  

% This model has found many applications in signal processing such as
% tracking ($\latent_t$ represents the state vector of a target at time $t$ and
% $\Obs_t$ the associated noisy range bearing measurement) \citep{jazwinski2007stochastic},
% financial problems ($\latent_t$ represents the volatility of a financial time
% series) \citep{shepard1999filtering} or image segmentation ($\latent_t$ represents the class
% associated to a noisy observed pixel $\Obs_t$) \citep{derrode-wp-sp}.
% % While the HMC model is quite simple, it involves many challenges.

\subsection{Pairwise Markov chains}
\label{sec:pairwise_triplet_mc}
In HMCs, the latent process is Markovian, \ie~the latent
variable $\latent_t$ depends only on $\latent_{t-1}$. It can be relevant to consider
latent variables that depend on more than one previous latent variable.
It is also valid for the observations $\obs_{t}$, which can depend on more than
one previous latent variable.
For example, in the case of time series, the observations $\obs_{t}$ can depend
on the previous observation $\obs_{t-1}$ and the latent variable $\latent_{t-1}$.
We introduce the Pairwise Markov Chain (PMC)~\citep{pieczynski2003pairwise,
derrode2004signal, le2008fuzzy} model that relax the 
Markovianity assumption of the HMC. 
PMCs generalize the HMC
by considering the joint process of $\{\latent_t, \obs_{t}\}_{t\in \mathbb{N}}$
as a Markov chain.
The joint distribution of the sequence of observations and latent variables
is given by
\begin{equation}
    \label{eq:pmc_intro}
    \p(\latent_{0:T}, \obs_{0:T} ) \overset{\text{PMC}}{=} 
    \p(\latent_0, \obs_0) \prod_{t=1}^{T}  \p(\latent_{t}, \obs_{t}|\latent_{t-1}, \obs_{t-1})
    \text{,} \quad \text{ for all } T \in \NN  \text{.}
\end{equation}

The use of such models has been proposed
in past contributions, in simpler contexts where the sequence $\latent_{0:T}
\leftarrow \lab_{0:T}$ represents a series
of labels for the sequence of observations $\obs_{0:T}$. 
It has been shown that when the PMC model
is stationary, it is possible to propose an unsupervised estimation 
method to estimate jointly  $\theta$ and $\lab_t$ from $\obs_{0:T}$ provided
that the distribution of the observation given the hidden 
states is restricted to a set of classical distributions
such as the Gaussian one~\citep{gorynin2018assessing}.
Several questions then arise: how can we use the structure of PMCs
as generative models for modeling $\p(\obs_{0:T})$?
Can these models be adapted to unsupervised classification
scenarios where $\p(\lab_{0:T}, \obs_{0:T})$ is parameterized by deep neural networks?
% In essence, by integrating the versatility of DNNs with probabilistic models, 
% we can not only create models that predict and classify,
% but also generate new sequences that are statistically consistent with the
% original data.  
In next chapters we will focus on these questions.

% The stationary assumption can be relaxed by considering the TMC model with a third discrete
% latent process~\citep{lanchantin2004unsupervised}; in this case, the new process models, 
% the non-stationarity of the pair $(\lab_{0:T},\obs_{0:T})$ and
% the complete triplet model can also be estimated through an unsupervised
% procedure \citep{lanchantin2004unsupervised,gorynin2018assessing}.
% Finally, it is also possible to consider a large class
% of conditional distributions for the observations by the introduction of 
% copulas~\citep{derrode2013unsupervised, derrode2016unsupervised}. 


% PMCs have been traditionally applied to sequential classification tasks,
% especially in simpler contexts where the sequence $\latent_{0:T}
%  \leftarrow \lab_{0:T}$ represents a series
% of labels for the sequence of observations $\obs_{0:T}$.
% Several questions then arise: how can we use the structure of PMCs
% as generative models for modeling $\p(\obs_{0:T})$?
% Can these models be adapted to unsupervised classification
% scenarios where $\p(\lab_{0:T}, \obs_{0:T})$ is parameterized by deep neural networks?
% % In essence, by integrating the versatility of DNNs with probabilistic models, 
% % we can not only create models that predict and classify,
% % but also generate new sequences that are statistically consistent with the
% % original data.  
% In next chapters we will focus on these questions.



\subsection{Sequential generative models for Bayesian classification}
\label{sec:seq_gen_models}
We can introduce an additional level of complexity with Triplet Markov Chains 
(TMCs)~\citep{wp-cras-chaines3,pieczynski2005triplet}
for classification tasks. TMCs provide a refined framework in which we can model
not only the sequence of observations $\obs_{0:T}$ and their associated labels
$\lab_{0:T}$, but also incorporate an auxiliary sequence $\latent_{0:T}$, 
enriching the relationships within the data.
TMC  have been mainly used
with a discrete  auxiliary sequence $\latent_{0:T}$~\citep{gorynin2018assessing,
lanchantin2008unsupervised,pieczynski2007multisensor}.
In this thesis, we will focus on the case where the sequence $\latent_{0:T}$ is continuous.
Thus, we consider the joint distribution $\p(\lab_{0:T}, \obs_{0:T}, \latent_{0:T})$,
for all $T \in \mathbb{N}$, given by
% The joint distribution $\p(\lab_{0:T}, \obs_{0:T}, \latent_{0:T})$, 
% for all $T \in \mathbb{N}$, is given by
\begin{equation}
    \label{eq:tmc_intro}
    \p(\lab_{0:T}, \latent_{0:T}, \obs_{0:T} ) \overset{\text{TMC}}{=} 
    \p(\lab_0,  \latent_0, \obs_0) \prod_{t=1}^{T}  \p(\lab_{t}, 
    \latent_{t}, \obs_{t} |\lab_{t-1}, \latent_{t-1}, \obs_{t-1})
    \text{.}%  \text{ for all } T \in \NN  \text{.}
\end{equation}

In a supervised context, the sequence of labels $\lab_{0:T}$ is known, the
TMC can be seen as a PMC with an augmented representation of latent variables,
\ie~$\obs_t \leftarrow (\lab_t, \latent_t)$, for all $t \in \mathbb{N}$.
 However, the TMC
model becomes more interesting when the sequence $\{\lab_t\}_{t\in \mathbb{N}}$
corresponds to an unobserved  physical process of interest, and $\{\latent_t\}_{t\in
\mathbb{N}}$ is treated as a separate, distinct process.
However, the TMC model can be used for semi-supervised and unsupervised
classification tasks, where the labels $\lab_{0:T}$ are unobserved or
partially observed. In an unsupervised learning~\citep{lanchantin2004unsupervised}, 
we have to estimate the
parameters of the model which takes into account the interpretability of
$\lab_{0:T}$ and also the different roles of $\lab_{0:T}$ and $\latent_{0:T}$.
While in a semi-supervised context, the labels $\lab_{0:T}$ are partially observed, and
we look for estimating the missing labels associated to each sequence.
In both semi-supervised and unsupervised context, TMCs
combined with DNN can provide a more refined approach to Bayesian classification, 
which is one of the main objectives of this thesis.\\



% \katy{HERE}
% The use of such models has been proposed
% in past contributions. It has been shown that when the PMC model
% is stationary, it is possible to propose an unsupervised estimation 
% method to estimate jointly  $\theta$ and $\lab_t$ from $\obs_{0:T}$ provided
% that the distribution of the observation given the hidden 
% states is restricted to a set of classical distributions
% such as the Gaussian one~\citep{gorynin2018assessing}.
% The stationary assumption can be relaxed by considering the TMC model with a third discrete
% latent process~\citep{lanchantin2004unsupervised}; in this case, the new process models, 
% the non-stationarity of the pair $(\lab_{0:T},\obs_{0:T})$ and
% the complete triplet model can also be estimated through an unsupervised
% procedure \citep{lanchantin2004unsupervised,gorynin2018assessing}.
% Finally, it is also possible to consider a large class
% of conditional distributions for the observations by the introduction of 
% copulas~\citep{derrode2013unsupervised, derrode2016unsupervised}. 




% \subsection{Challenges in Sequential Data Modeling}
% \label{sec:discussion}
% In this section, we discuss questions that arise from 
% all the concepts and models presented in this chapter.
% These questions will be addressed in this thesis.






% \paragraph*{Choice of the Transition Distributions: }
% \label{sec:choice_transition_distributions}
% Both PMC and TMC models give rise to a new question from a modeling point of
% view: how to choose the transition distributions in~\eqref{eq:pmc_intro} and
% \eqref{eq:tmc_intro} to capture the dynamics of the system
% for a given application?
% % Different factorizations of the transition distributions can be considered.
% For example, in the case of time series, the observations $\obs_{t}$ can only depend
% on the previous observation $\obs_{t-1}$ and  latent variables $\lab_{t-1}$.
% In this case, the transition distribution can be factorized as
% \begin{equation*}
%     \label{eq:pmc_factorization}
%     \p(\lab_{t}, \obs_{t}|\lab_{t-1}, \obs_{t-1}) = 
%     \p(\lab_{t}|\lab_{t-1} \obs_{t-1}, \obs_{t}) \p(\obs_{t}|\lab_{t-1}, \obs_{t-1}) \text{.}
% \end{equation*}


% Chapters~\ref{chap:pmc}, 
% ~\ref{chap:semi_supervised_pmc_tmc}, 
% and~\ref{chap:unsp_pmc_tmc}
% present different choices of the transition distributions
% in~\eqref{eq:pmc_intro}, and~\eqref{eq:tmc_intro} 
% for different applications and different modeling objectives.
% % Chapter~\ref{chap:medical_perspectives}

% % Chapter~\ref{chap:pmc}
% % Chapter~\ref{chap:semi_supervised_pmc_tmc}
% % Chapter~\ref{chap:unsp_pmc_tmc}
% % Chapter~\ref{chap:medical_perspectives}

% \subsection*{2. Impact of PMC on the Distribution of Observations}
% \label{sec:impact_pmc}
% The introduction of the PMC model, which relaxes the Markovianity assumption of
% the HMC, has an impact on the distribution of the observations.
% Natural questions arise:  how to evaluate this impact?
% Can we compare the distributions $\p(\obs_{0:T})$
% induced by the HMC~\eqref{eq:hmc_intro}, 
% and the PMC~\eqref{eq:pmc_intro}?\\
% Notice that an evaluation of the impact of the TMC model on $\p(\obs_{0:T})$ is relevant
% when a physical interpretation of the latent variables $\lab_{0:T}$ is considered.
% Otherwise, the TMC model can be seen as a PMC as discussed in



% Chapter~\ref{chap:pmc} 
% presents a comparison of the distributions $\p(\obs_{0:T})$
% induced by the HMC~\eqref{eq:hmc_intro},
% and the PMC~\eqref{eq:pmc_intro} from a theoretical point of view, 
% with a particular case of the PMC model.
% This allows us to compare the ability of the HMC and the PMC to model $\p(\obs_{0:T})$.



% \subsection*{3. Neural Networks in PMC and TMC Models}
% \label{sec:neural_networks_pmc_tmc}
% For applications like image segmentation, some assumptions are introduced in
% the PMC and TMC models~\citep{derrode2004signal,gorynin2018assessing}, 
% which can be restrictive and may reduce their modeling capabilities.
% Exploiting the fact that (deep) NNs can act as universal approximators
% (see Subsection~\ref{subsec:neural_networks}), 
% a deep PMC (or TMC) could learn complex, non-linear relationships in the data more effectively
% than a traditional PMC.
% Can we propose a general parameterized approach for PMC~\eqref{eq:pmc_intro}
% and TMC~\eqref{eq:tmc_intro} models with NNs?


% Chapters~\ref{chap:pmc}, 
% ~\ref{chap:semi_supervised_pmc_tmc}, 
% and~\ref{chap:unsp_pmc_tmc}
% present general parameterizations of PMC~\eqref{eq:pmc_intro}
% and TMC~\eqref{eq:tmc_intro} models
% which include NNs.


% \subsection*{4. Interpretation of Latent Variables}
% \label{sec:interpretation_latent_variables}
% Bayesian estimation in PMC and TMC models can be challenging
% when the latent variables $\lab_{0:T}$ need to be interpreted.
% This challenge is particularly
% significant when dealing with highly parameterized models. 
% To maintain interpretability in the parameter estimation of
%  models~\eqref{eq:pmc_intro}
% and~\eqref{eq:tmc_intro},
% and to leverage the interpretability of an established model like HMC, 
% different approaches can be considered.


% In Chapters~\ref{chap:semi_supervised_pmc_tmc}, 
% and~\ref{chap:unsp_pmc_tmc}, 
% we consider the case where the latent variables $\lab_{0:T}$ are
% interpreted as labels in a specific application.


% \subsection*{5. Computational Cost of PMC and TMC Models}
% \label{sec:computational_cost}
% The computational cost of the Bayesian estimation algorithms for PMC and TMC 
% models can be high, specially with high-dimensional observations. 
% Moreover, incorporating NNs into PMC and TMC models can increase the
% computational cost.
% Can we propose a general parameterized approach for PMC~\eqref{eq:pmc_intro}
% and TMC~\eqref{eq:tmc_intro} models (with NNs) that is 
% computationally efficient for high-dimensional observations?


% In Chapters~\ref{chap:pmc}, 
% ~\ref{chap:semi_supervised_pmc_tmc}, 
% and~\ref{chap:unsp_pmc_tmc},
% we propose a Variational Bayesian approach for general 
% parameter estimation in PMC and TMC models that is computationally efficient
% for high-dimensional observations.


% \subsection*{6. RNNs and Markov Chains Models}
% The comparison of generative RNNs with Markovian models
% can be relevant since they are generative models for sequential data.
% Can we compare their ability to model the distribution of observations 
% based on a sequence of random or deterministic variables? 

% A partial answer to this question is given in~\citep{salaun2019comparing}, 
% where the authors compare the ability of generative RNNs and HMCs to model the
% distribution of observations based on a sequence of random variables. 
% This comparison is made from a theoretical point of view with a unification of
% the two models in a common framework.
% This model is called Generative Unified Model (GUM), which
% is a particular case of PMC, and reads
% \begin{equation}
%     \label{eq:GUM}
%     \p(\obs_{0:T},\lat_{0:T})  \overset{\rm GUM}{=} 
%     \p(\lat_0, \obs_0)\prod_{t=1}^T 
%     \p(\lat_t|\lat_{t-1},\obs_{t-1})\p(\obs_t|\lat_{t}) \text{, }  
%     \text{for all } T \in \NN  \text{.}
% \end{equation}
% However, the comparison of generative RNNs with more complex Markovian models
% remains an open question.
% % and it is an extension of question 2.
% % ~\ref{sec:impact_pmc}.



% Chapter~\ref{chap:pmc} 
% introduces the Pairwise Markov Chain (PMC) model
% as a general generative model for sequential data, 
% which includes the HMC, the RNN, and the GUM as particular cases.
% This allows us to compare the ability those models to model
% the distribution of observations, which is an extension of question 2.


% \subsection*{7. Dynamical Variational Autoencoder}
% \label{sec:dynamic_vae}
% Variational Autoencoders are deep generative models that 
% have been widely used to represent high-dimensional complex data through 
% a low-dimensional latent space. 
% In the context of sequential data, the VAE model can be extended
% such that the latent space is not only used to represent the observations but
% also the temporal dependencies between the observations.
% This derives an extension of  the VAE model to a dynamical VAE model.
% % ~\ref{sec:neural_networks_pmc_tmc}.
% Can we propose Deep Pairwise Markov Chains or Deep Triplet Markov Chains 
% as a dynamical VAE? (extension of question 3).

% All the proposed models in this thesis (Chapters~\ref{chap:pmc}, 
% ~\ref{chap:semi_supervised_pmc_tmc}, 
% and~\ref{chap:unsp_pmc_tmc})
%  are generative models for sequential data and since they are adapted to
% high-dimensional observations, they can be seen as dynamical VAEs.


% \subsection*{8. Learning Step}
% Parameter estimation depends on the availability of data, and the specific application.
% Different learning contexts can be considered: supervised, semi-supervised, and
% unsupervised learning. For example, in the case of unsupervised learning and image segmentation,
% the noisy observations $\obs_{0:T}$ are available, but the labels sequence
% $\lab_{0:T}$ are not.
% The question then arises: How feasible is parameter estimation in PMC and TMC
% models across different learning contexts and applications?

% Chapter~\ref{chap:pmc} presents a 
% parameter estimation algorithm for PMC models in the supervised learning context.
% Chapter~\ref{chap:semi_supervised_pmc_tmc} 
% introduces a parameter estimation algorithm for TMC models in the
% semi-supervised learning context.
% In Chapter~\ref{chap:unsp_pmc_tmc}, 
% we propose a parameter estimation algorithm for PMC and TMC models in the
% unsupervised learning context. All these algorithms are presented in the
% context of sequential classification.



% \newpage
\subsection{Organization of the thesis}

In this thesis, we address several fundamental questions that arise from the
concepts and models presented. These questions guide the research and structure
of the thesis, ensuring a comprehensive exploration of the topics. The key
questions include:

% - how to build powerful generative models from pairwise Markov models and do we
% have guarantee on their modeling power? 

% - how to exploit the generality of TMC
% for semi-unsupervised classification (supervised would coincide with the
% previous question)? 

% - when we use the previous models with deep parameterization
% (so highly parameterized model), how do we deal with the challenging
% unsupervised case where in addition to a model we have to learn the
% interpretability of the hidden labels? 

% - what are the challenges in AI for
% vascular surgery which could be adressed with the techniques presented in this
% manuscript (applied perspective).



\begin{itemize}
    \item How can we build powerful generative models from PMC
    models~\eqref{eq:pmc_intro}, 
    and what guarantees do we have on their modeling power?    
    \item In sequential Bayesian classification, when only a subset of
    labels is observed, how can we estimate the unobserved labels from the
    observations and the observed labels using a general TMC model~\eqref{eq:tmc_intro}? 
    Does the
    supervised case coincide with the previous question?
    \item How can PMC and TMC models be applied to unsupervised classification
    tasks, and what are the challenges of ensuring interpretability of hidden random
    variables in unsupervised classification?
    \item Can these models be adapted to different scenarios
    where the distributions~\eqref{eq:pmc_intro} and~\eqref{eq:tmc_intro}  
    are parameterized by deep neural networks?
    \item What adaptations to the VI algorithm are necessary for 
    general parameter estimation of these models?
    \item What are the challenges in AI for vascular surgery that could be addressed
    with the techniques presented in this thesis (applied
    perspective)?
\end{itemize}

To systematically address these questions, the thesis is structured to first
introduce the fundamental principles and challenges of deep learning, 
 followed by a detailed exploration of Bayesian
estimation and its application in understanding complex data structures. Later
sections focus on models for sequential data such as the PMC and TMC models,
underscoring their theoretical foundations and practical implications


Chapter~\ref{chap:pmc} 
introduces the PMC model as a generative model that can model complex
dependencies between observations and latent variables. We discuss how it can
serve as a general model from which different models such as the HMC and the RNN,
among others, can be derived as specific cases. Additionally, we present a
general parameterization of the PMC model, which includes deep parameterization
(DNNs). In the second part, we develop an adapted VI algorithm for general
parameter estimation of this model. We also explore the linear and stationary
Gaussian PMC model for a theoretical analysis of its generative power.



Chapter~\ref{chap:semi_supervised_pmc_tmc}
is dedicated to the semi-supervised learning problem. We propose a probabilistic
approach to deal with sequential Bayesian classification when only a subset of
labels is observed. The goal is to estimate the unobserved labels from the
observations and the observed labels using a general TMC model 
(which includes a deep parameterization). We introduce a
new adaptation of VI, enabling us to estimate the parameters of the
model and the unobserved labels.


Chapter~\ref{chap:unsp_pmc_tmc}
focuses on the unsupervised classification task. We propose PMC and TMC models
for estimating unobserved labels associated with a sequence
of observations. For each introduced model, an original unsupervised Bayesian
estimation method is proposed. In particular, it considers the interpretability
of the hidden random variables in terms of classification.


Finally, chapter 5
% ~\ref{chap:medical_perspectives}
presents a workflow adapted to the data provided by
the GEPFROMED group, and future perspectives that integrate the classical
neural network models with a probabilistic approach. We also discuss the
potential of the proposed models for the segmentation of medical data.

% \begin{itemize}
%     \item  you did not propose Monte Carlo techniques. You can insist on the
%     fact that VBI is suitable due to the assumed large dimension of z. By the
%     way, in the introduction, you can explain that Monte Carlo approximation of
%     the EM may be inefficient in large dimensions.
%     \item Example of the DKL gaussian case
% \end{itemize}

% \begin{remark}
%     In the case of DPMCs, their
%     gradients are computable from the backpropagation
%     algorithm~\citep{rumelhart1986learning} since $\pz$ and $\px$ 
%     are differentiable w.r.t. $\theta$. 
%     Moreover, the parameters $\phi$ of the variational distribution $\q$
%     are the parameters of the DNNs which are differentiable w.r.t. $\phi$.
%     Thus, Algorithm~\ref{algo:algo_train_dpmc_gen} can be applied to DPMCs.
% \end{remark}


\thispagestyle{empty}

% !TEx root = latex_avec_réduction_pour_impression_recto_verso_et_rognage_minimum.tex
\chapter{Generative hidden  Markov models}
\label{chap:pmc}

\localtableofcontents
\pagebreak

\section{Introduction}
This chapter introduces the \gls*{pmc} as a general
generative model that can be used to model
complex dependencies between the observations and the latent variables.
First, we recall  the 
\gls*{pmc} model~\citep{pieczynski2003pairwise}, 
and introduce how it can be used as a general model for generative modeling, 
from which the \gls*{hmc}~\citep{rabiner1989tutorial}, the \gls*{rnn}~\citep{medsker2001recurrent},
and the \gls*{gum}~\citep{salaun2019comparing} can be derived as particular cases.
% can be derived as particular cases.
Moreover, we present a general parameterization of the \gls*{pmc} model, 
which includes a deep parameterization (DNNs).
In the second part, we develop an adapted VI
 algorithm~\citep{jaakkola2000bayesian,Blei_2017}
for a general parameter estimation of this model, 
which can be applied to any \gls*{pmc} model, linear or not, Gaussian or not.
We provide some experimental results, 
demonstrate \gls*{pmc} as a generative model, 
and see how it compares to other popular models 
that use latent variables and DNNs.
To conclude, we  focus on a particular instance of the \gls*{pmc} model,
the linear, and stationary Gaussian \gls*{pmc}, 
for a theoretical analysis of the generative power of the \gls*{pmc} model.
This analysis is based on the expressivity 
of the \gls*{pmc}, 
\ie~the distribution of the observations generated by the model.


% \section{General Pairwise Markov Chain}
% \subsection{Model definition}

\section{The pairwise Markov chain as a unified model}
\label{sec:pmc_unified_model}

% \yohan{Maybe put 2/3 lines to introduce the chapter : here we first cast popular
% generative models into a general PMC model. Next we propose a parameter
% estimation method based on sequential VI\@. We compare our generalization with
% current generative models. Finally, we try to quantify the interest of such
% models.}
We recall the notation introduced in Chapter~\ref{chap:main_concepts},
$\obs_{0:T}=(\obs_{0}, \dots,\; \obs_{T})$, $\obs_t \in \mathbb{R}^{d_{\obs}}$, and
$\latent_{0:T}=(\latent_{0}, \dots,\; \latent_{T})$,  $\latent_t \in \mathbb{R}^{d_{\latent}}$
which are two sequences of observed and latent  random variables (r.v.), respectively, 
of length $T+1$.
This chapter focuses on the case where the latent variables 
are not interpretable as the labels of the observations. In other words,
our interest is a generative model, where the latent variables are 
just an intermediate step to create a complex distribution of the observations.
For that, we recall the joint distribution of the observed and latent variables
in the PMC model~\eqref{eq:pmc_intro},  which is given by
\begin{equation*}
    % \label{eq:pmc_general}
    \p(\latent_{0:T}, \obs_{0:T}) = \p(\latent_0,\obs_0)\prod_{t=1}^T 
    \p(\latent_t,\obs_t|\latent_{t-1},\obs_{t-1}) \text{, } 
    \quad  \text{for all } T \in \NN  \text{.}
\end{equation*}
% % which is a direct generalization of \gls*{hmc} where 
% where we %only assume 
% assume that the pair $(\latent_T,\obs_T)$ is
% Markovian~\citep{pieczynski2003pairwise, le2008fuzzy, morales2021variational}.
% In such a model, $\latent_T$ is not necessarily a Markov chain,
% and the observations can now be dependent given $\latent_T$.

From a modeling point of view, the choice of the transition distribution
$\p(\latent_t,\obs_t|\latent_{t-1},\obs_{t-1})$ is a thorny problem.
The transition distribution can be factorized
in different ways. The choice of the factorization depends on the specific problem
and the underlying  assumptions  about the dependencies between variables.
In practice, we need to choose a transition distribution that has an impact on
the relevance of the model $\p(\obs_{0:T})$ and is able to fit the data.
To that end, we consider the following factorization for the transition distribution:
\begin{align}
    \label{eq:pmc_gen_transition}
    \p(\latent_t,\obs_t|& \latent_{t-1},\obs_{t-1}) =
    \p(\latent_t|\latent_{t-1},\obs_{t-1})\p(\obs_t|\latent_{t-1:t},\obs_{t-1}) \text{.}
    % % \label{eq:pmc_gen2}
    % \p(\latent_t,\obs_t|& \latent_{t-1},\obs_{t-1}) =
    % \p(\latent_t|\latent_{t-1},\obs_{t-1:t})\p(\obs_t|\latent_{t-1},\obs_{t-1})\text{.}
\end{align}
The joint distribution of $(\obs_{0:T}, \latent_{0:T})$ reads
\begin{equation}
    \label{eq:pmc_gen_t}
    \p(\latent_{0:T}, \obs_{0:T}) =
    % \overset{\rm PMC}{=} 
    \p(\latent_0,\obs_0)\!\prod_{t=1}^T 
    \p(\latent_t|\latent_{t-1},\obs_{t-1})\p(\obs_t|\latent_{t-1:t},\obs_{t-1}) \text{, } \text{for all } T \!\in\! \NN  \text{.}
\end{equation}
This factorization assumes that the observation $\obs_t$ depends 
on the previous latent variables
$\latent_{t-1}$ and $\latent_t$, 
and not only on the current latent variable $\latent_t$
and the previous observation $\obs_{t-1}$.
While the distribution of latent variable $\latent_t$ is determined 
by the previous observation $\obs_{t-1}$ 
and the previous latent variable $\latent_{t-1}$.
The choice of this factorization is motivated by the fact that some popular
generative models based on latent variables can be derived from it,
such as the \gls*{hmc}, the \gls*{rnn}  and the \gls*{gum}~\citep{salaun2019comparing}.


% This unified  PMC is presented in the Section~\ref{sec:pmc_unified_model}.

% In \citep{salaun2019comparing}, the GUM is proposed as a unified framework
% to compare the expressivity of generative models based on latent variables,
% where the HMC and the RNN are considered
% as particular cases of the GUM and compared to the GUM.
Our objective is to cast the HMC, RNN and GUM generative models into
a more general one, the \gls*{pmc}. 
To this end, we recall the \gls*{hmc}~\eqref{eq:hmc_intro}
where the joint distribution of the observed and latent variables 
reads
\begin{equation*}
    \p(\obs_{0:T},\latent_{0:T}) \overset {\rm HMC}{=} 
    \p(\latent_0, \obs_0)\prod_{t=1}^T \p(\latent_t|\latent_{t-1}) 
    \p(\obs_t|\latent_{t}) \text{, } \; \text{for all } T \in \NN \text{.}
\end{equation*}
Here $(\obs_t, \latent_t)$  becomes conditionally independent of $\obs_{t-1}$ given $\latent_{t-1}$, and 
$\obs_t$, in addition, does not depend on $\latent_{t-1}$. %(Figure~\ref{fig:hmm}). 
% This equation describes the joint probability distribution of the
% observed variables and hidden variables in the HMC model. It
% involves the initial hidden state distribution $\p(\latent_0)$,
% the transition probabilities from $\latent_{t-1}$  to $\latent_t$  $(\p(\latent_t|\latent_{t-1}))$,
% and the observation probabilities $\p(\obs_t | \latent_t)$.
Similarly, the \gls*{gum} is a particular case of  
%~\citep{salaun2019comparing}   
the \gls*{pmc} %(Figure~\ref{fig:gum}) 
defined as follows
% \begin{align*}
%     \p(\latent_t,\obs_t|\latent_{t-1},\obs_{t-1}) =  \p(\latent_t|\latent_{t-1},\obs_{t-1})\p(\obs_t|\latent_{t}) \text{.}
% \end{align*}
% The joint distribution of the observed and latent variables in the \gls*{gum} is given by
\begin{equation*}
    \p(\obs_{0:T},\latent_{0:T})  \overset{\rm GUM}{=} \p(\latent_0, \obs_0)
    \prod_{t=1}^T \p(\latent_t|\latent_{t-1},\obs_{t-1})\p(\obs_t|\latent_{t}) 
    \text{, }  \text{for all } T \in \NN  \text{,}
\end{equation*}
where $\obs_t$ becomes conditionally
 independent of $(\obs_{t-1},\latent_{t-1})$. 
In the case of predicting future observations with RNNs,
a probabilistic approach seems more appropriate when we want to quantify 
the uncertainty associated with our prediction. To do this, we simply replace $\g$
by a parametric distribution $\p$  and $o_t$ by $\obs_{t+1}$
in Equation~\eqref{eq:rnn_v2}.
In addition, we use the transformation  $\latent_{t} \leftarrow \lat_{t-1}$   in 
equations~\eqref{eq:rnn_v1}-\eqref{eq:rnn_v2}
to obtain the following model:
\begin{align*}
    &\p(\obs_{0:T})\overset {\rm RNN}{=}  \p(\obs_{0}) \prod_{t=1}^T 
    \p(\obs_t|\latent_t	) \text{, } \text{for all } T \in \NN \text{,}\\
    &\p(\latent_t|\latent_{t-1},\obs_{t-1})\overset {\rm RNN}{=} 
    \delta_{\f(\latent_{t-1},\obs_{t-1})}(\latent_t) \text{, } \;  \latent_0\overset {\rm RNN}{=} 0 \text{, } \; \p(\obs_{0}|\latent_{0}) \overset {\rm RNN}{=}  \p(\obs_{0}) \text{.}
\end{align*}
Contrary to the \gls*{hmc} and the GUM, the \gls*{rnn} follows a different approach.
In the \gls*{rnn}, the latent variable $\latent_t$ is deterministically determined,
given the previous observation $\obs_{t-1}$ and the previous latent variable $\latent_{t-1}$.
With a slight abuse of notation,  $\p(\latent_t|\latent_{t-1},\obs_{t-1})$ 
coincides with the Dirac measure and is not a probability density function.
%  $\delta_{\f(\latent_{t-1},\obs_{t-1})}(\latent_t)$
% where $\f$ is a deterministic function.
The expression of $\latent_t$ relies on an activation function $\f$.
Similar to the HMC and the GUM, in the RNN $\obs_t$ depends on $\latent_t$ given the past.
We can see a common underlying framework that captures the joint probability
distributions of the observed and latent variables in these models.
The graphical structures of the  models are summarized in 
Figure~\ref{fig:graphical_models}.



\begin{figure}[htb]
  \begin{subfigure}[b]{0.48\linewidth}
    \centering
    \includegraphics[width=6cm]{Figures/Graphical_models/hmc.pdf}
    \caption{HMC}
    \label{fig:hmm}
    \vspace{1.1cm}
  \end{subfigure}
  \hfill
  \begin{subfigure}[b]{0.47\linewidth}
    \centering
    \includegraphics[width=6cm]{Figures/Graphical_models/rnn.pdf}
    \caption{RNN}
    \label{fig:rnn}
    \vspace{1.1cm}
  \end{subfigure}

  \begin{subfigure}[b]{0.48\linewidth}
    \centering
    \includegraphics[width=6.0cm]{Figures/Graphical_models/gum.pdf}
    \caption{GUM}
    \label{fig:gum}
  \end{subfigure}
  \hfill
  \begin{subfigure}[b]{0.48\linewidth}
    \centering
    \includegraphics[width=6.0cm]{Figures/Graphical_models/pmc.pdf}
    \caption{PMC}
    \label{fig:pmc}
  \end{subfigure}

  \caption{Conditional dependencies of the \gls*{hmc}, \gls*{rnn}, 
  \gls*{gum}, and \gls*{pmc}. In the \gls*{rnn}, 
  the hidden states $\latent_t$ are shown as diamonds to stress 
  that they are no source of stochasticity. 
  The \gls*{hmc}, \gls*{rnn}, and \gls*{gum} are particular cases of the \gls*{pmc}.}
  \label{fig:graphical_models}
\end{figure}



\begin{remark}
    \cite{salaun2019comparing} have proposed the GUM as a unified framework
    to compare the expressivity of generative models based on latent variables.
    The GUM can be seen as a stochastic version of the \gls*{rnn} which 
    includes popular generative models such as
    the Variational RNN~\citep{chung2015recurrent} and 
    the Stochastic RNN~\citep{fraccaro2016sequential} as particular cases, 
    with a latent variable  
    $\latent_t \leftarrow (\lat_{t},\latent_{t})$. 
\end{remark}


\section{Parameter estimation for general PMCs}
\label{sub:pmc_parameter_estimation}
In this section, we propose a VI approach to  estimate the parameters $\theta$ 
of general PMC models.
This new approach can be applied to any sequence $\obs_{0:T}$ of varying length $T$
and  does not require the knowledge of the latent variables $\latent_{0:T}$.
In addition, it is suitable for high dimensional models \citep{Blei_2017}.
First, we introduce a general parameterization of the PMC model
which includes a deep parameterization (via DNNs).
Next, we adapt the  (static) VI framework described in 
Subsection~\ref{subsec:vbi}  
for the sequential case with  PMCs.



% % \subsection{Monte Carlo Approximation}
% % Subsection~\ref{subsec:reparameterization_trick}
% % \eqref{eq:reparameterization_trick}


% % However, the samples depend on $\phi$; in order to ensure
% % that the Monte Carlo approximation remains differentiable w.r.t. $\phi$, a
% % sample $\latent_{0:T}^{(m)}$, for $m \in [1:M]$ where $M$ is the number of
% % samples. This sample has to be reparameterized as a differentiable function of
% % $\phi$ (reparameterization trick, see 
% % Section~\ref{subsec:reparameterization_trick}).\\ 
% % More precisely, a final sample $\latent_{0:
% % T}^{(m)}$ is obtained by
% % \begin{align}
% %     \label{eq:reparametrization_vpmc}
% %     \latent_{0:T}^{(m)} =& g(\phi,\epsilon_{0:T}^{(m)}) \text{,}
% % \end{align}
% % where $\epsilon^{(m)}$ is a random variable sampled from a distribution
% % which does not depend on $\phi$, and where $g$ is a differentiable function of $\phi$.\\
% Thus, the final approximation reads
% \begin{align}
%     \label{eq:elbo-gpmc-mc}
%     \hat{\Qgen}(\theta,\phi) &= -\frac{1}{M} \sum_{m=1}^M 
% \log \left(\frac{\q(\latent_{0:T}^{(m)}|\obs_{0:T})}{\p(\latent_{0:T}^{(m)},\obs_{0:T})}\right) \text{, } 
% \end{align}
% with $\latent_{t}^{(m)} \sim \q(\latent_{t}^{(m)}|\latent_{0:t-1}^{(m)},\obs_{0:T}) \text{, for }
%  m \!\in \![1\!:\!M] \text{.} $
% In addition, $\latent_{t}^{(m)}$ is a differentiable function of $\phi$, for  
% $m \in [1:M]$ and $t \in [0:T]$.
% Algorithm~\ref{algo:algo_train_dpmc_gen} summarizes the general estimation algorithm
% for  PMC models.
% % \begin{remark}
% %     In the case of DPMCs, their
% %     gradients are computable from the backpropagation
% %     algorithm~\citep{rumelhart1986learning} since $\pz$ and $\px$ 
% %     are differentiable w.r.t. $\theta$. 
% %     Moreover, the parameters $\phi$ of the variational distribution $\q$
% %     are the parameters of the DNNs which are differentiable w.r.t. $\phi$.
% %     Thus, Algorithm~\ref{algo:algo_train_dpmc_gen} can be applied to DPMCs.
% % \end{remark}



%  \begin{remark}
%     \label{rem:adam_pmc}
%     In the practical implementation of our algorithm~\ref{algo:algo_train_dpmc_gen} 
%     in Line~\eqref{eq:elbo_grad_vpmc},
%     we use the Adam optimizer. This choice is based on the optimizer's proven
%     effectiveness and efficiency in various machine learning 
%     applications~\citep{kingma2014adam}.
% \end{remark}
    
% \begin{example}
%     \label{example:gauss_variational}
% We consider the case of a  Gaussian variational distribution which
% satisfies 
% \begin{align*}
% \q(\latent_{t}|\latent_{0:t-1},\obs_{0:T})&=
% \q(\latent_{t}|\latent_{t-1},\obs_{t})\\
% &= \mathcal{N}\left(\latent_t; \mulatent , \diag(\siglatent) \right) 
% % \text{, where } [\mulatent , \siglatent] =  \qz(\latent_{t-1},\obs_{t}) \text{, } 
% \end{align*}
% where $ [\mulatent , \siglatent] =  \qz(\latent_{t-1},\obs_{t})$
% and $\qz$ is a differentiable vector function parameterized by $\phi$,
% which is assumed to be differentiable w.r.t. $\phi$; and $\diag(\;\cdot \;)$ denotes
% the diagonal matrix deduced from $\siglatent$. 
% For example, $\mulatent$ and $\siglatent$ represent the output of a neural network
% similar to the one presented in Figure~\ref{fig:dpmc_gaussian}.
% % ~\ref{ex:dpmc_gaussian}.


% In this case, a sample $\latent_{t}^{(m)}$, for all $i \in [1:M]$ and $t \in [0:T]$,
% can be reparameterized by using the reparameterization trick 
% % (see Section~\ref{subsec:reparameterization_trick}) 
% and  reads
% \begin{equation*}
% \latent_{t}^{(m)}=
% \mulatent(\latent_{t-1}^{(m)},\obs_t)+
% \diag(\siglatent(\latent_{t-1},\obs_t))^{\frac{1}{2}} 
% \times  \epsilon_t^{(m)} \text{,} \quad \quad \epsilon_t^{(m)} 
% \overset{i.i.d.}{\sim} \mathcal{N}(0,I) \text{.}
% \end{equation*}

% The  Monte Carlo approximation of the ELBO can be rewritten as
% $$\hat{\Qgen}(\theta,\phi)= - \frac{1}{M} \sum_{m=1}^M \log \left( \frac{\q(\latent_0^{(m)}|\obs_0)}{\p(\latent_0^{(m)},\obs_0)}\right) - 
% \frac{1}{M} \sum_{m=1}^M \sum_{t=1}^T \log \left(\frac{\q(\latent_t^{(m)}|\latent_{t-1}^{(m)},\obs_t)}{\p(\latent_t^{(m)},\obs_t|\latent_{t-1}^{(m)},\obs_{t-1})} \right)  $$
% and is next optimized w.r.t. $(\theta,\phi)$.\\
% \end{example}

% \begin{figure}[htb]
%     \centering
%     \includegraphics[width=0.5\textwidth]{Figures/Graphical_models/rep_trick_example.pdf}
%     \caption{Example reparameterization trick.}
%     \katy{Change this example}
%     \label{fig:rt_example}
% \end{figure}





\subsection{General parameterization of PMCs}
\label{sec:pmc_parameterization}
We propose a general parameterization of the PMC model that can be applied to
any PMC. Without loss of generality, we consider the transition distribution 
$\p(\latent_t,\obs_t|\latent_{t-1},\obs_{t-1})$
given in~\eqref{eq:pmc_gen_transition}.
A general parameterization allows us to consider different any (conditional) distributions
$\p(\latent_t|\latent_{t-1},\obs_{t-1})$ and $\p(\obs_t|\latent_{t-1:t},\obs_{t-1})$,
\eg~Gaussian distributions.
Thus, for fixed  distributions, the parameters are learned based on functions 
of the conditional variables. 
This parameterization  extends beyond linear functions and also includes the application of 
deep neural networks due to the universal approximation property
(see Section~\ref{sub:nn}).

Let~$\pz(\latent_{t-1}, \;\obs_{t-1})$ and~$\px(\latent_{t-1:t},\;\obs_{t-1} )$  
be two vector-valued functions 
of ~$(\latent_{t-1},\obs_{t-1})$ and of~$(\latent_{t-1:t},\obs_{t-1})$, respectively, 
that are assumed to be differentiable w.r.t. $\theta$. 
Let also $\eta(\latent; \param )$ and $\zeta(\obs;\param')$
be probability density functions (pdf) on $\mathbb{R}^{d_\latent}$ 
and $\mathbb{R}^{d_\obs}$, respectively,
whose parameters are given by the vectors $\param$ and $\param'$, respectively.
$\eta$ and $\zeta$ are assumed to be differentiable w.r.t. $\param$ and $\param'$, 
respectively.
Then, we parameterize the conditional distributions in~\eqref{eq:pmc_gen_t} 
as 
\begin{eqnarray}
% \label{pmc-transition}
% \p(\latent_t, \obs_t | \latent_{t-1}, \obs_{t-1}) = \p(\latent_t |\latent_{t-1},\obs_{t-1}) \p(\obs_t | \latent_{t-1:t},\obs_{t-1}) \text{,}\nonumber\\
\label{pmc-theta-1-gen}
\p(\latent_t|\latent_{t-1},\obs_{t-1})=\eta(\latent_t; \;  \pz(\latent_{t-1},\obs_{t-1}) )  \text{,} \\
\label{pmc-theta-2-gen}
\p(\obs_t|\latent_{t-1:t},\obs_{t-1})=\zeta(\obs_t;\; \px(\latent_{t-1:t},\obs_{t-1} ) ) \text{.}
\end{eqnarray}
In other words, $\pz$ (resp. $\px$)
describes the parameters of the (conditional) distribution $\eta$ 
(resp. $\zeta$).

\begin{example}
    \label{ex:gaussian}
    As an illustration, we consider $\eta$ as a multivariate  Gaussian distribution. 
    $\pz$  is the vector which contains the mean and the covariance matrix of $\eta$. 
    In this case, $\p(\latent_t|\latent_{t-1},\obs_{t-1})$ reads 
    \begin{align*}
        \p(\latent_t|\latent_{t-1},\obs_{t-1})  &= 
        \N\left(\latent_t;  \mulatentp , \Siglatentp \right)   \text{, where }   
        \left[ \mulatentp , \Siglatentp \right]= 
        \pz(\latent_{t-1},\obs_{t-1}) \text{,} 
    \end{align*}
    It shows how the mean and covariance matrix of this Gaussian distribution are derived
    from the values given by the function $\pz$, 
    which is assumed to be differentiable w.r.t. $\theta$.
% \katyobs{Pendiente hacer despues de ver el chap 4 para ver los ejemplos}
\end{example}

% \begin{remark} It is easy to verify  that from \eqref{eq:pmc_gen_t}, 
%     we deduce  the HMC \eqref{eq:hmc}
%     \begin{align}
%     p(\latent_t,\obs_t|& \latent_{t-1},\obs_{t-1}) =  %p(\latent_t|\latent_{t-1},\obs_{t-1})p(\obs_t|\latent_{t})
%     p(\latent_t|\latent_{t-1},\cancel{\obs_{t-1}})p(\obs_t|\latent_{t}\cancel{\obs_{t-1}}, \cancel{\latent_{t-1}}) \text{.}
%     \end{align}       
% \end{remark}

% \subsection{Deep Pairwise Markov Chain}
\paragraph{Deep pairwise Markov chain - }
\label{sec:dpmc}
A particular case of the general parameterization of the PMC model is the \gls*{dpmc},
where the parameterization of the transition distribution~$\p(\latent_t,\obs_t|\latent_{t-1},\obs_{t-1})$
presented in \eqref{pmc-theta-1-gen} and \eqref{pmc-theta-2-gen} is given by DNNs.
%  the distributions $\p(\latent_t | \latent_{t-1}, \obs_{t-1})$
% and  $\p(\obs_t | \latent_{t-1:t}, \obs_{t-1})$ 
Since DNNs can theoretically approximate any function which satisfies 
reasonable assumptions (see Section~\ref{subsec:neural_networks}),
% ~\citep{pinkus1999approximation}, 
our objective is to use them to
approximate any parameterization of $\eta$ and $\zeta$
of the distributions $\p(\latent_t | \latent_{t-1}, \obs_{t-1})$
and  $\p(\obs_t | \latent_{t-1:t}, \obs_{t-1})$, respectively.
In other words,  $\pz$ and $\px$ are the outputs of two \gls*{dnns}. 
For example, with $(\latent_{t-1},\obs_{t-1})$
and $(\latent_{t-1:t},\obs_{t-1})$ as inputs, respectively in~\eqref{pmc-theta-1-gen} and \eqref{pmc-theta-2-gen}.
The set of parameters $\theta$ now consists of the parameters of these 
\gls*{dnns}  (weights and biases). 
In this case, their gradients are computable from the backpropagation
algorithm~\citep{rumelhart1985learning,hecht1992theory} since $\pz$ and $\px$
are differentiable w.r.t. $\theta$ (see Section~\ref{sub:principle}).


% Note that a unique DNN is 
% used for $\pz$ (resp. $\px$) overtime. 

\begin{example}
    \label{ex:dpmc_gaussian}
    % \katyobs{check this example after the previous one}

    We recall the previous example~\ref{ex:gaussian}, where $\eta$ is a Gaussian distribution,
    In this case, the mean $\mulatentp$  and the covariance matrix 
    $\diag(\siglatentp)$ are the
    output of a neural network as illustrated in Figure~\ref{fig:dpmc_gaussian}.
    For example, $\mulatentp$ and $\siglatentp$ can be the output of a linear layer. 

    \begin{align*}
        \p(\latent_t|\latent_{t-1},\obs_{t-1})  &= \N\left(\latent_t;  \mulatentp , \diag(\siglatentp) \right)   \text{, where }   \left[ \mulatentp , \siglatentp \right]= \pz(\latent_{t-1},\obs_{t-1}) \text{.} 
    \end{align*}
    
    \begin{figure}[htb]
        \centering
        \includegraphics[width=0.8\textwidth]{Figures/Graphical_models/pmc_example.pdf}
        \caption{Illustration of a deep parameterization of the
        distribution~$\p(\latent_t|\latent_{t-1},\obs_{t-1})$, where the
        parameters~$\mulatentp$ and~$\siglatentp$ of the Gaussian distribution
        are the output of a DNN.}
        \label{fig:dpmc_gaussian}
    \end{figure}
    % \katyobs{Add an example of the activation function used for both the variance and the mean}
\end{example}

\subsection{Variational Inference for PMCs}
For PMCs, we can extend the ELBO given in~\eqref{eq:elbo} ,
which was formulated for static models to the sequential case.
We now define $\obs
\leftarrow \obs_{0:T}$, and $\latent \leftarrow \latent_{0:T}$.
Then the following inequality holds for any variational distribution 
$\q(\latent_{0:T}|\obs_{0:T})$,
\begin{align}
    \label{eq:elbo_general}
    \log(\p(\obs_{0:T})) & \geq - \int  \q(\latent_{0:T}|\obs_{0:T})
    \log  \left(\frac{\q(\latent_{0:T}|\obs_{0:T})}{\p(\obs_{0:T},\latent_{0:T})}\right) 
     {\rm d} \latent_{0:T}  = \Qgen(\theta,\phi) \text{,}
\end{align}
where  $\q$ depends on a set of parameters $\phi$.
In our sequential case, we choose the following variational distribution
\begin{align}
    \label{eq:var_dist}
    \q(\latent_{0:T}|\obs_{0:T})=& \q(\latent_{0}|\obs_{0:T}) 
    \prod_{t=1}^T \q(\latent_{t}|\latent_{0:t-1},\obs_{0:T}) \text{.} 
\end{align}
This general factorization, that is based on transitions 
$\q(\latent_{t}|\latent_{0:t-1},\obs_{0:T})$, captures 
the temporal dependencies inherent in sequential data.
This form involves a choice of a variational distribution
$\q(\latent_{t}|\latent_{0:t-1},\obs_{0:T})$,
and the parameters that govern this distribution remain constant across
time. % involves a choice of a variational distribution,so a  time-independent parameterization
The variational distribution $\q$ should respect the differentiability 
and computational tractability constraints.
For efficient optimization, $\q(\latent_{t}|\latent_{0:t-1},\obs_{0:T})$ should
be differentiable w.r.t. $\phi$ and 
should be chosen in a way that $\Qgen(\theta,\phi)$ is computable or can be
approximated (see Subsection~\ref{subsec:vbi}).
% This property is crucial for using
% gradient-based optimization algorithms, such as stochastic gradient descent, to
% learn the optimal parameters of the variational distribution $\q$.
% In addition, $\q$ should be chosen in a way that $\Qgen(\theta,\phi)$ is computable 
% or can be approximated (see Subsection~\ref{subsec:vbi}).
% % \katyobs{Check this paragraph}
Thus, the factorization of $\p(\latent_{0:T},\obs_{0:T})$ and 
$\q(\latent_{0:T}|\obs_{0:T})$  given by~\eqref{eq:pmc_gen_t}
and~\eqref{eq:var_dist}, respectively, allows us to rewrite the 
ELBO~\eqref{eq:elbo_general} as follows
\begin{align}
    \label{eq:elbo_vpmc}
    \Qgen(\theta,\phi) =& \L_1(\theta,\phi) +  \L_2(\theta,\phi)
\end{align}
where
\begin{align}
    \L_1(\theta,\phi) =& \; \E_{\q(\latent_0| \obs_{0:T} )} (  \log \p(\obs_0| \latent_0)) 
    \nonumber
    \\
    \label{eq:elbo_vpmc_l1}
    &+  
    \sum_{t=1}^T \E_{\q(\latent_t|\latent_{0:t-1},\obs_{0:T})}
    (  \log \p(\obs_t|\latent_{t-1:t},\obs_{t-1} ) )\text{,}\\
    \L_2(\theta,\phi) =& - \dkl(\q(\latent_0| \obs_{0:T} ) | |  \p(\latent_0))
    \nonumber
    \\
    \label{eq:elbo_vpmc_l2}
    & - \sum_{t=1}^T  \dkl(\q(\latent_t|\latent_{0:t-1},\obs_{0:T}) | | \p(\latent_t|\latent_{t-1},\obs_{t-1})) \text{.}
\end{align}
$\Qgen(\theta,\phi)$ involves the sum of two terms. The first term $\L_1(\theta,\phi)$ represents
a reconstruction term which measures the ability to reconstruct observations
according to the conditional likelihood $\p$ from the latent variables
distributed according to $\q$.
The second term $\L_2(\theta,\phi)$ involves a KLD term between 
the variational  $\q$ and the conditional prior $\p$ distributions,
which encourages $\q$ to be close to $\p$~\citep{kingma2014}.
% \subsection{Monte Carlo Approximation}
It remains to compute and optimize the 
ELBO~\eqref{eq:elbo_vpmc} w.r.t.~$(\theta,\phi)$ 
in order to estimate the parameters of the PMC model.
On one hand, the term $\L_2(\theta,\phi)$~\eqref{eq:elbo_vpmc_l2} 
involves the KLD between
$\q$ and $\p$.
% , which can often be integrated analytically~\citep{kingma2014}
% since $\q$ and $\p$ are assumed to be tractable distributions (\eg Gaussian
% distributions). 





\begin{algorithm}[htbp!]
    \caption{General parameter estimation for generative PMCs}
    \label{algo:algo_train_dpmc_gen}
  \begin{algorithmic}[1]
    \Require{$\obs_{0:T}$, the data; $\varrho$, the learning rate; $M$ the number of samples}
    \Ensure{$(\theta^*, \phi^*)$, sets of estimated parameters}
    \State Initialize the parameters $\theta^0$ and $\phi^0$
    \State $j\leftarrow 0$\label{line:start_vpmc}
    \While{\text{convergence is not attained}}
      \State Sample $\latent_0^{(m)}\sim q_{\phi^{{j}}}(\latent_0|\obs_{0:T})$,  for all  $1 \leq m \leq M$ 
      \State Sample $\latent_t^{(m)}\sim q_{\phi^{{j}}}(\latent_t|\latent_{0:t-1}^{(m)},\obs_{0:T})$,   for all  $1 \leq m \leq M$, for all $1 \leq t \leq T$ 
      \State Evaluate the loss $\widehat{\Qgen}(\theta^{{j}},\phi^{{j}})$ 
      from \eqref{eq:elbo_vpmc},
      ~\eqref{eq:elbo_vpmc_l2},
      and~\eqref{eq:elbo-pmc-mc-1}. 
      \label{line:evaluate_loss}
      \State{Compute the derivative of the loss function 
      $\nabla_{(\theta, \phi)} \widehat{\Qgen}(\theta,\phi)$.  
    %   from \eqref{eq:elbo-gpmc-mc}.
    }\label{line:derivate_pmc} 
      \State Update the parameters with gradient ascent
    \begin{equation}
    \begin{pmatrix}\theta^{(j+1)}\\\phi^{(j+1)}\end{pmatrix}=
    \begin{pmatrix}\theta^{{j}}\\\phi^{{j}}\end{pmatrix}
    + \varrho {\nabla_{(\theta, \phi)} \widehat{\Qgen}(\theta,\phi)}\Big|_{(\theta^{{j}},\phi^{{j}})}
    \label{eq:elbo_grad_vpmc}
    \end{equation}
    \State  $j\leftarrow j+1$
    \EndWhile
    \State  $\theta^{*} \leftarrow \theta^{{j}}$
    \State  $\phi^{*} \leftarrow \phi^{{j}}$
    \label{line:end_dtmc_vpmc}
  \end{algorithmic}
    % \vspace*{0.2cm}
  \end{algorithm}


% \begin{example}
%     \label{example:dkl_gaussian}
%     Given two univariate Gaussian distributions $p(z)$ and $q(z)$,
%     with parameters $\mu_p$, $\sigma_p^2$ and $\mu_q$, $\sigma_q^2$ respectively. 
%     The KL divergence between  $p$ and $q$ is given by
% $$\dkl(p \parallel q) = \frac{1}{2} 
% \left( \log \left( \frac{\sigma_q^2}{\sigma_p^2} \right) 
% + \frac{\sigma_p^2 + (\mu_p - \mu_q)^2}{\sigma_q^2} - 1 \right) \text{.}$$
% \end{example}

On the other hand, the term $\L_1(\theta,\phi)$~\eqref{eq:elbo_vpmc_l1} 
coincides with
expectations w.r.t.~$\q$ and can be approximated by Monte Carlo estimation.
% with samples $\latent_{0:T}^{(m)} \sim \q(\latent_{0:T}|\obs_{0:T})$, for $m \in [1:M]$.
For this, we use the reparameterization trick
presented in Subsection~\ref{subsec:reparameterization_trick},
which can be extended to the sequential case by considering
Equation~\eqref{eq:reparameterization_trick} for each time 
step $t$, as follows:
\begin{align}
    \label{eq:reparametrization_vpmc}
    \latent_{0:T}^{(m)} =& g(\phi, \epsilon_{0:T}^{(m)}) \text{, for } m \in [1:M] \text{.}
\end{align}
% where $\epsilon^{(m)}$ is a random variable sampled from a distribution
% which does not depend on $\phi$, and where $g$ is a differentiable function of $\phi$.\\
Thus, $\L_1(\theta,\phi)$~\eqref{eq:elbo_vpmc_l1}
can be approximated by
% \begin{align}
%     \label{eq:elbo-gpmc-mc}
%     \hat{\Qgen}(\theta,\phi) &= -\frac{1}{M} \sum_{m=1}^M 
% \log \left(\frac{\q(\latent_{0:T}^{(m)}|\obs_{0:T})}{\p(\latent_{0:T}^{(m)},\obs_{0:T})}\right) \text{, } 
% \end{align}
\begin{align}
    % \label{eq:elbo-pmc-mc}
    % \hat{\Qgen}(\theta,\phi) =&  \;  \hat{\L}_1(\theta,\phi) +  
    % \L_2(\theta,\phi)\\
    % % \hat{\L}_2(\theta,\phi)\\
    \label{eq:elbo-pmc-mc-1}
    \hat{\L}_1(\theta,\phi) =& \frac{1}{M} \sum_{m=1}^M  \log \p(\obs_0| \latent_0^{(m)})
    +\frac{1}{M} \sum_{m=1}^M  \sum_{t=1}^T  
    \log \p(\obs_t|\latent_{t-1:t}^{(m)},\obs_{t-1} ) \text{,}
    % \text{,}\\
    % \hat{\L}_2(\theta,\phi) = & - \dkl(\q(\latent_0| \obs_{0:T} ) | |  \p(\latent_0))\nonumber \\
    % \label{eq:elbo-pmc-mc-2}
    % &- \sum_{t=1}^T  \dkl(\q(\latent_t|\latent_{0:t-1},\obs_{0:T}) | | \p(\latent_t|\latent_{t-1},\obs_{t-1})) \text{.}
\end{align}
where $\latent_{t}^{(m)}$ is a differentiable function of $\phi$ that is sampled from
$\q(\latent_{t}|\latent_{0:t-1},\obs_{0:T})$, for $m \in [1:M]$ and $t \in [0:T]$.
Algorithm~\ref{algo:algo_train_dpmc_gen} 
summarizes the general estimation algorithm
for  general PMCs. Here, we learn the generative model by maximizing the
ELBO $\Qgen$ with respect to their parameters $\theta$ and $\phi$.



% \begin{remark} 
%     In the case of DPMCs, the parameters $\phi$ of the variational distribution $\q$ are the 
%      parameters of the DNNs which are differentiable w.r.t. $\phi$. Thus, 
%      Algorithm~\ref{algo:algo_train_dpmc_gen} 
%      can be applied to DPMCs. 
%  \end{remark}

\section{Experiments and results}
In this section, we first introduce a particular instance of the PMC model
which combines the deep PMC model (see Section~\ref{sec:dpmc}) 
and the stochastic RNN model (SRNN)~\citep{bayer2015learning, chung2015recurrent}.
From this instance, we derive different generative models for sequential data
with specific dependencies between latent and observed variables.
Finally, we compare their performance with
the stochastic RNN model on two datasets.


\subsection{Model description}
SRNN  architectures are specific instances of the PMCs,
% (see Section~\ref{sec:pmc_unified_model}).
which have demonstrated good experimental results~\citep{bayer2015learning, chung2015recurrent}, 
making it natural to compare them with their PMC extension. 
We introduce a model that combines the DPMC, and 
the SRNN models.
This generative (deep) PMC model consists of a latent process 
in an augmented dimension,  $\Latent_t \leftarrow (\Lat_t,\Latent_t)$,
the transition distribution now reads 
\begin{align}
    \label{eq:tmc-param}
    \p(\lat_t,\latent_t,\obs_t|\lat_{t-1},\latent_{t-1},\obs_{t-1}) = &\; \p(\lat_t|\lat_{t-1},\latent_{t-1},\obs_{t-1}) \p(\latent_t|\lat_{t-1:t},\latent_{t-1},\obs_{t-1}) \times \nonumber \\
    & \; \; \p(\obs_t|\lat_{t-1:t},\latent_{t-1:t},\obs_{t-1}) \text{.}
\end{align}
\begin{remark}
    Note that the previous equation  is nothing more than a particular case of the TMC model
    with transition~\eqref{eq:tmc-param}.    
    However, we consider it as a particular instance of the PMC model
    since $\lat_t$ and $\latent_t$, f←or all $t \in [0:T]$, are considered as latent variables
    with no physical interpretation.
\end{remark}
On the other hand, the variational distribution $\q$ defined in~\eqref{eq:var_dist}
is factorized as follows
\begin{align*}
    % \label{eq:var_pmc_latent}
    \q(\latent_{t}, \lat_{t} | \latent_{0:t-1},  \lat_{0:t-1},\obs_{0:T}) &= 
    \q(\latent_{t} | \latent_{0:t-1},  \lat_{0:t},\obs_{0:T}) 
    \q(\lat_{t} | \latent_{0:t-1},  \lat_{0:t-1},\obs_{0:T}) \text{.}  %\\
    % &= \q(\latent_{t} |\lat_{0:t},\obs_{t}) 
    % \q(\lat_{t} | \latent_{0:t-1},  \lat_{0:t-1},\obs_{0:T})
\end{align*}

% \paragraph{Parameterization - }
We consider the general parameterization presented in 
Subsection~\ref{sec:pmc_unified_model}. 
However, we now have an additional distribution because of the new variable $\lat_t$.
Let $\lambda$ be a distribution on $\lat_t$ parameterized by a differentiable (w.r.t. $\theta$) and 
vector valued function denoted as $\ph$ and which can depend on 
$(\lat_{t-1},\latent_{t-1},\obs_{t-1})$.
% % \ref{sec:pmc_unified_model}
% \katy{Aqui creo que es mejor que consideramos las mismas hipotesis presentadas en la param. general
% y asi solo anado una nueva funcion $\lambda$ para la nueva variable introducida}
% \katy{Change this parte wrt Yohan HDR because of the use of Z, H, and X and not $x_t$ and so on:
% Let $\lambda$, $\zeta$ and $\eta$  be distributions on $\lat_t$, $\latent_t$ and $\obs_t$, respectively,}
% and parameterized by differentiable (w.r.t. $\theta$) and vector valued functions denoted as
% $\ph$, $\pz$ and $\px$
% and which can depend on $(\lat_{t-1},\latent_{t-1},\obs_{t-1})$,
% $(\lat_{t-1:t},\latent_{t-1},\obs_{t-1})$ and on $(\lat_{t-1:t},\latent_{t-1:t},\obs_{t-1})$, respectively.
We recall that $\pz$ and $\px$ are defined in~\eqref{pmc-theta-1-gen}
and~\eqref{pmc-theta-2-gen}, respectively.
Thus, the parameterized transition~\eqref{eq:tmc-param} reads
\begin{equation*}
    % \label{eq:tmc-theta}
    \begin{aligned}
    \p(\lat_t|\lat_{t-1},\latent_{t-1},\obs_{t-1})  &= \lambda\left(\lat_t; \ph(\lat_{t-1},\latent_{t-1},\obs_{t-1})\right) \text{,} \\
    %\label{tmc-theta-2}
    \p(\latent_t|\lat_{t-1:t},\latent_{t-1},\obs_{t-1})&=\eta \left(\latent_t;\pz(\lat_{t-1:t},\latent_{t-1},\obs_{t-1})\right) \text{,}\\
    %\label{tmc-theta-3}
    \p(\obs_t|\lat_{t-1:t},\latent_{t-1:t},\obs_{t-1})&=\zeta\left(\obs_t;\px(\lat_{t-1:t},\latent_{t-1:t},\obs_{t-1}) \right) \text{.}
    \end{aligned}
\end{equation*}

In the context of SRNN architectures,
the variable $\lat_t$ represents a deterministic summary of the
past until time $t-1$, \ie~$\lat_t = \ph(\lat_{t-1},\latent_{t-1}, \obs_{t-1})$.
While $\latent_t$ corresponds to a
noisy version of $\lat_t$ (it is why we have split
the latent process in two).
Note that since $\Lat_{0:T}$ is deterministic given $(\latent_{0:T},\obs_{0:T})$,
its posterior distribution becomes trivial, and thus 
there is no need to consider a variational distribution for it.
The variational distribution $\q$ is  then parameterized as
\begin{align}
    \label{eq:pmc_qz}
    \q(\latent_t | \latent_{0:t-1},  \lat_{0:t},\obs_{0:T}) &= 
    \q(\latent_t | \lat_{t},\obs_{t})
    = \tau(\latent_t; \qz (\lat_{t},\obs_{t})) \text{,}
\end{align}
where $\tau(\latent;\qz)$ is a probability density function
on $\mathbb{R}^{d_\latent}$ 
whose parameters are given by $\qz$, 
which is differentiable w.r.t. $\phi$.
% $\qz$ is a differentiable vector valued function parameterized by $\phi$.
Following this reasoning and with a slight abuse of notation 
(where $\lambda$ coincides with the Dirac measure), we can incorporate several
degrees of generalization of the classical RNN  and of the SRNN
of \citet{chung2015recurrent}. 
The different deep PMC models we consider are defined in 
Table~\ref{tab:config-pmc} 
and are based on the specific dependencies of the involved random variables. 
Note that $\px$, $\ph$, $\pz$ and $\qz$ are now neural networks.

\begin{table}[!htpb]
    \begin{center}
    \small
    \begin{tabular}{|l|r|r|r|r|}
    % \begin{tabular}{l|rrr|}
    \cline{2-4}
    \multicolumn{1}{l}{}             & \multicolumn{3}{|c|}{\textbf{Parameterized function}}                                     \\ \hline
        \multicolumn{1}{|l|}{\textbf{Models}} & \multicolumn{1}{r|}{$\ph$} & \multicolumn{1}{r|}{$\pz$} & $\px$                 \\ \hline
    % \multicolumn{1}{|c|}{\diagbox{Model} {Parameterized function}}  & \multicolumn{1}{c|}{$s_{\theta}$} & \multicolumn{1}{c|}{$f_{\theta}$} & \multicolumn{1}{c|}{$g_{\theta}$} \\ \hline 
    RNN    &  $(\lat_{t-1},\obs_{t-1})$ & $\times$ & $\lat_t$ \\ \hline
    SRNN    &  $(\lat_{t-1},\latent_{t-1},\obs_{t-1})$ & $\lat_t$ & $(\lat_t,\latent_t)$ \\ \hline
    PMC-I    & $(\lat_{t-1},\latent_{t-1},\obs_{t-1})$  & $\lat_t$   & $(\lat_t,\latent_t,\obs_{t-1})$  \\ \hline
    PMC-II  & $(\lat_{t-1},\latent_{t-1},\obs_{t-1})$  &$\lat_t$ &  $(\lat_{t-1:t},\latent_{t},\obs_{t-1})$  \\ \hline
    PMC-III &  $(\lat_{t-1},\latent_{t-1},\obs_{t-1})$  & $\lat_t$  &  $(\lat_{t-1:t},\latent_{t-1:t},\obs_{t-1})$ \\ \hline
    PMC-IV &  $(\lat_{t-1},\latent_{t-1},\obs_{t-1})$  & $(\lat_t,\obs_{t-1})$  & $(\lat_{t-1:t},\latent_{t},\obs_{t-1})$ \\ \hline
    \end{tabular}
    \end{center}
    \caption{Configuration of the dependencies for different deep generative \gls*{pmc}s. 
    In each model, the sequence of latent variables $\{\Lat_t\}_{t \in \NN}$ 
    is treated as a deterministic variable given the observations. As a result, 
    $\eta$ coincides with the Dirac measure. The distribution $\lambda$ 
    is typically chosen to be Gaussian,
    while $\zeta$ depends on the nature of the observations. 
    Remember that in a classical \gls*{rnn}, $\{\Latent_t\}_{t \in \NN}$ 
    is not considered.}
    \label{tab:config-pmc} 
\end{table}


\subsection{Results}

% \subsection{Experiments}
\paragraph{Model configuration -}
In our experiments, the observed random variables are discrete,
and each $\Obs_t$ takes values in a binary space $\{0,1\}^{d_{\obs}}$.
% \katyobs{the last notation is correct?}
As a consequence, the distribution $\zeta$ coincides with the
Bernoulli distribution, and the output of $\px$ with its parameter.
For $\lambda$, we choose the Gaussian distribution and 
the output of $\pz$ corresponds to the mean and to the diagonal covariance matrix of 
the Gaussian distribution, which is summarized as follows
\begin{equation*}
    \begin{aligned}
    \lat_t &= \ph(\; \cdot \;) \text{,} \\
    \p(\latent_t|\; \cdot \;)  &= \N\left(\latent_t;  \mulatentp , \diag(\siglatentp) \right)   \text{, where }   \left[ \mulatentp , \siglatentp \right]= \pz(\; \cdot \;) \text{,} \\
    \p(\obs_t|\; \cdot \;)&= \Ber\left(\obs_t; \ropx \right) 
    \text{, where } \ropx  = \px(\; \cdot \;) \text{.}
    \end{aligned}
\end{equation*}
Here the notation $( \; \cdot \;)$ is used to avoid presenting a specific dependence between variables. 
These dependencies are specified for each model and are presented in 
Table~\ref{tab:config-pmc}.
The variational distribution $\q$ given in~\eqref{eq:pmc_qz}
is chosen as Gaussian, which satisfies
% \begin{align}
%     \label{eq:m3}
%     % \q(\latent_{t}|\lat_{0:t-1},\obs_{0:T}) &= \mathcal{N}\left(\latent_t;\qz(\lat_t,\obs_t)\right) \text{,}
%     \q(\latent_t|\lat_{t-1:t},\latent_{t-1},\obs_{t})&= \mathcal{N}\left(\latent_t;\qz(\lat_t,\obs_t)\right) \text{,}
% \end{align}
\begin{equation*}
    % \q(\latent_{t}|\latent_{0:t-1},\obs_{0:T}) = 
    \q(\latent_t|\lat_{t}, \obs_{t})  =
    \mathcal{N}\left(\latent_t; \mulatent, \diag(\siglatent) \right) \text{, where } 
    \left[ \mulatent, \siglatent \right] = \qz(\lat_{t},\obs_t) \text{.}
\end{equation*}
The functions   $\px$, $\pz$, $\ph$ and $\qz$ are implemented as
neural networks consisting of two hidden layers.
The rectified linear unit (ReLu) activation function is used for the hidden layers,
and the outputs of the neural networks are adapted according to their role. 
For example, the output of $\px$ is a layer of $d_\obs$ sigmoid functions 
due to the nature of the observations (binary values).
Additionally, the number of hidden units of each neural network coincides with the dimension $d_\latent$ 
of $\Latent_t$ and is different for each model and data set, which is specified in the next part.



\paragraph{Training - }
Each model was trained with stochastic gradient descent on the negative evidence 
lower bound using the Adam optimizer~\citep{kingma2014adam} with 
a learning rate of $0.001$.
% and a batch size of $512$ images.
% \katyobs{Check the batch size for both data sets}
The number of epochs was set to $100$ for both data sets.
% We have two configurations for the training of the models:
The number of hidden units of the neural networks ($d_\latent$) can be fixed for all the models,
or can be chosen by considering the number of parameters of the models to be compared
(\ie~the number of parameters are the same or close to). 

\paragraph{Evaluation - } 
The  performance of the models is evaluated in terms of the approximated ELBO
and log-likelihood of the observations on the test data set; 
we use a particle filter with the estimated variational distribution 
as importance distribution and $N=100$ particles.
% , see Algorithm~\ref{alg:particle_filter}.

% \katy{INTRODUCTION: Add the algorithm of the particle filter in Main Concepts section }


\paragraph{Image generation - }
The MNIST dataset~\citep{lecun1998mnist} contains  $60000$ (resp. $10000$) train (resp. test) 
$28 \times 28$  binary images. In this case, 
an observation $\Obs_t$ consists of a column of the image
and its dimensionality is $d_\obs=28$. The length of each sequence is $T+1=28$.
For this data set, we consider two configurations for the training of the models and are
summarized in Table~\ref{tab:config}. Config. 1 corresponds to the configuration
in which the number of hidden units of the neural networks is fixed $d_\lat=100$ for all the models. 
In Config.2, the number of hidden units of the neural networks
is chosen by considering the number of parameters of the models to be compared. 
We set $d_\lat=162$ , $d_\lat=100$, $d_\lat=95$, $d_\lat=79$, $d_\lat=78$ and 
$d_\lat=74$  for the RNN, SRNN, the
PMC-I, the PMC-II, the PMC-III, and the PMC-IV, respectively.  
We also set $d_\latent=3$ for each model and both configurations.

\begin{table}
    \begin{center}
        \begin{tabular}{|l|r|r|r|r|r|r|r|r|r|r|}
        \hline
        % \multirow{2}{*}{\diagbox{Model}{Data set}} 
        \multirow{3}{*}{Model}  &\multicolumn{4}{c|}{MNIST data set}  &\multicolumn{2}{c|}{\!\!Music data sets\!\!} \\
        \cline{2-7}
           &\multicolumn{2}{c|}{Config. 1} &\multicolumn{2}{c|}{Config. 2} &\multicolumn{2}{c|}{Config. 2}  \\
        % & \multicolumn{2}{c|}{MIDI} \\ 
        \cline{2-7}
         & \multicolumn{1}{c|}{$d_{\latent}$} & \multicolumn{1}{c|}{$d_{\lat}$}  & \multicolumn{1}{c|}{$d_{\latent}$} & \multicolumn{1}{c|}{$d_{\lat}$}   & \multicolumn{1}{c|}{$d_{\latent}$} & \multicolumn{1}{c|}{$d_{\lat}$}  
        % & \multicolumn{1}{c|}{$d_{\lat}$} & \multicolumn{1}{c|}{$d_{\latent}$}
        \\ \hline \hline
        RNN     & 3 & 100 & 3 & 162 & 300 & 562  \\ %\hline
        SRNN    & 3 & 100 & 3 & 100 & 300 & 300  \\ %\hline
        PMC-I   & 3 & 100 & 3 & 95 & 300 & 294  \\ %\hline
        PMC-II  & 3 & 100 & 3 & 79 & 300 & 278  \\ %\hline
        PMC-III & 3 & 100 & 3 & 78 & 300 & 260 \\ %\hline
        PMC-IV  & 3 & 100 & 3 & 74 & 300 & 272  \\ \hline
        \end{tabular}    
        \end{center}
        %\vspace{-0.4cm}
        \caption{Dimensions of latent variables for each Deep PMC.  
        $\ph$, $\pz$, $\px$  and $\qz$ are implemented as 
        neural networks with two hidden layers. 
        The number of neurons on each layer coincide with $d_{\lat}$.}        
    \label{tab:config}
\end{table}

Table \ref{tab:t1} presents the averaged ELBO 
and the averaged approximated log-likelihood  on the test set assigned by our models.
The results with the Config.1 (resp. Config. 2) 
show that PMC-IV (resp. PMC-II) has the higher averaged ELBO and 
averaged approximated log-likelihood compared to other models.
This indicates that the performance of the PMCs is better than of SRNN and RNN models.
An example of images generated from the estimated $\p(\obs_{0:t})$ of the PMC-II 
is shown in Figure~\ref{fig:images}.
  
\begin{table}[!htpb]
    % \small
    \begin{center}
    \begin{tabular}{|l|r|r|r|r|}
        \hline
        
        \multirow{2}{*}{Model}     &\multicolumn{2}{c|}{MNIST, config. 1}&\multicolumn{2}{c|}{MNIST, config. 2}\\ 
        \cline{2-5} 
    % \multicolumn{1}{|c|}{Model}   
    & \multicolumn{1}{c|}{ELBO} & \multicolumn{1}{c|}{approx. LL } & \multicolumn{1}{c|}{ELBO} & \multicolumn{1}{c|}{approx. LL } \\ 
    \hline \hline
    RNN    & -65,976 & -65,976& -65,700 & -65,700 \\ %\hline
    SRNN    & -67,248 & -64,760 & -67,222 & -64,762 \\ %\hline
    PMC-I   & -66,544 & -64,076 & -67,322 & -64,698 \\ %\hline
    PMC-II  & -66,784 & -64,201 & \textbf{-66,815} & \textbf{-64,255} \\ %\hline
    PMC-III & -66,518 & -63,876 & -67,513 & -64,876 \\ %\hline
    PMC-IV & \textbf{-66,150} & \textbf{-63,603} & -67,648 & -64,924 \\ \hline
    \end{tabular}    
    \end{center}
    %\vspace{-0.4cm}
    \caption{Averaged ELBO and approximated log-likelihood (approx. LL) of the observations 
    on the test set with two different configurations. 
    For the RNN, the ELBO coincides with the (exact) log-likelihood.}
    \label{tab:t1}
\end{table}


\begin{figure}[!htpb]
    \centering
    \centerline{\includegraphics[width=5.5cm]{Figures/mnist_digits.PNG}}
  \caption{Examples of generated images from estimated $\p(\obs_{0:t})$ for
  the MNIST  data set with the PMC-II model.}
  \label{fig:images}
  \end{figure}


\paragraph*{Polyphonic music generation - }
We also consider the polyphonic music data sets~\citep{bengio2013advances},
% \katyobs{Check this reference}
where three polyphonic music data sets are available, the classical piano music 
(Piano), the folk tunes (Nottingham) and the four-part chorales by J.S. Bach (JSB). 
The input consists of $88$ binary visible units that span the whole 
range of piano from A0 to C8  (\ie~$\obs_t \in \{0,1\}^{88}$).
In this case, we consider the Config. 2 for the training of the models (see Table~\ref{tab:config})
since it is a fairer comparison between the models.

We set $d_\latent=300$  for each model, and  $d_{\lat}=562$, 
$d_{\lat}=300$,  $d_{\lat}=294$, 
$d_{\lat}=278$, $d_{\lat}=260$ and  $d_{\lat}=272$ for the RNN, the SRNN, the
PMC-I, the PMC-II, the PMC-III and  the PMC-IV respectively. 
Table \ref{tab:t2} presents the results of the averaged ELBO and the averaged approximated log-likelihood
where the PMC-II (resp. PMC-IV) has the best performance compared to other models
on the Piano (resp. Nottingham and JSB) data set.


\begin{table}[!htpb]
    \begin{center}
    % \small
    \begin{tabular}{|l|r|r|r|r|}
    \hline
    \multirow{2}{*}{Model}  &\multicolumn{3}{c|}{Polyphonic music data sets}\\ 
    \cline{2-4} 
      & \multicolumn{1}{c|}{Piano} & \multicolumn{1}{c|}{Nottingham} & \multicolumn{1}{c|}{JSB} \\ 
      \hline \hline
    RNN    &  -10,52 & -23,89 & -10,77 \\ %\hline
    SRNN    &  -9,4011 & -13,2982 & -10,2739 \\ %\hline
    PMC-I    & -9,3077  & -11,3856   &-10,3126  \\ %\hline
    PMC-II  & \textbf{-8,8265}   &-14,8485 &  -10,2409  \\ %\hline
    PMC-III &  -9,2285  &-13,3900  &  -10,1103 \\ %\hline
    PMC-IV &  -9,4134  & \textbf{-10,6323}  & \textbf{ -9,2372} \\ \hline
    \end{tabular}
    \end{center}
    \caption{Approximated likelihoods on the polyphonic music data sets.
    For the RNN, the exact log-likelihood is computed.}
    \label{tab:t2} 
\end{table}


\section{Generative power of PMCs}
In this section, our objective is to analyze the previous models from a
theoretical point of view. We consider a linear and stationary Gaussian PMC, with
$d_\obs=1$.
 In a stationary Gaussian process, the statistical properties, like the mean and
covariance of the observations, do not change over time. This stationarity implies that the
covariance between two observations depends only on the time difference $k$ between
them. This analysis is then based on the associated covariance function
$ r_k = \Cov(\obs_{t},\obs_{t+k})$, for all $k \in \NN$, which
characterize the distribution $\p(\obs_{0:T})$ induced by each model.
% \katyobs{Here change}

% In this section, our objective is to introduce distributions $\p(\obs_{0:T})$ 
% induced by a linear and stationary Gaussian PMC.
% Such distributions are characterized by their
% covariance sequence 
% % specifically focusing on the covariance function 
% % $r_k = \Cov(\obs_{t},\obs_{t+k})$, for all $k \in \NN$. 
% % However, a direct comparison is not possible as $r_k$ lacks 
% % a closed-form expression without additional assumptions on the model.\\
% % To that end, 
% We first introduce a linear and stationary Gaussian PMC 
% with $d_\obs = 1$. 
% Next, we derive a closed-form expression for the covariance sequence $\{r_k\}_{k \in \NN}$.
% % This expression is then compared to the covariance of the linear and stationary Gaussian GUM.
% This will facilitate a theoretical analysis of the modeling
% power of the PMC.


\subsection{
    Linear and stationary Gaussian PMCs
}
\paragraph{Linear PMC -}
We consider the  case where  $\pz$ and $\px$ in
equations~\eqref{pmc-theta-1-gen}-\eqref{pmc-theta-2-gen} are 
vectorial linear  functions.
We have the following linear parameterization of the PMC
\begin{align}
    \label{eq:linear_pmc1}
    \p(\latent_0, \obs_0) &= \varsigma \big( (\latent_0, \obs_0); \;[ 0 ;  \Sigma_0 \;] \big) \text{,}\\
    \label{eq:linear_pmc2}
    \p(\latent_t|\latent_{t-1},\obs_{t-1})&= \eta(\latent_t; \;[ a\latent_{t-1}+ c\obs_{t-1}; \alpha ])  \text{,} \\%\quad \text{for all } t\in [1,T] \text{,}\\
    \label{eq:linear_pmc3}
    \p(\obs_t|\latent_{t-1:t}, \obs_{t-1})&= \zeta(\obs_t; \;[ b\latent_{t} + e \latent_{t-1}+f\obs_{t-1} ; \beta ])\text{,} %\quad \text{for all } t\in [0,T] \text{,}
\end{align}
where the notation $[\cdot \; ; \; \cdot]$
considers the first and second order of the initial distribution $\p(\latent_0, \obs_0)$,
of  $\p(\obs_t | \latent_{t-1:t}, \obs_{t-1})$,
and of $\p(\latent_t | \latent_{t-1}, \obs_{t-1})$.
The dimensions of the parameters $a$, $b$, $c$, $e$ and $f$ % and $\tilde{\gamma}$ 
are 
$d_{\latent}  \times  d_{\latent}$,
$1  \times  d_{\latent}$,
$d_{\latent}  \times  1 $,
$1 \times  d_{\latent}$, 
$1 \times 1 $,
respectively.
The covariance matrix $\alpha$ is a square matrix and $\beta \geq 0$.
In the initial distribution  $\p(\latent_0, \obs_0)$, $0$ 
is a $(d_{\latent} + 1 )$ zero vector and $\Sigma_0$ is a $(d_{\latent} + 1 )$
square covariance matrix given by
\begin{align*}
    \Sigma_0 = \begin{bmatrix} \eta & \tilde{\gamma}^\intercal \\ \tilde{\gamma}& r_0 \end{bmatrix} \text{.}
\end{align*}
The dimensions of $\eta$ and $\tilde{\gamma}$ are $d_{\latent}  \times d_{\latent} $ and $1 \times  d_{\latent}$,
respectively; and $r_0$ is scalar.
Thus, the set of parameters now is $\theta=(a,b,c,e,f,\alpha,\beta,\eta,\tilde{\gamma}, r_0)$.


\paragraph{Gaussian PMC -}
We now consider $\varsigma$, $\eta$ and $\zeta$ as Gaussian distributions so
the distribution $\p(\obs_{0:T})$ is a multivariate Gaussian distribution due to
the linear structure of the model. However, it is important to note that this
assumption does not result in any loss of generality; we employ it here for the
sake of clarity.
The covariance function $r_k$ associated
to $\p(\obs_{0:T})$ can be deduced from the covariance matrix 
$\Sigma_t$ associated to the distribution $\p(\latent_t,\obs_t)$.
First, an equivalent representation of~\eqref{eq:linear_pmc1}
-\eqref{eq:linear_pmc3}
is obtained by considering the first and second
 order moments of the pair $(\latent_t,\obs_t)$ given $(\latent_{t-1},\obs_{t-1})$. 
Since this distribution involves the product of two Gaussian 
distributions, one being linear
in the other and with results on
conditional Gaussian distributions, we obtain:
% do the product of two gaussian pdf and one is linear in the other so using
% conditional results on gaussian, the pair remain gaussian and its parameters
% read
% \katyobs{check the change of notation you did} 
\begin{align*}
    %\label{eq:expec_pmc}
    &\E\left( \begin{bmatrix} \latent_t ,  \obs_t \end{bmatrix}^\intercal | \latent_{t-1},\obs_{t-1}  \right)= 
    M \begin{bmatrix} \latent_{t-1} \\ \obs_{t-1} \end{bmatrix} \text{, }\\
    %\label{eq:cov_pmc}
    &\Var\left( \begin{bmatrix} \latent_t , \obs_t \end{bmatrix}^\intercal | \latent_{t-1},\obs_{t-1}  \right)= \Sigma_{t|t-1} \text{,}
\end{align*}
where
\begin{equation}
    \label{eq:trans-var-pmc}
    M= \begin{bmatrix} a & c \\ ba+e & bc+f \end{bmatrix} \text{, } \quad \Sigma_{t|t-1} = \begin{bmatrix} \alpha & \alpha b^\intercal \\ b\alpha & \beta+ b\alpha b^\intercal \end{bmatrix} \text{.}
\end{equation}
The covariance of the pair $(\latent_t,\obs_t)$
is given by $\Sigma_0 \times  (M^k)^\intercal$, which
can be easily deduced from the previous representation.
We also obtain the following expression for the covariance matrix 
associated to the distribution $\p(\latent_t,\obs_t)$
 $\Sigma_t$,
\begin{equation}
    \label{eq:recursion_sigma}
    \Sigma_t  = M \Sigma_{t-1} M^\intercal + \Sigma_{t|t-1} \text{,}
\end{equation}
% $$ \Sigma_t = M \Sigma_0 M^\intercal + \Sigma_{t|t-1} \text{,}$$
which is an immediate consequence %of the application
of the Lemma~\ref{prop:gaussian} 
in Appendix~\ref{chap:appendix}.
% in Appendix~\ref{sec:proofs}. 
% $$\Cov (\begin{bmatrix} \latent_t , \obs_t \end{bmatrix}^\intercal, \begin{bmatrix} \latent_{t+k} , \obs_{t+k} \end{bmatrix}^\intercal)
% =  \Sigma_0 \times  (M^k)^\intercal$$

    



\paragraph{Stationary PMC - }
In order to assure the stationarity of $\{\obs_t\}_{t \in \NN}$,
we consider directly that the process $\{\latent_t,\obs_t\}_{t\in \NN}$
is stationary.
Consequently, $\Sigma_0$  and $\Sigma_t$~\eqref{eq:recursion_sigma}
should satisfy the following equivalence
%  $\Var(\begin{bmatrix} \latent_0 , \obs_0 \end{bmatrix}^\intercal) = \Sigma_0=\Sigma_{t}$, 
% where $\Sigma_{t}$ is given in~\eqref{eq:recursion_sigma}. 
% The following
% expression holds for all $t \in \NN^*$, 
\begin{equation}
    \label{eq:constraints-pmc}
    \Sigma_0= M \Sigma_0 M^\intercal + \Sigma_{t|t-1} \text{.} 
\end{equation}  
This matrix equation describes 
a set of constraints on the parameters of the PMC model, 
which ensures the stationarity of the distributions
$\p(\latent_t,\obs_t)$ and $\p(\obs_t)$.
%the observed sequence $\{\obs_{t}\}_{t \in \NN}$.
% the process $\{\latent_t,\obs_t\}_{t\in \NN}$
% and then of $\{\obs_{t}\}_{t \in \NN}$.


% Since  $(\latent_t,\obs_t) | (\latent_{t-1},\obs_{t-1})  \sim \N( (\latent_{t},\obs_{t1}); \; M \begin{bmatrix} \latent_{t-1} \\ \obs_{t-1} \end{bmatrix}, \Sigma_{t|t-1})$, 
% then $\Cov ((\latent_t,\obs_t), (\latent_{t+k},\obs_{t+k}))$  
% is given by  $\Sigma_{z_t} (M^{\tau})^\top$ where $M$ and $\Sigma_{z_t}$ are defined in Equation \eqref{eq:matM} and \eqref{eq:sigma_zt}, respectively.\\
% Model \eqref{eq:pmc_gen_t} satisfies
% \begin{align}
%     \label{eq:h0}
%     \p(\latent_0,\obs_0)                    & = \N \left(\begin{pmatrix} \latent_0 \\ \obs_0  \end{pmatrix};\begin{bmatrix} 0 \\ 0  \end{bmatrix} , \begin{bmatrix} \eta & \gamma \eta \\ \gamma \eta & 1 \end{bmatrix} \right) \\
%     \label{eq:ht}
%     \p(\latent_t | \latent_{t-1} , \obs_{t-1} )   & = \N (\latent_t; a\latent_{t-1} + c\obs_{t-1} , \alpha ) \text{,}                                                                                                                       \\
%     \label{eq:xt}
%     %\nonumber
%     \p(\obs_t | \latent_{t-1:t} , \obs_{t-1} ) & = \N (\obs_t; b \latent_t + e\latent_{t-1} + \! f\obs_{t-1} , \beta )\text{,}
% \end{align}
% where $\theta=(a,b,c,e,f,\alpha,\beta,\eta,\gamma)$.
% The linear and Gaussian SRNN coincides with $e=f=0$, $\gamma=b$,
% while the linear and Gaussian \gls*{hmc} also satisfies $c=0$.

% \begin{remark}
%     Note that the linear HMC, RNN and GUM are particular cases of the linear PMC.
%     In particular, if we set
%     $\Cov(\latent_0,\obs_0)=\eta b^T$, 
%         \begin{itemize}
%             \item the HMC is obtained by setting $c=0$, $e=0$, $f=0$,
%             \item in an RNN, the transition between $(\latent_{t-1},\obs_{t-1})$ 
%             and $\latent_t$ is deterministic so $\alpha=0$. 
%             We have also set $e=0$, $f=0$. 
%             \item The GUM is obtained by setting $e=0$, $f=0$. 
%         \end{itemize}
%     \katyobs{Check this remark, add some details.}
% \end{remark}




    
% \begin{remark}
%     \yohan{not a good idea. I will put such a remark at the end, in the case
%     where you want to analyse d\_x >1; if we want to underline that the Gaussian
%     framework is not necessarily, you can put it in the introduction of the
%     section, by saying we consider gaussian distribution without loss of
%     generaility; the important this is the linear caracter of the model.}

%     The equations~\eqref{eq:trans-var-pmc} and~\eqref{eq:constraints-pmc}
%     can be obtained beyond the Gaussian assumption.
%     The problem relies on  $M$ and $\Sigma_{t|t-1}$, that are not explicitly given. 
%     This makes the direct deduction of  $r_k$
%     from the covariance of the joint distribution $\p(\latent_t,\obs_t)$ difficult.
%     To address this challenge, the stochastic realization theory can be 
%     introduced. This theory has been developed for the GUM in
%     a recent study~\citep{desbouvries2022expressivity}. %
%     % , which provides more details and results.
%     For the PMC, applying this theory can be more difficult due to the 
%     introduction of new dependencies through the parameters 
%     $(e,f)$.  However, a detailed exploration of this perspective is out
%     of the scope of this thesis.  Interested readers are 
%     encouraged to refer to \cite{desbouvries2022expressivity} for a comprehensive
%     analysis of the GUM, the HMC and the RNN.
%     % We refer the reader to Appendix~\ref{sec:stochastic_realization} 
%     % for some results based on the stochastic realization theory for the PMC;
%     % and~ for a detailed analysis of the GUM, the HMC and the RNN.
% \end{remark}
% % \yohan{With the stochastic realization theory, we can obtain the covariance function of the PMC.
% % but it is not necessarily a Gaussian distribution.}


\subsection{Theoretical analysis of PMCs}
% \katy{HERE}
We are interested in the modeling power of the PMC. 
For that, we characterize the covariance function
of the distribution $\p(\obs_{0:T})$ induced by the PMC,
and compare it with the one of the GUM presented in~\cite{salaun2019comparing}.
We focus on the case where the latent and observed variables 
are both scalar ($d_\latent = 1$ and $d_\obs = 1$).
The scalar case is interesting because it allows for a direct deduction of 
the covariance function derived from the PMC. 
% The PMC is described by \eqref{eq:trans-var-pmc}
% and satisfies \eqref{eq:constraints-pmc}.

For clarity, we set $r_0=1$, which means $\p(\obs_t)=\N(\obs_t;0;1)$,
for all $t \in \NN$. We also parameterize $\tilde{\gamma}=\gamma \eta$,
then the set of parameters is now given by $\theta=(a,b,c,e,f,\alpha,\beta,\eta,\gamma)$.
%             \item the HMC is obtained by setting $c=0$, $e=0$, $f=0$,
%             \item in an RNN, the transition between $(\latent_{t-1},\obs_{t-1})$ 
%             and $\latent_t$ is deterministic so $\alpha=0$. 
%             We have also set $e=0$, $f=0$. 
%             \item The GUM is obtained by setting $e=0$, $f=0$
% We start by presenting the covariance function of the GUM plugging the parameters
We start by presenting the covariance function of the GUM, 
% ($e=f=0$, and $\gamma=b$), 
where the matrix $M$ is diagonalizable. 
By plugging in $e=f=0$, and $\gamma=b$, the covariance function of the GUM is given by
% The covariance function of the GUM is given by
% With these assumptions, the matrix $M$ is diagonalizable in the GUM ($e=f=0$),
% and the covariance function is given by 
\begin{equation}
    \label{eq:covar-gum}
    \begin{aligned}
    % \Var(\Obs_t)&= \beta+b^2\eta \text{,} \quad \text{ for all } t \in \NN  \text{,}\\
    % % \Cov(\Obs_t,\Obs_{t+k})
    % \hat{r}
    % r_k = (a+cb)^{k-1} (a\eta b^2 + bc (\beta+ \eta b^2 )) \text{,} \quad \text{ for all } k \in \NN^*  \text{,}
    r_k & \overset{\text{GUM}}{=}  A^{k-1} B 
    % (\underbrace{a+cb}_{A})^{k-1} 
    % (\underbrace{a\eta b^2 + bc (\beta+ \eta b^2 )}_{B})
     \text{,} \text{ for all } k \in \NN^*  \text{,}
    \end{aligned}
\end{equation}
where $A=a+cb$, $B=a\eta b^2 + bc (\beta+ \eta b^2 )$,
and  $\Var(\Obs_t)= \beta+\eta b^2 = 1$, for all $t \in \NN$. 
The stationarity constraints are simplified to two constraints:
\begin{align*}
    \beta &= 1 - b^2 \eta \text{,}\\
    \alpha &= \big(1  - a^2 -2abc \big)\eta - c^2 \text{,}
\end{align*}  
In addition to the settings of the GUM, the covariance functions 
associated to a linear and stationary Gaussian HMC and RNN
are also derived.  
\begin{itemize}
\item \textbf{HMC -} with $c=0$ and  $r_k= a^{k} \eta b^2 $.
% which implies $A=a$ and $B=ab^2\eta$;
\item \textbf{RNN -} the transition between $(\latent_{t-1},\obs_{t-1})$ 
and $\latent_t$ is deterministic then $\alpha=0$. Moreover, 
$\latent_{0}=0$ and $\Obs_{0}$ is independent of $\latent_{0}$. 
Since $\Var(\Obs_t)= b^2 \beta+\eta = 1$, 
%Since in our framework 
%we first define $\p(\latentent_{0})$, we add an artificial observation $\Obs_{-1}$ such 
%that $\E(\Obs_{-1})=0$ and  $\Var(\Obs_{-1})=r_0$.
%Finally, since $\latentent_0=c\obs_{-1}$,
the constraint $\eta=c^2$ should also be satisfied to ensure that 
% to ensure that $\eta_t=\eta$ and 
$\Var(\Obs_t)=r_0=1$ for all $t \in \NN$. 
%the stationarity of $\{\Latent_t\}_{t \in \NN}$ and so of  $\{\Obs_t\}_{t \in \NN}$.
\end{itemize}




In order to extend this study for PMCs, we assume that 
$M$ is diagonalizable in the PMC, \ie~$M=PDP^{-1}$ with
\begin{eqnarray*}
    P &=& \begin{bmatrix} \frac{-a+bc+f+K}{2(ab+e)} & \frac{a-bc-f+K}{2(ab+e)} \\ 1 & 1  \end{bmatrix} \text{,} \\
    D&=& \begin{bmatrix} \frac {1}{2}(a+bc+f-K) & 0 \\ 0 & \frac {1}{2}(a+bc+f+K) \end{bmatrix} \text{,} \\
    P^{-1}&=& \begin{bmatrix} -\frac{ab+e}{K} & \frac{a-bc-f+K}{2K} \\ \frac{ab+e}{K} & \frac{-a+bc+f+K}{2K} \end{bmatrix} \text{,}
\end{eqnarray*}
where 
$$K=\sqrt{(a+bc+f)^2-4(af-ce)} \text{.} $$
Note that the condition $(a+bc+f)^2-4(af-ce) \geq 0$ is satisfied
since $M$ is diagonalizable. 
As a result,  we can deduce  $r_k$ for the PMC, which is summarized in the following proposition.
%  $r_k=\Cov(\obs_{t},\obs_{t+k})$, for all $k \in \NN$,
% which is summarized in the following proposition.

% From the previous relation, the following restrictions on the parameters are obtained:
% \begin{align*}
% % \label{eq:varcond1}
% \eta &= \alpha +  a\eta (a+2c\gamma) + c^2\\    
% % \label{eq:varcond2}
% \gamma \eta &= b \alpha +  (a^2b + 2abc \gamma + ae + af\gamma + ce\gamma)\eta  + (bc^2 + cf) \\
% % \label{eq:varcond3}
% 1 &= \beta + b^2 \alpha + ((ab + e)^2 + 2\gamma  (ab+e) (bc + f) )\eta+ (bc + f)^2
% \end{align*}

% \begin{equation}
%     \label{contrainte}
%     p(\obs_t)=\mathcal{N}(\obs_t;0;1) \text{, for all } 0 \leq t \leq T.
% \end{equation}

% We will show that the \gls*{pmc} can model more complex distributions than the \gls*{hmc}, \gls*{rnn} and SRNN, if the Linear and Gaussian assumptions are satisfied.
% The following proposition shows that the \gls*{pmc} can model more complex Gaussian distributions.
% \begin{proposition}
%     \label{prop-1}
%     Let $p(\obs_{0:T})$ be a Gaussian distribution satisfying for all positive
%     integers $T,t,k$
%     \begin{eqnarray}
%         \label{obs-pmm-e}
%         p(\obs_{0:T})&=&\mathcal{N}(\obs_{0:T};{\bf 0}; \tilde{\Sigma}) \text{,} \\
%         \label{cov-pmm-e}
%         \text{\rm cov}(\obs_{t},\obs_{t+k})&=& \left\{
%         \begin{matrix}
%             \scriptstyle
%             \noindent \tilde{A}^{k}     & \; \text{if } k \text{  is even} \\
%             \scriptstyle
%             {\tilde A}^{k-1} {\tilde B} & \; \text{otherwise.}
%         \end{matrix} \right. \text{,}
%     \end{eqnarray}
%     such that $\tilde{A}$ and $\tilde{B}$ defines
%     a valid covariance matrix.
%     Then for such $(\tilde{A},{\tilde B})$, it exists a set
%     of parameters $\theta=(a,b,c,e\neq 0,\alpha,\beta,\eta)$
%     such that $\p(\obs_{0:T})=p(\obs_{0:T})$.
% \end{proposition}
% This proposition shows that the linear and Gaussian PMM can model
% some Gaussian distributions which cannot be modeled by the previous
% linear and Gaussian \gls*{hmc}, \gls*{rnn} and SRNN.



\begin{proposition}
    \label{prop:cov}
    Let a linear and stationary (scalar) Gaussian PMC be defined by the transition and the conditional
    covariance matrices $M$ and $\Sigma_{t|t-1}$ in \eqref{eq:trans-var-pmc}
    and the initial covariance matrix 
    \begin{equation*}
    \Sigma_0 = \begin{bmatrix} \eta & \gamma \eta \\ \gamma \eta & 1 \end{bmatrix} \text{.} 
    \end{equation*}
    If $M$ is diagonalizable, the covariance function of $\{\Obs_t\}_{t \in \NN}$ reads 
    \begin{equation}
    \label{eq:cov-pmc}
    % \Cov(\Obs_{t},\Obs_{t+k})
    r_k= \overline{A}^{k} (\overline{B}+\frac{1}{2}) - \overline{C}^{k} (\overline{B}-\frac{1}{2})  \text{,}
    \end{equation}
    where 
    \begin{eqnarray*}
    \overline{A}&=& \frac{a + bc + f -K}{2}  \text{,} \\
    \overline{B} &=& \frac{a-bc-f-2 \gamma \eta (ab+e)}{2K}\text{,} \\
    \overline{C} &=&\frac{a + bc + f +K}{2} \text{,} \\
    K&=& \sqrt{(a+bc+f)^2 - 4(af  - ce )}
    \end{eqnarray*}
    and where the following stationarity constraints are satisfied:
    \begin{align*}
     b\eta+(ae+af\gamma+ce\gamma)+fc &=\gamma\eta \text{,}\\
     (1-a^2-2ac\gamma)\eta-c^2 & \geq 0 \text{,} \\
     1- b^2\eta- 2b\eta(\gamma-b)-e\eta(e+2f\gamma)-f^2 & \geq 0 \text{.}
    \end{align*}\\
\end{proposition}

\begin{proof}
    The proof relies on the assumption that  $M$ is diagonalizable, 
    which enables us to derive an explicit expression for the
    covariances of the pair $(\latent_t,\obs_t)$. 
    Then $r_k$ can directly be deduced from this expression that reads
    \begin{align*}
        % \label{eq:covariance}
        \Sigma_0 \times  (M^k)^\intercal & = \Cov (
            \begin{bmatrix} \latent_t , \obs_t \end{bmatrix}^\intercal, 
            \begin{bmatrix} \latent_{t+k} , \obs_{t+k} \end{bmatrix}^\intercal) \\
        &=\begin{bmatrix}
        \Cov(\latent_{t},\latent_{t+k}) &  \Cov(\latent_{t},\obs_{t+k})\\ 
        \Cov(\obs_{t},\latent_{t+k}) & \pmb{\Cov(\obs_{t},\obs_{t+k})}%=r_k}
        \end{bmatrix}\text{.}
    \end{align*}
    On the other hand, the stationarity constraints are given by \eqref{eq:constraints-pmc}.
    % , which were deduced by applying the Lemma~\ref{lem:recursion_covariance}.
    %  and the fact that the process $\{\latent_t,\obs_t\}_{t\in \NN}$ is stationary.
    We set $r_0=1$ and $\tilde{\gamma}=\gamma \eta$, so the following 
    stationary relation holds
    \begin{align*}
     \begin{bmatrix}
     \eta & \gamma \eta\\ 
     \gamma \eta & 1
    \end{bmatrix}  &=  \begin{bmatrix}
    \alpha & b\alpha \\ 
    b\alpha & \beta + b^2\alpha.
    \end{bmatrix} \; +    \begin{bmatrix}
    a & c\\ 
    ab+e & bc+f
    \end{bmatrix} 
     \begin{bmatrix}
    \eta &  \gamma \eta\\ 
    \gamma \eta & 1 \\ 
    \end{bmatrix}  
    \begin{bmatrix}
    a & ab+e\\ 
    c & bc+f
    \end{bmatrix}\text{.}
    \end{align*}
    Since the covariance matrix is symmetric and the diagonal elements are positive because
    they are variances, the set of 3 constraints are deduced directly from the previous relation.        
\end{proof}

\begin{remark}
    The stationarity of the distribution $\p(\obs_{0:T})$ 
    implies that
    its associated variance-covariance matrix $\Sigma^{\obs}_{T}$  is a Toeplitz matrix
    (\ie~the coefficients on each diagonal are equal) 
    fully determined by its first row given by the covariance  sequence
    $[r_0, r_1, \ldots,r_T]$, where $r_k$ is given by \eqref{eq:cov-pmc}, 
    for all $k \in \NN^*$.
    Thus, $\Sigma^{\obs}_{T}$   
    % depends on $\overline{A}$, $\overline{B}$ and $\overline{C}$
    reads as,  for all $T \in \NN^*$,

    % \begin{scriptsize}
        \begin{align*}
            % \displaystyle 
            % & \Cov(\Obs_{0:T},\Obs_{T})
            & \Sigma^{\obs}_{T}
            % (\overline{A}, \overline{B}, \overline{C})   
            \displaystyle  
            =
         \begin{bmatrix}
        1 & r_1 & r_2 & r_3 & \dots & r_T \\
        r_1 & 1 & r_1 & r_2 & \dots &\vdots \\
        r_2 & r_1 &  1 & r_1 &\ddots & \vdots\\
        r_3 & r_2&  r_1 & 1 &\ddots & \vdots\\
        \vdots &\ddots &  \ddots &\ddots &\ddots & \vdots\\
        r_T & \dots & r_3 & r_2 & r_1 & 1  
        \end{bmatrix}
        \text{.}
        \end{align*}
    % \end{scriptsize}
\end{remark}

%     \begin{scriptsize}
%         \begin{align*}
%             \displaystyle 
%             % & \Cov(\Obs_{0:T},\Obs_{T})
%             & \Sigma^{\obs}_{T}
%             % (\overline{A}, \overline{B}, \overline{C})   
%             \displaystyle  
%             \overset{PMC}{=} 
%             {\displaystyle \text{Toeplitz}
%             \big( [1, \overline{A} \big(\overline{B} + \frac{1}{2} \big) - \overline{C} \big(\overline{B} - \frac{1}{2} \big), \dots, 
%             \overline{A}^{T-1} \big(\overline{B} + \frac{1}{2} \big) - \overline{C}^{T-1} \big(\overline{B} - \frac{1}{2} \big) ]  \big)}
%             % \text{, for all } T \in \NN^* 
%             % \text{.} 
%             \\ \vspace*{0.4cm}
%             &= \begin{bmatrix}
%         1 & \overline{A}\big(\overline{B} + 
%         \frac{1}{2} \big)-\overline{C}\big(\overline{B} -\frac{1}{2} \big)
%         & \dots & \overline{A}^{T -1} \big(\overline{B} + \frac{1}{2} \big) 
%         - \overline{C}^{T -1} \big(\overline{B} - \frac{1}{2} \big) \\ 
%         \overline{A}\big(\overline{B} + \frac{1}{2} \big)-\overline{C}\big(\overline{B} - \frac{1}{2} \big)& 1 & \dots &\vdots \\
%         \vdots &\vdots &\ddots & \vdots\\
%         \overline{A}^{T-1} \big(\overline{B} + \frac{1}{2} \big) - \overline{C}^{T -1 } \big(\overline{B} - \frac{1}{2} \big) & \dots & \dots  &  1
%         \end{bmatrix}
%         \end{align*}
%     \end{scriptsize}
% \end{remark}


% \katy{Do I have to extend (proof and more details) this part similar to Achile's thesis ??????}


% \paragraph{Analysis - }
% The covariance functions associated to a linear and stationary 
% Gaussian  GUM, HMC and  RNN
% are particular cases of the covariance function of the PMC.
% % They are easily derived from the Proposition~\ref{prop:cov}
% % by setting $e=f=0$ and $\gamma=b$,
% % $c=0$ and $\alpha=0$, and $e=f=0$, respectively.
% For example, the covariance function $\{r_k\}_{k \in \NN}$ associated to 
% the linear and stationary
% Gaussian GUM~\eqref{eq:covar-gum}
%  is easily  derived from the Proposition~\ref{prop:cov}
% by setting $e=f=0$ and $\gamma=b$. 
% % Similarly, the covariance functions
% % associated to the linear and stationary Gaussian HMC and RNN
% % are also derived from the Proposition~\ref{prop:cov}
% % by setting $c=0$ and $\alpha=0$, respectively.

Proposition~\ref{prop:cov} shows that the PMC
generalizes the form of the covariance matrices of the GUM, HMC and RNN
by introducing the parameters $e$ and $f$.
However, it remains challenging to determine whether any covariance series 
in the form \eqref{eq:cov-pmc} can be generated by a PMC
because identifying $\overline{A}$, $\overline{B}$ and $\overline{C}$, 
in order to ensure that \eqref{eq:cov-pmc} represents a 
valid covariance series, is a complex problem. 
Nonetheless, we can exhibit some particular covariance functions 
that can be generated by a (particular) PMC but not by a GUM, HMC or RNN
as shown in the next proposition.


\begin{proposition}
    \label{prop:cov-pmc}
    Let $\tilde{A}$ and $\tilde{B}$ be two scalars, $r_0 =1$ and
    \begin{eqnarray}
    \label{eq:cov-pmm-e}
    r_k&=& \left\{
    \begin{matrix}
    \scriptstyle 
    \noindent \tilde{A}^{k}  & \; \text{if } k \text{  is even,}\\ 
    \scriptstyle
    {\tilde A}^{k-1} {\tilde B} & \; \text{otherwise.}
    \end{matrix} \right. %\text{.}
    \end{eqnarray}
    Then $\{r_k\}_{k \in \NN}$ is a covariance function if and only if
    \begin{equation}
    \label{eq:cond-A-B-tilde}
    -1  \leq \tilde A \leq  1 \quad \text{ and } \quad -\frac{ \tilde A^2 +1}{2} \leq \tilde{B} \leq \frac{ \tilde A^2 +1}{2} \text{,}
    \end{equation}
    and can be realized by a linear and stationary Gaussian PMC.\\
\end{proposition}


\begin{proof}
    The proof relies on the Carathéodory-Toeplitz
    theorem~\citep{akhiezer1965classical}  since $\Sigma^{\obs}_{T}$ is defined
    by a Toeplitz matrix with first row 
    $$[1,\tilde{B},\tilde{A}^2, \tilde{A}^2 \tilde{B}, \tilde{A}^4, \cdots].$$ 

    We analyze the series expansion of the covariance function to establish
    the necessary conditions for the positive semi-definiteness of $\Sigma^{\obs}_{T}$. 
    This theorem allows us to determine the values of $\tilde A$ and of
    $\tilde{B}$ in~\eqref{eq:cov-pmm-e} 
    such that $\Sigma^{\obs}_{T}$ is a valid covariance matrix.

    We deduce
    the constraints $-1  \leq \tilde A \leq  1$, 
    $ -\frac{ \tilde A^2 +1}{2} \leq \tilde{B} \leq \frac{ \tilde A^2 +1}{2} $. 
    Next, 
    % two cases provide the results exhibited in the proposition.
    setting $\gamma=b$, and~$f$ either as~$0$ or~$-a-bc$ (two particular cases of the PMC),
    we show that \eqref{eq:cov-pmc} coincides
    with \eqref{eq:cov-pmm-e}, with 
    \begin{eqnarray*}
        \left\{
       \begin{matrix}
       \tilde A=\sqrt{ce} \quad \text{ and } \quad \tilde{B}=b(c(1-b^2\eta)+e\eta)  & \;  \text{if } f = 0 \text{,} \\ 
       \tilde A=\sqrt{e^2\eta + a^2(1 -b^2\eta )}\quad \text{ and } \quad \tilde{B}= be\eta - a (1-b^2\eta) & \; \text{if } f = -a-bc  \text{.}
       \end{matrix} \right.
    \end{eqnarray*}
    Finally, for any $(\tilde A,\tilde{B})$ satisfying \eqref{eq:cond-A-B-tilde},
    we show that it is possible to find a set of parameters $(a,b,c,e,\eta,\alpha,\beta)$
    which satisfies the previous system and the stationarity constraints \eqref{eq:constraints-pmc}
    for both cases $f=0$ and $f=-a-bc$.
    For a detailed
    step-by-step proof, please refer to the Appendix~\ref{chap:appendix_22}.
\end{proof}

% \paragraph{Analysis -}
Proposition \ref{prop:cov-pmc} shows that it is possible 
to produce a covariance function
$$r_{k}=A^{k-1} B(k) \text{,}$$ 
with a switching $B(k)$ satisfying $B(k)=A$ if $k$ is even and
$B(k)=B$, otherwise.
The constraints on $A$ and $B$ with this switching are
$-1 \leq A \leq 1$ and $-\frac{A^2+1}{2} \leq B \leq \frac{A^2+1}{2}$ 
since $B(k)$ is as expression of $A$ and $B$. \\

This result can be compared with that of the GUM in the scalar 
case~\citep{salaun2019comparing}, 
that can produce any covariance function  given by~\eqref{eq:covar-gum},
$r_{k}=A^{k-1} B$,   with the constraints     $-1  \leq A  \leq 1 $ and 
$\frac{A-1}{2} \leq  B \leq \frac{A+1}{2}$.
In other words, this proposition shows that the linear and stationary Gaussian PMC
can model some Gaussian distributions which cannot be modeled by 
the previous linear and stationary Gaussian GUM.
% \katy{Check if it is correct and change the caption of the plot}
% This proposition shows that the linear and stationary Gaussian PMC 
% can model some Gaussian distributions which cannot be modeled 
% by the previous linear and stationary Gaussian GUM.
%-----------------------------------
% The following proposition shows that the linear and Gaussian PMC can model
% some Gaussian distributions which cannot be modeled by the previous
% linear and Gaussian HMC, RNN and GUM.


% \begin{figure}
%     \centering
%     \includegraphics[width=0.5\linewidth]{Figures/Graphical_models/comp_prop.pdf}
%     \caption{The parallelogram (light blue) coincides with all the multivariate 
%     centered Gaussian distributions with a covariance matrix which satisfy
%     $\Cov(\Obs_t,\Obs_{t+k})=A^{k-1} B(k)$, for all $k \in \NN$. 
%     Such distributions can be modeled by a GUM.
%     \katy{Check if this interpretation is correct 
%     and change the caption of this plot}}
%     \label{fig:comp_pmc}
% \end{figure}


% The parallelogram (light blue) coincides with all the multivariate 
% centered Gaussian distributions with a covariance matrix which satisfy 
% Cov(Xt, Xt+τ ) = Aτ−1B. Such distributions can be modeled by a GUM.
% The blue (resp. orange) areas (resp. curves) coincide with the value of
% A and B which can be taken by the HMM (resp. the RNN). 
% This results shows that the modeling power of the GUM is larger than 
% that of the HMM, which is larger than that of the RNN. 
% However, let us note that in the context of this study, 
% the RNN is finally defined by only 2 free parameters, 
% the HMM relies on 3 parameters and the GUM on 4 parameters.

% \begin{figure}[htb]
%     \begin{subfigure}[b]{0.48\linewidth}
%       \centering
%       \includegraphics[width=6cm]{Figures/Graphical_models/gum_prop.pdf}
%       \caption{GUM}
%       \label{fig:gum_prop}
%       \vspace{1.1cm}
%     \end{subfigure}
  
%     \begin{subfigure}[b]{0.48\linewidth}
%       \centering
%       \includegraphics[width=6cm]{Figures/Graphical_models/pmc_prop.pdf}
%       \caption{PMC}
%       \label{fig:pmc_prop}
%     \end{subfigure}  
%     \caption{xx}
%     \label{fig:comparison}
%   \end{figure}
  






\section{Conclusions}
This chapter was devoted to the development, study, comparison 
and application of a general generative model for sequential data 
based on the PMC model.
Our approach combined the advantages of the HMM, RNN and GUM models 
and encapsulated them in a single framework.
A new parameter estimation method based on the variational inference
framework was also presented for the general PMC model, 
which is computationally efficient and easy to implement.
Moreover, we presented a particular instance of the variational PMC model,
combining the PMC model and deep parameterizations. This model has been compared
with the RNN and SRNN models on the MNIST and polyphonic music data sets. The
results show that the performance of the deep PMCs is better than of SRNN and
RNN models.
We have also shown that the linear and stationary Gaussian PMC
can model some Gaussian distributions which cannot be modeled by
the previous linear and stationary Gaussian HMC, RNN and GUM.
% We also showed that our framework can be used for 
% supervised sequential learning problems, 
% where the proposed parameter estimation method is still valid.



\thispagestyle{empty}

% !TEX root = late\obs_avec_réduction_pour_impression_recto_verso_et_rognage_minimum.tex
\chapter{Triplet Markov models for semi-supervised classification}
\markboth{CHAPTER 3. TMCs FOR SEMI-SUPERVISED CLASSIFICATION}{Short right heading}
\label{chap:semi_supervised_pmc_tmc}


\localtableofcontents
\pagebreak

% \epigraph{
% ``It’s not an idea until you write it down.'' }{Ivan Sutherland}


\pagebreak

\section{Introduction}
\label{sec:introduction_ch3}
% \yohan{Here, I would directly consider that you want to apply previous models in
% the case where we have labels associated to each observation. So you have (z,x)
% which is replaced by (z,x,y) where y are the labels. And I would insist on the
% fact that there is no difficulties for the inference steps in the supervised
% case since x would be replaced by (x,y). However, two difficulties remain : 1.
% how to conceive a generative models for the observations AND the labels? 2. How
% to proceed to bayesian inference when the labels are partially observed (the
% case where there no observed will be Chapter 4).}

In this chapter, we want to extend the study we have done in the previous chapters
to the case where we have labels associated with each observation.
% We recall the notations and the problem statement,  
% we consider a sequence of random variables $\obs_{0:T}=(\obs_{0}, \dots, \obs_{T})$, 
% a sequence of labels $\lab_{0:T}=(\lab_{0}, \dots,\; \lab_{T})$ associated 
% to the previous sequence $\obs_{0:T}$, and a sequence of continuous latent variables
% $\latent_{0:T}=(\latent_{0}, \dots,\; \latent_{T})$.
Let us recall the sequence of random variables $\obs_{0:T}=(\obs_{0}, \dots, \obs_{T})$
and the sequence of labels $\lab_{0:T}=(\lab_{0}, \dots,\; \lab_{T})$ associated to 
the previous sequence $\obs_{0:T}$, 
where $\obs_t \in \mathbb{R}^{d_x}$, and $\lab_t \in \Omega=\{\omega_1,\dots,\omega_C\}$, 
with $C$ the number of classes.
We also consider a sequence of latent variables 
$\latent_{0:T}=(\latent_{0}, \dots,\; \latent_{T})$, where  
$\latent_t \in \mathbb{R}^{d_{\latent}}$.
% For example, we consider the problem of image segmentation, where
% the observations $\obs_{0:T}$ can represent a 
% noisy grayscale image while $\lab_{0:T}$ represents
% the original black and white image.
% The goal is to recover the original image from a noisy version of it.
% % \katyobs{Add an image here and in the introduction}


The objective associated to Bayesian classification 
consists in computing, for all $t$,
the posterior distributions  $p(\lab_t|\obs_{0:T})$. %given in~\eqref{eq:post_distrib}.
% Different approaches can be considered to address this problem for sequential
% data. 
The difficulty of the problem depends on the availability of the labels
associated with the observations. When the labels are observed, the problem is
referred to as supervised learning.
% Different approaches can be considered to address this problem
% for sequential data. 
% The difficulty of the problem depends on the availability of the labels
% associated with the observations. When the labels are observed, the problem is
% referred to as supervised learning. 
Chapter~\ref{chap:pmc} 
 was dedicated to a general
generative model based on PMCs, which can be used for 
supervised learning. This adaptation involves taking as observed variable the
pair of observations and labels $\obs_t \leftarrow (\obs_t, \lab_t)$, and
applying the adapted variational Algorithm~\ref{algo:algo_train_dpmc_gen} 
 discussed in the chapter (see Appendix~\ref{chap:appendix2} for more details). 
However, in many real-world applications, it is expensive or impossible to obtain
labels for the entire sequence $\obs_{0:T}$
due to various reasons, such as the high cost of labeling,
the lack of expertise, or the lack of time, etc.
The labels can be partially observed or not observed at all,
which leads to the semi-supervised and unsupervised learning problems, 
respectively.
Two main challenges arise in this context:
\begin{itemize}
    \item How to effectively design generative models that
not only generate  observations $\obs_t$, but also generate labels $\lab_t$? 
% The idea is to
% integrate the labels into the generative process so that they are consistent
% with the observations in terms of the underlying data distribution.
% \item Bayesian inference with partial labels: When labels are only partially
% observed or not observed at all, the challenge becomes more complicated. 
\item How to perform effective Bayesian inference under 
these  conditions?
%  of partially observed labels or no labels at all?
% ~\ref{chap:unsupervised_pmc_tmc}.
% These scenarios will be explored in detail in Chapter 4.
\end{itemize}


This chapter is devoted to the semi-supervised learning problem, 
and the unsupervised learning problem  will be addressed in the next chapter.
Here, the objective is to estimate the unobserved labels from the observations
and the observed labels. To that end, the TMC model is  considered (see Subsection~\ref{sec:seq_gen_models})
 in which we can model not only the
sequence of observations $\obs_{0:T}$, 
and their associated labels $\lab_{0:T}$, but also incorporate an 
auxiliary sequence $\latent_{0:T}$,  which can provide additional information
about the relationship between the observations and the labels.
We  propose a new adaptation of the VI algorithm
presented in the previous chapter, which enables us to estimate the parameters
of a general TMC model, and the unobserved labels. 
This general semi-supervised learning algorithm enables us to derive a variety
of (deep) generative models which have been applied to sequential Bayesian
classification problems. Finally, we consider the problem of image segmentation, where
the observations $\obs_{0:T}$ represent a 
noisy grayscale image while $\lab_{0:T}$ represent
the original black and white image.
The goal is to recover the original image from a noisy version of it.
We show that our approach outperforms the
state-of-the-art semi-supervised learning algorithms 
such as the~\gls*{vsl}~\citep{chen2019variational},
and the~\gls*{svrnn}~\citep{butepage2019predicting}.
% \yohan{Maybe you should explain that the TMC you already introduced in the
% introduction enables you to propose a full generative models on the labels and
% the observations (in the sense that existing models are conditional models ?)}

% \yohan{I would replace this section by : ``a brief description of the
% semi-supervised problem in generative models''}

% \newpage
\section{Semi-supervised estimation in general TMC}
% \section{General Triplet Markov Chain}
\label{chp:gen_tmc}
\subsection{General parameterization of the TMC}
\label{sec:general_param_tmc}
% \yohan{Think about it : is it better to introduce such parameterization before
% or after the ELBO? Because we never use the specifical parameterization in the
% bayesian inference section?}

% the choice of the transition distribution
% $\p(\triplet_t| \triplet_{t-1})$ is a thorny problem
% from a theoretical point of view, similar to the one encountered in the PMC model.
% The transition distribution can be factorized
% in different ways and derive different models.



We recall the TMC model given in Equation~\eqref{eq:tmc_intro}:
$$\p( \latent_{0:T}, \lab_{0:T}, \obs_{0:T} ) = \p(\latent_0,\lab_0,  \obs_0) 
\prod_{t=1}^T \p(\triplet_t| \triplet_{t-1}) \text{,}$$
where the triplet $\triplet_t=(\latent_t, \lab_t,  \obs_t)$.
Here, it is possible to have different factorizations of the transition
distribution $\p(\triplet_t| \triplet_{t-1})$. 

\begin{example}
    The following factorizations are  the possible choices for the transition
    distribution $\p(\triplet_t| \triplet_{t-1})$:
    \begin{align*}
        \p(\triplet_t| \triplet_{t-1}) 
         &= \p(\obs_{t}| \triplet_{t-1}) \p(\lab_{t}| \obs_{t}, \triplet_{t-1}) \p(\latent_{t}| \obs_{t}, \lab_{t}, \triplet_{t-1}) 
         \text{,}\\
        \p(\triplet_t| \triplet_{t-1}) &= \p(\obs_{t}| \lab_{t}, \triplet_{t-1}) \p(\lab_{t}| \triplet_{t-1}) 
        \p(\latent_{t}|\obs_{t},  \latent_{t}, \triplet_{t-1}) \text{,}\\
        \p(\triplet_t| \triplet_{t-1}) &= \p(\obs_{t}| \obs_{t}, \lab_{t}, \triplet_{t-1}) \p(\lab_{t}| \latent_{t}, \triplet_{t-1})
        \p(\latent_{t}|\triplet_{t-1}) \text{.}
    \end{align*}
    In the first example, $\obs_t$ depends on the triplet
    $\triplet_{t-1}$,  the label $\lab_t$ depends on the observation $\obs_t$
    and the triplet $\triplet_{t-1}$, and the latent variable $\latent_t$ depends on
    the observation $\obs_t$, the label $\lab_t$ and the triplet $\triplet_{t-1}$.
\end{example}
The choice of the factorization of the transition distribution depends on 
the specific application and the underlying model.
Thus,  we use a general notation for the associated
 conditional distributions
$\p(\obs_t |\;\cdot\;)$, $\p(\latent_t| \;\cdot\;)$ and $\p(\lab_t | \;\cdot\;)$ 
in order to avoid presenting a specific dependence between variables. 


In Chapter~\ref{chap:pmc}, 
we have introduced the probability density functions on $\mathbb{R}^{d_\obs}$, 
$\mathbb{R}^{d_\latent}$, as $\zeta$, 
and $\eta$, respectively 
(Equations~\eqref{pmc-theta-1-gen}, and~\eqref{pmc-theta-2-gen}).
They are introduced to describe a general parameterization of the PMC model.
% These functions are used to describe the parameters of the transition
% distributions for latent and observed variables within the PMC framework. 
As we extend these ideas to the TMC model in this chapter, 
we continue to use the functions to parameterize the transition distribution 
in the TMC model.
We also define $\vartheta$ as a probability distribution on $\Omega$, 
which is used to parameterize the transition distribution for the labels
and is differentiable w.r.t. their parameters.
The general parameterized model is described by:

\begin{align}
\label{eq:general_model_tmc}
    \p(v_t|v_{t-1}) &=  \p(\obs_t|\;\cdot\;) \; \p(\latent_t|\;\cdot\;) \; \p(\lab_t|\;\cdot\;) \text{,}\\
\label{eq:pz}
    \p(\latent_t | \;\cdot\;) &= \eta(\latent_t; \; \pz(\;\cdot\;) ) \text{,}\\
\label{eq:py}
    \p(\lab_t | \;\cdot\;) &= \vartheta(\lab_t; \; \py(\;\cdot\;)) \text{,}\\
    \label{eq:px}
    \p(\obs_t | \;\cdot\;) &= \zeta(\obs_t; \; \px(\;\cdot\;) ) \text{,}
\end{align}
where  $\py$, $\px$ and $\pz$ are 
vector-valued functions that are assumed to be
differentiable w.r.t. $\theta$. 
% In other words, $\py$, $\px$ and $\pz$
% describe the parameters of the (conditional) distribution $\vartheta$, 
% $\zeta$ and $\eta$, respectively.
% We also assume that $\zeta$, $\vartheta$ and $\eta$ are differentiable w.r.t. 
% their parameters.

% \yohan{The problem with this notation is that we have no idea on the dependencies 
% on the conditional variables and so on the chosen factorization. 
% So if you want to avoid to make a choice on the factorization, explain 
% it better (for eg: in function of the chosen factorization, 
% $z_t$ can depend on $x_t$,...)}


\begin{example}
    % \yohan{explain in this example the factorization you have chosen and the
    % interest of such model (binar classification with continuous latent and
    % observation variables,), so a generalization of the TMC of pieczynski where
    % the latent variable is now continuous. For simplicity I will just assume the
    % classical hidden markov chain with an additional latent variable in this
    % example and explain that y is the pixel, x the noisy observation and z a
    % variable which enables to ``learn'' the nature of the noise}

    For the sake of clarity, let us explore a specific application of the
    TMC model where the labels $\lab_t$ are binary ($\Omega=\{\omega_1,\omega_2 \}$).
    The observations satisfy $\obs_t \in \mathbb{R}$, and $\latent_t \in \mathbb{R}$, 
    for all $t$.
    This setup is particularly useful in image processing tasks such as 
     noise reduction and classification,
     where $\lab_t$ represents a pixel's classification (\eg~object vs. background),
     $\obs_t$ is the observed noisy pixel value, and $\latent_t$ models the latent
     variables that influence the observation's noise characteristics.
    Thus, we can extend the TMC model proposed in~\citep{pieczynski2005triplet},
     by incorporating a continuous latent variable $\latent_t$.
    In particular, the model can be described as
    % \begin{align*}
    %     \p(\lab_t | \lab_{t-1},\, \obs_{t-1}, \latent_{t-1} ) 
    %     &= \vartheta(\lab_t;\; \py(\lab_{t-1},\, \obs_{t-1}, \latent_{t-1})) 
    %     \text{,}\\
    %     \p(\obs_t| \lab_{t-1} )   
    %     &= \zeta(\obs_t;\; \px(\lab_{t-1}) ) \text{,}\\
    %     \p(\latent_t| \obs_{t-1},\, \lab_{t-1})
    %     &=  \eta(\latent_t;\pz(\obs_{t-1},\, \lab_{t-1})  )  
    % \end{align*}
    \begin{align*}
        \py(\lab_{t-1},\, \obs_{t-1},\, \latent_t) &= 
        {\rm sigm}(a_{\lab_{t-1}} \obs_{t-1} + b_{\lab_{t-1}} \latent_t + c_{\lab_{t-1}})
        \text{,} \\
        \px(\obs_{t}) 
        &= \big[d_{\lab_{t}},\, \sigma_{\lab_{t}} \big] \text{,} \\
        \pz(\obs_{t-1},\, \lab_{t-1}) &=  
        \big[ e_{\lab_{t-1}}\obs_{t-1},\, \sigma_{\lab_{t-1}}'\big] 
        \text{,} \\
        \vartheta(\lab_t; \rho)&= \Ber\left(\lab_t; \rho \right) 
        \text{,} \\
        \zeta(\obs_t; s = [\mu,\, {\sigma}] ) &=     
        \mathcal{N}(\obs_t; \mu,\, {\sigma}^2 )
        \text{,} \\
        \eta(\latent_t; s' = [\mu',\, {\sigma'}])  
        &= \N\left(\latent_t;  \mu' ,\, {\sigma'}^2 \right)   
        \text{,}
    \end{align*}
where ${\rm sigm}(v)=1/(1+\exp(-v)) \in [0,\,1]$ is the sigmoid function.
Note that, for example, the notation $d_{\lab_{t}}$ means that
the parameter $d$ depends on the label $\lab_{t}$, 
\ie~$\p(\obs_t|\lab_{t} = \omega_j) = \N(\obs_t; d_{\omega_j},\, \sigma_{\omega_j}^{\obs})$.
The set of parameters is then given by:
\begin{align*}
    \theta &= \big ( a_{\omega_i},\, b_{\omega_i},\, c_{\omega_i},\, 
    d_{\omega_j},\, 
    \sigma_{\omega_j},\, e_{\omega_i},\, \sigma_{\omega_i}'
    | (\omega_i,\omega_j) \in \Omega^2 \big ) \text{.} 
\end{align*}
This parameterization can be easily extended to
the multi-class cases with $C>2$ by replacing $\pyun$
by a vector of the ${\rm softmax}$ function, 
and $\vartheta(\lab_t;\rho)$ 
by the categorical distribution described
by the $C$ components of  a vector $\rho$. 
% \katyobs{Check if it is a good example}
\end{example}

\begin{remark}
    \label{rem:general_param}
    The parameterization of the TMC model is very general and can be used to derive 
    a variety of models. 
    Similarly to the PMC model, the functions $\py$, $\px$ 
    and $\pz$  can also be parameterized by deep neural networks, where 
    the parameters $\theta$ encompass the 
    weights and biases of the neural networks.
    We will refer to this model as the \gls*{dtmc}
    % deep TMC (d-TMC)
     model.
    % For example, the Variational Sequential Labeler (VSL)~\citep{chen2019variational}
    % and the Semi-Supervised Variational Recurrent Neural Network (SVRNN)~\citep{butepage2019predicting}
    % are particular instances of the TMC model.
    % In the next sections, we present three particular instances of the TMC model.
\end{remark}
       



\subsection{A brief description of the semi-supervised problem}
In many practical scenarios, obtaining complete label information for all data
points is often infeasible. Consequently, we frequently encounter 
situations where only a
subset of the labels is observed. This incomplete labeling poses significant
challenges for effective model training and inference.  
% In this section, we address the problem of semi-supervised Bayesian
%  estimation in general TMCs given in~\eqref{eq:tmc_intro}.
%  ~\eqref{eq:tmc_intro}.
To clarify our approach, we decompose the sequence of labels  
$\lab_{0:T}$ into observed and hidden components:
$$\lab_{0:T}  = (\labl, \labu) \text{,}$$ 
where $\labl=\{\lab_t\}_{t\in \LL}$  (resp. $\labu=\{\lab_t\}_{t\in \U}$) 
is the set of observed (resp. hidden) labels.
Here,  $\LL$ (resp. $\U$) denotes the set of time indices where labels are 
observed (resp. hidden).
We assume that $\LL \cap \U = \emptyset$ and $\LL \cup \U = \{0,\dots,T\}$.
For example, if $T=5$, and labels are observed at time steps $0,1,2$, 
then $\LL=\{0,1,2\}$ and $\U=\{3,4,5\}$.
Thus, our observed data is  $(\obs_{0:5}, \lab_0, \lab_1, \lab_2)$, and 
the hidden labels are $(\lab_3, \lab_4, \lab_5)$.


% Our  main objectives are to estimate the parameters $\theta$ of the TMC model
% and to compute the posterior distribution of the hidden labels $\labu$.
% % is to train relevant generative models 
% % based on TMCs~\eqref{eq:tmc}, and to look for estimating 
% % the missing labels 
% % associated to each sequence. 
% The introduction of a continuous latent process $\latent_{0:T}$ is 
% interesting from a modeling point of view, however
% a direct application would involve the computation of intractable integrals.

Here, our goal is to estimate the parameters $\theta$  of the TMC model
from  $(\obs_{0:T},\labl)$,
and compute the posterior distribution of the hidden labels $\lab_t$,
for all $t \in \U$. 
The likelihood of the observed data $(\obs_{0:T},\labl)$ reads
\begin{align}
    \label{eq:likelihood_semi}
    \p(\obs_{0:T}, \labl)=\sum_{\lab_s \text{, }  s \in \U} 
    \int \p(\latent_{0:T},\lab_{0:T},\obs_{0:T}) {\rm d}\latent_{0:T} \text{,}
\end{align}
and the posterior distributions, for all $t \in \U$, are defined as 
\begin{align}
    \label{eq:post_distrib_semi}
    p(\lab_t|\obs_{0:T},\labl)=\frac{\sum_{\lab_s \text{, }  
    s \in \U \backslash \{t\}} \int 
    p(\latent_{0:T},\lab_{0:T},\obs_{0:T}) {\rm d}\latent_{0:T}  } 
    { \sum_{\lab_s \text{, }  s \in \U} \int p(\latent_{0:T},\lab_{0:T},\obs_{0:T}) {\rm d}\latent_{0:T}} 
    \text{.}
\end{align}

Equations~\eqref{eq:likelihood_semi} and~\eqref{eq:post_distrib_semi}
involve integrals w.r.t. the latent variables.
% , which are not exactly computable
% in general.
% ~\eqref{eq:likelihood_semi}, 
% and posterior distributions~\eqref{eq:post_distrib_semi} 
%  involve  integrals w.r.t. the latent variables.
Consequently, they are not exactly computable in general. 
To that end, we have proposed a VI approach
presented in Chapters~\ref{chap:main_concepts}
and~\ref{chap:pmc}.
However, the algorithms cannot be applied directly to this case 
since partial observations of the
sequence $\lab_{0:T}$ result in hidden labels.
As consequence, the variational distribution has to be adapted 
to the case where the observed variables are 
$(\obs_{0:T}, \labl)$
and the latent variables are $(\latent_{0:T}, \labu)$.
We deal with both discrete and continuous latent variables,
which is a challenging problem since the variational distribution
has to be factorized in order to be tractable, and 
has to be independent of the time step $t$ 
in order to have a general
model able to be applied to different contexts.



% \katy{HERE}
% \begin{remark}
%     \label{rem:factorization_tmc}
%     % Our main interest is to propose a general model, 
%     % which enables us to derive a variety of models.
%     The following factorizations of the transition distribution are 
%     some few examples, and there are other potential factorizations. 
%     The choice of which to use will be driven by the specific application and 
%     the underlying model.
%     \begin{align*}
%         \p(\triplet_t| \triplet_{t-1}) &= \p(\obs_{t}| \triplet_{t-1}) \p(\lab_{t}| \obs_{t}, \triplet_{t-1}) \p(\latent_{t}| \obs_{t}, \lab_{t}, \triplet_{t-1}) \text{,}\\
%         \p(\triplet_t| \triplet_{t-1}) &= \p(\obs_{t}| \lab_{t}, \triplet_{t-1}) \p(\lab_{t}| \triplet_{t-1}) 
%         \p(\latent_{t}|\obs_{t},  \latent_{t}, \triplet_{t-1}) \text{,}\\
%         \p(\triplet_t| \triplet_{t-1}) &= \p(\obs_{t}| \obs_{t}, \lab_{t}, \triplet_{t-1}) \p(\lab_{t}| \latent_{t}, \triplet_{t-1})
%         \p(\latent_{t}|\triplet_{t-1}) \text{.}
%     \end{align*}
    
%     % This choice of the factorization
%     % can derive different models. 
%     % , such as the Variational Sequential Labeler (VSL)~\citep{chen2019variational}
%     % and the Semi-Supervised Variational Recurrent Neural Network (SVRNN)~\citep{butepage2019predicting}.
% \end{remark}


% In Chapter~\ref{chap:pmc}, we have presented the PMC model
% as a generative model for sequential data.
% This model relies on a general parameterization of the two
% distributions ($\p(\obs_t|\lab_{t-1:t}, \obs_{t-1})$ 
% and $\p(\lab_t|\lab_{t-1},\obs_{t-1} )$)
% involved in the transition distribution
% $\p(\obs_t,\lab_t|\obs_{t-1},\lab_{t-1})$, which 
% defines the joint distribution $\p(\obs_{0:T},\lab_{0:T})$,
% \begin{equation*}
%     \label{pmc_semi}
%      \p(\lab_{0:T},\obs_{0:T})=\p(\lab_0,\obs_0)\prod_{t=1}^T \p(\lab_t,\obs_t|\lab_{t-1},\obs_{t-1}) \text{.} 
% \end{equation*}


% However, the choice of these distributions 
% cannot be obvious in practice and can have
% an impact on the performance of the classification. 
% When we introduce the PMC the observation $\obs_t$ not only depends on $\lab_t$
% but can also depend on $\lab_{t-1}$ (and/or $\obs_{t-1}$).
% A consequence one has to deal with during the inference process is that
% $\lab_{t-1}$ can also be associated to a class related to the observation
% $\obs_t$.
% The goal of this section is to consider three processes, 
% the observations $\obs_{0:T}$, the labels $\lab_{0:T}$ and a third 
% process $\latent_{0:T}$. 
% This third auxiliary and continuous process  $\latent_{0:T}$
% aims at complexifying the distribution
% $\p(\lab_{0:T},\obs_{0:T})$~\citep{bayer2015learning}.
% % The rationale behind this auxiliary process
% % is that even if $\p(\latent_{0:T})$ and 
% % $\p(\obs_{0:T}|\latent_{0:T})$ are two elementary distributions, the resulting
% % distribution $\p(\obs_{0:T})=\int \p(\latent_{0:T})\p(\obs_{0:T}|\latent_{0:T}) {\rm d} \latent_{0:T}$ 
% % can be complex~\citep{bayer2015learning}. 
% In our classification context, the third latent process   can be 
% used to implicitly estimate the nature of 
% the distributions $\p(\obs_t|\lab_{t-1:t},\obs_{t-1})$ 
% and $\p(\lab_t|\lab_{t-1}, \obs_{t-1})$
% of our presented PMC; or to model the process $(\lab_{0:T}, \obs_{0:T})$ 
% since  $\p(\lab_{0:T},\obs_{0:T})=\int \p(\latent_{0:T})\p(\lab_{0:T},\obs_{0:T}|\latent_{0:T}) {\rm d}\latent_{0:T}$.
% Thus, we propose to consider a generalization of the PMC model 
% by introducing a third continuous process $\latent_{0:T}$.
% This generative model is  the Triplet Markov Chain
% model~\eqref{eq:tmc}, 
% which was previously presented in Section~\ref{chp:gen_tmc}.

% \katy{END HERE}


\section{Semi-supervised Variational Inference for TMCs}
In this section, we explore the semi-supervised variational inference method
applied to  general TMCs.
We start by the ELBO, and the formulation of the variational
distribution. Finally, we propose an algorithm to estimate the parameters
of the TMC model in the semi-supervised context.

\subsection{ELBO for semi-supervised learning}

We consider the variational distribution 
$\q(\latent_{0:T}, \labu|\obs_{0:T}, \labl)$. 
The ELBO of the log-likelihood~\eqref{eq:likelihood_semi} reads
\begin{align}
\label{eq:elbo_seq}
\Qsemi(\theta,\phi) \!&=\!  - \! \sum_{\substack{\lab_s, \\ s \in \U} } \int  \q(\latent_{0:T}, \labu|\obs_{0:T}, \labl) %\times \nonumber \\ & \quad \quad
\log \left(\frac{\q(\latent_{0:T},\labu  |\obs_{0:T},\labl)}{\p(\latent_{0:T}, \lab_{0:T},\obs_{0:T})}\right)  {\rm d} \latent_{0:T} 
\text{.}
\end{align}
% where  the joint distribution $\p(\latent_{0:T},\obs_{0:T}, \lab_{0:T})$ 
% is defined by the TMC model.
Let us now discuss on the 
computation of 
\eqref{eq:elbo_seq}.
First, it is worthwhile to remark 
that it does not depend on the choice of the generative model.
Any parameterized TMC model~\eqref{eq:general_model_tmc}-\eqref{eq:py}
can be used 
since $ \p(\latent_{0:T},\obs_{0:T}, \lab_{0:T})$ is 
defined by the transition distribution $\p(\triplet_t| \triplet_{t-1})$
and the initial distribution $\p(\triplet_0)$.
Thus, its computation only depends
on the choice of the variational
distribution $\q(\latent_{0:T}, \labu|\obs_{0:T}, \labl)$, 
which can be factorized in two different ways.
% \katyobs{For sake of clarity, we will omit
% the initial distribution of the variables at time $t=0$.}

The first factorization is given by
\begin{align}
\label{eq:fact-1}
\q(\latent_{0:T}, \labu | \obs_{0:T}, \labl)= & \q^{0} \times  \prod_{t=1}^T  \q(\latent_t|\latent_{0:t-1},\lab_{0:t-1},\obs_{0:T},\lab_{t+1:T}^{\LL}) \times \nonumber \\ 
& \prod_{\substack{t\geq 1\\ t \in \U}}^T  \q(\lab_t|\lab_{0:t-1},\latent_{0:t},\obs_{0:T},\lab_{t+1:T}^{\LL}) \text{,}
\end{align}
where  $\lab_{0:t-1}=(\lab_{0:t-1}^{\U},\lab_{0:t-1}^{\LL})$, and 
\begin{eqnarray*}
    \q^{0}&=& \left\{
    \begin{matrix}
    q(\latent_0| \obs_{0:T},\labl)  & \; \text{if }  t=0 \in \LL \text{,}\\ 
    \q(\latent_0| \obs_{0:T},\labl) \q(\lab_{0}|\latent_0, \obs_{0:T},\labl) & \; \text{otherwise.}
    \end{matrix} \right. %\text{.}
\end{eqnarray*}
While the second one coincides  with 
\begin{align}
\label{eq:fact-2}
\q(\latent_{0:T}, \labu|\obs_{0:T}, \labl)=&  \q^{0} \times  \prod_{t=1}^T  \q(\latent_t|\latent_{0:t-1},\lab_{0:t},\obs_{0:T},\lab_{t+1:T}^{\LL}) \times \nonumber \\ 
& \prod_{\substack{t\geq 1\\ t \in \U}}^T  \q(\lab_t|\lab_{0:t-1},\latent_{0:t-1},\obs_{0:T},\lab_{t+1:T}^{\LL}) \text{,}
\end{align}
and
\begin{eqnarray*}
    \q^{0}&=& \left\{
    \begin{matrix}
    q(\latent_0| \obs_{0:T},\labl)  & \; \text{if } t=0 \in \LL \text{,} \\ 
    \q(\latent_0|\lab_0, \obs_{0:T},\labl) \q(\lab_{0}|\obs_{0:T},\labl) & \; \text{otherwise.}
    \end{matrix} \right. %\text{\\}
\end{eqnarray*}


Once the variational distribution is chosen, and the generative model is fixed
(\ie~the factorization of the transition distribution
is fixed
% ~\eqref{eq:general_model_tmc}-\eqref{eq:py}
), 
the ELBO $\Qsemi(\theta,\phi)$ in~\eqref{eq:elbo_seq} can be rewritten as
\begin{align}
\label{eq:elbo_seq2}
\Qsemi(\theta,\phi) = %\beta 
\L^{\LL}(\theta,\phi) + \L^{\U}(\theta,\phi) \text{,}
\end{align}
where
% for the sake of clarity, we assume that $t=0$ is observed, \ie~$0 \in \LL$,
\begin{align}
\label{eq:elbo_seq_obs}
\L^{\LL}(\theta,\phi) \;=\; &  \sum_{\substack{ t \in \LL} } \Big[ 
\E_{\q(\latent_{t}| \cdot ) } (\log p(\obs_t| \; \cdot \; )
+\log p(\lab_t| \; \cdot \; ))
% \nonumber \\& 
- \dkl (\q(\latent_{t}|\; \cdot \; )|| \p(\latent_{t}|\; \cdot \; )) 
\Big] \text{,} \\ 
\label{eq:elbo_seq_hidden}
\L^{\U}(\theta,\phi) = & \sum_{\substack{ t \in \U} } \Big[
\E_{\q(\latent_{t}, \lab_{t}| \cdot ) } \log p(\obs_t| \; \cdot \; )
- \dkl (\q(\latent_{t}|\; \cdot \; )|| \p(\latent_{t}|\; \cdot \; ))
\nonumber \\& \quad  \quad \quad
- \dkl (\q(\lab_{t}|\; \cdot \; )|| \p(\lab_{t}|\; \cdot \; ))
\Big] \text{.}
\end{align}
% where $\beta$ is the weight of the supervised term $\L^{\LL}$, 
% and it is a hyperparameter of the model.

$\L^{\LL}$ and $\L^{\U}$ can be seen as the ELBOs associated to the 
observed and hidden labels, respectively.
% The decomposition~\eqref{eq:elbo_seq2} can be seen as a generalization to the 
% sequential case of a XXXX


% \begin{remark}
%     $\L^{\LL}$ and $\L^{\U}$ can be seen as the ELBOs associated to the 
%     observed and hidden labels, respectively.
%     % In fact, the proposed ELBO  $\Qsup(\theta,\phi)$~\eqref{eq:elbo_sup_pmc}
%     % in Chapter~\ref{chap:pmc}, for supervised classification 
%     % (all labels are observed) with the PMC model,
%     % can be seen as a particular case of $\L^{\LL}$.
%     % Indeed, if we assume that all labels are observed, 
%     % then $\U=\emptyset$ and $\L^{\LL}=\Qsup$,
%     % with specific choices of the variational distribution
%     % and the generative model.
% \end{remark}

\subsection{Learning semi-supervised TMCs}
Now, it remains to compute the ELBO 
$\Qsemi(\theta,\phi)$~\eqref{eq:elbo_seq2}, ,
which is not tractable in general.
Moreover, we also deal with both discrete and continuous latent variables. 
We now present how to easily approximate the ELBO
in this case.
%  of discrete and continuous latent variables.


\paragraph*{Continuous latent variables: } 
The ELBO $\Qsemi(\theta,\phi)$
involves computation of the expectation according to 
the variational distribution,
$\q(\latent_{t}|\;\cdot \;)$,
which is often intractable.
We thus propose to use a Monte-Carlo approximation based on the reparameterization 
trick for continuous latent variables (see Section~\ref{subsec:optimization_vae}) 
similar to the one used in the PMC model.
This technique allows us to sample from the variational distribution
% These techniques allow us to sample from the variational distributions    
$\q(\latent_{t}|\; \cdot \;)$.
% $\q(lab_t|\; \cdot \;)$, respectively.
By sampling in this way, we obtain $M$ differentiable samples 
$\latent_{0:T}^{(m)}$. 
% given by
% \begin{align}
%     \label{eq:reparametrization_vtmc}
%     \latent_{0:T}^{(m)} =& g(\phi ,\epsilon^{(m)}) \text{, for all } m = 1, \dots, M \text{,}
% \end{align}
% where $\epsilon^{(m)}$ is a random variable sampled from a distribution 
% which does not depend on $\phi$, and where $g$ is a differentiable function of $\phi$.
The $M$ samples $\latent_{0:T}^{(m)}$ are used to approximate the expectations. After this, 
an optimization algorithm can be used to estimate the parameters
% \item Handling of the continuous latent variables: We use the reparameterization trick
(See example~\ref{ex:gaussian_case}).\\

\paragraph*{Discrete latent variables: } 
% Discrete variables pose several challenges 
% in optimization and learning as 
% they are fundamentally non-differentiable. 
% % In most optimization algorithms
% % used in practice, the gradients are computed and used to update the parameters.
% Thus, it is not trivial to perform optimization with respect to the parameters of 
% the categorical distributions $\p(\lab_t|\; \cdot \;)$ and
%  $\q(\lab_t|\; \cdot \;)$.
A static semi-supervised model with discrete latent variables 
has been proposed in~\citep{kingma2014semi} and solves this problem by
marginalizing out $\lab$ over all the labels. 
However, this approach is not tractable
when numerous labels are involved.
In Chapter~\ref{chap:main_concepts},
we have presented  the use of discrete variables 
in a VI framework (see Subsection~\ref{subsec:vbi})
The  Straight-Through Gumbel-Softmax estimator provides a way to relax discrete variables, 
making them differentiable and amenable to gradient-based optimization. 
In addition, the expectation with respect to the variational distribution
$\q(\lab_t|\; \cdot \;)$ is evaluated with a single relaxed 
sample~\citep{andriyash2018improved,jang2016categorical}.\\


In summary, this approach combines the (classical)
reparameterization trick for continuous latent variables
and the G-S trick for discrete latent variables, in order
to obtain differentiable samples from the variational distributions
$\q(\latent_{t}|\; \cdot \;)$ and $\q(\lab_{t}|\; \cdot \;)$, respectively.
These samples are used to approximate the ELBO \eqref{eq:elbo_seq2}, 
making it computationally feasible for optimization. 
In addition, the $\dkl$ terms in \eqref{eq:elbo_seq_obs} and \eqref{eq:elbo_seq_hidden}
can be computed analytically since the variational distribution is assumed to be tractable.
% ~\cite{maddison2016concrete, jang2016categorical} 
% By using this Monte Carlo approximation, we can efficiently perform variational 
% inference %in models with both continuous and discrete latent variables, 
% facilitating the training and inference processes.
Algorithm~\ref{algo:algo_train_tmc_semi} summarizes the proposed approach, 
where we represent the hidden labels as a 
stochastic vector, and the observed labels as a one-hot vector.
In the case of S-T Gumbel-Softmax, in the forward pass,
 line~\ref{line:sample_gumbel_softmax}
is followed by an argmax operation to discretize the samples
 (see Remark~\ref{rem:gumbel_softmax}).


\begin{algorithm}[htbp!]
    \caption{General parameter estimation for TMCs in semi-supervised classification context}
    \label{algo:algo_train_tmc_semi}
  \begin{algorithmic}[1]
    \Require{$(\obs_{0:T}, \labl )$, the data where 
    $\lab_t$ is one-hot encoded, for all $t \in \LL$; $\varrho$, the learning rate; 
    $M$ the number of samples, 
    $\tau$ the temperature parameter }
    \Ensure{$(\theta^*, \phi^*)$, sets of estimated parameters}
    \State Initialize the parameters $\theta^0$ and $\phi^0$
    \State $j\leftarrow 0$\label{line:start_vtmc_semi}
    \While{\text{convergence is not attained}}
      \State Sample $\latent_0^{(m)}\sim q_{\phi^{{j}}}(\latent_0|\; \cdot \;)$,  
      for all  $1 \leq m \leq M$.
      \State Sample $\latent_t^{(m)}\sim q_{\phi^{{j}}}(\latent_t|\latent_{0:\cdot}^{(m)},\; \dots)$,   for all  $1 \leq m \leq M$, for all $1 \leq t \leq T$. 
    %   \Statex{\textbf{Forward pass:}}
    %   \State Sample $\lab_t^{G-M} \sim q_{\phi^{{j}}}(\lab_t|\latent_{0:\cdot}^{(m)}, \; \dots)$, 
    %   using the Gumbel-Max trick, for all $t \in \U$. 
    % \Statex{\textbf{Backward pass:}}  
    \State Sample $\lab_t^{G-S} \sim q_{\phi^{{j}}}(\lab_t|\lab_{0:\cdot}^{G-M}, \; \dots)$,
        using the Gumbel-Softmax trick, for all $t \in \U$, 
        with temperature $\tau$. \label{line:sample_gumbel_softmax}
    \State Evaluate the (approximated) loss $\widehat{\Qsemi}(\theta^{{j}},\phi^{{j}})$
        with the samples $\latent_{0:T}^{(m)}$ and ${\lab_t}^{G-S}$, for all $t \in \U$.
        % from \eqref{elbo_seq2}-\eqref{elbo_seq_hidden}. \label{line:evaluate_loss_semi}
    \State{Compute the derivative of the loss function
      $\nabla_{(\theta, \phi)} \widehat{\Qsemi}(\theta,\phi)$ with the 
      samples $\latent_{0:T}^{(m)}$ and ${\lab_t}^{G-S}$, for all $t \in \U$.
    %   from \eqref{eq:eq:elbo_seq2}-\eqref{eq:eq:elbo_seq_hidden}.
    }\label{line:derivate_tmc_semi} 
      \State Update the parameters with gradient ascent
    \begin{equation}
    \begin{pmatrix}\theta^{(j+1)}\\\phi^{(j+1)}\end{pmatrix}=
    \begin{pmatrix}\theta^{{j}}\\\phi^{{j}}\end{pmatrix}
    + \varrho {\nabla_{(\theta, \phi)} \widehat{\Qsemi}(\theta,\phi)}\Big|_{(\theta^{{j}},\phi^{{j}})}
    \label{eq:elbo_grad_vtmc}
    \end{equation}
    \State  $j\leftarrow j+1$
    \EndWhile
    \State  $\theta^{*} \leftarrow \theta^{{j}}$
    \State  $\phi^{*} \leftarrow \phi^{{j}}$
    \label{line:end_dtmc_tmc_semi}
  \end{algorithmic}
    % \vspace*{0.2cm}
\end{algorithm}

% \begin{remark}
%     In Line~\ref{eq:elbo_grad_vtmc}, we use the Adam optimizer~\citep{kingma2014adam}
%     as in Algorithm~\ref{algo:algo_train_dpmc_gen}..
% \end{remark}

% which is nothing more than expectation 
% according to $\q(\latent_{0:T}, \labu|\; \obs_{0:T}, \labl)$.
% We thus propose to use a Monte-Carlo approximation based on the reparameterization trick in order
% to obtain a differentiable approximation $\hat{Q}(\theta,\phi)$ of $Q(\theta,\phi)$. 

%  we use the classical reparameterization trick to sample sequentially 
% according to the continuous distribution $ q(\latent_t|\latent_{0:t-1},\lab_{0:t},\obs_{0:T},\lab_{t+1:T}^{\LL})$ (or 
%  $q(\latent_t|\latent_{0:t-1},\lab_{0:t},\obs_{0:T},\lab_{t+1:T}^{\LL})$) (see  Example~\ref{example:gauss_variational}),
% while we use the Gumbel-Softmax (G-S) 
% trick~\citep{maddison2016concrete, jang2016categorical} to sample according to $q(\lab_t|$ $\lab_{0:t-1},\latent_{0:t},\obs_{0:T},\lab_{t+1:T}^{\LL})$ (or
% $q(\lab_t|$ $\lab_{0:t-1},\latent_{0:t-1},\obs_{0:T},\lab_{t+1:T}^{\LL})$) since
% the labels are discrete.

%Since  only a subset of labels is observed $\labl$, the set of hidden 
%labels $\labu$ is treated as latent variables and variational inference 
%involves finding a lower bound on the marginal likelihood of the observed data $\obs_{0:T}$ and $\labl$.
%The variational lower bound is given by: 
% So the latent variables $\latent_{0:T}$ and $\labu$ are hidden variables that are not 
% directly measurable or observable, but are inferred from the observed data  $\obs_{0:T}$ and $\labl$. 
%\begin{align}
 %   \label{eq:elbo_semi_tmc}
 %   \log(\p(\obs_{0:T}, \labl) )   \geq  & -  \int \sum_{\labu}  \q(\latent_{0:T}, \labu|\obs_{0:T}, \labl) \times \nonumber \\ 
 %   & \quad \quad  \log \left(\frac{\q(\latent_{0:T},\labu  |\obs_{0:T})}{\p(\latent_{0:T},\lab_{0:T}, \obs_{0:T})}\right)  {\rm d} \latent_{0:T} \nonumber \\
 %   & =  Q(\theta,\phi) \text{,}  
%\end{align}
%where $\phi$ denotes the parameters of the variational distribution 
%$\q(\latent_{0:T},\labu|\obs_{0:T}, \labl)$.

%Since our model has both discrete and continuous latent variables, 
%the approximation of the ELBO in Eq.~\eqref{eq:elbo_semi_tmc} becomes more complex.
%To that end,  we can use the Gumbel-Softmax (G-S) 
%trick~\cite{maddison2016concrete, jang2016categorical} and the reparametrization
%trick~\cite{kingma2013auto} to approximate $Q(\theta,\phi)$ simultaneously.
%For the continuous latent variables $\latent_{0:T}$, the reparametrization 
%trick introduced in section \ref{subsec:varinf} is still valid.
%On the other hand, for the discrete latent variables $\labu$, 
%the G-S trick enables to obtain a differentiable approximation to the discrete
%categorical distribution. 

%It remains to choose a factorization of the variational distribution
%$\q(\latent_{0:T}, \labu| \obs_{0:T}, \labl)$, which is a crucial step in variational inference.
%Different models can be obtained by choosing different factorizations 
%of the variational distribution, which will be discussed 
%in the next section \ref{subsec:particular_cases}. For example, we can first consider  
%$\q(\latent_{0:T}, \labu| \obs_{0:T}, \labl) =  \q(\latent_{0:T}| \obs_{0:T}, \lab_{0:T} ) \q(\labu| \obs_{0:T}, \labl )$ 
%and then consider a mean-field variational distribution.



% The computational complexity of using ELBO with both discrete and continuous 
% latent variables depend on the specific model and the size of the data. 
% In general, the Gumbel-Softmax trick involves sampling from a Gumbel
% distribution and applying a softmax function, which tends to be 
% computationally expensive. However, recent advances in hardware and 
% software have made such calculations feasible and efficient.


% \subsection{Estimation of the hidden labels}
\paragraph*{Estimation of $\lab_{t}$, for all $t \in \U$: } 

% Once we have an estimate $\phi^*$
% of $\phi$ of the model with Algorithm~\ref{algo:algo_train_tmc_semi},
% we can classify the hidden labels $\labu$, for all $t \in \U$.
% This is done by using the variational distribution
% $q_{\phi^*}(\lab_t|\; \cdot \;)$. 
% Thus, we sample from the variational distribution
% $\q(\lab_t|\; \cdot \;)$, for all $t \in \U$,
% and obtain a complete sequence of  labels $\lab_{0:T}$.

Once we have an estimate \(\phi^*\) of \(\phi\) of the model with Algorithm~\ref{algo:algo_train_tmc_semi},
we can approximate the hidden labels \(y_{0:T}^{\mathcal{H}}\), for all \(t \in
\mathcal{H}\). This can be done by using either the variational approximation
\(q_{\phi^*}(y_t \mid \cdot)\) or an importance sampling approach with weighting.
In the variational approximation method, we sample from the variational
distribution \(q_{\phi}(y_t \mid \cdot)\), for all \(t \in \mathcal{H}\), and
obtain a complete sequence of labels \(\hat{y}_{0:T}\). Alternatively, using the
importance sampling approach, we would sample from the proposal distribution and
weight the samples to obtain an approximation of the hidden labels. This method
can provide a more accurate estimation, especially when the variational
approximation is not sufficiently close to the true posterior.


\section{Experiments}
\label{sec:simulation}
% \begin{example}
%     We present the proposed approach for the case of a TMC with Gaussian latent variables
%     and Bernoulli hidden variables,
%     %  which is a particular case with $C=2$,  
%     \ie
%     $\lab_t$ takes values in $\Omega = \{1,0\}$  
%     with probabilities $\pi_1 =\roqy$ $(\lab_t = 1)$
%      and $\pi_2 = 1- \roqy$ $(\lab_t = 0)$.
%     $\latent_t$ is a Gaussian random variable
%     with mean $\mulatent$ and variance $\siglatent$.\\
%     Thus, the variational distributions $\q(\latent_t|\; \cdot \;)$ 
%     and $\q(\lab_t|\; \cdot \;)$
%     are specified as follows:
%     \begin{align*}
%         \q(\latent_t|\; \cdot \;)= &
%         \q(\latent_{t}|\latent_{t-1},\obs_{t})\\
%         =& \mathcal{N}\left(\latent_t; \mulatent(\latent_{t-1},\obs_{t} ) , 
%         \diag(\siglatent(\latent_{t-1},\obs_{t})) \right) \text{,}\\
%         \q(\lab_t| \; \cdot \;)=& 
%         \q(\lab_{t}|\latent_{t-1},\obs_{t}) \text{, for }  t \in \U \\
%         =& \Ber(\lab_t; \roqy(\latent_{t-1},\obs_{t})) \text{,}
%     \end{align*}
%     where $\mulatent$, $\siglatent$ and $\roqy$ are differentiable 
%     functions w.r.t. $\phi$
%     and $\diag(\;\cdot \;)$ denotes  the diagonal matrix 
%     deduced from the values of $\siglat$.  

%     First, a sample $\latent_{t}^{(m)}$, for all $i \in [1:M]$ and $t \in [0:T]$,
%     can be reparameterized  by using the reparameterization trick, 
%     \begin{equation*}
%         \latent_{t}=\mulatent(\latent_{t-1}^{(m)},\obs_t)+
%         \diag(\siglatent(\latent_{t-1},\obs_t))^{\frac{1}{2}} 
%         \times  \epsilon_t^{(m)} \text{,} \quad \quad \epsilon_t^{(m)} \overset{\text{\iid}}{\sim} \mathcal{N}(0,I) \text{.}
%     \end{equation*}


%     Now, we consider the hidden labels $\lab_t$. 
%     For simplicity, we neglect the temperature $\tau$ in the following 
%     discussion; in practice, however, the temperature parameter can be useful.
%     Specifically, $\lab_t =1$ if 
%     $$\log \pi_1 + G_1 > \log \pi_2 + G_2 \text{,}$$
%     where $G_1$ and $G_2$ are i.i.d. samples from the Gumbel 
%     distribution $\text{Gumbel}(0,1)$, 
%     and made  the $\argmax$  explicit through the inequality, for  all $t \in \U$,
    
%     \begin{eqnarray}
%         \label{eq:example_semi}
%         \lab_t^{G-M}&=& \left\{
%         \begin{matrix}
%         \noindent 1  & \; \text{if } \log \roqy -\log(1-\roqy)  + G_1 - G_2>0 \text{, }\\ 
%         0 & \; \text{otherwise.}
%         \end{matrix} \right. %\text{.}
%     \end{eqnarray}
%     The Gumbel-Softmax trick allows us to obtain a differentiable approximation of the
%     $\argmax$ function, \ie
%     \begin{align*}
%         \lab_t^{G-S} &=  \sigma(\log \roqy -\log(1-\roqy)  + G_1 - G_2) 
%         \text{, for all } t \in \U \text{,}
%     \end{align*}
%     where $\sigma$ is the sigmoid function since $\lab_t$ takes values in $\{0,1\}$
%     and the sigmoid is a particular case of the softmax function.
%     % $\lab_t^{G-S}$ is a differentiable approximation of $\lab_t^{G-M}$. 
%     The Kulback-Leibler divergence terms in 
%     \eqref{eq:elbo_seq_obs} and \eqref{eq:elbo_seq_hidden}
%     can be computed analytically and the reconstruction term dependents on the generative model.
%     The corresponding Kullback-Leibler divergence terms with Bernoulli
%     % and Gaussian distributions
%     distributions can be calculated analytically, 
%     for all $t \in \U$, as follows:
%     \begin{align*}
%         \dkl(\q(\lab_t|\cdot)|| \p(\lab_t|\cdot)) &= 
%         \sum_{c\in \Omega} \log ((\roqy)^{c} (1-\roqy)^{(1-c)})
%         - \log ((\ropy)^{c} (1-\ropy)^{(1-c)}) \text{,} \\
%     \end{align*}
%     where $\ropy$ is the probability of $\lab_t$ according to $\p(\lab_t|\cdot)$, 
%     \ie~$\ropy = \p(\lab_t=1|\cdot)$.

%     % Then the  approximation of the  general ELBO 
%     % can be rewritten as
%     % \begin{align*}
%     %     \widehat{\Qsemi}(\theta,\phi)= & - \frac{1}{M} \sum_{i=1}^M   
%     %     \log \left( \frac{\q(\latent_0^{(m)}|\obs_0, \lab_0 )}{\p(\latent_0^{(m)},\obs_0, \lab_0)}\right)
%     %     - \frac{1}{M} \sum_{i=1}^M  \sum_{\substack{t\geq 1\\ t \in \U}} 
%     %     \log \left(\q(\lab_t|\latent_{t-1}^{(m)},\obs_t) \right)   \\
%     %     & - \frac{1}{M} \sum_{i=1}^M \sum_{t=1}^T 
%     %     \log \left(\frac{\q(\latent_t^{(m)}|\latent_{t-1}^{(m)},\obs_t)}
%     %     {\p(\latent_t^{(m)},\obs_t, \lab_t|\latent_{t-1}^{(m)},\obs_{t-1}, \lab_{t-1})} \right)
%     % \end{align*}
%     % which is next optimized w.r.t. $(\theta,\phi)$.
% \end{example}

In this section, we present the practical applications and 
effectiveness of the TMC  model in a semi-supervised learning framework. We
start by comparing our deep TMC models with existing probabilistic and deep
learning models to highlight their advantages in terms of flexibility.
Next, we detail binary data generation experiments that will be used
for model comparison. Finally, we discuss the implementation of the semi-supervised 
classification task and present the results obtained with each model variant. 


\subsection{DTMC vs  existing models}
\label{subsec:particular_cases}
We have presented a general framework for semi-supervised learning
with TMCs. It depends on the choice of
the generative model, which is described by the transition distribution
$\p(\triplet_t|\triplet_{t-1})$. It
has an impact on the performance of the model 
for a specific task  
(classification, prediction, detection, or generation). 
The choice of the variational distribution  
$\q(\latent_{0:T}, \labu|\obs_{0:T}, \labl)$ is also crucial
since it has an impact on the computational complexity of the model.
% is a crucial step in variational inference.
Different models can be obtained by choosing different factorizations
of the variational distribution. A general factorization is given by
\begin{align*}
        \q(\latent_t | \;\cdot\;) &=  \tau(\latent_t; \qz (\;\cdot \;)) \text{,}\\
        \q(\lab_t | \;\cdot\;) &=  
        \varsigma(\lab_t;  \qy(\;\cdot\;)) \text{, for } t\in \U\text{,}
\end{align*}
where $\varsigma(\lab_t; \cdot )$ (resp. $\tau(\latent_t; \cdot )$ )
is a probability distribution on $\Omega$
(resp. probability density function on $\mathbb{R}^{d_\latent}$).
$\qy$ and $\qz$ are assumed to be differentiable functions w.r.t. $\phi$ 
(remember that $(\cdot)$ 
denotes a non-specified dependence between the variables of the model).
In the Deep TMC model, the set of parameters $(\theta, \phi)$  
of the generating  and the variational distributions 
can be described by deep neural networks.\\

Now, we present two popular models in the literature, 
which have been proposed for semi-supervised classification tasks.
First, we present a variation
of the Variational Sequential Labeler~\citep{chen2019variational} model 
based on our general model;
and then we present the
Semi-supervised Variational Recurrent Neural Network model
proposed by~\cite{butepage2019predicting}.
Both models are considered as particular cases of the proposed framework
and will be used in the experimental section to compare 
the performance of the proposed model. 

% \subsection{Modified variational sequential labeler}
On one hand, the VSL is a semi-supervised learning model
for sequential data which has originally been proposed for the 
sequence labeling tasks in natural language processing, 
that is based on conditional VAEs~\citep{pagnoni2018conditional}.
% where at each time step $t$, the observation $\obs_t$ is generated
% according its associated context $u_t$, which consists of the observations
% other than $\obs_t$. 
% The lower bound of the log-likelihood at each time step $t$ is given by
% \begin{align*}
%     \log \p(\obs_t | u_t) &\leq 
%     \E_{\q(\latent_t|\obs_{0:T})} 
%     \left[ \log \p(\obs_t | \latent_t, u_t)\p(\latent_t|u_t) \p(\lab_t|\latent_t, u_t)   \right]
%     \text{, for all } t \in \LL \text{.}\\
%     \log \p(\obs_t | u_t) &\leq 
%     \E_{\q(\latent_t, \lab_t |\obs_{0:T})} 
%     \left[ \log \p(\obs_t | \latent_t, u_t)\p(\latent_t|u_t) \p(\lab_t|\latent_t, u_t)  \right]
%     \text{, for all } t \in \U  \text{.}
% \end{align*}
We propose a variation of this model by considering a modified version
where the context depends on the previous observation $\obs_{t-1}$ 
and the current latent variable $\latent_t$ 
(more details are given in the Appendix~\ref{chap:appendix3}).
We refer to it as the modified Variational Sequential Labeler (mVSL)
and the associated generative model is given by
\begin{align*}
% \label{eq:vsl}
\p(v_t|v_{t-1}) \overset{\rm mVSL}{=}  \p(\obs_t|\latent_t) 
\p(\lab_t|\latent_t) \p(\latent_t|\obs_{t-1}, \latent_{t-1})  
\text{.}
\end{align*}
While the associated variational distribution
satisfies factorization \eqref{eq:fact-1}
with
\begin{align}
    \q(\latent_t|\latent_{t-1},\lab_{t-1},\obs_{0:T},\lab_{t+1:T}^{\LL})
    &=\q(\latent_t|\obs_{0:T}) \text{,} \\
    \q(\lab_t|\lab_{t-1},\latent_t,\obs_{0:T},\lab_{t+1:T}^{\LL})
    &=\p(\lab_t|\latent_t) \text{, for all } t \in \U \text{.}
\end{align}   
In this case, the ELBO \eqref{eq:elbo_seq2} reduces to
\begin{align*}
    % \label{eq:elbo_vsl}
    \Qsemi(\theta,\phi) \overset{\rm mVSL}{=}& 
    \sum_{t \in \LL} \E_{\q(\latent_{t}| \obs_{0:T})} \left(
     \log\p(\lab_{t}|\latent_{t}) \right) + \nonumber \\
    % & \E_{\q(\latent_0|\obs_{0:T})} \log \p(\obs_0|\latent_0) -
    % \dkl(\q(\latent_0|\obs_{0:T})||\p(\latent_0)) + \nonumber \\
    & \! \sum_{t=0}^T\!\!
    \Bigg[ \E_{\q(\latent_{t}| \obs_{0:T})} \log \p(\obs_t|\latent_t) \nonumber \\
    & -    \dkl(\q(\latent_t|\obs_{0:T})|| \p(\latent_t|\obs_{t-1}, \latent_{t-1} ))  \Bigg] 
    \text{,}
\end{align*}
where $\obs_{-1} =\latent_{-1} = \emptyset$.
% \begin{align}
% \label{eq:elbo_vsl}
% Q(\theta,\phi) \overset{\rm mVSL}{=}& 
% \sum_{t \in \LL} \int \q(\latent_{t}| \obs_{0:T}) \log\p(\lab_{t}|\latent_{t}) {\rm d} \latent_t + \nonumber \\
% & \q(\latent_0|\obs_{0:T}) \log \p(\obs_0|\latent_0) + \nonumber \\ 
% \sum_{t=0}^T \int \q(\latent_{t}| \obs_{0:T}) \Bigg[ \log \p(\obs_t|\latent_t) -\nonumber\\ 
% & 
% \log \left(\frac{\q(\latent_t|\obs_{0:T})}{p(\latent_t|\obs_{t-1}, \latent_{t-1} )} \right)  \Bigg]  {\rm d} \latent_t 
% \text{.}
% \end{align}
% Additionally, it can be observed that it consists of two terms and 
% that the previous assumptions enable us to interpret it as an expectation 
% according to $\q(\latent_{0:T}|\obs_{0:T})$. 
% Thus, it is not necessary to sample discrete variables according to 
% the G-S trick. Moreover, a regularization term $\beta$ can be introduced 
% in the second part of the ELBO in 
% order to encourage good performance on labeled data 
% while leveraging the context of the noisy observations during reconstruction.
% While this model simplifies the inference, 
% it should be noted that in the generative process, 
% the observation $\obs_t$ is conditionally independent of its associated label and may not
% be adapted to some applications.

%Moreover, the VSL simplifies the ELBO by setting $\q(\lab_t| \latent_t) = \p(\lab_t| \latent_t)$, which enables the use of classical variational inference with only continuous latent variables. To further improve performance, they also introduce a regularization term into the loss function that encourages good performance on labeled data while leveraging the context of the noisy image during reconstruction. 

% \subsection{Semi-supervised variational RNN}
\label{sec:svrnn}
On the other hand, the generative model used in the SVRNN model
is a particular case of the TMC model where the latent variable
$\latent_t$ consists of the pair $\latent_t=(z'_t, h_t)$. The associated
transition distribution reads:
% \begin{align}
% \label{eq:svrnn}
%  \p(v_t|v_{t-1}) = \p(\lab_t|v_{t-1}) \p(\latent_t|\lab_t, v_{t-1}) \p(\obs_t|\lab_t,\latent_t,v_{t-1}) \text{,}
% \end{align}
\begin{align*}
% \label{eq:svrnn}
 \p(v_t|v_{t-1})  \overset{\rm \scriptscriptstyle SVRNN }{ = }\p(\lab_t|v_{t-1}) \p(\latent_t|\lab_t, v_{t-1}) \p(\obs_t|\lab_t,\latent_t,v_{t-1}) \text{,}
\end{align*}
where 
\begin{eqnarray*}
\p(\lab_t|v_{t-1})&=& \p(\lab_t|h_{t-1})\text{,} \\
\p(\latent_t|\lab_t,v_{t-1})&=&\delta_{f_{\theta}(z'_t,\lab_t,\obs_t,h_{t-1})}(h_t) \times \p(z'_t|\lab_t, h_{t-1}) \text{,} \\
\p(\obs_t|\lab_t,\latent_t,v_{t-1})&=& \p(\obs_t|\lab_t, z'_t, h_{t-1}) \text{,}
\end{eqnarray*}
and where $f_{\theta}$ is a deterministic, 
\ie~the variable $\latent'_t$ is a stochastic latent variable 
and $h_t$ is deterministically given by 
$h_t = f_{\theta}( z'_t, \obs_t, \lab_t, h_{t-1})$, 
where $f_{\theta}$ is a function parameterized 
by a RNN, for example. 
The variational distribution $\q(\latent_{0:T}, \labu|$ $ \obs_{0:T}, \labl)$ satisfies
the factorization \eqref{eq:fact-2}
with
\begin{align*}
 q(z'_t|\latent_{t-1},\lab_{t},\obs_{0:T},\lab_{t+1:T}^{\LL})=  \q(z'_t| \obs_t, \lab_t, h_{t-1})\text{,} \\
q(\lab_t|\lab_{t-1},\latent'_{t-1},\obs_{0:T},\lab_{t+1:T}^{\LL})= 
\q(\lab_t| \obs_t, h_{t-1}) \text{.}
\end{align*}
The ELBO of the SVRNN model is given by
\begin{align*}
    % \label{eq:elbo_svrnn}
    \Qsemi(\theta,\phi) \overset{\rm \scriptscriptstyle SVRNN }{=}& \quad
    \L^{\LL}(\theta,\phi) + \L^{\U}(\theta,\phi) + J^{\LL}(\theta,\phi) \text{,}
\end{align*}
where 
\begin{align}
    \L^{\LL}(\theta,\varphi) = \sum_{t\in \LL}
     & \E_{\q(\latent'_t| \obs_t, \lab_t, h_{t-1})} 
       \log \p(\obs_t|\lab_t, \latent'_t, h_{t-1}) 
       + \log(\p(\lab_t | h_{t-1} ))    \nonumber \\ 
    &  - \dkl (\q(\latent'_t|\obs_t, \lab_t, h_{t-1})||
    p(\latent'_t|\lab_t, h_{t-1} ))  \text{,} \\
    \L^{\U}(\theta,\varphi) =&  \sum_{t\in \U}
     \E_{\q(\latent'_t, \lab_t| \obs_t, h_{t-1})} 
       \log \p(\obs_t|\lab_t, \latent'_t, h_{t-1})     \nonumber \\ 
    & - \dkl (\q(\latent'_t|\obs_t, \lab_t, h_{t-1})) \nonumber  \\
    & - \dkl (\q(\lab_t| \obs_t, h_{t-1})|| \p(\lab_t|h_{t-1} )) \text{,}\\
    J^{\LL}(\theta,\phi) = & \sum_{t\in \LL} 
    \E_{\tilde{p}(\lab_t, \obs_t)}
    \log(\p(\lab_t | h_{t-1} ) \q(\lab_t| \obs_t, h_{t-1})) \text{,}
\end{align}
where $\tilde{p}(\lab_t, \obs_t)$, for $t\in \LL$, 
denotes the empirical distribution of the data.
Their final ELBO does not coincide with \eqref{eq:elbo_seq2}. 
The reason why is that they derive it 
from the static case~\citep{jang2016categorical} and add 
a penalization term $J^{\LL}(\theta,\phi)$  that encourages 
$\p(\lab_t|h_{t-1})$ and $\q(\lab_t| \obs_t, h_{t-1})$ 
to be close to the empirical distribution of the data.
Since $\Lat_t$ is deterministic given $(z'_t, \obs_t, \lab_t, h_{t-1})$,
its posterior distribution becomes trivial, and thus 
there is no need to consider a variational distribution for it.


\subsection{Binary data generation}
\label{subsec:data_generation}
We used the Binary Shape Database
\footnote{\url{http://vision.lems.brown.edu/content/available-software-and-databases}}.
% ~\citep{binaryimg} 
and focused 
on both \textit{cattle}-type and \textit{camel}-type images. 
To transform these images into a $1$-D signal ( $\obs_{0:T}$ ), 
we used a Hilbert-Peano filling curve~\citep{sagan2012space}. 
To evaluate the models presented in Section~\ref{subsec:particular_cases},
we introduced non-linear blurring to highlight their ability to learn and
correct for signal corruption. 
% Specifically, we applied general stationary noise (resp. stationary multiplicative noise)  
% to the \textit{cattle}-type (resp. \textit{camel}-type) image. 
More precisely, we generated an artificial noise for the \textit{cattle}-type by
generating $\obs_t$ according to
\begin{equation}
    \label{eq:noise_eq1}
    \obs_t| \lab_{t},\obs_{t-1} \sim \mathcal{N}\Big(\sin(a_{\lab_t}+\obs_{t-1});
    \sigma^2\Big),
\end{equation}
where $a_{\omega_1}=0$ , $a_{\omega_2} = 0.4$ and $\sigma^2=0.25$. 
% The generated image is shown in Figure\ref{fig:res_cow40}(a).
We now consider the \textit{camel}-type image which is corrupted 
with a stationary multiplicative noise (non-elementary noise) given by
\begin{equation}
\label{eq:noise_eq2}
    \obs_t|\lab_t,\latent_t \sim\mathcal{N}\left(a_{\lab_t};b_{\lab_t}^2\right) * \latent_t,
\end{equation}
where $\latent_t\sim\mathcal{N}(0, 1)$, $a_{\omega_1}=0\text{, } a_{\omega_2} = 0.5$ and $b_{\omega_1}=b_{\omega_2}=0.2$. 

The generated images are presented in Figure~\ref{fig:res_cow40}(a) 
and Figure~\ref{fig:res_camel60}(a), respectively. 
% More details about the image generation process are
% available in~\cite{gangloff2023deep}. 
Additionally, we randomly selected
pixels $\lab_t \in \labl$, with a percentage of the pixels being labeled, and the rest considered unobserved or hidden (\textit{e.g.} Figure~\ref{fig:res_cow40}(c) 
and Figure~\ref{fig:res_camel60}(c)). %(\textit{e.g.} $40\%$ or $60\%$).




\subsection{Semi-supervised binary image segmentation}
% In this section, we present the
% semi-supervised binary image segmentation  problem. 
Our goal is to recover the 
segmentation of a binary image $(\Omega=\{\omega_1,\omega_2\})$
from the noisy observations
$\obs_{0:T}$ when a partial segmentation $\labl$ is available.
In particular, $\vartheta(\lab_t; \cdot )$ (resp. $\varsigma(\lab_t;\cdot)$) 
is set  as a Bernoulli distribution with parameters $\ropy$ (resp. $\roqy$). 
As for the distribution  $\zeta(\obs_t; \cdot )$ 
(resp. $\eta(\latent_t;\cdot)$ and  $\tau(\latent_t;\cdot)$), 
we set it as a Gaussian distribution with parameters 
$[ \muobs , \diag(\sigobs) ]$ (resp.  $[ \mulatentp , \diag(\siglatentp)]$
 and $[ \mulatent , \diag(\siglatent) ]$),
% \begin{align*}
%     \p(v_t|v_{t-1}) &=  \p(\obs_t|\cdot) \p(\latent_t|\cdot)\p(\lab_t|\cdot) \nonumber\\
%     \p(\obs_t | \cdot) &= \N(\obs_t; \mu_{px, t}, {\rm diag}(\sigma_{px,t}) ) \\
%     \p(\latent_t | \cdot) &= \N(\obs_t; \mu_{pz, t}, {\rm diag}(\sigma_{pz,t}) ) \\
%     \p(\lab_t | \cdot) &= \Ber(\lab_t;  \rho_{py,t}) \text{,}
% \end{align*}
where ${\rm diag(.)}$ denotes the diagonal matrix deduced from the values 
of $\sigma_{\cdot,t}$.
% and the parameters are $\theta = \{\mu_{pz,t}, \sigma_{pz,t}, \mu_{px,t}, \sigma_{px,t}, \rho_{py,t}\}$  
% and $\phi = \{ \mu_{qz,t}, \sigma_{qz,t} , \rho_{qy,t}\}$.

In our simulations, we consider three particular cases of this deep TMC model
which read as follows:
\begin{align}
    \label{eq:TMC_I}
    \p(\triplet_t|\triplet_{t-1}) \overset{\rm TMC-I}{=}  \quad &
    \p(\lab_t|\lab_{t-1}) 
    \p(\latent_t|\latent_{t-1}) 
    \p(\obs_t|\lab_t,\latent_t) \text{,}\\
    \label{eq:TMC_II}
    \p(\triplet_t|\triplet_{t-1}) \overset{\rm  TMC-II}{=}  \quad &
    \p(\lab_t|\lab_{t-1}, \obs_{t-1}) 
    \p(\latent_t|\latent_{t-1}) 
    \p(\obs_t|\lab_t,\latent_t)\text{,} \\
    \label{eq:TMC_III}
    \p(\triplet_t|\triplet_{t-1}) \overset{\rm TMC-III}{=}  \quad&
    \p(\lab_t|\lab_{t-1}, \obs_{t-1}) 
    \p(\latent_t|\latent_{t-1}) 
    \p(\obs_t|\lab_t,\latent_t, \obs_{t-1})
    \text{.}
\end{align} 
The TMC-I~\eqref{eq:TMC_I} model assumes a Markovian distribution 
for the labels and the latent variables aim at 
learning the distribution of the noise given the label and 
the latent variable.
In the TMC-II~\eqref{eq:TMC_II}, and TMC-III ~\eqref{eq:TMC_III}
models the Markovianity assumption
for the labels is relaxed. The TMC-III model  also
considers the previous observation $\obs_{t-1}$
as an additional input to the distribution of the observation $\obs_t$.

In order to capture temporal dependencies 
in the input data and to have an efficient computation of the 
variational distribution for the \gls*{dtmc} models, 
we use a deterministic function to generate $\tilde{h}_t$ which  
takes as input $(\obs_t, \lab_t, \latent_t, \tilde{h}_{t-1})$. 
After this, the variational distribution $\q(\latent_{0:T}, \labu|$ $ \obs_{0:T}, \labl)$ satisfies the factorization \eqref{eq:fact-2}
with $\q(\latent_t|\obs_t, \lab_t,\tilde{h}_{t-1} )$ and $\q(\lab_t|\obs_t, \tilde{h}_{t-1})$. 
In the TMC-I case, the parameters are given by:
\begin{align*}
    &[ \muobs , \sigobs ]  = \px(\lab_t, \latent_t),\\
    &[ \mulatentp , \siglatentp]  = \pz(\latent_{t-1}), \\
    &\ropy  = \py(\lab_{t-1}),\\
     &[\mulatent , \siglatent] = \qz( \obs_t, \lab_t,\tilde{h}_{t-1})\text{, }\\
    &\roqy  = \qy( \obs_t, \tilde{h}_{t-1}) \text{.}
\end{align*} 
In the TMC-II and TMC-III cases, the parameters are given  in the same way,
except that $\px$ and $\qy$ take $\obs_{t-1}$ as an additional input.



\subsection{Results}
\label{subsec:results}
Each model was trained using stochastic gradient descent to 
optimize the negative associated ELBO, with the Adam optimizer~\cite{kingma2014adam}. 
The neural networks $\psi_{(\cdot)}^{(\cdot)}$ were designed with two hidden 
layers using rectified linear units and appropriate outputs, such as linear, softplus, 
and sigmoid. To ensure a fair comparison, we matched the total number 
of parameters of all models to be approximately equal. 
As a result, the number of hidden units for each hidden layer differs 
for each model. In fact, the SVRNN, TMC-I, TMC-II, TMC-III, 
and VLS models have 14, 25, 25, 24, and 30 hidden units, respectively.
We used an RNN cell to generate $\tilde{h}_t$ (resp. $h_t$) 
for the \gls*{dtmc} (resp. SVRNN) models. 
In the VLS model, we used the parameterization approach for 
$\q(\latent_t|\obs_{0:T})$ presented in~\cite{chen2019variational}, 
which involves using an RNN cell and with a regularization term equal to $0,1$.
We also added a penalization term used in the SVRNN to the ELBO of the 
TMC-I, TMC-II, and TMC-III models that was presented in Section~\ref{sec:svrnn}.



The performance of the models is evaluated in terms of
the error rate (ER) of the reconstruction of the unobserved pixels, 
which are estimated by using the variational approximation approach.
Table~\ref{tab:error_rates} presents the average of the error rates 
obtained for reconstructing unobserved pixels on all the 
\textit{name}-type images. 
The notation \textit{name $\%$} is used to indicate the specific 
image set and the percentage of unobserved labels in the image. 
As shown in the table, the deep TMC models consistently outperform 
the VSL and the SVRNN, achieving a lower average error rate for each image set. 

% \input{Figures/table_semi}
\vspace{-0.1cm}

\begin{table}[!htpb]
    \begin{center}
    % \small
    \begin{tabular}{|l|r|r|r|r|}
    \hline
    \multirow{2}{*}{Model}  &\multicolumn{3}{c|}{Data sets and \% of unlabeled pixels}\\ 
    \cline{2-4} 
      & \multicolumn{1}{c|}{Cattle 40\%} & \multicolumn{1}{c|}{Cattle 60\%} & \multicolumn{1}{c|}{Cammel 60\%} \\ 
      \hline \hline
      \multicolumn{1}{|l|}{VSL}      & 20,59\% & 22,38\% & 18,82\% \\ \hline
      \multicolumn{1}{|l|}{SVRNN}    & 14,92\% & 20,12\% & 16,80\% \\ \hline
      \multicolumn{1}{|l|}{\gls*{dtmc}-I}   & 3,50\%  & 6,44\%  & 4,50\%  \\ \hline
      \multicolumn{1}{|l|}{\gls*{dtmc}-II} & \textbf{2,95\%}                  & \textbf{5,53\%}                  & \textbf{4,25\%}                 \\ \hline
      \multicolumn{1}{|l|}{\gls*{dtmc}-III} & 3,21\%  & 6,09\%  & 4,59\%  \\ \hline
      \end{tabular}
      \vspace{-0.2cm}
      \caption{Average error rates of the
       reconstruction of the unobserved pixels on different 
       sets of images with different percentages of unobserved pixels.}      
    \label{tab:error_rates}
    \end{center}
\end{table}

Moreover, our algorithm achieves superior performance for both noises.
Figure~\ref{fig:res_cow40} (resp. Figure~\ref{fig:res_camel60}) 
displays the performance of our proposed algorithms compared to the
 VSL and the SVRNN on a \textit{cattle}-type (resp. \textit{camel}-type) 
 image with $60\%$ (resp.  $60\%$) of unobserved labels.
In particular, we observe that in the VSL model, 
the error is mainly due to the misclassification of the black pixels 
(Figure~\ref{fig:res_cow40}(d) 
and Figure~\ref{fig:res_camel60}(d)). 
While for the SVRNN, the error results from the misclassification 
of the two classes (Figure~\ref{fig:res_cow40}(e) 
and Figure~\ref{fig:res_camel60}(e)).
% Additionally, we observe that when dealing with elementary noise, 
% the performance of the VLS model is superior to that of SVRNN. 
% However, this capability is lost as we increase the percentage 
% of unobserved labels, even with elementary noise.

\input{Figures/cow}
\input{Figures/cattle}

 
% \subsection{Discussion}
% \label{subsec:discussion}

\newpage
\section{Conclusions}
\label{sec:conclusion}
% In this section, we proposed a  general semi-supervised generative
% latent variable model.
% In particular, by considering the TMC model, we have shown that it is possible
% to obtain a wide variety of generative models and
% to estimate them in the common framework of
% variational inference in the case where only
% a part of the observations are labelled.
% Our general model learns to represent discrete labels, continuous features observations 
% over time  able to classify, predict labels,
% and also generate new sequences of features.
% The experiments demonstrate the effectiveness of the proposed 
% approach in achieving state-of-the-art performance on 
% the task of binary image segmentation.

In this chapter, we presented a semi-supervised latent variable generative model.
By exploring the TMC model, we have illustrated the
feasibility of creating a diverse set of generative models based
on the VI. This approach is particularly advantageous when dealing
with data sets in which only a subset of the observations are labeled. The model
we propose is capable of learning and representing a wide range of data features.
It can effectively handle discrete labels and continuous feature observations
over time, providing capabilities to classify, predict labels, and generate new
feature sequences. This versatility makes the model particularly suitable for
complex temporal data scenarios. The results of our experiments support the
effectiveness of our approach in achieving good performance in the
task of binary image segmentation.

\thispagestyle{empty}

% !TEX root = latex_avec_réduction_pour_impression_recto_verso_et_rognage_minimum.tex
\chapter{Deep Markov models for unsupervised classification}
\markboth{CHAPTER 4. DEEP MCs FOR UNSUPERVISED CLASSIFICATION}{Short right heading}

\label{chap:unsp_pmc_tmc}

\localtableofcontents
\pagebreak

\section{Introduction}
% \yohan{Here you have to explain that the challenges is that we do not have any y
% in the database. So one could apply the previous methodology, but it does not
% guarantee the interpretability of the estimated labels.}



In the previous Chapters~\ref{chap:pmc}
and~\ref{chap:semi_supervised_pmc_tmc}, 
we have introduced the PMC and TMC models as frameworks for generative models,
supervised and semi-supervised classification.
In this chapter, we consider the problem of unsupervised classification
where only the sequence of observations $\obs_{0:T}$ is observed,
and that we want to estimate the sequence of hidden labels $\lab_{0:T}$.
We recall that the estimation of $\lab_t$ from $\obs_{0:T}$, for all $t$, $0\leq t \leq T$, relies 
on the  unknown posterior distribution $p(\lab_t|\obs_{0:T})$,
% ~\eqref{eq:post_distrib}. 
\begin{equation*}
  % \label{eq:post_distrib}
  \p(\lab_t|\obs_{0:T})=
  \frac{\sum_{\lab_{0:t-1},\lab_{t+1:T}} \p(\obs_{0:T}, \lab_{0:T}) } 
  {\sum_{\lab_{0:T}} \p(\obs_{0:T}, \lab_{0:T})} \text{,}
\end{equation*} 
which can be derived from the distribution $\p(\lab_{0:T},\obs_{0:T})$
or $\p(\latent_{0:T},\lab_{0:T},\obs_{0:T})$ since 
$\p(\lab_{0:T},\obs_{0:T})=\int \p(\latent_{0:T},\lab_{0:T},\obs_{0:T}) 
\mathrm{d}\latent_{0:T}$.
% The distribution $\p(\obs_{0:T}, \lab_{0:T})$ can be also seen as a marginal distribution of 
% a distribution in augmented dimension $\p(\latent_{0:T},\lab_{0:T},\obs_{0:T})$.
Thus, we continue to consider the PMC and TMC models, where their 
associated conditional distributions can be parameterized by 
universal approximators (DNNs) under the constraint
that $\lab_{0:T}$ is an interpretable hidden process. 
As we will see, this particular constraint requires us to review previous
techniques to include the learning of an interpretable label.


% In terms of modeling, a direct consequence is 
% that~\eqref{eq:pmc_intro_uns}-\eqref{eq:tmc_intro}
%  do not require 
% any additional assumption on the involved distributions. 

% in the spirit of the Variational Auto-Encoders (VAEs) \citep{kingma2013auto}.
% However, while VAEs and their extensions
% \citep{chung2015recurrent,gregor2015draw}
% aim at building powerful generative
% models (\ie~an expressive 
% probability distribution $\p(\obs_{0:T})$ on the observations), 
% Next, a main advantage of embedding
% the DNN framework into a probabilistic framework
% is that it is possible to derive 
% unsupervised Bayesian estimation algorithms to jointly
% estimate $\theta$ and $\lab_t$, for all $t$.
% The counterpart of this generalization is that
% the resulting models can be highly parameterized
% in such a way that the final estimated models can suffer 
% from a lack of interpretability as compared to the simple 
% HMC~\eqref{eq:hmc_intro}.
% Thus,  starting from a simple but interpretable
% model \textcolor{black}{(\ie~a model where the hidden process of interest 
% is interpretable as defined earlier)}, we include this constraint 
% in our parameterized models and their associated Bayesian inference algorithms.
% Our models are based on the declination of
% the general TMCs \eqref{eq:tmc_intro} in three versions
% and aim at modeling different kinds of problems:





This chapter is organized in three parts.
First, we give up the latent variable $\latent_t$ and
consider a PMC model~\eqref{eq:pmc_intro_uns} 
without any latent variable.
We directly parameterize the joint distribution
$\p(\lab_{0:T},\obs_{0:T})$ of a PMC.
We continue considering a DNN parameterization and 
an ad hoc procedure based on a pretraining of DNNs which aims at
transforming a simple and interpretable model such as~\eqref{eq:hmc_intro}
into a complex probabilistic architecture while keeping this interpretability constraint.
We show that it is possible to adapt existing  Bayesian inference algorithms
to our models and the VI framework is not necessary in the PMC case.


Next, we reintroduce the continuous latent variables $\latent_{0:T}$, and propose
a modified VI framework to estimate the parameters of the model
which takes into account the interpretability of $\lab_{0:T}$ and also 
the different roles of $\lab_{0:T}$ and $\latent_{0:T}$. 
We also propose a Sequential Monte Carlo algorithm~\citep{doucet2009tutorial} 
based on the previous variational framework to obtain the final estimates of $\lab_t$.
% Finally,  we propose an alternative use of the latent process, where our
% objective is to characterize explicitly the relationship
% between the pair $(\lab_t,\obs_t)$ and the past observations $\obs_{t-1}$ when
% $\latent_{0:T}$ is deterministic given the observations. Thus,  a closed-form
% expression of $\p(\lab_t,\obs_t|\lab_{t-1},\obs_{t-1})$ is available contrary to
% the general TMC introduced before.
% A direct advantage of the resulting TMC model is that
% it can be interpreted as the combination of a PMC model \eqref{eq:pmc_intro_uns}
% with an RNN~\citep{rumelhart1986learning,
% mikolov2014learning}, and that the distributions of interest can be computed
% exactly, without any approximation.
% \begin{itemize}
% \item first, we consider a model
% in which we directly parameterize the joint distribution
% $\p(\lab_{0:T},\obs_{0:T})$ of a PMC.
% % (\ie~we consider a TMC \eqref{eq:tmc_intro}
% % without any latent process $\latent_{0:T}$). 
% We continue considering a DNN parameterization, we show that it is possible to adapt
% existing Bayesian inference algorithms,
% we propose and ad-hoc procedure based on a pretraining of DNNs which aims at
% transforming a simple and interpretable model such as \eqref{eq:hmc_intro} 
% into a complex probabilistic architecture while keeping this interpretability constraint;
% \item in our second version of TMCs,
% we reintroduce a continuous latent process
% $\latent_{0:T}$. The aim of this continuous process is to learn
% the nature of the distribution of $(\lab_{0:T},\obs_{0:T})$; 
% even if the distributions underlying $p(\latent_{0:T},\lab_{0:T},\obs_{0:T})$ 
% are simple distributions (\eg~  Gaussian distributions for
% the continuous r.v.), the implicit marginal one
% $\p(\lab_{0:T},\obs_{0:T}) = \int p(\latent_{0:T},\lab_{0:T},\obs_{0:T}) {\rm d}\latent_{0:T}$ can become
% complex and more relevant than a direct parameterization
% of $\p(\lab_{0:T},\obs_{0:T})$.
% %\hugo{}{on insère le mot clé \emph{implicit distribution} [YI
% %2018 Semi implicit VI?]}.
% However, due to the continuous nature of the latent
% process, the distributions of interest
% cannot be computed exactly \textcolor{black}{in general}; thus, we
% modify the variational Bayesian inference framework
% \citep{Jordan99anintroduction} in order to propose
% a parameter estimation algorithm which takes into
% account the interpretability constraint of $\lab_{0:T}$ but also the different roles of $\lab_{0:T}$ and $\latent_{0:T}$;
% we finally propose a Sequential Monte Carlo (SMC) algorithm \citep{doucet2009tutorial} based on the previous variational framework to obtain the final estimates of $\lab_t$;
% \item in our last version of the TMC model, 
% we propose an alternative use of the latent process $\latent_{0:T}$; here, our objective
% is to \textcolor{black}{ characterize explicitly 
% the relationship between the pair $(\lab_t,\obs_t)$ 
% and the past observations $\obs_{t-1}$: when
% $\latent_{0:T}$ is deterministic given the observations,
% a closed-form expression of $\p(\lab_t,\obs_t|\lab_{t-1},\obs_{t-1})$ is available 
% contrary to the general TMC introduced before.
% %introduce an explicit long dependency 
% %on the observations to model
% %the joint process $(\lab_{0:T},\obs_{0:T})$. To that end, the latent process $\latent_{0:T}$ becomes deterministic given the observations $\obs_{0:T}$. 
% A direct advantage of the resulting TMC
% model is that it can be interpreted as the combination of a PMC model \eqref{eq:pmc_intro_uns} 
% with a Recurrent Neural Network (RNN) \citep{rumelhart1986learning, mikolov2014learning} and that the distributions of interest can be computed exactly, without any approximation.}
% %while preserving the interpretability of $\lab_t$.}
% \end{itemize}
For each model, we perform simulations to evaluate to what extent our
generalized models lead to a better estimation of the hidden states $\lab_t$.
Most of the simulations on synthetic and real data are run in the context of
unsupervised image segmentation 
(as in Chapter~\ref{chap:semi_supervised_pmc_tmc}).
% We show that our deep
% parameterizations and the training procedure that we propose always improve the
% segmentation accuracy. The results then pave the way towards a new approach for
% unsupervised signal processing with general hidden Markov models.




% \yohan{Next paragraph. Chapter 3: Unsupervised Learning\\
%     This should be merged with the introduction of this chapter. 
%     At this point :\\
% * we have already discussed of general parameterization of PMC and TMC;\\
% * VI has been explained.\\
% So you just had to deal with the problem of this chapter and specificites 
% ie the interpretabilty of labels when we use deep parameterization :\\
% * so we start with deep PMC without latent variables z to see if we can generalize simple HMCs. 
% Inference is simple for that case (no need to VI) \\
% but the challenge is to keep the interpretability of a simple HMC for eg
% * we reintroduce the latent variable and VI}

% The use of such models has been proposed
% in past contributions. It has been shown that when the PMC model
% is stationary, it is possible to propose an unsupervised estimation 
% method to estimate jointly  $\theta$ and $\lab_t$ from $\obs_{0:T}$ provided
% that the distribution of the observation given the hidden 
% states is restricted to a set of classical distributions
% such as the Gaussian one~\citep{gorynin2018assessing}.
% The stationary assumption can be relaxed by considering the TMC model with a third discrete
% latent process~\citep{lanchantin2004unsupervised}; in this case, the new process models, 
% the non-stationarity of the pair $(\lab_{0:T},\obs_{0:T})$ and
% the complete triplet model can also be estimated through an unsupervised
% procedure \citep{lanchantin2004unsupervised,gorynin2018assessing}.
% Finally, it is also possible to consider a large class
% of conditional distributions for the observations by the introduction of 
% copulas~\citep{derrode2013unsupervised, derrode2016unsupervised}. 


% In summary, very general PMC or TMC models can be implicitly built
% from the introduction of an additional latent process or the use 
% of copulas~\citep{lanchantin2011unsupervised, li2019adaptive}.
% %Consequently, and up to our best knowledge, although PMCs and TMCs have been introduced in a very general context,
% %their application for 
% %unsupervised estimation 
% %(\ie~ the joint estimation of $\theta$ and of $\lab_t$ from $\obs_{0:T}$) 
% %has been possible by making specific assumptions. First, the $\latent_{0:T}$ process has always been considered as discrete in the literature when $\lab_{0:T}$ is also discrete.
% %Second, we note that most of the literature on PMCs tends towards the choice of a restricted set of classical
% %distributions for the conditional distributions of the observations $(\obs_{0:1})$
% %(\eg~ a Gaussian one whose parameters depend on $\lab_{0:1}$),
% %in this case they can be easily estimated by popular estimation algorithms \citep{gorynin2018assessing}. A notable exception is the line of work which considers the introduction of copulas in the class-conditional distributions~\citep{derrode2013unsupervised, derrode2016unsupervised},
% %this can lead to much more complex noise models. Note also that in TMC literature, because of the much richer modeling possibilities, it is not hard to find non-Gaussian TMCs~\citep{lanchantin2011unsupervised, li2019adaptive}.
% %}
% %\textcolor{}{Third, practical TMC models of the literature are \emph{stationary} processes in augmented dimensions offering a way of dealing with a non-stationary process (\emph{i.e.} estimating the hidden states $\lab_{0:T}$ and the stationarity regimes) by the introduction of an additional auxiliary process. This is one of the most appealing properties of TMCs but the stationarity assumption of the overall model can be restrictive in practice. For example, a stationary TMC has been used to handle a non-stationary PMC in~\citep{lanchantin2004unsupervised}. Another theoretical example, by pushing the same logic one step further, is that a non-stationary TMC $(\lab_{0:T}, \latent_{0:T}, \obs_{0:T})$ (with discrete $\lab_{0:T}$ and $\latent_{0:T}$) could be handled the same way by introducing a fourth process $\pmb{w}$; this gives rise to a stationary $(\h, \z, \pmb{w}, \x)$. We recall the reader that the assumption that, for a discrete $\lab_{0:T}$,
% %$(\lab_{0:T},\obs_{0:T})$ (resp. $(\latent_{0:T},\lab_{0:T},\obs_{0:T})$, where $\latent_t$ is discrete) is a
% %stationary process means that the distribution
% %$\p(\lab_{0:T},\obs_{0:T})$ (resp.  $\p(\latent_{0:T},\lab_{0:T},\obs_{0:T})$) is directly described by the initial distribution $\p(\lab_{0:1},\obs_{0:1})=\p(\lab_{0:1})\p(\obs_{0:1}|\lab_{0:1})$
% %(resp. $\p(\lab_{0:1},\latent_{0:1},\obs_{0:1})=\p(\lab_{0:1},\latent_{0:1})\p(\obs_{0:1}|\lab_{0:1},\latent_{0:1})$), see for example \citep{pieczynski2003pairwise, gorynin2018assessing}.
% %}
% % These distributions coincide with a
% % discrete distribution on $\Omega\times\Omega$ and a conditional
% % continuous one on $\mathbb{R}^{d_{\obs}} \times \mathbb{R}^{d_{\obs}}$, respectively; they are
% % thus easier to model in the sense that 
% % the conditioning  does not depend on a continuous r.v. 
% In addition, the PMC and TMC models are presented for unsupervised classification
% but with an alternative way to build such models
% which allows, for example, to describe explicitly, with
% any distribution, the probability distribution of the
% observation given the hidden states. Moreover, we also 
% consider the case where the third latent process $\latent_{0:T}$
% is continuous when $\lab_{0:T}$ remains discrete. The application 
% is twofold since it leads to models with implicit but
% potentially complex noise and it can also be considered
% as an extension of discrete TMCs which is able to model continuous non-stationarity. 
% Finally, our construction also enables
% us to cast directly in our models the powerful deep learning framework.

\newpage
\section{PMCs for unsupervised classification}
\label{sec:generalParam}

%sec-pmc
% We also introduce a sequence
% of labels $\lab_{0:T}=(\lab_{0}, \dots,\; \lab_{T})$ associated to the previous sequence $\obs_{0:T}$. 
% We will assume that  $\obs_t \in \mathbb{R}^{d_x}$, while the label
% $\lab_t$ is discrete, so  $\lab_t \in \Omega=\{\omega_1,\dots,\omega_C\}$, 
% where $C$ is the number of classes.


In this section, we do not consider the latent variable $\latent_t$ in order to
build a solution on a model without latent variables, which is already
challenging due to the absence of the labels $\lab_t$ associated to 
the observations $\obs_t$.
We adapt the PMC model 
discussed in Chapter~\ref{chap:pmc} 
to the unsupervised classification problem, where
the pair $(\latent_t,\obs_t)$ is replaced by $(\lab_t,\obs_t)$, 
where $\lab_t$ is a discrete r.v.
% where $\lab_t$ is a discrete r.v. associated to the observation $\obs_t$.
This modification addresses the need for interpretable models. 

The PMC model reads
\begin{equation}
  \label{eq:pmc_intro_uns}
  \p(\lab_{0:T},\obs_{0:T}) = \p(\lab_0)
  \prod_{t=1}^T \p(\lab_t, \obs_t |\lab_{t-1}, \obs_{t-1}) \text{,}
\end{equation}
where  the factorization of the transition distribution is given by
\begin{align}
  \label{eq:pmc_gen}
  \p(\lab_t,\obs_t|& \lab_{t-1},\obs_{t-1}) =
  \p(\lab_t|\lab_{t-1},\obs_{t-1})\p(\obs_t|\lab_{t-1:t},\obs_{t-1}) \text{.}
\end{align}

We also define the~\gls*{semipmc}, a particular instance of the PMC model, 
where the observation $\obs_t$ does not depend on $\lab_{t-1}$, 
given $(\lab_t,\obs_{t-1})$, \ie~
\begin{align}
  \p(\lab_t,\obs_t|& \lab_{t-1},\obs_{t-1}) =
   \p(\lab_t|\lab_{t-1},\obs_{t-1}) \p(\obs_t|\lab_t,\obs_{t-1}) \text{.}
\end{align}
This model is particularly interesting in the context of unsupervised classification, 
where the interpretability problem may be easier.
Figure~\ref{fig:pmc_graphs} illustrates the graphical representation of the PMC model 
and its particular instances that we consider in this section, 
\ie~the SPMC, and the HMC models.

\begin{figure}[htb]
    \begin{subfigure}[b]{0.3\linewidth}
      \centering
      \includegraphics[width=4cm]{Figures/Graphical_models/hmc_un.pdf}
      \caption{HMC}
      \label{fig:dhmcin}
      % \vspace{1.1cm}
    \end{subfigure}
    \hfill
    \begin{subfigure}[b]{0.3\linewidth}
      \centering
      \includegraphics[width=4cm]{Figures/Graphical_models/spmc_un.pdf}
      \caption{SPMC}
      \label{fig:dpmccn1}
      % \vspace{1.1cm}
    \end{subfigure}
    \hfill
    \begin{subfigure}[b]{0.3\linewidth}
      \centering
      \includegraphics[width=4.0cm]{Figures/Graphical_models/pmc_un.pdf}
      \caption{PMC}
      % \vspace{1.1cm}
      \label{fig:dpmccn2}
    \end{subfigure}
    \caption{Graphical representations of the HMC, SPMC, and PMC models.}
    % The white circles (resp. gray squares) represent the hidden (resp. observed) r.v. $\lab_t$ (resp. $\obs_t$).}
\label{fig:pmc_graphs}

\end{figure}
% In summary, a homogeneous process refers to the invariance of the 
% probability distribution with respect to the starting point in time, 
% while a stationary process refers to the constancy of the statistical properties 



% \subsection{General parameterization of PMCs}

% \yohan{already done in chapter 2. You can just explain that we consider the pair
% $(x,y)$ as a PMC as a generalization of the HMC for classification and we keep in
% mind that the objective is to propose complex parameterization}
We revisit the general parameterization of the PMC model introduced in
Chapter~\ref{chap:pmc} to adapt it to the unsupervised classification problem.
We parameterize the
conditional distributions in \eqref{eq:pmc_gen} as
\begin{eqnarray}
\label{pmc-theta-1}
\p(\lab_t|\lab_{t-1},\obs_{t-1})=\vartheta(\lab_t;\pyun(\lab_{t-1},\obs_{t-1})) \text{, } \\
\label{pmc-theta-2}
\p(\obs_t|\lab_{t-1:t},\obs_{t-1})=\zeta(\obs_t;\pxun(\lab_{t-1:t},\obs_{t-1})) \text{.}
\end{eqnarray}

% \begin{remark}
%   % In Section~\ref{sec:pmc_parameterization}, 
%   % we have introduced a general parameterization of the PMC model
%   % which is different from the one presented here.
%   In this section, we consider a general parameterization adapted to the
%   unsupervised classification problem with two different dicrete and 
%   continuous processes
%   $\lab_{0:T}$ and $\obs_{0:T}$, respectively.  
%   In Section~\ref{sec:pmc_parameterization}
%   the parameterization is adapted to the model that generates new observations $\obs_t$
%   with (non-interpretable) continuous latent variables $\latent_{0:T}$.
% \end{remark}


% \begin{example}
%   \begin{align*}
%       \py(\lab_{t-1},\, \obs_{t-1},\, \latent_t) &= 
%       {\rm sigm}(a_{\lab_{t-1}} \obs_{t-1} + b_{\lab_{t-1}} \latent_t + c_{\lab_{t-1}})
%       \text{,} \\
%       \px(\lab_{t}) 
%       &= \big[d_{\lab_{t}},\, \sigma_{\lab_{t}} \big] \text{,} \\
%       \pz(\obs_{t-1},\, \lab_{t-1}) &=  
%       \big[ e_{\lab_{t-1}}\obs_{t-1},\, \sigma_{\lab_{t-1}}'\big] 
%       \text{,} \\
%       \vartheta(\lab_t; \rho)&= \Ber\left(\lab_t; \rho \right) 
%       \text{,} \\
%       \zeta(\obs_t; s = [\mu,\, {\sigma}] ) &=     
%       \mathcal{N}(\obs_t; \mu,\, {\sigma}^2 )
%       \text{,} \\
%       \eta(\latent_t; s' = [\mu',\, {\sigma'}])  
%       &= \N\left(\latent_t;  \mu' ,\, {\sigma'}^2 \right)   
%       \text{,}
%   \end{align*}

% \end{example}

\begin{example}
  Let us show that this general parameterization
  includes the classical HMC with independent Gaussian noise (HMC-IN).
  Let us assume that $\Omega=\{\omega_1,\omega_2 \}$ and
  $\obs_t \in \mathbb{R}$. 
  % We denote $\mathcal{N}(\obs;m;\sigma^2)$ the Gaussian
  % distribution with mean $m$ and variance $\sigma^2$ taken at point $\obs$,
  % ${\rm Ber}(\lab;v)$ the Bernoulli distribution with parameter $v$ such that
  % ${\rm Ber}(\omega_1;v)=v$ and ${\rm sigm}(z)=1/(1+\exp(-z)) \in [0,1]$ the
  % sigmoid function. 
  In this case, the HMC-IN model can be described as
  \begin{align}
    \label{param-1}
    \py(\lab_{t-1},\, \obs_{t-1},\, \latent_t) &= 
    {\rm sigm}( b_{\lab_{t-1}})
    \text{,} \\
    \label{param-2}
    \px(\lab_{t}) 
    &= \big[d_{\lab_{t}},\, \sigma_{\lab_{t}} \big] \text{,} \\
    \label{param-32}
    \vartheta(\lab_t; \rho)&= \Ber\left(\lab_t; \rho \right) 
    \text{,} \\
    \label{param-4}
    \zeta(\obs_t; s = [\mu,\, {\sigma}] ) &=     
    \mathcal{N}(\obs_t; \mu,\, {\sigma}^2 )
    \text{,} 
\end{align}

%   \begin{align}
%     \vartheta(\lab_t|\; \lab_{t-1},\obs_{t-1} \;)&= \Ber\left(\lab_t; \ropy \right) 
% \text{,} \quad \text{ where } 
% \ropy  = \py(\lab_{t-1},\obs_{t-1}) \text{,}\\
% \zeta(\obs_t| \lab_{t-1:t},\obs_{t-1} ) &= 
% \mathcal{N}(\obs_t; \muobs,\, {\sigobs}^2 ) 
% \text{,} \quad \text{ where}
% \left[ \muobs,\, \sigobs \right]= \px(\lab_{t-1:t},\obs_{t-1}) \text{,}
% \end{align}
% with
% \begin{align}
%   \py(\lab_{t-1},\, \obs_{t-1}) &= 
%   {\rm sigm}\left(b_{\lab_{t-1}}\right)  \text{,} \\
%   \px(\lab_{t-1:t},\obs_{t-1}) &= \big[d_{\lab_{t}},\, \sigma_{\lab_{t}}^{\obs}  \big] \text{.}
% \end{align}

  % \begin{eqnarray}
  % \label{param-1}
  % \pyun(\lab_{t-1},\obs_{t-1})&=&
  % {\rm sigm}\left(b_{\lab_{t-1}}\right)  \text{,} \\
  % \label{param-2}
  % \pxun(\lab_{t-1:t},\obs_{t-1}) &=& \begin{bmatrix} d_{\lab_t}; 
  % %\label{param-3}
  % \sigma_{\lab_t}\end{bmatrix} \text{,} \\
  % \label{param-32}
  % \vartheta(\lab;v)&=& {\rm Ber}(h,v) \text{,} \\
  % \label{param-4}
  % \zeta\left(\obs;v'=\left[v'_1;v'_2\right]\right)
  % &=& \mathcal{N}\left(\obs;v'_1;(v'_2)^2\right) \text{.} 
  % \end{eqnarray}
  % Here, the notation $[\cdot, \cdot]$ refers to a vectorial output
  % and ${\rm sigm}(z)=1/(1+\exp(-z)) \in [0,1]$ to the sigmoid function.
  Indeed, ~\eqref{param-1}-~\eqref{param-2}
  only depend on $\lab_{t-1}$ and on $\lab_t$, respectively. 
  Thus, we have  $\p(\lab_t=\omega_1|\lab_{t-1}=\omega_i)={\rm sigm}(b_{\omega_i})$ 
  and $\p(\obs_t|\lab_t=\omega_j)=\mathcal{N}(\obs_t;d_{\omega_j};\sigma^2_{\omega_j})$. 
  Finally, the set of parameters is given by
  $\theta=(b_{\omega_i},d_{\omega_j},\sigma_{\omega_j}|(\omega_i,\omega_j) \in \Omega \times \Omega)$.
  % This parameterization can be easily extended to
  % the multi-class cases with $C>2$ by replacing $\pyun$
  % in \eqref{param-1} by a vector of softmax function.
  % ,
  % $$\pyun(\lab_{t-1},\obs_{t-1})=\left[\frac{e^{b_{\omega_1,\lab_{t-1}}}}{\sum_{j=1}^C e^{b_{\omega_j,\lab_{t-1}}}}; \cdots;\frac{e^{b_{\omega_C,\lab_{t-1}}}}{\sum_{j=1}^C e^{b_{\omega_j,\lab_{t-1}}}} \right] \text{,} $$ 
  % and $\vartheta(\lab;v)$ by the categorical distribution described
  % by the $C$ components of the vector $v$. 
  As a further illustrative example in the binary case,
  it is possible to start from this particular parameterization of HMCs to derive a linear and Gaussian PMC model in which we introduce
  dependencies on $\obs_{t-1}$ and $\lab_{t-1}$. In this case,
  $\vartheta$ and $\zeta$ are unchanged but $\pyun$
  and $\pxun$ now read as
  \begin{eqnarray}
  \label{param-1_bis}
  \pyun(\lab_{t-1},\obs_{t-1})&=&{\rm sigm}\left(a_{\lab_{t-1}}\obs_{t-1}+b_{\lab_{t-1}}\right) \text{,} \\
  \label{param-2_bis}
   \pxun(\lab_{t-1:t},\obs_{t-1}) &=& \begin{bmatrix} c_{\lab_{t-1},\lab_{t}}\obs_{t-1}+ d_{\lab_{t-1},\lab_{t}}; \
  %\label{param-3}
  \sigma_{\lab_t,\lab_{t-1}}\end{bmatrix} \text{.} 
  \end{eqnarray}
  The set of parameters is now given by 
  $\theta=(a_{\omega_i},b_{\omega_i},c_{\omega_j,\omega_i},
  d_{\omega_j,\omega_i}, \sigma_{\omega_j,\omega_i}|
   (\omega_j,\omega_i) \in \Omega^2 )$.
  As we will see later, these models play a critical role in the construction of parameterization based on DNNs. Indeed, despite their simple form, they generally provide an interpretable classification. 
  %The parameters related to $\pxun$ can be updated exactly  from \eqref{eq:EMq} while those related to $\pyun$ are updated according to the gradient update rule~\eqref{update-GEM}. Note that, in the SPMC model,  the coefficients describing $\pxun$ in \eqref{param-2} no longer depend on $\lab_{t-1}$.     
\end{example}

We now show that under this framework it is possible to derive an unsupervised
estimation algorithm which approximates the ML estimate of $\theta$,
%and which computes exactly the posterior
%distributions $\p(\lab_t|\obs_{0:T})$
no matter the choice of the parameterization $\pyun$ and $\pxun$.
% This contribution is based on the
% preliminary result~\citep{gangloff2021unsupervised} but has been revised and
% extended. 
In particular, we use a direct ML approach rather than an EM one (see
Remark~\ref{rem-EM}) and introduce a pretraining approach for deep
parameterizations. This pretraining approach is a novel contribution that will
be detailed in the next sections. 
Once $\theta$ has been estimated, we resort to the classical
estimation of the posterior distributions $\p(\lab_t|\obs_{0:T})$.
%\begin{remark}
%\label{rk:multiclass1}
%\textcolor{}{In Eqs.~\ref{param-1} and~\ref{param-1_bis}, we illustrate the deep parametrization in the case of the binary hidden random variables $\lab_t$ and scalar observations $\obs_t$, $\forall k$. The results could be straightforwardly extended to the multi-class ($C > 2$) and multi-channel ($d_{\obs} > 1$) case considering a one-hot encoding for $\lab_t$, a dot product between $\lab_{t-1}$ and $\obs_{t-1}$ and transforming the sigmoid function into the softmax function.}
%\end{remark}



\subsection{Bayesian inference for PMCs}
\label{sec:inference_pmc}
% \yohan{you should make a connection with Chap 2 : explain that in the case where
% the hidden variable is discrete one can compute exactly the likelihood so we do
% not need to resort to VBI; equivalently it is because the optimal variational
% distribution $q(y_{0:t}|x_{0:t})$  is computable.}



\subsubsection{Estimation of $\theta$} 


Since the hidden variable $\lab_t$ is discrete, the likelihood
$\p(\obs_{0:T})$ can be computed exactly. 
This accessibility is a key point in the estimation of $\theta$. 
Here, the VI method is not necessary to approximate the likelihood (equivalently, 
the optimal variational distribution is available).
% , and 
% equivalently, the optimal variational distribution $q(\lab_{0:T}|\obs_{0:T})$ 
% is available.
Given the differentiability of the functions $\pyun$, $\pxun$, $\vartheta$, and
$\zeta$, we can propose a gradient ascent method on the likelihood $\p(\obs_{0:T})$ 
to approximate 
the ML estimate of $\theta$. This gradient ascent method is based on the sequential computation 
of  $\alpha_{\theta,t}(\lab_t)=\p(\lab_t,\obs_{0:t})$,
for all $t$, $0 \leq t \leq T$,
from which we deduce the likelihood
\begin{equation}
\label{likelihood-pmc}
\p(\obs_{0:T})=\sum_{\lab_{T}} \alpha_{\theta,T}(\lab_{T}) \text{.}
\end{equation}
Based on the Markovian property of \eqref{eq:pmc_intro_uns} and on the general 
parameterization \eqref{pmc-theta-1}-\eqref{pmc-theta-2}, the
coefficients $\alpha_{\theta,T}(\lab_{T})$ can be computed 
recursively from~\citep{pieczynski2003pairwise} as 
\begin{equation}
\label{eq:alpha}
\alpha_{\theta,t}(\lab_t)=\sum_{\lab_{t-1}}\alpha_{\theta,t-1}(\lab_{t-1}) \vartheta(\lab_t;\pyun(\lab_{t-1},\obs_{t-1})) 
\zeta(\obs_t;\pxun(\lab_{t-1:t},\obs_{t-1})) \text{.}
\end{equation}
Consequently, the gradient of the likelihood $\p(\obs_{0:T})$ 
(or equivalently that of the log-likelihood) w.r.t.~$\theta$ can be
deduced from that of $\alpha_{\theta,t}$,
which is itself sequentially computable
by using the decomposition~\eqref{eq:alpha}  
because \(p(y_t, x_{0:t}) = \sum_{y_{t-1}} p(y_{t-1:t}, x_{0:t}) = 
\sum_{y_{t-1}} p(y_{t-1:t}, x_{0:
t-1}) p(y_t, x_t \mid y_{t-1}, x_{t-1})\)
This sequential structure has the advantage
that numerical auto-differentiation methods
can be used to compute such gradients in practice~\citep{NEURIPS2019_9015}.
%computed sequentially by using the decomposition of $\alpha_{\theta,t}$; 
%\textcolor{}{indeed, \eqref{eq:alpha} enables
%us to compute the gradient of $\alpha_{\theta,t}$ in function of that of $\alpha_{\theta,t-1}$ }
%Moreover, we consider maximizing the log-likelihood w.r.t.~$\theta$, since it is equivalent and computationally more appealing than maximizing the likelihood w.r.t.~$\theta$.
The estimation of $\theta$ can thus be deduced
from an iterative gradient ascent method based on a learning
rate $\epsilon$ and, for example, on the update
\begin{equation}
\label{grad-likelihood}
\theta^{(j+1)}=\theta^{(j)} + \epsilon {\nabla_{\theta} \log{\p(\obs_{0:T})}}\Big|_{\theta=\theta^{(j)}} \text{.}
\end{equation}
The unsupervised estimation of $\theta$ is summarized in Algorithm~\ref{algo:algo_theta_pmc}.
The gradients can be computed automatically through auto-differentiation tools,
\eg~JAX by~\cite{jax2018github}.



\begin{remark}
\label{rem-EM}
Generally, the parameter estimation
procedure for a probabilistic model with hidden r.v. is based on the 
EM algorithm~\citep{dempster1977maximum} 
(see Algorithm~\ref{algo:em_algorithm}
 in Appendix~\ref{chap:appendix}).
It relies on the computation of 
$$Q(\theta,\theta^{(j)})=\E_{p_{\theta^{(j)}}(\lab_{0:T}|\obs_{0:T})}\big(\log p_{\theta}(\lab_{0:T},\obs_{0:T}) \big)$$ 
followed by the maximization of $Q(\theta,\theta^{(j)})$ w.r.t.~$\theta$.
However, for general parameterizations \eqref{pmc-theta-1}-\eqref{pmc-theta-2},
the maximization step cannot be computed analytically. In this case, it is
possible to use a gradient-EM approach to replace the maximization step, but it
is then strictly equivalent and computationally more demanding than computing
the gradient of the log-likelihood~\citep{xu1996convergence,
balakrishnan2017statistical} as we propose in~\eqref{grad-likelihood}. Finally,
for particular parameterizations for which the maximization step is computable,
the comparison between these two approaches is an open question and is out of
scope of this thesis.
\end{remark}

% \begin{remark}
% \label{rem:link_pmc_generative}
% \katy{Link with PMC presentation in Chapter 2??}
% \end{remark}

\begin{algorithm}[htbp!]
  \caption{Unsupervised estimation of $\theta$ in general PMC models.}
  \label{algo:algo_theta_pmc}
  \begin{algorithmic}[1]
  %\KwData{$\pmb{x}_{0:T}_{I}$, the observed image}
  \Require{A realization $\obs_{0:T}$, a set of estimated parameters $\theta^*$}
  \Ensure{$\theta^*$, a set of estimated parameters}
  \State  $j=0$\label{line:start_dpmc}
  \While{\text{ convergence of   $\log p_{\theta^{(j)}}(\obs_{0:T})$ is not attained}}
  \State  Compute $\log\alpha_{\theta^{(j)},t}(\lab_t)$ and $\nabla_{\theta}\log\alpha_{\theta^{(j)},t}(\lab_t)\Big|_{\theta=\theta^{(j)}}$, for all $\lab_t \in \Omega$, 
  for all $0 \leq t \leq T$, with \eqref{eq:alpha}
  \State Compute $\log p_{\theta^{(j)}}(\obs_{0:T})$ and $\nabla_{\theta}\log p_{\theta^{(j)}}(\obs_{0:T})\Big|_{\theta=\theta^{(j)}}$, with~\eqref{likelihood-pmc}

  \State Set $\theta^{(j+1)}=\theta^{(j)} + \epsilon {\nabla_{\theta} \log{\p(\obs_{0:T})}}\Big|_{\theta=\theta^{(j)}}$\label{update-GEM} 
  \State $j\leftarrow j+1$
  \EndWhile
  \State  $\theta^{*} \leftarrow \theta^{(j)}$
\end{algorithmic}
  % \vspace*{0.2cm}
\end{algorithm}


\subsubsection{Estimation of $\lab_t$}
Once we have obtained an estimate $\theta^*$ of $\theta$, it remains to compute
$p_{\theta^*}(\lab_t|\obs_{0:T})$, for all $t$. Since we deal with particular PMCs, it
can be done by following the steps of~\citet{pieczynski2003pairwise}, \ie~by
using the Markovian property of~\eqref{eq:pmc_intro_uns} 
and by introducing the
backward coefficients
$\beta_{\theta^*,t}(\lab_t)= p_{\theta^*}(\obs_{t+1:T}|\lab_t,\obs_t)$, for all $t$,
with $\beta_{\theta^*,T}(\lab_{T})=1$.
These coefficients can be computed sequentially from
\begin{equation}
\label{eq:beta}
\beta_{\theta^*,t-1}(\lab_{t-1})= 
\sum_{\lab_{t}} \beta_{\theta^*,t}(\lab_{t})
\vartheta(\lab_t;\pyop(\lab_{t-1},\obs_{t-1})) 
\zeta(\obs_t;\pxop(\lab_{t-1:t},\obs_{t-1})) \text{.}
\end{equation}
Thus, we deduce
% \yohan{next equations $\theta^*$ should be replaced by $\theta$} 
\begin{eqnarray}
\label{eq:pair_post_margin}
p_{\theta^*}(\lab_{t-1:t}|\obs_{0:T}) &\propto&
\alpha_{\theta^*,t-1}(\lab_{t-1}) \times  \beta_{\theta^*,t}(\lab_t) \times
\vartheta(\lab_t;\pyop(\lab_{t-1},\obs_{t-1})) \times \nonumber\\
&&  \;\;  \zeta(\obs_t;\pxop(\lab_{t-1:t},\obs_{t-1}))   \text{,} \\
\label{eq:post_margin}
p_{\theta^*}(\lab_t|\obs_{0:T})&=&\sum_{\lab_{t-1}} p_{\theta^*}(\lab_{t-1:t}|\obs_{0:T}).
\end{eqnarray}
The computation of the MAP estimate
of $\lab_t$ is summarized in Algorithm~\ref{algo:algo_hk_pmc}.
% \katyobs{check acronym MAP with other sections} 


%In this section, we present  a general Bayesian inference algorithm for the PMCs. Our objective is to compute $\p(\lab_{t}|\obs_{0:T})$ for all $t$ in model  \eqref{eq:PMC},  which satisfies \eqref{pmc-theta-1}-\eqref{pmc-theta-2}, and we use the general expressions derived in ~\citep{pieczynski2003pairwise}, which are still valid here. 
%These expressions are a direct extension of the Forward-Backward algorithm, so using  the Markovian property of $\p(\lab_{0:T},\obs_{0:T})$ in \eqref{eq:pmc_gen},  we set  $\alpha_{\theta,t}(\lab_t)=\p(\obs_1,\cdots,\obs_t,\lab_t)$ 
%and $\beta_{\theta,t}(\lab_t)= \p(\obs_{k+1},\cdots,\obs_{0:T}|\lab_t,\obs_t)$, $\beta_{\theta,T}(\lab_{T})=1$, for all $t$, $1 \leq t \leq T$. The following expression are derived,
%\begin{align}
%\label{eq:alpha}
%&    \alpha_{\theta,t}(\lab_t)=\sum_{\lab_{t-1}}\alpha_{\theta,t-1}(\lab_{t-1}) \p(\lab_t,\obs_t|\lab_{t-1},\obs_{t-1}) \text{,  for all } k \text{, } 1 \leq t \leq T \text{,}\\
%\label{eq:beta}
%& \beta_{\theta,t-1}(\lab_{t-1})= \sum_{\lab_{t}} \beta_{\theta,t}(\lab_{t})\p(\lab_t,\obs_t|\lab_{t-1},\obs_{t-1}) \text{,  for all } k \text{, }  K >  k \pxuneq 1 \text{;}
%\end{align}
%where  $\p(\lab_{t},\obs_{0:t}|\lab_{t-1},\obs_{t-1})=  
%\vartheta(\lab_t;\pyun(\lab_{t-1},\obs_{t-1})) \times \zeta(\obs_t;\pxun(\lab_t,\lab_{t-1},\obs_{t-1}))$. We finally deduce $\p(\lab_t|\obs_{0:T})$ using the fact that

%\begin{align}
%\label{eq:pair_post_margin}
%& \p(\lab_{t-1},\lab_t|\obs_{0:T}) \propto
%\alpha_{\theta,t-1}(\lab_{t-1}) \times  \beta_{\theta,t}(\lab_t) \times  \p(\lab_t,\obs_t|\lab_{t-1},\obs_{t-1}) \text{,}
%\end{align}
%and it is given by
%\begin{equation}
%\p(\lab_t|\obs_{0:T})=\sum_{\lab_{t-1}} \p(\lab_{t-1},\lab_t|\obs_{0:T}).
%\label{eq:post_margin}
%end{equation}

%The aim is to estimate the  unknown parameter $\theta$  from a realization $\obs_{0:T}$, \cite{pieczynski2003pairwise} proposed  a generalization of the classical Iterative Conditional Estimation (ICE) as a method of parameter estimation for stationary PMC models. 
%In this article, we propose a maximum likelihood estimation approach which is a variant of the Expectation-Maximisation (EM) algorithm~\citep{dempster1977maximum}. 
%The E-step of the EM algorithm computes $Q(\theta,\theta^{(j)})={\rm E}_{\theta^{(j)}}(\log(p_{\theta}(\lab_{0:T},\obs_{0:T}))|\obs_{0:T}) \text{,}$ and the M-step consist of maximizing over $\theta$. 
%The two steps are repeated until the convergence. 
%In our case $Q(\theta,\theta^{(j)})$ reads, 
%\begin{align}
%\label{eq:EMq}
%&Q(\theta,\theta^{(j)})= 
%\sum_{t=2}^T \sum_{\lab_{t-1},\lab_t} p_{\theta^{(j)}}(\lab_{t-1},\lab_{t}|\obs_{0:T}) \log(\vartheta(\lab_t;\pyun(\lab_{t-1},\obs_{t-1}))\zeta(\obs_t;\pxun(\lab_t,\lab_{t-1},\obs_{t-1}))) \text{.}
%\end{align}

%We have that  $Q(\theta,\theta^{(j)})$ can be exactly computed  regardless of the parametrization \eqref{pmc-theta-1}-\eqref{pmc-theta-2} (E-step), since $p_{\theta^{(j)}}(\lab_{t-1},$ $ \lab_{t}|\obs_{0:T})$ is computable.  The next step is to maximize $Q(\theta,\theta^{(j)})$ w.r.t.~$\theta$ (M-step).  However, the M-step  is not feasible
%in general PMC models, with the exception of simple models (\emph{e.g.} Gaussian and linear PCM). The Gradient EM (GEM) algorithm~\citep{balakrishnan2017statistical} deals with this problem by considering a gradient step towards the maximum of the expectation calculated in the E-step.  In this case, the computation of the gradient of  $Q(\theta,\theta^{(j)})$ w.r.t.~$\theta$ can be done since $\vartheta(\lab;v)$ and $\zeta_{\obs}(\obs;v')$  (resp. $\pyun$ and 
%$\pxun$) are differentiable w.r.t.~$v$ and $v'$ (resp. w.r.t.~$\theta$).  Thus, according to ~\cite{balakrishnan2017statistical},   the parameter update reads, 

%\begin{equation}
%\label{update-GEM}
%\theta^{(j+1)}=\theta^{(j)} + \epsilon {\nabla_{\theta} Q(\theta,\theta^{(j)})}\Big|_{\theta=\theta^{(j)}} \text{,}
%\end{equation}
%where $\epsilon$ is the learning rate. 
%Finally,  Algorithm \ref{algo:algo_1} summarizes the inference and parameter estimation processes for our general PMC models.

% \begin{algorithm}[htbp!]
% \DontPrintSemicolon
% \KwData {A realization $\obs_{0:T}$, a learning rate $\epsilon$, an initial set of parameters $\theta^{(0)}$} 
% %% \phantom{Data:}~~$\theta^{(0)}$, an initial set of parameters}
% \KwResult{$\theta^*$, a set of estimated parameters}
% $j=0$\label{line:start_dpmc}\\
% \While{\text{ convergence of   $\log p_{\theta^{(j)}}(\obs_{0:T})$ is not attained}}{
% Compute $\log\alpha_{\theta^{(j)},t}(\lab_t)$ and $\nabla_{\theta}\log\alpha_{\theta^{(j)},t}(\lab_t)\Big|_{\theta=\theta^{(j)}}$, for all $\lab_t \in \Omega$, for all $0 \leq t \leq T$, with \eqref{eq:alpha}\\
% Compute $\log p_{\theta^{(j)}}(\obs_{0:T})$ and $\nabla_{\theta}\log p_{\theta^{(j)}}(\obs_{0:T})\Big|_{\theta=\theta^{(j)}}$, with~\eqref{likelihood-pmc}
% \\
% Set $\theta^{(j+1)}=\theta^{(j)} + \epsilon {\nabla_{\theta} \log{\p(\obs_{0:T})}}\Big|_{\theta=\theta^{(j)}}$\label{update-GEM} \\
% $j\leftarrow j+1$
% }
% $\theta^{*} \leftarrow \theta^{(j)}$\\
% \caption{}
% \label{algo:algo_theta_pmc}
% \end{algorithm}





\begin{algorithm}[htbp!]
  \caption{Unsupervised estimation of $\lab_t$ in general PMC models.}
  \label{algo:algo_hk_pmc}
  \begin{algorithmic}[1]
  %\KwData{$\pmb{x}_{0:T}_{I}$, the observed image}
  \Require{A realization $\obs_{0:T}$, a set of estimated parameters $\theta^*$}
  \Ensure{$\hat{{\lab}}_{0:T}$, the estimated hidden r.v.}\\
  Compute $\alpha_{\theta^{*},t}(\lab_t)$, for all $\lab_t \in \Omega$, for all $0 \leq t \leq T$, with~\eqref{eq:alpha}\\
  Compute $\beta_{\theta^{*},t}(\lab_t)$, for all $\lab_t \in \Omega$, for all $0 \leq t \leq T$, with~\eqref{eq:beta}\\
  Compute $p_{\theta^{*}}(\lab_{t-1:t}|\obs_{0:T})$, for all $\lab_{t-1:t} \in \Omega \times \Omega$, for all $0 \leq t \leq T$, with \eqref{eq:pair_post_margin}\\
  Compute $\hat{\lab}_t= \argmax p_{\theta^{*}}(\lab_t|\obs_{0:T})$, for all $0 \leq t \leq T$, with~\eqref{eq:post_margin}\label{line:end_dpmc}  
  \end{algorithmic}
  % \vspace*{0.2cm}
\end{algorithm}

% \begin{remark}
% % \yohan{explain that the gradients can be computed due to the sequential
% % formules; and so that in practice, it is possible to compute them automaticcaly
% % through auto differenciation and next using any optimizer}
% The gradients can be computed automatically through auto-differentiation tools
% (\eg~ JAX by~\cite{jax2018github}), and then
% used with any optimizer, \eg~ Adam~\citep{kingma2014adam} in Line~\eqref{update-GEM}.
% % The Adam optimizer~\citep{kingma2014adam} 
% % is used instead of 
% % the vanilla gradient update shown in Line~\eqref{update-GEM} 
% % of Algorithm~\ref{algo:algo_hk_pmc}.
% % % ~\ref{rem:adam_pmc}.
% % FROM HERE 
% % \yohan{should be included in the previous remark. Actually, it means that the
% % only interest of the EM approach is when we have closed-form updates}
% % In some rare cases (which exclude the deep models introduced in 
% % Section~\ref{sec:deeppmc}), the maximization of $Q(\theta,\theta^{(j)})$ 
% % admits closed form formulas which are then used. 
% % TO HERE\\
% % Otherwise, JAX~\citep{jax2018github} is used as the auto-differentiation tool.
% \end{remark}

% \subsection{Gaussian and linear PMC}
% \label{mod:m1}
% \textbf{check that no copy paste remains from}\citep{gangloff2021unsupervised}


\subsection{Deep PMCs for unsupervised classification}
\label{sec:deeppmc}
We consider the particular parameterization $\pyun$ and $\pxun$ of the
distributions $\vartheta$ and $\zeta$, respectively, where $\pyun$ and $\pxun$
are the outputs of two DNNs with $(\lab_{t-1},\obs_{t-1})$ and $(\lab_{t-1:t},
\obs_{t-1})$ as inputs, respectively (as in Section~\ref{sec:dpmc}).
% ). Thus, the
% set of parameters $\theta$ consists of the weights and biases of these DNNs.
Note that a unique DNN is used for $\pyun$ (resp. $\pxun$) overtime.

Since $\pyun$ and $\pxun$ are differentiable w.r.t.~$\theta$ and their gradients
are computable from the backpropagation algorithm~\citep{rumelhart1986learning},
Algorithm~\ref{algo:algo_theta_pmc} can be directly applied to estimate $\theta$.
However, due to the large number of parameters of these architectures, some
problems tend to appear in practice. In particular, a random initialization of
$\theta$ can lead to convergence issues for the optimization of $\log \p(\obs_{0:
T})$. More importantly, the final r.v. $\lab_t$ learned by such a model may no
longer be interpretable, \ie~it is not ensured that $\lab_t$ coincides with the
original class associated to $\obs_t$. In other words, a direct application of
Algorithm~\ref{algo:algo_theta_pmc} tends to return a final model which gives
poorer results than the simple models described in Section~\ref{sec:generalParam} 
in terms of classification, as it considers $\lab_t$ as a latent variable rather
than an interpretable label.


We propose a two-step solution based on a constrained output layer and on a
pretraining which aims at initializing properly $\theta$. This solution relies
on a simple model such as the linear and Gaussian PMC described in Section
\ref{sec:generalParam} where the linear functions $\pyun$ and $\pxun$ in
\eqref{param-1_bis}-\eqref{param-2_bis} can be seen as the output layer of an
elementary DNN with no hidden layer. Rather than directly training the DNN
associated to $\pyun$ and $\pxun$, we first estimate the linear PMC model
\eqref{param-1_bis}-\eqref{param-2_bis} with Algorithm~\ref{algo:algo_theta_pmc}
before adding intermediate layers. % and residual connections. 
These layers are
next pretrained from the classification obtained with the elementary model, and
are finally finely trained with our ML approach.


%In this section, our aim is to present the contributions of the PMCs models combined with DNNs in the context of  unsupervised binary image; In other words, the functions $\pyun$ and $\pxun$ are parametrized by DNNs. 
%This gives rise to the Deep-PMC (DPMC) and the Deep-SPMC (DSPMC) models.
%In the deep models, $\pyun$ and $\pxun$ are parametrized by (deep) neural networks with rectified linear activation functions for intermediate hidden layers and linear or square function for the output layer; and the gradient of $Q(\theta,\theta^{(j)})$ w.r.t.~$\theta$ is deduced from  those of $\pyun$ and $\pxun$, which are computable with the backpropagation algorithm~\citep{rumelhart1986learning}. Thus,  the update rule given by \eqref{update-GEM} is performed.\\

%In practice, a random initialization of the parameters would lead to convergence problems in the optimization of $Q(\theta,\theta^{(j)})$ and $\lab_t$ is no longer interpretable because we do not ensure that the learned latent random variable $\lab_t$ coincides with the  original class associated to the observation $\obs_t$. In the next section, a two-step solution (output layer constraint - pretraining) is proposed to deal with this problem. \\

\subsubsection{Constrained output layer}
\label{sec:constrained_archi}
%From now on, we will denote $\theta_{\fr}$ (resp. $\theta_{\ufr}$)
%the set of frozen (resp. unfrozen) parameters. $\theta_{\fr}$ coincides
%with the parameters associated to the output of the DNN while 
%$\theta_{\ufr}$ describes the parameters of the hidden layer.
The main idea of our constrained training step is to make coincide a subset of $\theta$ %$\theta_{\ufr}$ 
with the parameters of an elementary linear (equivalently a non-deep) PMC model 
\eqref{param-1_bis}-\eqref{param-2_bis}
which is assumed to provide
an interpretable classification.
In other words, we first estimate an elementary linear PMC model with Algorithm~\ref{algo:algo_theta_pmc}, 
and we denote the set of 
associated parameters $\theta_{\fr}$, in the sense that these parameters are next \emph{frozen} and will not be further 
updated. We next consider this linear layer as the output layer of a DNN where
the other parameters are denoted $\theta_{\ufr}$, which
are \emph{unfrozen} in the sense that they have not been estimated yet. 


\begin{figure}[H]
  \begin{minipage}[c]{\textwidth}
    \centering
    {
      \includegraphics[width=0.75\textwidth]{Figures/Graphical_models/frozen_un.pdf}
    }
    \end{minipage}%
    \vspace{.3cm}
    % \begin{minipage}[c]{\textwidth}
    % {
    %   $\Sigma=\pyun(\lab_{t-1},\obs_{t-1},\lab_{t-1}\obs_{t-1})
    %   ={\rm sigm} (\textcolor{black}{\gamma_1}
    %   l^3_1+ \textcolor{black}{\gamma_2}l^3_2+ 
    %   \textcolor{black}{\gamma_3}l^3_3+ \textcolor{black}{\kappa})$,
    %   where the last layer parameters $ \{\gamma_1,\gamma_2,\gamma_3,\kappa\}$ are frozen to
    %   $\gamma_1=b_{\omega_2}-b_{\omega_1},
    %   \gamma_2=a_{\omega_2}-a_{\omega_1},
    %   \gamma_3=a_{\omega_1}$ and $\kappa=b_{\omega_1}$.      
    % }
    % \end{minipage}
  % \includegraphics[width=0.75\textwidth]{Figures/Graphical_models/frozen_un.pdf}
  % \caption{Graphical representation of the frozen PMC model.}
  \caption{DNN architecture with constrained output layer for $\pyun$ with two hidden layers.
  $\Sigma=\pyun(\lab_{t-1},\obs_{t-1},\lab_{t-1}\obs_{t-1}) ={\rm sigm}
  (\textcolor{black}{\gamma_1} l^3_1+ \textcolor{black}{\gamma_2}l^3_2+
  \textcolor{black}{\gamma_3}l^3_3+ \textcolor{black}{\kappa})$, where the last
  layer parameters $ \{\gamma_1,\gamma_2,\gamma_3,\kappa\}$ are frozen to
  $\gamma_1=b_{\omega_2}-b_{\omega_1}, \gamma_2=a_{\omega_2}-a_{\omega_1},
  \gamma_3=a_{\omega_1}$ and $\kappa=b_{\omega_1}$.\\
  The parameters $\theta_{\fr}$ are related to the output layer which 
  computes the function $\pyun$ of the linear PMC model \eqref{param-1_bis}.
  Due to the one-hot encoding of the discrete r.v. $\lab_{t-1}$
  ($\lab_{t-1}=\omega_1 \leftrightarrow \lab_{t-1}=0$ and $\lab_{t-1}=\omega_2 \leftrightarrow \lab_{t-1}=1$),
  this parameterization 
  is equivalent to that of \eqref{param-1_bis}
  up to the  given
  correspondence between $\theta_{\fr}=(\gamma_1, \gamma_2, \gamma_3,\kappa)$
  and $(a_{\omega_1},a_{\omega_2},b_{\omega_1},b_{\omega_2})$. \textcolor{black}{When the number 
  of classes $C$ increases, the size of the first
  and last layer increases due to the one-hot encoding of $\lab_{t-1}$.}
  %The residual connections between the inputs and the last hidden and frozen layer are emphasized.
  Linear activation functions are used in the last hidden layer in red.
  %Note that the non-linear ReLU activations following the white hidden layer nodes are not represented for clarity.}
  %Example of the output-layer constraint for a DNN $f_{\pmb{\theta}}$: the parameters of the output linear  $\theta_{\fr}$ have been estimated from the related non-deep model parameters (Eq.~\ref{pmc-theta-1}) and are then frozen.}
  }
  \label{fig:constrained_archi}
\end{figure}
Figure~\ref{fig:constrained_archi} describes an example 
of a constrained DNN architecture for the function $\pyun$
when $\Omega=\{\omega_1,\omega_2\}$ and $\mathbb{R}^{d_{\obs}} = \mathbb{R}$,
without loss of generality. 

% The output layer of the DNN computes the function $\pyun$ 
% of the linear PMC model \eqref{param-1_bis} and is given by
% \begin{align*}
%   \Sigma=\pyun(\lab_{t-1},\obs_{t-1},\lab_{t-1}\obs_{t-1})
%   ={\rm sigm} (\textcolor{black}{\gamma_1}
%   l^3_1+ \textcolor{black}{\gamma_2}l^3_2+ 
%   \textcolor{black}{\gamma_3}l^3_3+ \textcolor{black}{\kappa}),
% \end{align*}
% where the last layer parameters $ \{\gamma_1,\gamma_2,\gamma_3,\kappa\}$ are frozen to
% $\gamma_1=b_{\omega_2}-b_{\omega_1},
% \gamma_2=a_{\omega_2}-a_{\omega_1},
% \gamma_3=a_{\omega_1}\text{ and }
% \kappa=b_{\omega_1}.$





\subsubsection{Pretraining by backpropagation}
\label{sec:pretraining_backprop}
It remains to estimate 
the parameters $\theta_{\ufr}$ of the intermediate hidden
layers. The idea 
is to initialize them in a such way
that the initial DPMC coincides with
the elementary one; in other words, and
due to the previous step,
the output of the newly added hidden layers aims at coinciding with the 
identity function after the pretraining.
After initializing randomly $\theta_{\ufr}$,
our pretraining step aims at minimizing
cost functions $C_{\pyun}$ and $C_{\pxun}$ which involve the pre-classification $\hat{\lab}_{0:T}^{\pre}$. 
Typically, 
the cost function $C_{\pyun}$ is the
averaged overtime cross-entropy
between the output of the DNN $\pyun$
and $\hat{\lab}_{t}^{\pre}$ and $C_{\pxun}$ is
the mean square error between the output
of $\pxun$ and the parameters of the elementary
linear models associated to $\hat{\lab}_{t-1:t}^{\pre}$
(see Equation~\eqref{param-2_bis}). The minimization of
these cost functions w.r.t.~$\theta_{\rm ufr}$ is done with the backpropagation algorithm.
%Before computing the ML estimate of 
%these parameters, we initialize them by running a supervised (pre)training which relies on a classification $\hat{\h}_{0:T}^{\pre}$ obtained
%from the elementary linear model which aims
%at ensuring that the r.v. $\lab_t$ are interpretable. To that end, 
%based on the backpropagation algorithm. This procedure makes use of the classification $\hat{\lab_{0:T}}_{\pre}$ obtained
%with the elementary model and of a given cost function in order to ensure that the final r.v. $\lab_t$ are interpretable. This backpropagation procedure starts by making sure that the newly added hidden layers implement the identity function using a residual learning idea. Indeed, at the beginning of the backpropagation $\theta_{\ufr}$ is randomly initialized around the null vector and residual connections between the inputs and the last hidden and frozen layer are set.
%The cost function $\mathcal{C}_f$ for the
%pretraining of $\pyun$ is typically the cross-entropy between
%the first classification $\hat{\lab_{0:T}}_{\pre}$ and the output
%$\pyun$\hugo{ while that for}{. The cost function for the} pretraining of $\pxun$, $\mathcal{C}_g$,
%coincides with the mean square error
%etween the observations $\obs_t$ and the output of $\pxun$.
Finally, once $\theta_{\ufr}$ has been properly initialized, it is fine-tuned with  Algorithm~\ref{algo:algo_theta_pmc} which approximates the ML estimate of $\theta$. 
Algorithm~\ref{algo:algo_train_dpmc} summarizes the two estimation steps specific to the DNN parameterization.

\begin{remark}
In order to estimate the parameters 
of our deep PMC, we have used a reverse approach w.r.t. the pretraining approaches proposed at the beginning of 2010s to help
supervised learning in DNN~\citep{erhan2010does}.
Indeed, due to the large number of parameters in these architectures,~\citep{Mohamed-DBN, Glorot, deep-SPM} 
have suggested to first pretrain in an unsupervised way a DNN from a generative probabilistic model which shares common parameters with the original DNN
(\eg~a Deep Belief Network). The backpropagation algorithm for supervised estimation is 
next initialized with the
(approximated) ML estimate of this probabilistic model.
Here, we have started to pretrain our architecture in a supervised
way with a pre-classification and
next embedded it in our original probabilistic model in which we compute an approximation
of the ML estimate. 
\end{remark}


\begin{algorithm}[htbp!]
  \caption{A general estimation algorithm for deep parameterization of PMC models.}
  \label{algo:algo_train_dpmc}
  \begin{algorithmic}[1]
  \Require{$\obs_{0:T}$, the observation}
  \Ensure{$\hat{{\lab}_{0:T}}$, the final classification}
  \Statex{\textbf{Linear model: initialization of the output layer of $\pyun$ and $\pxun$  ($\S$ \ref{sec:constrained_archi})}}
  \\Initialize randomly $\theta_{\fr}^{(0)}$  \label{line:nondeep1} \\
  Estimate $\theta_{\fr}^*$ using Algorithm~\ref{algo:algo_theta_pmc} with $\theta_{\fr}^{(0)}$\\
  Estimate $\hat{\lab}_{0:T}^{\pre}$ using Algorithm~\ref{algo:algo_hk_pmc} with $\theta_{\fr}^*$ \label{line:nondeep3}
  \Statex{\textbf{Pretraining of $\theta_{\ufr}$ ($\S$ \ref{sec:pretraining_backprop})}}
  \\ Compute $\theta_{\ufr}^{*}$
  using Algorithm~\ref{algo:algo_theta_pmc} with $\theta^{(0)}=(\theta_{\fr}^*, \theta_{\ufr}^{(0)})$  ($\theta_{\fr}^*$ is not updated)\\
  Compute $\hat{\lab}_{0:T}$ using Algorithm~\ref{algo:algo_hk_pmc} with $\theta^{*}=(\theta_{\fr}^*, \theta_{\ufr}^{*})$
  \Statex{\textbf{Complete deep model: fine-tuning}}
  \\ Compute $\theta_{\ufr}^{*}$
    using Algorithm~\ref{algo:algo_theta_pmc} with $\theta^{(0)}=(\theta_{\fr}^*, \theta_{\ufr}^{(0)})$  ($\theta_{\fr}^*$ is not updated)\\
  Compute $\hat{\lab}_{0:T}$ using Algorithm~\ref{algo:algo_hk_pmc} with $\theta^{*}=(\theta_{\fr}^*, \theta_{\ufr}^{*})$
  \end{algorithmic}
  % \vspace*{0.2cm}
\end{algorithm}


\subsection{Simulations}
\label{sec:pmc}
We illustrate the performance of our models with the same 
binary image segmentation problem as in Chapter~\ref{chap:semi_supervised_pmc_tmc}.
% We illustrate the gain of our general parameterization w.r.t. an elementary
% HMC-IN by considering a problem of unsupervised binary image segmentation 
% ($\Omega=\{\omega_1,\omega_2\}$) from noisy observations.
% Similarly to Subsection~\ref{subsec:data_generation}
% and 
In order to highlight our unsupervised approach,
we consider the cattle-type images of the Binary Shape Database. %
% \footnote{\url{http://vision.lems.brown.edu/content/available-software-and-databases}}. 
The images are transformed into a $1$-D signal $\obs_{0:T}$ with a Hilbert-Peano
filling curve~\citep{sagan2012space}.
%Throughout the article, the contributions of the new models are illustrated, after they have been formally introduced,
%by considering simulations in which we perform unsupervised 
%binary image segmentation ($\Omega=\{\omega_1,\omega_2\}$)
%using 
They are  blurred with a noise which exhibits non-linearities 
to highlight the ability of the generalized PMC models to 
learn such a signal corruption
% \footnote{The code to reproduce the experiments is available at \url{https://github.com/HGangloff/deep_hidden_markov_models/}}. More precisely, we generate an artificial noise by
generating $\obs_t$ according to~\eqref{eq:noise_eq1}, with
$a_{\omega_1}=0$, $\sigma^2=0.25$ and $a_{\omega_2}$ is a varying parameter 
(see Subsection~\ref{subsec:data_generation}). 
% \begin{equation}
%     \label{eq:noise_eq11}
%     \obs_t| \lab_{t},\obs_{t-1} \sim \mathcal{N}\Big(\sin(a_{\lab_t}+\obs_{t-1});
%     \sigma^2\Big),
% \end{equation}
 %sigma=0.5

%The parameters introduced in the all following examples on synthetic data are considered unknown to meet the case of unsupervised segmentation.
We next focus on two kinds of parameterizations of distributions $\vartheta$ and
$\zeta$ which coincide with \eqref{param-32}-\eqref{param-4}. Each
parameterization is applied to the SPMC and PMC models (see Figure~\ref{fig:pmc_graphs}). 
First, we consider a linear parameterization (SPMC and PMC) based
on \eqref{param-32}-\eqref{param-2_bis}. The second parameterization is a deep
one (DSPMC and DPMC) and relies on one (unfrozen) hidden layer with $100$
neurons and the ReLU activation function. For this architecture, we apply the
training constraints discussed in Paragraph \ref{sec:deeppmc}.

%\hugo{}{ Second} a deep \hugo{one}{parameterization} (\hugo{}{case of} DSMPCs and DPMCs) \hugo{still 
%based on ?-?}{with probability distributions based on}  \hugo{for the nature
%of distributions but on one (unfrozen) hidden layer 
%with  $100$ neurons and an ouput layer whose size coincides with 
%that of the input of the neural network 
%according to the constraint discussed in
%paragraph ? for the parameterization
%of these distributions}{ and a DNN parameterized with one (unfrozen) hidden layer 
%with $100$ neurons and ReLU activation function. The input and output layers of the DNN have the same size. }
%Moreover, our neural networks consist of one unfrozen
%hidden layer with $100$ neurons and one frozen
%hidden layer 

%\begin{scenario}[Non-linear correlated noise] The hidden image $\lab_{0:T}$ is the \emph{cattle}-type image of
%the Binary Shape Database. Each observation is simulated as
%\begin{equation}
%    X_t \sim \mathcal{N}\Big(\sin(a_{\lab_t}+\obs_{t-1});
%    \sigma^2\Big).
%    \label{eq:noise_eq11}
%\end{equation}
%where $a_{\omega_2}$ is a varying parameter and we set $a_{\omega_1}=0$ $\sigma^2=0.25$. %sigma=0.5
%\label{sce:pmc1}
%\end{scenario}

In Figure~\ref{fig:nonlin_corr_pmc_sce1_a}, we display the averaged error rates
for each model over all the selected images as a function of $a_{\omega_2}$.
Figure~\ref{fig:nonlin_corr_pmc_sce1_b} displays the results of the
classifications for a particular image of the database. As it can be observed,
although the same Gaussian distribution $\zeta$ is used both models, the general
PMC framework that we introduced leads to a great improvement of the elementary
HMC model. Next, the deep parameterized models (DPMC and DSPMC) are the most
accurate models and are able to capture the complexity by improving the results
of their non-deep counterpart. More importantly, note that the gain obtained
with our DPMC and DSPMC models does not require any further modeling effort in
the sense that they are a particular parameterization in our general framework.

%the experiment of Scenario~\ref{sce:pmc1} which consists in unsupervised segmentations with varying highly non-linear noise levels.
%It is clear, from Figure~\ref{fig:nonlin_corr_pmc_sce1} that the  models seem the most accurate models, able to capture the noise complexity. Each of the model improves the result of their non-deep counterpart.
%Importantly, as this defines one of the main motivations behind our work; 

\input{Figures/nonlin_corr_pmc_sce1}

%%%%%%%From now on, we will consider two classes of PMC models, in the first one we consider the linear and Gaussian PMC and SPMC models described above. The second one is the class of the Deep-PMC (DPMC) and the Deep-SPMC (DSPMC) models.






% \section{Triplet Markov Chain for unsupervised learning}

% Our previous PMC models rely on a general parameterization of the two distributions $\vartheta$ and $\zeta$. 
% However, the choice of these distributions  is not obvious in practice and has 
% an impact on the performance of the classification. 
% The goal of this section is to consider a third latent and continuous process  $\latent_{0:T}$ which
% aims at complexifying the distribution
% $\p(\lab_{0:T},\obs_{0:T})$.


% The rationale behind this auxiliary process is that even if $\p(\latent_{0:T})$ and 
% $\p(\obs_{0:T}|\latent_{0:T})$ are two elementary distributions, the resulting
% distribution $\p(\obs_{0:T})=\int \p(\latent_{0:T})\p(\obs_{0:T}|\latent_{0:T}) {\rm d} \latent_{0:T}$ 
% can be complex \citep{bayer2015learning}. 
% So in our context, the third latent process can be 
% used to implicitly estimate the nature of 
% the distributions $\vartheta$ and $\zeta$ of our PMC or
% to model and learn the continuous non-stationarity of  the process $(\lab_{0:T}, \obs_{0:T})$ 
% since $\p(\lab_{0:T},\obs_{0:T})=\int \p(\latent_{0:T})\p(\lab_{0:T},\obs_{0:T}|\latent_{0:T}) {\rm d}\latent_{0:T}$.



% \begin{eqnarray}
%     \label{tmc-theta-1}
%     \p(\latent_t|\triplet_{t-1}) &=& \eta \left(\latent_t; pz(\triplet_{t-1})\right) \text{,}\\
%     \label{tmc-theta-2}
%     \p(\lab_t| \latent_{t}, \triplet_{t-1})&=&\vartheta \left(\lab_t;ph(\latent_{t}, \triplet_{t-1})\right) \text{, } \\
%     \label{tmc-theta-3}
%     \p(\obs_t|\lab_t, \latent_{t},\triplet_{t-1})&=&\zeta\left(\obs_t;px(\lab_t, \latent_{t},\triplet_{t-1})\right) \text{.}
% \end{eqnarray}
% Remark that if  $pz$ does not depend on $\triplet_{t-1}$,
% and if $ph$ and $px$ are independent of $\latent_{t-1:k}$,
% the distribution $\p(\lab_T,\obs_T)$ coincides with that of a PMC built from \eqref{pmc-theta-1}-\eqref{pmc-theta-2}.


\section{TMCs for unsupervised classification}
\label{sec-tmc}
%-------------------------------
% Triplet Markov Chains
%-------------------------------
% \katy{
% * estimation problem is a particular case of Chap 3 where we have all the observations. So a part of the optimal variational distribution is computable (q(y|z,x))
% * but we introduce a modified ELBO to take into account the difference between z and y which are both unobserved.
% * pretraining step for TMC
% }

% \yohan{the interesting topics of this part : the ELBO is a particular case of
% the semi-supervsied case; a part of the optimal variational distribution can be
% computed exactly; the ELBO is modified in order to take into account the
% difference between z and y. We can approach it as we did before; we can adapt
% the pretraining procedure.}

% = 
% \int \p(\latent_{0:T})\p(\obs_{0:T}|\latent_{0:T}) {\rm d} \latent_{0:T}$.
% The rationale behind this auxiliary process
% is that even if $\p(\latent_{0:T})$ and 
% $\p(\obs_{0:T}|\latent_{0:T})$ are two elementary distributions, the resulting
% distribution $\p(\obs_{0:T})=\int \p(\latent_{0:T})\p(\obs_{0:T}|\latent_{0:T}) {\rm d} \latent_{0:T}$ 
% can be complex~\citep{bayer2015learning}. 

In this section, we extend the integration of a third latent process into our
PMC model. This third continuous latent process $\latent_{0:T}$ can be used to implicitly estimate 
the nature of the distributions $\vartheta$ and $\zeta$ of our PMC or to model and learn the
continuous non-stationarity of the process $(\lab_{0:T}, \obs_{0:T})$ since
$\p(\lab_{0:T},\obs_{0:T})=\int \p(\latent_{0:T}) \p(\lab_{0:T},\obs_{0:
T}|\latent_{0:T}) {\rm d}\latent_{0:T}$. 
However, this integration poses computational challenges 
because a direct computation of the integrals w.r.t.~$\latent_t$ in~\eqref{eq:alpha}
 and~\eqref{eq:beta} is intractable.
Consequently, the likelihood and posterior distributions, 
$\p(\obs_{0:T})$ and $\p(\lab_t|\obs_{0:T})$ are no longer exactly computable in
general. 
Here, we derive a new estimation algorithm based on 
VI (see Section~\ref{subsec:vbi}), 
where the ELBO is a particular case of the semi-supervised case presented in
Chapter~\ref{chap:semi_supervised_pmc_tmc}.
Moreover, a part of the variational distribution $\q$ can be computed
explicitly, which allows adjustments to be made in the model
learning phase. We also propose a modified version of the ELBO, 
which improves the interpretability of the labels by distinguishing them 
from the latent variables.





% A third latent process can be 
% used to implicitly estimate the nature of 
% the distributions $\vartheta$ and $\zeta$ of our PMC or
% to model and learn the continuous non-stationarity of  
% the process $(\lab_{0:T}, \obs_{0:T})$  since 
% $\p(\lab_{0:T},\obs_{0:T})=\int \p(\latent_{0:T})
% \p(\lab_{0:T},\obs_{0:T}|\latent_{0:T}) {\rm d}\latent_{0:T}$. 
% %implicitly estimate these distributions in addition to their parameters
% %by the introduction of a third latent
% %auxiliary process $\latent_{0:T}$ which aims 
% %at complexifying the distribution
% %$\p(\lab_{0:T},\obs_{0:T})$. 
% %Indeed, the rationale behind this auxiliary process is
% %the following~\citep{bayer2015learning}.
% %Assume that a r.v. $x \in \mathbb{R}$ follows an unknown distribution $p(\obs)$ while $z \in \mathbb{R}$
% %follows an elementary one $p(z)$ (\eg~ the Gaussian distribution).
% %Denoting 
% %$Q_X$ (resp. $Q_Z$) the cumulative density function of $\obs$ (resp. of $\latent$) and observing that the r.v. $Q_X(\obs)$ and $Q_Z(z)$ both follow the uniform distribution on the unit interval,
% %then the r.v. $Q_X^{-1}(Q_Z(z))$ admits $p(\obs)$ as pdf.
% %In other words, whatever the distribution of $\latent$, it is possible to model
% %an unknown distribution $p(\obs)$
% %via an auxiliary r.v. $\latent$ and a joint distribution
% %$p(x,z)=p(z)p(x|z)$, provided
% %$p(x|z)$ is well chosen and close
% %to $\delta_{Q_X^{-1}(Q_Z(z))}(\obs)$.
% %\textcolor{red}{Alors à la fin, pourquoi a-t-on astucieusement ajouté de la complexité ?!}
% %\textcolor{}{A second interpretation of this continuous $\latent_{0:T}$ process is that is offers a way to model and learn the continuous non-stationarity of the $(\lab_{0:T}, \obs_{0:T})$. Indeed, $\p(\lab_{0:T},\obs_{0:T},\latent_{0:T})=\p(\latent_{0:T})\p(\lab_{0:T},\obs_{0:T}|\latent_{0:T})$ and, if $\latent_{0:T}$ is continuous, the stationarity of $(\lab_{0:T},\obs_{0:T})$ can vary continuously.}
% However, the introduction of a continuous latent process $\latent_{0:T}$ is
% interesting from a modeling point of view but makes 
% Algorithm~\ref{algo:algo_theta_pmc}
% and \ref{algo:algo_hk_pmc} uncomputable. Indeed, a direct
% application would involve the computation of intractable integrals 
% in~\eqref{eq:alpha} and~\eqref{eq:beta} 
% w.r.t.~$\latent_t$. 
% Consequently, the likelihood and posterior distributions, 
% $\p(\obs_{0:T})$ and $\p(\lab_t|\obs_{0:T})$ are no longer exactly computable in
% general. In order to estimate $\theta$ and $\lab_t$, we derive a new estimation
% algorithm based on VI (see Section~\ref{subsec:vbi})
% % variational Bayesian inference
% which consists in maximizing a lower bound of the likelihood $\p(\obs_{0:T})$.
% % After reviewing the principle of variational inference and introducing its
% % extension to TMCs, 
% We propose a parameterization TMC models
% %(similarly to the general parameterization of PMCs \ref{sec:generalParam})
% as well as a parameter estimation algorithm. Our algorithm relies on the
% optimization of an objective function deduced from the VI
% framework, but it also enforces the interpretability of $\lab_{0:T}$ by modifying
% the classical lower bound used in VI.


% %In this section we reconsider the continuous
% %latent process $\latent_{0:T}$ associated to the TMC model \eqref{tmc}.
% %Note that when the r.v. $\latent_t$ are discrete, we obtain a particular
% %case the models described in \eqref{sec-pmc} where the hidden r.v. 
% %$(\lab_t,\latent_t)$ belong to an augmented discrete space. However, due
% %to the continuous nature of $\latent_t$,



% %In the previous section we parameterized the distributions we chose to characterize the joint distribution $p(\obs_{0:T}, \lab_{0:T})$. In this section, we want to incorporate a third continuous process $\latent_t$ and learn the distributions that characterize  the joint distribution $p(\obs_{0:T}, \lab_{0:T}) = \int p(\obs_{0:T}, \lab_{0:T}, \latent_{0:T}) \d \latent$. The Triplet Markov Chains (TMCs) allow us to incorporate this third process and generalize  the HMCs and the PMCs models. The TMCs offer similar advantages and superior modeling capabilities since  the assumptions  that $\lab_{0:T}$ or  $(\obs_{0:T}, \lab_{0:T})$ are Markov chains are not required. 

% %Let $\lab_{0:T}$ and $\obs_{0:T}$ be defined as before and let us introduce a sequence of auxiliary latent random variables $\latent_{0:T} = (\latent_1,\cdots,\latent_t)$, where $\latent_t \in \mathbb{R}^d$ for all $t$, $1 \leq t \leq T$. In the TMC model, $(\latent_{0:T},\lab_{0:T},\obs_{0:T})$ is a Markov chain thus
% %\begin{equation}
% %    p(\latent_{0:T},\lab_{0:T},\obs_{0:T})=p(\lab_1,\latent_1,\obs_1)\prod_{t=2}^Tp(\lab_t,\latent_t,\obs_t|\lab_{t-1},\latent_{t-1},\obs_{t-1}).
% %    \label{eq:tmc_general}
% %\end{equation}
% %where
% %\begin{align}
% %\label{eq:tmc_gen2}
% %p(\lab_t,\latent_t,\obs_t|\lab_{t-1},\latent_{t-1},\obs_{t-1}) =
% %p(\lab_t| \latent_{t}, \lab_{t-1},\obs_{t-1},\latent_{t-1} )p(\obs_t|\lab_t, \latent_{t},\lab_{t-1},\obs_{t-1},\latent_{t-1}) p(\latent_t|\lab_{t-1},\obs_{t-1}, \latent_{t-1}) \text{.}
% %\end{align}



\subsection{Variational Inference for general TMCs}
\label{sec:inference_tmc}

% \yohan{here we have a connection with the semi-supervised case where all the
% labels are unobserved. So you can already refer to the previous elbo and
% explaining that it as a simpler form and that the optimal variational
% distribution q(y|x,z) is computable}

In Chapter~\ref{chap:semi_supervised_pmc_tmc}, 
we have introduced a VI framework for the case where the labels are partially observed.
In this section, we consider the unsupervised case where all the labels are unobserved.
Thus, the ELBO is simpler than in the semi-supervised case~\eqref{eq:elbo_seq}, 
and a  part of the optimal variational distribution can be computed 
exactly (Proposition~\ref{prop:prop1}).

Let us recall the notation  $\triplet_t=(\lab_t,\latent_t,\obs_t)$  
for the triplet.
The TMC~\eqref{eq:tmc_intro}
%\eqref{tmc-theta-1}-\eqref{tmc-theta-3}
can be seen as a PMC~\eqref{pmc-theta-1}-\eqref{pmc-theta-2}
in augmented dimension,
\ie~a PMC where $(\latent_{0:T},\lab_{0:T})$ plays 
the role of the hidden process. 
If $\latent_{0:T}$ were a discrete process, it would be possible to apply 
directly the Bayesian
inference framework developed in Section 
\ref{sec:inference_pmc}.
However, the continuous nature of $\latent_t$ involves intractable 
integrals to compute sequentially the 
equivalent of~\eqref{eq:alpha},
% , \emph{i.e.},
% \begin{equation*}
% \p(\lab_t,\latent_t,\obs_t)=\int \sum_{\lab_{t-1}} \p(\triplet_t|\triplet_{t-1}) \p(\lab_{t-1},\latent_{t-1},\obs_{t-1}) {\rm d}\latent_{t-1} \text{,}
% \end{equation*}
and therefore $\p(\obs_{0:T})$.
To overcome this issue, we introduce a variational distribution 
$\q(\latent_{0:T},\lab_{0:T}|\obs_{0:T})$, 
% %  % ---- begin YOHAN
% %  Let $\q(\latent_{0:T},\lab_{0:T}|\obs_{0:T})$ be a \emph{variational} distribution  
% parameterized by a set of parameters $\phi$. Observing that the Kullback-Leibler Divergence (KLD)
%  between $\q(\latent_{0:T},\lab_{0:T}|\obs_{0:T})$ and $\p(\latent_{0:T},\lab_{0:T}|\obs_{0:T})$ is positive,
%  \begin{align*}
%  &\dkl\left(\q(\lab_{0:T}, \latent_{0:T}|\obs_{0:T}) || \p(\latent_{0:T},\lab_{0:T}|\obs_{0:T})\right)\nonumber \\
%  & \quad \quad \quad \quad =\sum_{\lab_{0:T}} \int \q(\latent_{0:T},\lab_{0:T}|\obs_{0:T}) \log\left(\frac{\q(\latent_{0:T},\lab_{0:T}|\obs_{0:T})}{\p(\latent_{0:T},\lab_{0:T}|\obs_{0:T})}\right)\ {\rm d}  \latent_{0:T} \geq 0 \text{,}
%  \end{align*}
 and  deduce the \gls*{elbo} of the TMC model for the unsupervised case:
 \begin{align*}
 \log \p(\obs_{0:T}) & \geq \sum_{\lab_{0:T}} 
 \int \q(\latent_{0:T},\lab_{0:T}|\obs_{0:T})  
 \log\left(\frac{\p(\latent_{0:T},\lab_{0:T},\obs_{0:T}) }
 {\q(\latent_{0:T},\lab_{0:T}|\obs_{0:T})}\right)  {\rm d}\latent_{0:T} 
%\\ &=  \E_{q_{\phi}(\latent_{0:T},\lab_{0:T}|\obs_{0:T})}\big[\log p_{\theta}(\obs_{0:T}|\latent_{0:T},\lab_{0:T})\big] - \dkl(q_{\phi}(\latent_{0:T},\lab_{0:T}|\obs_{0:T})||p_{\theta}(\latent_{0:T},\lab_{0:T})) 
\\ & \; \; = \Qunsup(\theta,\phi) \text{.} 
 \end{align*}

%  \yohan{Next paragraph already said in previous chapters check if it is necessary
%  }
% Equality holds if $\q(\latent_{0:T},\lab_{0:T}|\obs_{0:T})=\p(\lab_{0:T},
% \latent_{0:T}|\obs_{0:T})$. When the posterior distribution $\p(\lab_{0:T},
% \latent_{0:T}|\obs_{0:T})$ is computable, the alternating maximization w.r.t.
% $\theta$ and $\q$ of the ELBO
% $\Qunsup(\theta,\phi)$, coincides with the EM
% algorithm~\citep{variational-EM}. However, here, the posterior distribution
%  is not computable in general because the latent variables $\latent_{0:T}$ are
% continuous. Thus, we maximize $\Qunsup(\theta,\phi)$ w.r.t.~$\theta$ and $\phi$
%  for a given class of
% distributions $\q$. The choice of the variational distribution $\q(\lab_{0:T},
% \latent_{0:T}|\obs_{0:T})$ is critical as we stated in Sections~\ref{subsec:vbi}
% and~\ref{sub:pmc_parameter_estimation}.
% % The variational distribution $\q$ should be close to $\p$ 
% % but should also be chosen in a such way that the associated ELBO can be exactly computed or
% easily approximated while remaining differentiable w.r.t.~$(\theta,\phi)$. 
In the context of TMCs with a discrete and continuous latent process, 
Proposition~\ref{prop:prop1} exploits the observation that

\begin{equation}
\label{backward-decomposition}
\p(\lab_{0:T}|\latent_{0:T},\obs_{0:T})=\p(\lab_{T}|\latent_{0:T},\obs_{0:T})
 \prod_{t=1}^T \p(\lab_{t-1}|\lab_t,\latent_{0:T},\obs_{0:T})
\end{equation}
is computable (see Appendix~\ref{app:prop1}) and 
shows that it is optimal
(in the sense of the value of the ELBO) to
restrict the choice of $\q(\latent_{0:T},\lab_{0:T}|\obs_{0:T})$ 
to that of $\q(\latent_{0:T}|\obs_{0:T})$. 
% In the case where $\lab_{0:T}$ is discrete, Proposition \ref{prop:prop1} shows
% that a part of the variational distribution can be optimized exactly.
%can be computed
%from a direct adaptation of \eqref{eq:pair_post_margin}.

%The resulting ELBO is then optimal.
% ---- end YOHAN
%The idea of VI is to compute a variational distribution $\q(\latent_{0:T},\lab_{0:T}|\obs_{0:T})$
%as close as possible of $\p(\latent_{0:T},\lab_{0:T}|\obs_{0:T})$ . 
%To that end, a measure of dissimilarity between distributions  $\q(\latent_{0:T},\lab_{0:T}|\obs_{0:T})$  and  $\p(\latent_{0:T},\lab_{0:T}|\obs_{0:T})$ called Kullback-Leibler  divergence ($\dkl$) which is defined as 
%\begin{equation}
%\dkl(\q(\latent_{0:T}, \lab_{0:T}|\obs_{0:T}) || %\p(\latent_{0:T},\lab_{0:T}|\obs_{0:T}))=\sum_{\lab_{0:T}} \int %q(\latent_{0:T},\lab_{0:T}|\obs_{0:T}) %\log\left(\frac{\q(\latent_{0:T},\lab_{0:T}|\obs_{0:T})}{\p(\%z,\lab_{0:T}|\obs_{0:T})}\right)\ {\rm d}  \latent_{0:T} %\geq 0 
%\text{,}
%\end{equation}
%where $\dkl(\q(\latent_{0:T}, \lab_{0:T}|\obs_{0:T}) || \p(\latent_{0:T},\lab_{0:T}|\obs_{0:T})) = 0$ if and only if $\q(\latent_{0:T},\lab_{0:T}|\obs_{0:T})=\p(\latent_{0:T},\lab_{0:T}|\obs_{0:T})$, otherwise it is positive.  Then, we try to minimize the $\dkl$ with respect to  $\q(\latent_{0:T},\lab_{0:T}|\obs_{0:T})$, for a given family of distributions $\q(\latent_{0:T},\lab_{0:T}|\obs_{0:T})$. This dissimilarity can be rewritten as
%\begin{align*}
%\dkl(\q(\latent_{0:T}, \lab_{0:T}|\obs_{0:T}) || \p(\latent_{0:T},\lab_{0:T}|\obs_{0:T})) &= - \sum_{\lab_{0:T}} \int q(\latent_{0:T},\lab_{0:T}|\obs_{0:T}) \log\left(\frac{\p(\latent_{0:T},\lab_{0:T},\obs_{0:T})}{\q(\latent_{0:T},\lab_{0:T}|\obs_{0:T})}\right) {\rm d}  \latent_{0:T} + \log \p(\obs_{0:T})\\
%&=  - \Qunsup(\theta,\varphi) + \log \p(\obs_{0:T}) \text{,}
%\end{align*}
%where $\Qunsup(\theta,\varphi)$ is called the Evidence Lower Bound (ELBO) and it also satisfies $\log \p(\obs_{0:T}) \geq \Qunsup(\theta,\varphi)$.   This optimization problem requires $\p(\obs_{0:T})$ which is intractable, however, it is independent of $\log \p(\obs_{0:T})$ which means that minimization of the $\dkl$ and maximization of the ELBO are equivalent from an optimization perspective. On other words,  this approach  aims at maximizing $\Qunsup(\theta,\varphi)$ w.r.t.~$(\theta,\varphi)$ for a given parametric family of distributions $\q(\latent_{0:T},\lab_{0:T}|\obs_{0:T})$~\citep{blei2017variational} [Variational EM].


\begin{proposition}
\label{prop:prop1}

Let us denote $\Qunsup(\theta,\phi)$ and $\Qunsup^{\rm opt}(\theta,\phi)$,
the ELBOs associated to the variational 
distributions
$$\q(\latent_{0:T},\lab_{0:T}|\obs_{0:T})=\q(\latent_{0:T}|\obs_{0:T})\q(\lab_{0:T}|\latent_{0:T},\obs_{0:T})$$
and 
$$\q^{\rm opt}(\latent_{0:T},\lab_{0:T}|\obs_{0:T})=\q(\latent_{0:T}|\obs_{0:T})\p(\lab_{0:T}|\latent_{0:T},\obs_{0:T})\text{,}$$
respectively. \\
Then, for any $(\theta,\phi)$, we have 
\begin{equation}
\label{prop1_ineq}
\log \p(\obs_{0:T}) \geq \Qunsup^{\rm opt}(\theta,\phi) \geq \Qunsup(\theta,\phi),
\end{equation}
where
\begin{align}
\label{elbo-opt}
\Qunsup^{\rm opt}(\theta,\phi)=Q_0^{\rm opt}(\theta,\phi)+\sum_{t=1}^T Q_{t-1,t}^{\rm opt}(\theta,\phi)+Q_{0:T}^{\rm opt}(\theta,\phi) \text{,}
\end{align}
and where
\begin{align}
\label{elbo-1}
Q_0^{\rm opt}(\theta,\phi)&=\int  \sum_{\lab_{0}}\q(\latent_{0:T}|\obs_{0:T})p_{\theta}(\lab_0|\latent_{0:T},\obs_{0:T}) \log\p(\triplet_0) {\rm d}  \latent_{0:T} \text{,} \\
%Q_{t-1,t}^{\rm opt}(\theta,\phi)\!\!&=\!\!\!\!
%\int\!\!\!\! \sum_{\lab_{t-1:t}}\!\!\!\log\left(\frac{\eta(\latent_t; \pz(\triplet_{t-1})) \vartheta(\lab_t;\pyun(\latent_{t},\triplet_{t-1}))  \zeta(\obs_t;\pxun(\lab_t, \latent_{t},\triplet_{t-1}))  }{p_{\theta}(\lab_{t-1}|\lab_{t}, \latent_{0:T}, \obs_{0:T}) \q(\latent_{0:T}|\obs_{0:T})\hugo{}{\text{here or}}}\right)\! p_{\theta}(\lab_{t-1:t}|\latent_{0:T},\obs_{0:T}) \q(\latent_{0:T}|\obs_{0:T}){\rm d}  \latent_{0:T} \text{,} \\
Q_{t-1,t}^{\rm opt}(\theta,\phi)&=
\int  \sum_{\lab_{t-1:t}}\q(\latent_{0:T}|\obs_{0:T})p_{\theta}(\lab_{t-1:t}|\latent_{0:T},\obs_{0:T}) \times \nonumber \\
 & \quad \log\left(\frac{\p(\triplet_t|\triplet_{t-1})}{p_{\theta}(\lab_{t-1}|\lab_{t}, \latent_{0:T}, \obs_{0:T}) \q(\latent_{0:T}|\obs_{0:T})}\right) {\rm d}  \latent_{0:T} \text{,} \\
\label{elbo-3}
Q_{0:T}^{\rm opt}(\theta,\phi)&= - \int
    \sum_{\lab_{T}}\q(\latent_{0:T}|\obs_{0:T}) \p(\lab_{T}|\latent_{0:T},\obs_{0:T}) \log\p(\lab_{T}|\latent_{0:T},\obs_t){\rm d}  \latent_{0:T}.
\end{align}
%\begin{align}
%\label{elbo-opt}
%    \Qunsup^{\rm opt}(\theta,\phi) = & \sum_{t=1}^T \int \sum_{\lab_{t-1},\lab_t}p_{\theta}(\lab_t,\lab_{t-1}|\latent_{0:T},\obs_{0:T}) \q(\latent_{0:T}|\obs_{0:T})\log(\zetat(\obs_t)) {\rm d}  \latent_{0:T} \nonumber \\ 
%    &- \sum_{t=1}^T \int \sum_{\lab_{t-1},\lab_t}p_{\theta}(\lab_t,\lab_{t-1}|\latent_{0:T},\obs_{0:T}) \q(\latent_{0:T}|\obs_{0:T})\log\left(\frac{\varthetat(\lab_t)\etat(\latent_t)}{p_{\theta}(\lab_{t-1}|\lab_{t}, \latent_{0:T}, \obs_{0:T})\q(\latent_{0:T}|\obs_{0:T})}\right) {\rm d}  \latent_{0:T} \text{;}
%\end{align}
%and $\varthetat(\lab_t)$, $\etat(\latent_t)$, $\zetat(\obs_t)$ are short abbreviations of the functions  given in   \eqref{tmc-theta-1}-\eqref{tmc-theta-3}.
% \begin{align}
% \label{elbo-opt-1}
% \E_{\q^{\rm opt}(\latent_{0:T},\lab_{0:T}|\obs_{0:T})}\big[\log p_{\theta}(\obs_{0:T}|\latent_{0:T},\lab_{0:T}) \big] & = XXXX    \text{,} \\
% \label{elbo-opt-2}
% \dkl(\q^{\rm opt}(\latent_{0:T},\lab_{0:T}|\obs_{0:T})||p_{\theta}(\latent_{0:T},\lab_{0:T})) &  = XXXX
% \end{align}
\end{proposition}
A proof of Proposition~\ref{prop:prop1} is given in Appendix~\ref{app:prop1}. 
%Note that all the terms 
%in the integrals are computable
%(see Appendix \ref{app:prop1} for the
%computation of $\p(\lab_{t-1:t}|\obs_{0:T},\latent_{0:T})$
%and so of $\p(\lab_{t-1}|\lab_t,\obs_{0:T},\latent_{0:T})$).
%and relies on the backward decomposition 
%\begin{equation}
%\label{backward-decomposition}
%\p(\lab_{0:T}|\obs_{0:T},\latent_{0:T})=\p(\lab_{T}|\obs_{0:T},\latent_{0:T})
% \prod_{t=1}^T \p(\lab_{t-1}|\lab_t,\obs_{0:T},\latent_{0:T}),
%\end{equation}
%where each term $\p(\lab_{t-1}|\lab_t,\obs_{0:T},\latent_{0:T})$
%is computable. 
The practical computation of these integrals
will be described later with the modified objective function.
%Note that all the terms in \eqref{elbo-1}-\eqref{elbo-3} are known\hugo{but the practical computation of the integrals
%will be described later}{, we will describe later the practical computations of the integrals}.



\subsection{Estimation algorithm for TMCs}
Following the approach that we have developed for PMC models, 
we extend our parameterization framework to the 
distributions of TMC models. 
As a direct extension of Section~\ref{sec:generalParam}, functions $\pyun$ and
$\pxun$ can now depend on $\latent_{t-1:t}$. 
We present a particular parameterization of TMCs models derived from the general one
introduced in Section~\ref{sec:general_param_tmc}.
The transition distribution is factorized as follows:
\begin{equation}
  \label{tmc-trans}
  %\p(\latent_t,\lab_t,\obs_t|\latent_{t-1},\lab_{t-1},\obs_{t-1})=\p(\latent_t|\latent_{t-1},\lab_{t-1},\obs_{t-1})\p(\lab_t|\latent_t,\latent_{t-1},\lab_{t-1},\obs_{t-1})\p(\obs_t|\lab_t,\latent_t,\latent_{t-1},\lab_{t-1},\obs_{t-1}) 
    \p(\triplet_t|\triplet_{t-1})=\p(\latent_t|\triplet_{t-1})\p(\lab_t|\latent_t,\triplet_{t-1})\p(\obs_t|\lab_t,\latent_t,\triplet_{t-1})
    \text{.}
\end{equation} 
Thus, the parameterization of the TMC model
given by Equations~\eqref{eq:general_model_tmc}-\eqref{eq:py}
is now given by
\begin{eqnarray}
  \label{tmc-theta-1}
  \p(\latent_t|\triplet_{t-1}) &=& \eta \left(\latent_t; \pz(\triplet_{t-1})\right) \text{,}\\
  \label{tmc-theta-2}
  \p(\lab_t| \latent_{t},\triplet_{t-1})&=&\vartheta \left(\lab_t;\pyun(\latent_{t},\triplet_{t-1})\right) \text{, } \\
  \label{tmc-theta-3}
  \p(\obs_t|\lab_t, \latent_{t},\triplet_{t-1})&=&\zeta\left(\obs_t;\pxun(\lab_t, \latent_{t},\triplet_{t-1})\right) \text{.}
\end{eqnarray}
%Again, no stationary assumption needs to be made in the framework we derive \cite{}
% We next propose a general estimation method 
% based on a variational distribution 
% $\q(\latent_{0:T}|\obs_{0:T})$, for estimating $\theta$ and $\p(\lab_t|\obs_{0:T})$, for all $t$, which takes into account the interpretability constraint.
% %'Yohan'
% %we first introduce a general parameterization of our TMC models; next we derive a variational Bayesian inference algorithm for the estimation of their parameters and for the estimation of $\lab_t$ from $\obs_{0:T}$.
% %'Yohan'
% %Our aim is to present a  general parameterization of TMCs models as we did for PMCs models (Section \ref{sec:generalParam}). Similar to PMCs,  we consider the functions $\pyun(\cdot)$, $\pxun(\cdot)$ and $\pz(\cdot)$ which depend on $\theta$ and are differentiable w.r.t.~$\theta$. Thus, the transition distribution of the TMC is parameterized through $\pyun$, $\pxun$ and $\pz$ as follows:
% \subsubsection{General Parameterization of TMCs}
%
\begin{remark}
  If  $\pz$ does not depend on $\triplet_{t-1}$,
  and if $\pyun$ and $\pxun$ are independent of $\latent_{t-1:t}$,
  the distribution $\p(\lab_{0:T},\obs_{0:T})$ coincides with that of a 
  PMC built from~\eqref{pmc-theta-1}-~\eqref{pmc-theta-2}.
\end{remark}

%where $\zeta(\obs;v')$ and $\eta(z;\hat{v})$ (resp. $\vartheta(\lab;v)$) are given classes of  probability density functions on $\mathbb{R}$ (resp. a probability distribution on $\Omega$) and differentiable w.r.t.~$v$ and $\hat{v}$ (resp. $v'$).



%Let $f_{\theta,h}(\lab_{t-1},\obs_{t-1})$ and $f_{\theta,x}(\lab_t,\lab_{t-1},\obs_{t-1})$ be functions parameterized by an unknown parameter  $\theta$. We assume that these functions are differentiable w.r.t.~$\theta$.  [...]

%, a generalization from the literature on TMCs where only discrete auxiliary processes have been considered, see for example~\cite{lanchantin2011unsupervised}. For comparison, in Section~\ref{sec:tmc_models}, we will consider TMCs with discrete random variables. In such case, we will have $\latent_t\in\Lambda=\{\vartheta_1,\cdots,\vartheta_L\},\forall k.$ \\

%\begin{remark}[Why being interested by continuous auxiliary random variables ?]
%A note not to forget to say how it generalizes. Do we talk about some kind of implicit distribution set up ? (see for example https://arxiv.org/pdf/1702.08235.pdf or https://arxiv.org/pdf/1805.11183.pdf)
%\end{remark}









% \textbf{Remark:} Note that another more straightforward approach for inference could be used. Using the estimated variational distribution we are able to estimate an approximate posterior marginal $q(\lab_t|\obs_{0:T}),\forall k,$ with
% \begin{align}
%     q(\lab_t|\obs_{0:T})&=\int_{\latent_{0:T}}\d\latent_{0:T} p_{\theta}(\lab_t|\latent_{0:T},\obs_{0:T})q_{\phi}(\latent_{0:T}|\obs_{0:T}),\\
%     &\approx\frac{1}{N}\sum_{m=1}^M
%     p(\lab_t|\latent_{0:T}^n,\obs_{0:T}),
% \end{align}
% where we sample N i.i.d trajectories $z^n\sim q_{\phi}(\latent_{0:T}|\obs_{0:T}),\forall n\in\{1,\dots,N\}$.


%------------------------
% Volver a leer para poner en un lugar correcto
%Note that, to the best of our knowledge, only the di-MTMC model (\emph{\emph{i.e.}} when $\latent_{0:T}$ is a discrete auxiliary random variable) described below has already been introduced in the literature under slightly different forms. 
%One can cite for example~\cite{lanchantin2004unsupervised}, \cite{lapuyadelahorgue2010unsupervised} or \cite{boudaren2017unsupervised}. In these studies, the models offer the possibility to introduce non-stationarities in the parameters of the hidden and observed processes when $\latent_{0:T}$ is discrete. In our article, we generalize this case by allowing $\latent_{0:T}$ to be a continuous random variable and by allowing transition functions to be parameterized by neural networks.
%indeed, $\p(\lab_{t-1:t}|\latent_{0:T},\obs_{0:T})$ can be computed from 
%a direct adaptation of \eqref{eq:pair_post_margin}
%while the other terms coincide to the model or the variational distribution. The practical computation 
%of these integrals will described later.


%\textcolor{orange}{Écriture de $\Qunsup^{\rm opt}$ sous forme d'une XEnt et DKL qui généralise les VAEs de manière séquentielle et les beta-VAEs quand on aura rajouté les Beta dès le paragraphe suivant}
%\begin{align}
%    \Qunsup^{\rm opt}(\theta,\phi) = & \sum_{\lab_{0:T}} \int \p(\lab_{0:T}|\latent_{0:T},\obs_{0:T})\q(\latent_{0:T}|\obs_{0:T})\log(\prod_{t=1}^T\zetat(\obs_t)) {\rm d}  \latent_{0:T} \nonumber \\ 
%    &- \sum_{\lab_{0:T}} \int \p(\lab_{0:T}|\latent_{0:T},\obs_{0:T})\q(\latent_{0:T}|\obs_{0:T})\log\left(\frac{\prod_{t=1}^T\varthetat(\lab_t)\etat(\latent_t)}{\q(\latent_{0:T},\lab_{0:T}|\obs_{0:T})}\right) {\rm d}  \latent_{0:T}, \\
%    &= \mathrm{XEnt}(\p(\lab_{0:T}|\latent_{0:T},\obs_{0:T})\q(\latent_{0:T}|\obs_{0:T}),\prod_{t=1}^T \p(\obs_t|\obs_{t-1},\latent_{0:T},\lab_{0:T})) \\
%    &- \mathrm{DKL}(\p(\lab_{0:T}|\latent_{0:T},\obs_{0:T})\q(\latent_{0:T}|\obs_{0:T}) || \prod_{t=1}^T \p(\lab_t,\latent_t|\lab_{t-1},\latent_{t-1},\obs_{0:T})),\nonumber\\
%    &= \mathrm{XEnt}(\p(\lab_{0:T}|\latent_{0:T},\obs_{0:T})\q(\latent_{0:T}|\obs_{0:T}); \p(\obs_{0:T}|\latent_{0:T},\lab_{0:T})) - \mathrm{DKL}(\p(\lab_{0:T}|\latent_{0:T},\obs_{0:T})\q(\latent_{0:T}|\obs_{0:T}) || \p(\latent_{0:T},\lab_{0:T}|\obs_{0:T})).
%\end{align}
%\textcolor{orange}{
%Peut-être que les trois lignes au dessus sont à ajouter à la démo en Appendixe et la dernière forme est la forme à mettre dans la proposition 1 ?}\\

%\textcolor{orange}{
%Si on introduit $\tilde{p}(\latent_{0:T},\lab_{0:T}|\obs_{0:T})=\p(\lab_{0:T}|\latent_{0:T},\obs_{0:T})\q(\latent_{0:T}|\obs_{0:T})$ on a, dans $\Qunsup^{\rm opt}$ de l'équation au dessus, une expression totalement symmétrique en $\lab$ et $\z$ ce qui pourrait bien justifier l'introduction des $\beta$ qui suit juste après, pour l'interprétabilité.
%On pourrait également garder la forme séquentielle qui a peut-être plus de sens ?! :
%}
%\begin{align}
%    \Qunsup^{\rm opt}(\theta,\phi)
%    &= \mathrm{XEnt}(\prod_{t=1}^T\tilde{\p}(\lab_t,\latent_t|\lab_{t-1},\latent_{t-1},\obs_{0:T}),\prod_{t=1}^T \p(\obs_t|\obs_{t-1},\latent_{0:T},\lab_{0:T})) \\
%    &- \mathrm{DKL}(\prod_{t=1}^T\tilde{\p}(\lab_t,\latent_t|\lab_{t-1},\latent_{t-1},\obs_{0:T}) || \prod_{t=1}^T \p(\lab_t,\latent_t|\lab_{t-1},\latent_{t-1},\obs_{0:T})),\nonumber.
%\end{align}


%In the variational inference framework, we use a lower bound of the log model evidence  ($\log \p(\obs_{0:T}) $), which is called \emph{Evidence Lower Bound} (ELBO)~\citep{blei2017variational} and is denoted  as $\Qunsup(\theta,\varphi)$,%, and  whatever $\q$, we deduce that 
%\begin{align}
%\log \p(\obs_{0:T}) &\geq  \Qunsup(\theta,\varphi) \\
%&= \sum_{\lab_{0:T}} \int \q(\latent_{0:T},\lab_{0:T}|\obs_{0:T})  \log\left(\frac{\p(\latent_{0:T},\lab_{0:T},\obs_{0:T}) }{\q(\latent_{0:T},\lab_{0:T}|\obs_{0:T})}\right)  {\rm d}\latent_{0:T} \\
%&=  \E_{\latent_{0:T},\lab_{0:T}\sim q_{\phi}(\latent_{0:T},\lab_{0:T}|\obs_{0:T})}[
%    \log p_{\theta}(\obs_{0:T}|\latent_{0:T},\lab_{0:T})] 
%    - \dkl(q_{\phi}(\latent_{0:T},\lab_{0:T}|\obs_{0:T})||p_{\theta}(\latent_{0:T},\lab_{0:T}))\text{.} 
%\end{align}
\subsubsection{Joint estimation of $\theta$ and $\phi$}
\label{sec:joint_estimation}
Classical variational inference algorithms
aim at maximizing the ELBO~\eqref{elbo-opt}
when the objective is to estimate the parameters of a generative model, 
\ie~a model in which we do not focus on the interpretability of the hidden r.v. 
but rather on the modeling power of the distribution $\p(\obs_{0:T})$.
% ~\ref{sec:variational_autoencoder})
Consequently, in our case,  a direct maximization of \eqref{elbo-opt}
does not guarantee the
interpretability of the r.v. $\lab_{0:T}$. 
The problem is all the more critical that our hidden process is split
into an interpretable one, $\lab_{0:T}$, and an
auxiliary one, $\latent_{0:T}$.
To that end, we propose an adaptation and an interpretation to the sequential case of two 
techniques introduced in the machine learning 
community~\citep{higgins2017beta, kingma2014semi}. The first one relies on a reinterpretation of the ELBO \eqref{elbo-opt} as the sum of a 
reconstruction and a KLD terms; this last one is next penalized. 
The second technique consists in adding a penalizing term to the resulting ELBO which aims at
strengthening the distinct role of $\lab_{0:T}$ and of $\latent_{0:T}$ and exploiting the result of previous classifications obtained with an available
model.
%such that o discussed in Section \ref{sec-pmc}\hugo{}{[ça pourrait être n'importe quel autre modèle qui fournit une presegmentation ?]}.

%In this subsection, we present a parameter estimation procedure which solves the interpretability issue of $\lab_t$. Thus,  we will optimize a modified and generalized   version of the ELBO
%$\Qunsup^{\rm opt}(\theta,\phi)$ based on introducing $\beta_1 \geq 0$ and $\beta_2 \geq 0$,

%2Algorithm~\ref{algo:tmc_elbo_opt} proposes an approach based on a gradient ascent method of the MC approximation of



% \begin{align}
%     \mathcal{L}(\theta,\phi)
%     &=\E_{q^{\rm opt}_{\phi}(\latent_{0:T},\lab_{0:T}|\obs_{0:T})}[
%     \log p_{\theta}(\obs_{0:T}|\latent_{0:T},\lab_{0:T})]
%     - \beta_1 {\dkl(q^{\rm opt}_{\phi}(\latent_{0:T},\lab_{0:T}|\obs_{0:T})||p_{\theta}(\latent_{0:T},\lab_{0:T}))}.
%     \label{eq:beta_elbo}
% \end{align}

% %The idea of variational methods is to cast inference as an optimization problem, then, we maximize the loss function of Equation~\eqref{eq:beta_elbo} w.r.t $\theta$ and $\phi$.
% %In practice, the expectations of the loss function are approximated with a Sequential Monte Carlo method~\citep{doucet2009tutorial}. 
% %We can write $q_{\phi}(\latent_{0:T},\lab_{0:T}|\obs_{0:T}) = q_{\phi}(\latent_{0:T}|\obs_{0:T}) q_{\phi}(\lab_{0:T}|\latent_{0:T},\obs_{0:T})$ and thus we can sample $(\latent_{0:T},\lab_{0:T})\sim q_{\phi}(\latent_{0:T},\lab_{0:T}|\obs_{0:T})$ in two steps. First, we sample  $\latent_{0:T} \sim q_{\phi}(\latent_{0:T}|\obs_{0:T})$,  followed by $\lab_{0:T}\sim q_{\phi}(\lab_{0:T}|\latent_{0:T},\obs_{0:T})$.
% %If we  set $q_{\phi}(\lab_{0:T}|\latent_{0:T},\obs_{0:T})=p_{\theta}(\lab_{0:T}|\latent_{0:T},\obs_{0:T})$, this distribution is analytically computed from parameters upon which we optimize with the Forward-Backward algorithm (Section~\ref{sec:inference_tmc}).
% %This procedure can be seen as a generalized EM algorithm where the Expectation (E) step is not performed exactly~\citep{neal1998view} and the maximization (M) step is performed using gradients~\citep{balakrishnan2017statistical}.
% %Algorithm~\ref{algo:tmc_elbo_opt} sketches the procedure for the ELBO maximization with the Gradient EM algorithm.% Note that the derivative of the loss function is computed with the autodifferentiation tool JAX~\citep{jax2018github}.
% %\begin{remark}Because of the particular model structure, the exact (E) step could be performed for $\theta$, \emph{\emph{ie}} setting $q_{\phi}(\lab_{0:T}|\latent_{0:T},\obs_{0:T})=p_{\theta^{{j}}}(\lab_{0:T}|\latent_{0:T},\obs_{0:T})$ is possible. However the latter choice did not give good convergence results in practice.
% %We hypothesize that such performance issues are due to alternating exact (E) step for some parameters and approximate (E) step for others. \end{remark}
% \subsection{Training procedures for the Deep TMC models}
% \label{sec:deeptmc}
% Just like the PMC models, the TMC models can be extended with DNNs to relax the parameterization assumption of the probabilistic distributions: $\pyun$ and $\pxun$ are parameterized with DNNs to form Deep TMCs (DTMCs). Moreover, 
% since a variational approximation is considered in general TMCs, a DNN is also used to model the parameters of the distribution $\q$. % It is in this framework that this section details the specific training procedures and constraints to solve the training and interpretability issue in the DTMC models. This section is then the counterpart of Section~\ref{sec:deeppmc} which deals with the same related issues in Deep PMCs.
%\subsubsection{Penalizing the loss function}


\subsubsection{The $\beta$-ELBO}
We first start with an alternative decomposition of the ELBO~\eqref{elbo-opt}.

\begin{corollary}
\label{corollary1}
Let us factorize 
$$\p(\latent_{0:T},\lab_{0:T},\obs_{0:T})=\overline{p}_{\theta}(\latent_{0:T},\lab_{0:T}|\obs_{0:T})\tilde{p}_{\theta}(\obs_{0:T}|\latent_{0:T},\lab_{0:T})$$
with
\begin{align}
\label{tilde-p}
\tilde{p}_{\theta}(\obs_{0:T}|\latent_{0:T},\lab_{0:T})&=\p(\obs_0|\lab_{0},\latent_0)\prod_{t=1}^T \zeta \left(\obs_t;\pxun(\lab_t, \latent_{t},\triplet_{t-1})\right) \text{,} \\
\label{barre-p}
\overline{p}_{\theta}(\latent_{0:T},\lab_{0:T}|\obs_{0:T})&=\p(\lab_0,\latent_0)\prod_{t=1}^T 
\eta\left(\latent_t; \pz(\triplet_{t-1})\right) \vartheta \left(\lab_t;\pyun(\latent_{t},\triplet_{t-1})\right).
\end{align}

Then 
\begin{equation}
\label{elbo-opt-2}
    \Qunsup^{\rm opt}(\theta,\phi) = \mathcal{L}_1(\theta,\phi) +  \mathcal{L}_2(\theta,\phi) \text{,}
\end{equation}    
where 
\begin{align}
\label{L-1}
\mathcal{L}_1(\theta,\phi) &={\E}_{\q^{\rm opt}(\latent_{0:T},\lab_{0:T}|\obs_{0:T})}\left( \log \tilde{p}_{\theta}(\obs_{0:T}|\latent_{0:T},\lab_{0:T})\right)\text{,} \\
\label{L-2}
 \mathcal{L}_2(\theta,\phi) &= - \dkl \left(\q^{\rm opt}(\latent_{0:T},\lab_{0:T}|\obs_{0:T})|| \overline{p}_{\theta}(\latent_{0:T},\lab_{0:T}|\obs_{0:T})\right)  \text{.}
\end{align}
\end{corollary}

This decomposition can be seen as
a generalization to the sequential case of the decomposition proposed
for the $\beta$-VAE in~\citep{higgins2017beta}. Indeed, $\Qunsup^{\rm opt}$ involves
the sum of $(i)$ a reconstruction term $\mathcal{L}_1$ between 
$\q^{\rm opt}$ and $\tilde{p}_{\theta}$ 
which measures the ability 
to reconstruct observations $\obs_{0:T}$
according to the conditional likelihood
$\tilde{p}_{\theta}$
from the latent r.v. $(\latent_{0:T},\lab_{0:T})$ 
distributed according to $\q^{\rm opt}$;
$(ii)$ a KLD term $\mathcal{L}_2$
between the variational distribution and the conditional
prior $\overline{p}_{\theta}$.
However, contrary to the static
case, our decomposition involves 
$\tilde{p}_{\theta}(\obs_{0:T}|\latent_{0:T},\lab_{0:T})$ 
and $\overline{p}_{\theta}(\latent_{0:T},\lab_{0:T}|\obs_{0:T})$
rather than $\p(\obs_{0:T}|\latent_{0:T},\lab_{0:T})$  
and $\p(\latent_{0:T},\lab_{0:T})$, respectively. 
Indeed, except if $T=0$, the latter two distributions are no longer computable, 
which makes the classical ELBO decomposition impractical.
%in which case the distributions coincide.

The idea underlying our $\beta$-ELBO is to penalize the KLD term $\mathcal{L}_2(\theta,\phi)$. To understand why, let us detail 
the expression of $\mathcal{L}_1(\theta,\phi)$ and of $\mathcal{L}_2(\theta,\phi)$. First,
using \eqref{tilde-p} and \eqref{tmc-theta-3},
$\mathcal{L}_1(\theta,\phi)$ reads 
\begin{align}
\label{L-1_decomposed}
\mathcal{L}_1(\theta,\phi)=& {\E}_{\q^{\rm opt}(\lab_0,\latent_0|\obs_{0:T})}\left(\log \p(\obs_t|\lab_0, \latent_0) \right) +
\nonumber\\ & \quad  
\sum_{t=1}^T {\E}_{\q^{\rm opt}(\lab_t,\latent_t|\lab_{t-1},\latent_{0:t-1},\obs_{0:T})}\left(\log
%\underbrace{\zeta(\obs_t;\pxun(\lab_t, \latent_{t},\triplet_{t-1}))}_
{\p(\obs_t|\lab_t,\latent_t,\triplet_{t-1})}
\right).
\end{align}
Following this decomposition, it can be seen that at each time step $t$,
the maximization of \eqref{L-1_decomposed} encourages the model to interpret
the latent r.v. $(\lab_t,\latent_t)$ as those which explain the best the observation
$\obs_t$ given the past.
%From Eq.~\eqref{L-1}, it can be seen that this term encourages latent r.v. which explains the best the observed r.v., $\obs_{0:T}$. Now, with the decomposition of Eq.~\eqref{L-1_decomposed}, we see that the latent r.v. are in fact chosen so that, at each time step $t$, they explain the best $\obs_t$, while also taking into account the previous time steps.
On the other hand, using \eqref{barre-p} and
\eqref{tmc-theta-1}-\eqref{tmc-theta-2},
the maximization of  %$\mathcal{L}_2(\theta,\phi)$ reads
\begin{align}
\label{L-2_decomposed}
\mathcal{L}_2(\theta,\phi)=-&
\dkl \left(\q^{\rm opt}(\lab_0,\latent_0|\obs_{0:T})|| \p(\latent_0,\lab_0) %\p(\lab_t|\latent_{0})
\right)
- 
\nonumber\\& \quad 
\sum_{t=1}^T \dkl \left(\q^{\rm opt}(\lab_t,\latent_t|\lab_{t-1},\latent_{0:t-1},\obs_{0:T})|| 
\p(\lab_t,\latent_t|\triplet_{t-1})
%\eta(\latent_t; \pz(\triplet_{t-1})) \vartheta(\lab_t;\pyun(\latent_{t},\triplet_{t-1}))
\right)
\end{align}
tends to push the posterior variational distribution at each time step
to be close to the conditional prior distribution $\p(\lab_t,\latent_t|\triplet_{t-1})$.
%Similarly, Eq.~\eqref{L-2} can be interpreted as forcing the variational distribution on the latent to be close to a prior distribution. Then, from Eq.~\eqref{L-2_decomposed}, it is clear that such an constraint to match a prior is made sequentially: for each time step $t$, the variational distribution is forced to match a prior, taking into account all the preceding r.v.
As in~\citep{higgins2017beta}, we penalize $\mathcal{L}_2(\theta,\phi)$ via the introduction of a scalar $\beta_1$. Since a part of the latent r.v. has
to be interpretable, and that the interpretability of hidden r.v.
is not conditioned by the observations, the interest of this term is to
force the posterior distribution $\q^{\rm opt}$ to take into account the prior term at each time step.
In other words, this penalization term
aims at limiting the impact of the observations 
on the interpretability of the hidden r.v.,
particularly in problems where $\obs_t$ is a very noisy version of $\lab_t$.

%Such a weight in the ELBO has been introduced in~ and is now widely used, giving rise to the $\beta$-ELBO. It enables to tune how much the variational distribution should stick to its prior. Enforcing a stronger prior can a valuable asset in context where the noise is high. Indeed, with a high $\beta_1$, we encourages latent r.v. to match incorporate prior knowledge and we give less importance to latent r.v. explaining well the observations, which might lead to poor results at high noise levels. 

\subsubsection{Cross-entropy penalization}
%It remains to distinguish the role $\latent_t$ and $\lab_{0:T}$
%The previous step adapted from the $\beta$-VAE aims at 
%pushing the interpretability of the latent r.v. However, in our case, only $\lab_t$ needs to be interpretable.
We finally complete our objective function 
to guide the estimation process
into distinguishing the role of $\lab_{0:T}$ and of $\latent_{0:T}$ 
in order to obtain better interpretable estimations of $\lab_t$.
We assume that we have at our disposal
a pre-classification $\lab_{0:T}^{\pre}$. 
%obtained
%from one of the models described in Section \ref{sec-pmc}\hugo{}{[à nouveau pourquoi seulement les modèles là pourraient fournir une presegmentation ?]}.
Next, introduce
the KLD between the empirical distribution 
deduced from this pre-classification,
$p^{\rm emp}(\lab_{0:T})=\delta_{\lab_{0:T}^{\rm pre}}(\lab_{0:T})$, 
and the marginal variational distribution 
$$\q(\lab_{0:T}|\obs_{0:T})=\int \q^{\rm opt}(\latent_{0:T},\lab_{0:T}|\obs_{0:T}){\rm d}\latent_{0:T},$$
which aims itself at approximating the true posterior distribution $\p(\lab_{0:T}|\obs_{0:T})$. Thus, the objective is to push the variational distribution $\q$ to take into account the interpretable labels obtained from an already interpretable pre-classification through the negative cross-entropy 
\begin{equation}
\label{L-3}
 \mathcal{L}_3(\theta,\phi) = \E_{p ^{\rm emp}(\lab_{0:T})} \left(\log \q(\lab_{0:T}|\obs_{0:T}) \right)=\log\q(\lab^{\rm pre}_{0:T}|\obs_{0:T}),
\end{equation}
%${\rm E}_{p(\hat{\lab}_{0:T})}(\log(\q(\lab_{0:T}|\obs_{0:T})))$  
see for example~\citep{kingma2014semi,klys2018learning,kumar2021learning}. This additional term
is next penalized by a scalar $\beta_2$ which controls the proximity 
of the pre-classification with the variational posterior distribution.

%we define another term that will be part of the final loss function.
%It consists of a penalizing cross-entropy term between the empirical distribution of available labelled data (for example presegmentated data), $p(\hat{\lab_{0:T}})$, and the posterior distribution of $\lab_{0:T}$, $p_{\theta}({\lab_{0:T}}|\obs_{0:T})$, which is defined as $\mathcal{H}(p(\hat{\lab_{0:T}});p_{\theta}(\lab_{0:T}|\obs_{0:T}))= -\E_{p(\hat{\lab_{0:T}})}\Big[\log p_{\theta}(\lab_{0:T}|\obs_{0:T}) \Big]$. Intuitively, the control of this term aims at XXXXX à justifier
%A scalar $\beta_2$ is also introduced to control how much we need to stick to the empirical distribution. It is a hyperparameter that we manually adjust.
%Adding such terms in the loss function can be found in several works related to self-supervised learning or disentanglement learning

Finally, we obtain a new objective function 
%Summarizing the two previous constraints, the
%generalized version of the ELBO $\Qunsup^{\rm opt}(\theta,\phi)$ reads [A REECRIRE, \textcolor{orange}{dépend de la forme choisie pour Fopt}]
%modified loss function for the TMC models in the rest of the article is defined by
\begin{align}
\label{eq:beta_elbo_xent}
    \mathcal{L}(\theta,\phi)
    &= \mathcal{L}_1(\theta,\phi) + \beta_1 \mathcal{L}_2(\theta,\phi) + \beta_2 \mathcal{L}_3(\theta,\phi) \text{,}
    \end{align}
where  
$\mathcal{L}_1(\theta,\phi)$, $\mathcal{L}_2(\theta,\phi)$ and
$\mathcal{L}_3(\theta,\phi)$
are defined in \eqref{L-1}, \eqref{L-2}
and \eqref{L-3}, respectively.
%\begin{align}  
%\label{L-1}
%\mathcal{L}_1(\theta,\phi) &= \E_{\p(\lab_{0:T}|\latent_{0:T},\obs_{0:T})\q(\latent_{0:T}|\obs_{0:T})}(
%    \log \tilde{p}_{\theta}(\obs_{0:T}|\latent_{0:T},\lab_{0:T}))\text{,} \\
%\mathcal{L}_2(\theta,\phi)& = - {\dkl(\p(\lab_{0:T}|\latent_{0:T},\obs_{0:T})\q(\latent_{0:T}|\obs_{0:T})||\overline{p}_{\theta}(\latent_{0:T},\lab_{0:T}|\obs_{0:T}))} \text{,} \\
%\label{L-3}
% \mathcal{L}_3(\theta,\phi) &= \E_{p ^{\rm emp}(\lab_{0:T})} %(\log(\q(\lab_{0:T}|\obs_{0:T})))=\log(\q(\h^{\rm pre}_{0:T}|\obs_{0:T})) \text{.}
%\end{align}
If we set $\beta_1=1$ and $\beta_2=0$, then
$\mathcal{L}(\theta,\phi)$ coincides with
the ELBO $\Qunsup^{\rm opt}(\theta,\phi)$ in \eqref{elbo-opt-2}.

\subsubsection{Monte Carlo approximation}
It remains to compute and optimize~\eqref{eq:beta_elbo_xent} in practice.
$\mathcal{L}_1(\theta,\phi)$
and $\mathcal{L}_2(\theta,\phi)$ coincide
with mathematical expectations according to  $\q^{\rm opt}(\latent_{0:T},\lab_{0:T}|\obs_{0:T})=\q(\latent_{0:T}|\obs_{0:T})\p(\lab_{0:T}|\latent_{0:T},\obs_{0:T})$.
Using  expressions \eqref{L-1_decomposed}-\eqref{L-2_decomposed},
expectations according to $\p(\lab_{0:T}|\obs_{0:T},\latent_{0:T})$
are exactly computable.
Thus, $\mathcal{L}_1(\theta,\phi)$ and $\mathcal{L}_2(\theta,\phi)$
rely on the approximate computation of expectations according to $\q(\latent_{0:T}|\obs_{0:T})$. 
It can be also noted that $$\q(\lab_{0:T}|\obs_{0:T})
%=\E_{\q(\latent_{0:T}|\obs_{0:T})}(\q(\lab_{0:T}|\obs_{0:T},\latent_{0:T}))
=\E_{\q(\latent_{0:T}|\obs_{0:T})}(\p(\lab_{0:T}|\latent_{0:T},\obs_{0:T})),$$  
then  $\mathcal{L}_3(\theta,\phi)$ 
also relies on an expectation according to 
same distribution $\q(\latent_{0:T}|\obs_{0:T})$ as $\mathcal{L}_1(\theta,\phi)$ and $\mathcal{L}_2(\theta,\phi)$.
Consequently, Monte Carlo estimates based on \iid samples 
$\latent_{0:T}^{(m)} \sim \q(\latent_{0:T}|\obs_{0:T})$ are  estimates 
of $\mathcal{L}_1(\theta,\phi)$, $\mathcal{L}_2(\theta,\phi)$
and $\mathcal{L}_3(\theta,\phi)$.
The choice of the variational distribution is given by the following
factorization  $\q(\latent_{0:T}|\obs_{0:T})=\q(\latent_0|\obs_{0:T})\prod_{t=1}^T \q(\latent_t|\latent_{0:t-1},\obs_{0:T})$. 
% First, we set conditional distributions $\q(\latent_t|\latent_{0:t-1},\obs_{0:T})$ in
% order to obtain samples according to $\q(\latent_{0:T}|\obs_{0:T})$ sequentially.
Next, $\q(\latent_t|\latent_{0:t-1},\obs_{0:T})$ is chosen such that it is
possible to use the reparameterization trick to have a final sample 
$\latent_{0:T}^{(m)}$, which as a
differentiable function of $\phi$.
(see Subsection~\ref{subsec:reparameterization_trick}).
% . More precisely, the final sample $\latent_{0:T}^{(m)}$
% can be written as
% As we previously mentioned in ,
% \begin{equation}
% \label{reparameterization}
% \latent_{0:T}^{(m)} =\pi \left(\phi, {\pmb \epsilon}_{T}^{(m)}\right) \text{,}
% \end{equation}
% where $\pmb{\epsilon}_{T}^{(m)}$
% is a sequence of random samples which does not depend
% on $\phi$ and $\pi$ is a differentiable function w.r.t.~$\phi$.
% As an illustrative example, a sample $z^{(m)}$ according to Gaussian distribution with mean $\phi_1$ and standard deviation $\phi_2$
% can be reparameterized as a differentiable
% function of $(\phi_1,\phi_2)$ via $z^{(m)}=\phi_1+\phi_2 {\epsilon}^{(m)}$,
% where ${\epsilon}^{(m)} \sim \mathcal{N}(0,1)$. This sampling technique is
% referred to as the \emph{reparameterization trick}~\citep{kingma2014}.
Finally, we obtain the following estimate
of $\mathcal{L}(\theta,\phi)$ in \eqref{eq:beta_elbo_xent} 
given by
\begin{equation}
\label{L-approx}
\widehat{\mathcal{L}}(\theta,\phi)=\widehat{\mathcal{L}}_1(\theta,\phi)+\widehat{\mathcal{L}}_2(\theta,\phi)+\widehat{\mathcal{L}}_3(\theta,\phi) \text{,}
\end{equation}
where
%$\mathcal{L}_1(\theta,\phi)$, $\mathcal{L}_2(\theta,\phi)$ and
%$\mathcal{L}_3(\theta,\phi)$ in
%\eqref{L-1}-\eqref{L-3},
\begin{align}
\label{L-1-approx}
\widehat{\mathcal{L}}_1(\theta,\phi)&=\frac{1}{N} \sum_{m=1}^M {\E}_{\p(\lab_{0:T}|\latent_{0:T}^{(m)},\obs_{0:T})} \left( \log \tilde{p}_{\theta}(\obs_{0:T}|\latent_{0:T},\lab_{0:T}^{(m)}) \right) \text{,} \\ 
\label{L-2-approx}
\widehat{\mathcal{L}}_2(\theta,\phi)&=\frac{1}{N} \sum_{m=1}^M {\E}_{\p(\lab_{0:T}|\latent_{0:T}^{(m)},\obs_{0:T})} \left(\log\left( \frac{\overline{p}_{\theta}(\latent_{0:T},\lab_{0:T}^{(m)}|\obs_{0:T})}{\p(\lab_{0:T}|\latent_{0:T}^{(m)},\obs_{0:T})\q(\latent_{0:T}^{(m)}|\obs_{0:T})} \right) \right)  \text{,} \\
\label{L-3-approx}
\widehat{\mathcal{L}}_3(\theta,\phi)&= 
\log \left(\frac{1}{N} \sum_{m=1}^M \p(h^{\rm pre}_{0:T}|\latent_{0:T}^{(m)},\obs_{0:T}) \prod_{t=1}^T \p(\lab_{t-1}^{\rm pre}|\lab_t^{\rm pre},\latent_{0:T}^{(m)},\obs_{0:T})\right) \text{,}
%\frac{1}{ K} 
%    \sum_{t=1}^T \sum_{c=1}^C \delta_{\hat{\lab}_t}^{\omega_c}\log q_{\phi}(\lab_t=\omega_c|\x).
\end{align}
where the remaining expectations are
computed from 
\eqref{backward-decomposition} and from \eqref{tilde-p}-\eqref{barre-p}
and where samples $\latent_{0:T}^{(m)}$ satisfy the reparameterization concept.
The complete estimation algorithm 
is described in Algorithm~\ref{algo:tmc_elbo_opt}.


\begin{algorithm}[htbp!]
  \caption{Parameter estimation  in general  TMCs.}
  \label{algo:tmc_elbo_opt}
\begin{algorithmic}[1]
  \Require{$\obs_{0:T}$, the data; $\epsilon$, the learning rate; $M$ the number of samples}
  \Ensure{$(\theta^*, \phi^*)$, sets of estimated parameters}
  \State Initialize the parameters $\theta^0$ and $\phi^0$
  \State $j\leftarrow 0$\label{line:start_dtmc}
  \While{\text{convergence is not attained}}
    \State Sample $\latent_0^{(m)}\sim q_{\phi^{{j}}}(\latent_0|\obs_{0:T})$,  for all  $1 \leq m \leq M$ 
    \State Sample $\latent_t^{(m)}\sim q_{\phi^{{j}}}(\latent_t|\latent_{0:t-1}^{(m)},\obs_{0:T})$,   for all  $1 \leq m \leq M$, for all $1 \leq t \leq T$ 
    \State{Compute $p_{\theta}(\lab_{t-1}|\lab_t,\latent_{0:T}^{(m)},\obs_{0:T})$, for all $\lab_{t-1:t} \in \Omega \times \Omega$,  for all  $1 \leq m \leq M$, for all $1 \leq t \leq T$}
    \State Evaluate the loss $\widehat{\mathcal{L}}(\theta^{{j}},\phi^{{j}})$ from \eqref{L-approx}-\eqref{L-3-approx}
    \State{Compute the derivative of the loss function $\nabla_{(\theta, \phi)} \widehat{\mathcal{L}}(\theta,\phi)$  from \eqref{L-approx}-\eqref{L-3-approx} }
    \State Update the parameters with gradient ascent
  \begin{equation}
  \begin{pmatrix}\theta^{(j+1)}\\\phi^{(j+1)}\end{pmatrix}=
  \begin{pmatrix}\theta^{{j}}\\\phi^{{j}}\end{pmatrix}
  + \epsilon {\nabla_{(\theta, \phi)} \widehat{\mathcal{L}}(\theta,\phi)}\Big|_{(\theta^{{j}},\phi^{{j}})}
  \label{eq:elbo_grad}
  \end{equation}
  \State  $j\leftarrow j+1$
  \EndWhile
  \State  $\theta^{*} \leftarrow \theta^{{j}}$
  \State  $\phi^{*} \leftarrow \phi^{{j}}$
  \label{line:end_dtmc}
\end{algorithmic}
  % \vspace*{0.2cm}
\end{algorithm}


\subsubsection{Estimation of $\lab_t$}
\label{estimation-h-tmc}
%In this part we consider that $\theta^*$ and $\varphi^*$ are known,  \emph{ie} they are estimated or given. Then  it remains to compute the following expression, 
% Once an estimate of  $(\theta^*,\varphi^*)$ has
% been computed, it remains to compute
Once we have obtained an estimate ${\theta^*}$
of $\theta$, we focus on the computation
of $p_{\theta^*}(\lab_t|\obs_{0:T})$, 
\begin{equation}
\label{posterior-tmc}
p_{\theta^* }(\lab_t|\obs_{0:T})=\int_{\latent_{0:T}} p_{\theta^*}(\lab_t|\latent_{0:T},\obs_{0:T})p_{\theta^*}(\latent_{0:T}|\obs_{0:T}) \d\latent_{0:T} \text{,}
\end{equation}
where $p_{\theta^*}(\lab_t|\latent_{0:T},\obs_{0:T})$ 
is computable from a direct extension 
of \eqref{eq:alpha} and \eqref{eq:beta}-\eqref{eq:post_margin}
(see the proof of Proposition~\ref{prop:prop1}).
%by replacing the products $\vartheta(\lab_t;\pyun) \times \zeta(\obs_t;\pxun)$ by $\vartheta(\lab_t;\pyun) \times \zeta(\obs_t;\pxun) \times \eta( \latent_t; \pz)$.
Since \eqref{posterior-tmc} is intractable, we propose an MC estimate $\hat{p}_{\theta }(\lab_t|\obs_{0:T})$ deduced from the sequential importance resampling mechanism~\citep{livredoucetshort}
and based on the observation
that $p_{\theta^*}(\latent_{0:T}|\obs_{0:T}) \propto
p_{\theta^*}(\obs_{0:T},\latent_{0:T})$ is
known up to a constant. Indeed, $p_{\theta^*}(\obs_{0:T},\latent_{0:T})$ 
can also be computed from a direct extension of \eqref{likelihood-pmc}-\eqref{eq:alpha}.
We thus introduce the estimated variational distribution
$$\qs(\latent_{0:T}|\obs_{0:T})=
\qs(\latent_0|\obs_{0:T})\prod_{t=1}^T \qs(\latent_t|\latent_{0:t-1},\obs_{0:T})$$
as importance distribution due to its proximity with $\p(\latent_{0:T}|\obs_{0:T})$.
Finally, rewriting \eqref{posterior-tmc} as 
\begin{equation}
p_{\theta^*}(\lab_t|\obs_{0:T})=\frac{\E_{q_{\phi^*}(\latent_{0:T}|\obs_{0:T})}\left(
    \frac{p_{\theta^*}(\lab_t|\latent_{0:T},\obs_{0:T})p_{\theta^*}(\latent_{0:T},\obs_{0:T})}{q_{\phi^*}(\latent_{0:T}|\obs_{0:T})}\right)}{\E_{q_{\phi^*}(\latent_{0:T}|\obs_{0:T})}\left(\frac{p_{\theta^*}(\obs_{0:T})}{q_{\phi^*}(\latent_{0:T}|\obs_{0:T})}\right)} \text{,}
\end{equation}
%Note that $\p(\lab_t,\lab_{t-1}|\latent_{0:T},\obs_{0:T})$ %in \eqref{}-\eqref{} 
%can be computed from a direct extension of \eqref{eq:alpha}-\eqref{eq:beta} in which the products 
%$\vartheta \times \zeta $ are replaced 
%by $\vartheta \times \zeta \times \eta$, given in \eqref{tmc-theta-1}-\eqref{tmc-theta-3}. From this expression we deduce $\p(\lab_t|\latent_{0:T},\obs_{0:T})=\sum_{\lab_{t-1}} \p(\lab_{t-1},\lab_t|\latent_{0:T},\obs_{0:T})$.  
we compute the sequential MC sampler~\citep{doucet2009tutorial} presented
in Algorithm~\ref{algo:tmc_inf} consisting of the sequential application of three elementary steps (sampling, weighting and resampling). Note that any improvement of this sequential MC algorithm can be used~\citep{Fearnhead-smoothing}.
%[Doucet].

%the distribution $p_{\theta}(\lab_{0:T}|\latent_{0:T},\obs_{0:T})$ is analytically computable using the Forward-Backward algorithm, as in the PMCs, where $(\latent_{0:T},\obs_{0:T})$ is treated as the observations. Indeed, we similarly set  $\alpha_{\theta,t}(\lab_t)=\p(\latent_1,\obs_1, ,\cdots,\latent_t,\obs_t,\lab_t)$ 
%and ~$\beta_{\theta,t}(\lab_t)= \p(\latent_{k+1},\obs_{k+1},\cdots,\latent_{T},\obs_{0:T}|\lab_t,\latent_t,\obs_t)$, for all $t$, $1 \leq t \leq T$. Based on the distribution \eqref{tmc-trans} and on the parameterization \eqref{tmc-theta-1}- \eqref{tmc-theta-3}, the coefficients $\alpha_{\theta,t}(\lab_t)$ and $\beta_{\theta,t}(\lab_t)$  can be computed recursively as in the PMCs,
%\begin{align}
%\alpha_{\theta,t}(\lab_t)= \sum_{\lab_{t-1}} & \alpha_{\theta,t-1}(\lab_{t-1}) \eta( \latent_t; \pz(\lab_{t-1},\obs_{t-1}, \latent_{t-1})) \vartheta(\lab_t;\pyun(\latent_{t}, \lab_{t-1},\obs_{t-1},\latent_{t-1})) \nonumber\\ & \times\zeta(\obs_t;\pxun(\lab_t, \latent_{t},\lab_{t-1},\obs_{t-1},\latent_{t-1}))
%\label{eq:alpha_triplet}\\
%\beta_{\theta,t-1}(\lab_{t-1})=  \sum_{\lab_{t}} & \beta_{\theta,t}(\lab_{t})\eta( \latent_t; \pz(\lab_{t-1},\obs_{t-1}, \latent_{t-1})) \vartheta(\lab_t;\pyun(\latent_{t}, \lab_{t-1},\obs_{t-1},\latent_{t-1}))\nonumber\\ & \times \zeta(\obs_t;\pxun(\lab_t, \latent_{t},\lab_{t-1},\obs_{t-1},\latent_{t-1}))
%\label{eq:beta_triplet}
%\end{align}
%and we finally deduce
%\begin{align}
%\p(\lab_{t-1},\lab_t|\latent_{0:T},\obs_{0:T}) \propto &
%\; \alpha_{\theta,t-1}(\lab_{t-1})  \beta_{\theta,t}(\lab_t)  \eta( \latent_t; \pz(\lab_{t-1},\obs_{t-1}, \latent_{t-1})) \vartheta(\lab_t;\pyun(\latent_{t}, \lab_{t-1},\obs_{t-1},\latent_{t-1})) \nonumber\\ & \times\zeta(\obs_t;\pxun(\lab_t, \latent_{t},\lab_{t-1},\obs_{t-1},\latent_{t-1})) \text{,}
%\label{eq:pair_marginal_\lab_triplet}\\
%\p(\lab_t|\latent_{0:T},\obs_{0:T})= &\sum_{\lab_{t-1}} \p(\lab_{t-1},\lab_t|\latent_{0:T},\obs_{0:T})
%\label{eq:marginal_\lab_triplet}\text{.}
%\end{align}

%The distribution $p_{\theta}(\lab_t|\obs_{0:T})$ can be computed from $p_{\theta}(\lab_t|\obs_{0:T})=\int_{\latent_{0:T}} p_{\theta}(\lab_t|\latent_{0:T},\obs_{0:T})p_{\theta}(\latent_{0:T}|\obs_{0:T}) \d\latent_{0:T}$. However, an estimate of $p_{\theta}(\latent_{0:T}|\obs_{0:T})$ is required. For this purpose we introduce a variational distribution $q_{\phi}(\latent_{0:T}|\obs_{0:T})$, then $p_{\theta}(\lab_t|\obs_{0:T})$ reads

%\begin{align}
 %   p_{\theta}(\lab_t|\obs_{0:T})&=\int_{\latent_{0:T}} p_{\theta}(\lab_t|\latent_{0:T},\obs_{0:T})p_{\theta}(\latent_{0:T}|\obs_{0:T}) \d\latent_{0:T}, \nonumber\\
  %  &=\frac{\E_{\latent_{0:T}\sim q_{\phi}(\latent_{0:T}|\obs_{0:T})}\left[
 %   \frac{p_{\theta}(\lab_t|\latent_{0:T},\obs_{0:T})p_{\theta}(\latent_{0:T},\obs_{0:T})}{q_{\phi}(\latent_{0:T}|\obs_{0:T})}\right]}{\E_{\latent_{0:T}\sim q_{\phi}(\latent_{0:T}|\obs_{0:T})}\left[
 %   \frac{p_{\theta}(\obs_{0:T})}{q_{\phi}(\latent_{0:T}|\obs_{0:T})}\right]}\nonumber
    %\\
    %&\propto  \E_{\latent_{0:T}\sim q_{\phi}(\latent_{0:T}|\obs_{0:T})}\left[ \frac{p_{\theta}(\lab_t|\latent_{0:T},\obs_{0:T})p_{\theta}(\latent_{0:T},\obs_{0:T})}{q_{\phi}(\latent_{0:T}|\obs_{0:T})}\right]
%\end{align}

%We will use a  SMC approach to sample $z^{(i)}$ from  $q_{\phi}(\latent_{0:T}|\obs_{0:T})$. Thus, we can define the associated  weights $w^{(i)}\propto\frac{p_{\theta}(\latent_{0:T}^{(i)},\obs_{0:T})}{q_{\phi}(\latent_{0:T}^{(i)}|\obs_{0:T})}$.    Note that $p_{\theta}(\lab_t|\latent_{0:T},\obs_{0:T})$ can be computed using the Equation~\eqref{eq:marginal_\lab_triplet}.  We  can have the following expression 

%\begin{align}
%\label{eq:p\lab_\obs_tmc}
%    p_{\theta}(\lab_t|\obs_{0:T})&\approx\sum_{i=1}^M p_{\theta}(\lab_t|\latent_{0:T}^{(i)},\obs_{0:T})w^{(i)}.
% \end{align}

%In particular, we assume that the variational distribution factorizes as $q_{\phi}(\latent_{0:T}|\obs_{0:T}) = q_{\phi}(\latent_1| \obs_{0:T})  \prod_{t=2}^T q_{\phi}(\latent_t|\latent_{1:t-1} \obs_{0:T})$.  The associated weights can be calculated sequentially, for all $k\in\{1,\dots,T\}$ and $i\in\{1,\dots,N\}$, according  to

%\begin{align}
 %   w^{(i)}_t&\propto\frac{p_{\theta}(\latent_{1:t}^{(i)},\obs_{1:t})}{q_{\phi}(\latent_{1:t}|\obs_{0:T})},\nonumber\\
 %   &\propto w^{(i)}_{t-1}\frac{p_{\theta}(\latent_t^{(i)},\obs_t|\latent_{1:t-1}^{(i)},\obs_{1:t-1})}
 %   {q_{\phi}(\latent_t^{(i)}|\latent_{1:t-1}^{(i)},\obs_{0:T})},\nonumber\\
 %   &\propto w^{(i)}_{t-1}\frac{p_{\theta}(\latent_{1:t}^{(i)},\obs_{1:t})}
 %   {p_{\theta}(\latent_{1:t-1}^{(i)},\obs_{1:t-1})q_{\phi}(\latent_t^{(i)}|\latent_{1:t-1}^{(i)},\obs_{0:T})}.
 %   \label{eq:weight_tpm}
%\end{align}

%Then, we draw $M$ samples $\latent_t^{(i)}$ from the variational distribution $q_{\phi}(\latent_t|\latent_{1:t-1} \obs_{0:T})$, for all $t$,  by using  the reparameterization trick~\citep{kingma2013auto}.
%Additionally, in the last equation the likelihood with augmented states appears  which is  exactly computable by using Equation~\eqref{eq:alpha_triplet}, we have  $p_{\theta}(\latent_{1:t}^{(i)},\obs_{1:t})=\sum_{\lab_t}\alpha_{\theta,t}(\lab_t)$, for all $t$. Thus we can estimate $p_{\theta}(\lab_t|\obs_{0:T})$ given in Equation \eqref{eq:p\lab_\obs_tmc}, for all $t$.\\


\begin{algorithm}[htbp!]
  \caption{A Sequential Monte Carlo algorithm for Bayesian classification in general TMC.}
  \label{algo:tmc_inf}
\begin{algorithmic}[1]
  \Require{$\obs_{0:T}$, the observation; a set of parameters $(\theta^*, \phi^*)$; $M$, the number of samples}
  \Ensure{$\widehat{{\lab}_{0:T}}$ the final classification}
  \State  Sample $\latent_0^{(m)}\sim q_{\phi^*}(\latent_0|\obs_{0:T})$, 
  \State Compute $w_0^{(m)} = \frac{p_{\theta^*}(\latent_0^{(m)}, \obs_0)}{q_{\phi^*}(\latent_0|\obs_{0:T})}$ 
  $W_0^{(m)} = w_0^{(m)} / \sum_{m=1}^M w_0^{(m)} $, for all $1 \leq m \leq M$
  \For{$t \gets 1$ to $T$}
  \State Sample $\latent_t^{(m)}\sim q_{\phi^*}(\latent_{0:t}|\latent_{0:t-1},\obs_{0:T})$, for all  $1 \leq m \leq M$ 
  \State Compute
  \begin{equation*}
    w_t^{(m)}=w_{t-1}^{(m)} \frac{p_{\theta^*}(\latent_{t}^{(m)},\obs_{0:t})}
    {p_{\theta^*}(\latent_{0:t-1}^{(m)},\obs_{t-1})q_{\phi^*}(\latent_t^{(m)}|\latent_{0:t-1}^{(m)},\obs_{0:T})}\text{, for all}  1 \leq m \leq M  
  \end{equation*}
  \State Compute $W_t^{(m)}=w_{t}^{(m)}/ \sum_{m=1}^M w_{t}^{(m)}$, for all  $1 \leq m \leq M$ 
    \If{Resampling}
      \State Sample $l^{(m)} \sim p(l=j)=W_t^{(j)}$, for all  $1 \leq m \leq M$ 
      \State Set $\latent_{0:t}^{(m)}=\latent_{0:t}^{(l^{(m)})}$ and $W_t^{(m)}=1/M$  for all  $1 \leq m \leq M$
    \EndIf
    \EndFor
  \State Compute $p_{\theta^*}(\lab_{t-1:t}|\latent_{0:T}^{(m)},\obs_{0:T})$, for all $\lab_{t-1:t} \in \Omega \times \Omega$, for all $1\leq t \leq T$, using the extension of  \eqref{eq:pair_post_margin}
    %Compute $p_{\theta^*}(\lab_t|\latent_{0:T}^{(m)},\obs_{0:T})=\sum_{\lab_{t-1}} p_{\theta^*}(\lab_{t-1},\lab_t|\latent_{0:T}^{(m)},\obs_{0:T})$, $\forall i$
  \State Compute $\hat{p}_{\theta^*}(\lab_t|\obs_{0:T}) =\sum_{m=1}^M W_t^{(m)} p_{\theta^*}(\lab_t|\latent_{0:T}^{(m)},\obs_{0:T})$, for all $\lab_t \in \Omega$, for all $1\leq t \leq T$
  \State $\hat{\lab}_t= \argmax \hat{p}_{\theta^*}(\lab_t|\obs_{0:T})$,  for all $1\leq t \leq T$
\end{algorithmic}
  % \vspace*{0.2cm}
\end{algorithm}


\subsection{Deep TMCs for unsupervised classification}
\label{sec-deep-tmc}
Let us now focus on Deep TMCs for unsupervised classification.
We adapt the two-step procedure described in Section \ref{sec:deeppmc}. The main
difference with Section \ref{sec:deeppmc} is that the input of our DNN can now depend on the latent r.v. $\latent_t$; in addition, 
due to the VI framework that we have proposed in the previous section, we also consider that the conditional variational distribution $\q(\latent_t|\latent_{0:t-1},\obs_{0:T})$ at the
core of our estimation algorithm is parameterized by a DNN.


\subsubsection{Constrained ouput layer}
\label{constraint-tmc}
The first step is a direct adaptation of Section \ref{sec:constrained_archi} and relies on the preliminary estimation of a non-deep TMC model.
More precisely,  Algorithm~\ref{algo:tmc_inf}
is applied to estimate the parameter of a linear TMC model
(\ie~a TMC which is a direct extension of \eqref{param-1_bis}-\eqref{param-2_bis} or equivalently a deep TMC model with no hidden layer). Note that since $\latent_{0:T}$ does not need to 
be interpretable, $\q(\latent_t|\latent_{0:t-1},\obs_{0:T})$ are
already parameterized by a DNN in the linear TMC models. Next, the DNNs, which parameterize
$\pz$, $\pyun$ and $\pxun$, are built according to the same scheme of Figure~\ref{fig:constrained_archi}, except that the input and the hidden layer before the output also consists of $\latent_{t-1}$ or of $\latent_{t-1:t}$.
We thus obtain a set of frozen and unfrozen parameters.

\subsubsection{Pretraining of the unfrozen parameters}
\label{tmc-unfrozen}
The next step consists in pretraining
the unfrozen parameters of the intermediate hidden layers in order to mimic the
estimated linear TMC.
%and those
%of the variational distribution $\q$.
We use the same approach as the one developed in Section
\ref{sec:pretraining_backprop}
which relies on a pre-classification $\hat{\lab}_{0:T}^{\rm pre}$, but we now take into account the fact
that $\latent_t$ is not observed.
Since the objective of the r.v. $\latent_t$ is 
to encode the corresponding observation $\obs_t$ through the DNN related to $\q$,
we first sample $\latent_{0:T}$ according to the 
previously estimated variational distribution
$\q(\latent_{0:T}|\obs_{0:T})$;
%with the reparameterization trick of Section \ref{sec:joint_estimation}; 
we next use the components $\latent_{t-1:t}$ or $\latent_t$ 
%(which are functions of $\phi$)
as inputs of the DNNs $\pz$, $\pyun$ and $\pxun$.
Finally, as in Paragraph \ref{sec:pretraining_backprop}, we apply 
the backpropagation algorithm in order
to minimize an adapted cost function w.r.t.
%$(\theta_{\rm ufr},\phi)$
$\theta_{\rm ufr}$
which depends on $\hat{\lab}_{0:T}^{\rm pre}$.
%Foolowing our approach for Deep PMCs, we propose to embed DNNs in the TMC models to define the Deep TMC models. In the Deep TMC architectures, $\pyun$ and $\pxun$ are parameterized by a DNN. Since the latter can theoretically approximate any function, we use them in order to approximate any parameterization of $\pyun$ and $\pxun$, which would be automatically learnt, thus increasing the modeling power of the models. Note that, in the Deep TMC models that we propose, $\pzun$ will not parameterized by a DNN.
%In Section~\ref{sec:deeppmc}, we presented a step-by-step training strategy for preserve the interpretability of the results in Deep PMCs. Of course, this problem still occurs in Deep TMCs
%and we still rely on the same training strategy, which is nevertheless extended to take into account the auxiliary latent process $\latent_{0:T}$.
%First, the last layer constraint introduced in Section \ref{sec:constrained_archi} is also used to estimate the parameters of the last layer of $\pyun$ ad $\pxun$ from the related TMC model with linear $\pyun$ and $\pxun$, this concerns the set of parameters $\theta_{\fr}$ which we introduced earlier. Once $\theta_{\fr}$ have been initialized, it is frozen and the rest of the parameters $\theta_{\ufr}$ concerning $\pyun$ and $\pxun$ are then pretrained
%with backpropagation.
%For this step, the classification $\hat{\lab}_{0:T}^{\pre}$, obtained
%with the non-deep TMC model is used. The observations $\obs_{0:T}$ are used in entry of both backpropagation procedures since the latter goes through $\q$ to generate, with reparameterization trick, a realization of $\latent_{0:T}$ (needed as inputs for $\pyun$ and $\pxun$). Then, as before for Deep PMCs, in the backpropagation procedure which goes through $\pyun$, 
%the cost function, $\mathcal{C}_f$, is the cross-entropy between
%the classification $\hat{\lab}_{0:T}^{\pre}$ and the output of
%$\pyun$. In the backpropagation procedure which goes through $\pxun$, the cost function, $\mathcal{C}_{g}$, is the mean square error, computed between the observations $\obs_t$ and the output of $\pxun$.
%Finally, the unfrozen  parameters of $\pyun$, $\pxun$, $\pzun$ and $\q$ will be fine-tuned with an ELBO maximization process, leading to the final estimates $\theta_{\ufr}^*$ and $\phi^*$.
%Algorithm \ref{algo:algo_train_dmtmc} summarizes the procedure and Figure \ref{fig:pretrain_dmtmc} provides a graphical representation of the deep parameterization of $\pyun$ within DTMC models.
%\begin{remark}
%If we mentally discard the sequential case, the pretraining procedure we propose in order to obtain a first estimate of the parameters 
%of Deep TMCs coincides with the training via backpropagation of deterministic Conditional Autoencoders~\citep{han2021universal}. This happens because our pretraining with backpropagation discards the probabilistic formulation. Then, refering to Figure \ref{fig:pretrain_dmtmc}, we can interpret $\q$ has  an encoder network and $\pyun$ as a decoder network, the latent space is composed of the $\latent_t$ variable, the conditioning labels is the $\lab_t$ variable. Note that if we discard only the sequential case, it is also clear that Deep TMCs can be seen as conditional Variational Autoencoders~\citep{sohn2015learning}, in this case, with a stochastic encoder and a stochastic decoder. However training via backpropagation with the proposed loss functions is not a common approach with such models.
%\end{remark}
\begin{figure}[htbp!]
  \centering
  \includegraphics[width=0.9\textwidth]{Figures/Graphical_models/pretra_unf.pdf}
  \caption{Graphical and condensed representation of the parameterization of $\pyun$ in the DTMC models. \emph{r.t.} stands for reparameterization trick. 
  The dashed arrows represent the fact that some variables are copied. For clarity, we do not represent the block $\pyun$ which is similar
  to Figure~\ref{fig:constrained_archi}, up 
  to the introduction of $\latent_{t-1:t}$.
  %entries of $\pyun$ which consists of products of $\lab_{t-1}$, $\latent_{t-1:t}$ or $\obs_{t-1}$, due to the output layer constraint. Residual connections between the $\pyun$ layer inputs and the last hidden layer of $\pyun$ are also omitted. See Figure~\ref{fig:constrained_archi} for details.
  }
  \label{fig:pretrain_dmtmc}
\end{figure}
Figure~\ref{fig:pretrain_dmtmc}
 summarizes our pretraining procedure for function $\pyun$ and 
 the final estimation procedure is described in Algorithm~\ref{algo:algo_train_dmtmc}.



\begin{algorithm}[htbp!]
  \caption{A general estimation algorithm for deep parameterizations of TMC models}
  \label{algo:algo_train_dmtmc}
  \begin{algorithmic}[1]
    \Require{$\obs_{0:T}$, the observation; $\q$ a class of variational distribution}
    \Ensure{$\widehat{{\lab}}_{0:T}$ the final classification}
  \Statex{\textbf{Initialization of the output layer of $\pzun$, $\pyun$
  and $\pxun$}}
  \State Estimate $(\theta_{\fr}^*,\tilde{\phi})$ and $\hat{\lab}_{0:T}^{\pre}$ with Algorithm~\ref{algo:tmc_elbo_opt}-\ref{algo:tmc_inf}, using the related non-deep TMC model 
  \Statex \textbf{Pretraining of $\theta_{\ufr}$}  
  \State $\theta_{\ufr}^{(0)} \leftarrow$ ${\rm Backprop}(\hat{\lab}_{0:T}^{\pre},\obs_{0:T},\theta_{\fr}^*,\tilde{\phi},\mathcal{C}_{\pzun}, \mathcal{C}_{\f}, \mathcal{C}_{\g})$
  \Statex \textbf{Fine-tuning of the complete model} 
  \State Compute $(\theta_{\ufr}^{*}, \phi^{*})$ with Lines \ref{line:start_dtmc}-\ref{line:end_dtmc} of Algorithm~\ref{algo:tmc_elbo_opt}
  \State Compute $\widehat{\lab}_{0:T}$ with Algorithm~\ref{algo:tmc_inf}
  \end{algorithmic}
\end{algorithm}


\subsection{Simulations}
We continue to illustrate 
the performance of our models with the same 
binary image segmentation problem as
Section \ref{sec:pmc}.
% Since Section \ref{sec:pmc} was devoted to 
% the evaluation of deep parameterizations,
We focus our experiments on the relevance
of the latent process $\latent_{0:T}$. 
To that end, we focus on a particular TMC model in which the role of the latent process 
$\latent_{0:T}$ is to complexify the conditional distribution $\zeta$ of the noise but not $\vartheta$.
We first present the particular model and next the results. 
$\beta_1$ and $\beta_2$ are tuned manually
by taking into account the characteristics 
of the studied models.

%We present simulations to highlight the interest of the deep and continuous TMC parameterization. The generalization offered by the TMC framework enables a wide variety of models and very few of them have been explored in the TMC literature. We start this section by defining a specific TMC model and its variants that we will use in the experiments. The models will be presented for the case of the binary segmentation that we treat in the experiment sections, \emph{\emph{i.e.}}, $\Omega=\{\omega_1,\omega_2\}$. Additionally, we will also consider $\vartheta=\{\vartheta_1,\vartheta_2\}$ when the auxiliary random process $\latent_{0:T}$ is discrete. 

\subsubsection{The minimal TMCs}
\label{sec:tmc_models}
% \katy{Put this subsection in chp 3}
In order to highlight the role of $\latent_{0:T}$ 
w.r.t. the other characteristics of our models,
we introduce the Minimal TMC (MTMC) model which exhibits a reduced number of direct dependencies.
In this model, $\latent_{0:T}$ is an independent process
and given $\latent_{0:T}$, $(\lab_{0:T},\obs_{0:T})$ is a HMC where only the observations depend on $\latent_t$; 
in other words, $\pzun$ in \eqref{tmc-theta-1} does not
depend on $\triplet_{t-1}$, $\pyun$ in \eqref{tmc-theta-2} only depends
on $(\lab_{t-1})$ and  $\pxun$ in \eqref{tmc-theta-3}
only depends on $(\latent_t,\lab_{t})$. 
The joint distribution of $\triplet_{0:T}$ can be rewritten as 
\begin{equation}
\label{joint-tmmc}
\p(\triplet_{0:T})=\underbrace{\prod_{t=0}^T 
\eta(\latent_t; \pz)}_{\p(\latent_{0:T})} \underbrace{\p(\lab_0) \prod_{t=1}^T \vartheta(\lab_t;\pyun(\lab_{t-1}))}_{\p(\lab_{0:T}|\latent_{0:T})=\p(\lab_{0:T})} 
\underbrace{\prod_{t=0}^T \zeta(\obs_t;\pxun(\latent_t,\lab_{t}))\text{,}}_{\p(\obs_{0:T}|\latent_{0:T},\lab_{0:T})}
\end{equation}
With this model, the latent process  $\latent_{0:T}$
affects the conditional distribution of the observations.

We next consider three instances of MTMCs. The first one is the continuous linear MTMC in which $\latent_t \in \mathbb{R}$ are distributed according to standard normal distribution (so $\eta$ is the Gaussian distribution and $\pz=[0;1]$), $\pyun$, $\pxun$, $\vartheta$ and $\zeta$ coincide with our first illustrative example in Section \ref{sec:generalParam}, see \eqref{param-1}-\eqref{param-2}, up to the dependency in $\latent_t$. We also consider a deep version of the MTMC (DMTMC) in which  $\pxun$ is
parameterized by a DNN (with one hidden layer of $100$ neurons and ReLU activation function). For both continuous versions of the MTMC, we use the variational distribution 
\begin{equation}
    q_{\phi}(\latent_{0:T}|\obs_{0:T})=\prod_{t=1}^Tq_{\phi}(\latent_t|\latent_{t-1},\obs_t)=\prod_{t=1}^T
    \mathcal{N}(\latent_t;\nu_{\phi}(\latent_{t-1},\obs_t)).
    \label{eq:tmc_simple_q}
\end{equation}
where $\nu_{\phi}(\latent_{t-1},\obs_t)$ is parameterized by a DNN with one hidden layer of $100$ neurons and a ReLU activation function.

The motivation underlying this choice of variational distribution 
is that $\latent_{0:T}$
is an independent process and that $\obs_t$ only depends
on $(\lab_t,\latent_t)$ given the past; consequently, it is
reasonable to assume that the posterior distribution
of $\latent_t$ only depends on $\latent_{t-1}$ and $\obs_t$. In addition, more complex variational distributions tend to be more difficult to estimate. And indeed, it has been observed that the choice of the variational distribution does not impact the results in the case
of Scenario \eqref{joint-tmmc}, see Appendix~\ref{app:var_distrib}.
Finally, we also consider
a discrete version of the MTMC (di-MTMC)
in which $\latent_t \in \{\nu_1,\nu_2\}$ is 
discrete~\citep{gorynin2018assessing, li2019adaptive, chen2020modeling}. 
For this model, Algorithm~\ref{algo:algo_theta_pmc}
and  \ref{algo:algo_hk_pmc}
 can be directly applied in the augmented 
 space $\{\omega_1,\omega_2\} \times \{\upsilon_1,\upsilon_2\}$.

%In this section we focus on a particular TMC model, which we call the Minimal Triplet Markov Chain (MTMC). Despite the fact that it exhibits a reduced number of direct dependencies, it still offers interesting generalization properties. The graphical representation is given in Figure~\ref{fig:tmc_graphs}. 
%More particularly, we study three versions of the MTMC model: the continuous Minimal Triplet Markov Chain (MTMC), the Deep Minimal Triplet Markov Chain (DMTMC) and the discrete Minimal Triplet Markov Chain (di-MTMC).

%\begin{figure}[h!]
%\centering
%\begin{subfigure}{0.3\textwidth}
%\centering
%\input{tikz/mtmc.tex}
%\caption{Graphical representation of the MTMC family of models. The light gray hexagons represent the auxiliary random variable which is continuous in the MTMC and discrete in the di-MTMC.  The other graphical elements follow the convention from Figure~\ref{fig:pmc_graphs}.
%}
%\label{fig:tmc_graphs}
%\end{figure}

%\paragraph{The continuous Minimal Triplet Markov Chain - } 
%\label{sec:mtmc}
%We introduce the continuous Minimal Triplet Markov Chain (MTMC) model in the continuous case. Let us consider the following particular factorization of Equation~\eqref{eq:tmc_general},
%\begin{equation}
%    p_{\theta}(\latent_{0:T},\lab_{0:T},\obs_{0:T})=p_{\theta}(\lab_1)p_{\theta}(\latent_1)p_{\theta}(\obs_1|\lab_1,\latent_1)\prod_{t=2}^Tp_{\theta}(\lab_t|\lab_{t-1})p_{\theta}(\latent_t)p_{\theta}(\obs_t|\lab_{t},\latent_{t})\text{.}
%    \label{eq:mtmc}
%\end{equation}

%In this case, the transitions between the hidden states of $\lab_{0:T}$ are parameterized by a discrete transition matrix,  $\eta$ is a standard Gaussian distribution $\mathcal{N}(\latent_t;0,I_d)$ and  $\zeta$ is a Gaussian one $\mathcal{N}(\obs_t;v_1, v_2^2)$, where $\pxun(\lab_t,\latent_{t}) =  \begin{bmatrix} v_1, v_2^2 \end{bmatrix} = \begin{bmatrix} a_{\lab_t}\latent_{t}+ b_{\lab_t},\
%\sigma_{\lab_t}\end{bmatrix}$. 
%We will also consider the following variational distribution to approximate $p_{\theta}(\latent_{0:T}|\obs_{0:T})$:
%\begin{equation}
%    q_{\phi}(\latent_{0:T}|\obs_{0:T})=\prod_{t=1}^Tq_{\phi}(\latent_t|\obs_t)=\prod_{t=1}^T
%    \mathcal{N}(\latent_t;f_{\phi}(\obs_t)).
%   \label{eq:tmc_simple_q}
%\end{equation}

%The latter choice for $q_{\phi}$ corresponds to the commonly called \emph{mean-field} assumption and $\latent_{0:T}$ samples are drawn with reparameterization trick. 

% \textbf{\textcolor{orange}{Le reste du paragraphe The continuous Minimal Triplet Markov Chain a été supprimé car ces approximations sont données de manière plus générale dans Monte Carlo approximation}}
% Let us now explicit all the terms of $\mathcal{L}(\theta,\phi)$ (Equation~\ref{eq:beta_elbo_xent}) in the MTMC case, up to the initial term discarded for brevity. With $\latent_{0:T}^1,\dots,\latent_{0:T}^M$, \iid samples from $q_{\phi}(\latent_{0:T}|\obs_{0:T})$ we have:
% \begin{align}
%     \E_{\latent_{0:T},\lab_{0:T}\sim \p(\lab_{0:T}|\latent_{0:T},\obs_{0:T})\q(\latent_{0:T}|\obs_{0:T})}[
%     \log p_{\theta}(\obs_{0:T}|\latent_{0:T},\lab_{0:T})] &\approx \frac{1}{N}\sum_{m=1}^M\sum_{t=2}^T\sum_{\lab_t}p_{\theta}(\lab_t|\latent_{0:T}^n,\obs_{0:T})\log p_{\theta}(\obs_t|\lab_t,z^n_t),\\
%     \KL(\p(\lab_{0:T}|\latent_{0:T},\obs_{0:T})\q(\latent_{0:T}|\obs_{0:T})||p_{\theta}(\latent_{0:T},\lab_{0:T}|\obs_{0:T})) &\approx \frac{1}{N}\sum_{m=1}^M\sum_{t=2}^T\sum_{\lab_{t-1},\lab_t}p_{\theta}(\lab_t,\lab_{t-1}|\latent_{0:T}^n,\obs_{0:T})\log\frac{q_{\phi}(\latent_t^n|\obs_t)p_{\theta}(\lab_t|\lab_{t-1},\latent_{0:T}^n\obs_{0:T})}{p_{\theta}(\lab_t|\lab_{t-1})p_{\theta}(\latent_t^n)},\\
%     \mathcal{H}(p(\hat{\lab_{0:T}});p_{\theta}(\lab_{0:T}|\obs_{0:T})) &\approx -\frac{1}{ K} 
%     \sum_{t=1}^T\left( \delta_{\hat{\lab}_t}^{\omega_1}\log q_{\phi}(\lab_t=\omega_1|\obs_{0:T})+
%     \delta_{\hat{\lab}_t}^{\omega_2}\log q_{\phi}(\lab_t=\omega_2|\obs_{0:T})\right).
% \end{align}
% In the last equation, $\delta$ denotes the Kronecker delta function. Once those quantities are computed, the steps for the parameter estimation procedure for MTMCs are given in Algorithm~\ref{algo:tmc_elbo_opt}.\\

%\paragraph{The Deep Minimal Triplet Markov Chain - }
%Similarly to Section~\ref{sec:deeppmc}, we wish to increase the modeling capabilities of the TMCs by introducing a parameterization involving a DNN. This way, within the MTMC model, we can chose that the parameters of the Gaussian distribution $\p(\obs_t|\lab_t,\latent_t)$ are the outputs of $\pxun$ which becomes a DNN. We thus define a new model that we call the Deep Minimal Triplet Markov Chain (DMTMC). Such a deep parameterization is motivated by the results obtained with DPMC models.

%\begin{remark}
%Unlike DPMCs, we do not propose to parameterize $\pyun$ with a DNN in the DMTMC model. We found out that such parameterization led to failing parameter estimation procedures. This might be linked with the interpretability issue in general TMCs that motivates many development in this work.
%\end{remark}

%\paragraph{The discrete Minimal Triplet Markov Chain - }
%Finally, let us define the discrete MTMC (di-MTMC) model as follows: in the MTMC model we choose $\latent_{0:T}$ to be a discrete random variable. The model joint distribution of di-MTMC is then:
%\begin{equation}
%    p_{\theta}(\latent_{0:T},\lab_{0:T},\obs_{0:T})=p_{\theta}(\lab_1|\latent_1)p_{\theta}(\latent_1)p_{\theta}(\obs_1|\lab_1,\latent_1)\prod_{t=2}^Tp_{\theta}(\lab_t|\lab_{t-1},\latent_t)
%    p_{\theta}(\latent_t)p_{\theta}(\obs_t|\lab_{t},\latent_{t})\text{.}
%    \label{eq:dimtmc}
%\end{equation}
%In this model we set  $\eta$ as a Bernoulli distribution $\mathcal{B}\rm{er}(\latent_t;s)$,  $\zeta$ as a Gaussian distribution $\mathcal{N}(\obs_t;v_1,v_2^2)$ where $\pxun(\lab_t, \latent_t) = \big[ v_1, v_2^2  \big]= \big[ a_{\lab_t,\latent_t}, \sigma_{\lab_t}\big]$ and $s = 0.5$. A transition matrix is used to parameterize the transitions between the extended hidden process $\pmb{v}=(\latent_{0:T},\lab_{0:T})$.
%In di-MTMCs, $p_{\theta}(\pmb{v}|\obs_{0:T})$ is exactly computable and we do not need to resort to a variational approximation. Indeed, it then suffices to apply the Algorithm~\ref{algo:algo_1} with $\pmb{v}$ as the hidden r.v.
%The original and auxiliary posterior distributions can then be recovered by marginalization $p_{\theta}(\lab_{0:T}|\obs_{0:T})=\sum_{\latent_{0:T}}p_{\theta}(\latent_{0:T},\lab_{0:T}|x)$ and $p_{\theta}(\latent_{0:T}|\obs_{0:T})=\sum_{\lab_{0:T}}p_{\theta}(\latent_{0:T},\lab_{0:T}|x)$. Therefore, the developments of Section~\ref{sec:deeptmc} does not concern the di-MTMC model.

%\begin{remark}
%\label{rq:discrete_tmc}
%Interestingly, there is no specific constraints needed to ensure the interpretability in the inference for di-MTMCs. The constraint of a discrete $\latent_{0:T}$ seems strong enough to ensure interpretable results.
%\end{remark}
% \subsection{The Minimal TMCs}
% \label{sec:mtmc_models}
% We start with the 
% where the choice of parameters describes the transition:
% \begin{align}
%     \label{eq:mTMC}
%     \p(v_t|v_{t-1}) \!\overset{\rm mTMC}{=} \!\p(\lab_t|\lab_{t-1}) \p(\latent_t|\latent_{t-1}) \p(\obs_t|\lab_t,\latent_t) \text{.}
% \end{align}
% So this model assumes a Markovian distribution 
% for the labels and the latent variables aim at 
% learning the distribution of the noise given the label and 
% the latent variable.
% In order to capture temporal dependencies in the input data and to have an efficient computation of the variational distribution for the d-mTMC model, we use a deterministic function to generate $\tilde{h}_t$ which  
% takes as input $(\obs_t, \lab_t, \latent_t, \tilde{h}_{t-1})$. 
% Then the variational distribution $\q(\latent_{0:T}, \labu|$ $ \obs_{0:T}, \labl)$ satisfies the factorization \eqref{eq:fact-2}
% with $\q(\latent_t|\obs_t, \lab_t,\tilde{h}_{t-1} )$ and $\q(\lab_t|\obs_t, \tilde{h}_{t-1})$. 

% In order to highlight the role of $\latent_{0:T}$ 
% w.r.t. the other characteristics of our models,
% we introduce the MTMC model which exhibits a 
% reduced number of direct dependencies.

% The MTMC model is defined by the following factorization of the 
% joint distribution of $(\obs_{0:T},\latent_{0:T},\lab_{0:T})$:
% \begin{equation}
%     % \label{joint-tmmc}
%     \p(\triplet_{0:T})\overset{\rm mTMC}{=} \underbrace{\prod_{t=0}^T 
%     \eta(\latent_t;\pzun)}_{\p(\latent_{0:T})} 
%     \underbrace{\p(\lab_0) \prod_{t=1}^T \vartheta(\lab_t;\pyun(\lab_{t-1}))}_{\p(\lab_{0:T}|\latent_{0:T})=\p(\lab_{0:T})} 
%     \underbrace{\prod_{t=0}^T 
%     \zeta(\obs_t;\pxun(\latent_t,\lab_{t}))\text{,}}_{\p(\obs_{0:T}|\latent_{0:T},\lab_{0:T})}
% \end{equation}
% where  $\latent_{0:T}$ is an independent process
% and given $\latent_{0:T}$, $(\lab_{0:T},\obs_{0:T})$ is an HMC 
% where only the observations depend on $\latent_t$.
% In other words,   $\pxun$ in \eqref{eq:px}
% only depends on $(\latent_t,\lab_{t})$,  
% $\pzun$ in \eqref{eq:pz} does not
% depend on $(\triplet_{t-1}, \lab_t, \obs_t)$, 
% and $\pyun$ in \eqref{eq:py} only depends
% on $(\lab_{t-1})$.
% With this model, the latent process  $\latent_{0:T}$
% affects the conditional distribution of the observations.


%In this section we focus on a particular TMC model, which we call the Minimal Triplet Markov Chain (MTMC). Despite the fact that it exhibits a reduced number of direct dependencies, it still offers interesting generalization properties. The graphical representation is given in Figure~\ref{fig:tmc_graphs}. 
%More particularly, we study three versions of the MTMC model: the continuous Minimal Triplet Markov Chain (MTMC), the Deep Minimal Triplet Markov Chain (DMTMC) and the discrete Minimal Triplet Markov Chain (di-MTMC).

%\begin{figure}[h!]
%\centering
%\begin{subfigure}{0.3\textwidth}
%\centering
%\input{tikz/mtmc.tex}
%\caption{Graphical representation of the MTMC family of models. The light gray hexagons represent the auxiliary random variable which is continuous in the MTMC and discrete in the di-MTMC.  The other graphical elements follow the convention from Figure~\ref{fig:pmc_graphs}.
%}
%\label{fig:tmc_graphs}
%\end{figure}

%\paragraph{The continuous Minimal Triplet Markov Chain - } 
%\label{sec:mtmc}
%We introduce the continuous Minimal Triplet Markov Chain (MTMC) model in the continuous case. Let us consider the following particular factorization of Equation~\eqref{eq:tmc_general},
%\begin{equation}
%    p_{\theta}(\latent_{0:T},\lab_{0:T},\obs_{0:T})=p_{\theta}(\lab_1)p_{\theta}(\latent_1)p_{\theta}(\obs_1|\lab_1,\latent_1)\prod_{t=2}^Tp_{\theta}(\lab_t|\lab_{t-1})p_{\theta}(\latent_t)p_{\theta}(\obs_t|\lab_{t},\latent_{t})\text{.}
%    \label{eq:mtmc}
%\end{equation}

%In this case, the transitions between the hidden states of $\lab_{0:T}$ are parameterized by a discrete transition matrix,  $\eta$ is a standard Gaussian distribution $\mathcal{N}(\latent_t;0,I_d)$ and  $\zeta$ is a Gaussian one $\mathcal{N}(\obs_t;v_1, v_2^2)$, where $\pxun(\lab_t,\latent_{t}) =  \begin{bmatrix} v_1, v_2^2 \end{bmatrix} = \begin{bmatrix} a_{\lab_t}\latent_{t}+ b_{\lab_t},\
%\sigma_{\lab_t}\end{bmatrix}$. 
%We will also consider the following variational distribution to approximate $p_{\theta}(\latent_{0:T}|\obs_{0:T})$:
%\begin{equation}
%    q_{\phi}(\latent_{0:T}|\obs_{0:T})=\prod_{t=1}^Tq_{\phi}(\latent_t|\obs_t)=\prod_{t=1}^T
%    \mathcal{N}(\latent_t;f_{\phi}(\obs_t)).
%   \label{eq:tmc_simple_q}
%\end{equation}

%The latter choice for $q_{\phi}$ corresponds to the commonly called \emph{mean-field} assumption and $\latent_{0:T}$ samples are drawn with reparameterization trick. 

% \textbf{\textcolor{orange}{Le reste du paragraphe The continuous Minimal Triplet Markov Chain a été supprimé car ces approximations sont données de manière plus générale dans Monte Carlo approximation}}
% Let us now explicit all the terms of $\mathcal{L}(\theta,\phi)$ (Equation~\ref{eq:beta_elbo_xent}) in the MTMC case, up to the initial term discarded for brevity. With $\latent_{0:T}^1,\dots,\latent_{0:T}^M$, \iid samples from $q_{\phi}(\latent_{0:T}|\obs_{0:T})$ we have:
% \begin{align}
%     \E_{\latent_{0:T},\lab_{0:T}\sim \p(\lab_{0:T}|\latent_{0:T},\obs_{0:T})\q(\latent_{0:T}|\obs_{0:T})}[
%     \log p_{\theta}(\obs_{0:T}|\latent_{0:T},\lab_{0:T})] &\approx \frac{1}{N}\sum_{m=1}^M\sum_{t=2}^T\sum_{\lab_t}p_{\theta}(\lab_t|\latent_{0:T}^n,\obs_{0:T})\log p_{\theta}(\obs_t|\lab_t,z^n_t),\\
%     \KL(\p(\lab_{0:T}|\latent_{0:T},\obs_{0:T})\q(\latent_{0:T}|\obs_{0:T})||p_{\theta}(\latent_{0:T},\lab_{0:T}|\obs_{0:T})) &\approx \frac{1}{N}\sum_{m=1}^M\sum_{t=2}^T\sum_{\lab_{t-1},\lab_t}p_{\theta}(\lab_t,\lab_{t-1}|\latent_{0:T}^n,\obs_{0:T})\log\frac{q_{\phi}(\latent_t^n|\obs_t)p_{\theta}(\lab_t|\lab_{t-1},\latent_{0:T}^n\obs_{0:T})}{p_{\theta}(\lab_t|\lab_{t-1})p_{\theta}(\latent_t^n)},\\
%     \mathcal{H}(p(\hat{\lab_{0:T}});p_{\theta}(\lab_{0:T}|\obs_{0:T})) &\approx -\frac{1}{ K} 
%     \sum_{t=1}^T\left( \delta_{\hat{\lab}_t}^{\omega_1}\log q_{\phi}(\lab_t=\omega_1|\obs_{0:T})+
%     \delta_{\hat{\lab}_t}^{\omega_2}\log q_{\phi}(\lab_t=\omega_2|\obs_{0:T})\right).
% \end{align}
% In the last equation, $\delta$ denotes the Kronecker delta function. Once those quantities are computed, the steps for the parameter estimation procedure for MTMCs are given in Algorithm~\ref{algo:tmc_elbo_opt}.\\

%\paragraph{The Deep Minimal Triplet Markov Chain - }
%Similarly to Section~\ref{sec:deeppmc}, we wish to increase the modeling capabilities of the TMCs by introducing a parameterization involving a DNN. This way, within the MTMC model, we can chose that the parameters of the Gaussian distribution $\p(\obs_t|\lab_t,\latent_t)$ are the outputs of $\pxun$ which becomes a DNN. We thus define a new model that we call the Deep Minimal Triplet Markov Chain (DMTMC). Such a deep parameterization is motivated by the results obtained with DPMC models.

%\begin{remark}
%Unlike DPMCs, we do not propose to parameterize $\pyun$ with a DNN in the DMTMC model. We found out that such parameterization led to failing parameter estimation procedures. This might be linked with the interpretability issue in general TMCs that motivates many development in this work.
%\end{remark}

%\paragraph{The discrete Minimal Triplet Markov Chain - }
%Finally, let us define the discrete MTMC (di-MTMC) model as follows: in the MTMC model we choose $\latent_{0:T}$ to be a discrete random variable. The model joint distribution of di-MTMC is then:
%\begin{equation}
%    p_{\theta}(\latent_{0:T},\lab_{0:T},\obs_{0:T})=p_{\theta}(\lab_1|\latent_1)p_{\theta}(\latent_1)p_{\theta}(\obs_1|\lab_1,\latent_1)\prod_{t=2}^Tp_{\theta}(\lab_t|\lab_{t-1},\latent_t)
%    p_{\theta}(\latent_t)p_{\theta}(\obs_t|\lab_{t},\latent_{t})\text{.}
%    \label{eq:dimtmc}
%\end{equation}
%In this model we set  $\eta$ as a Bernoulli distribution $\mathcal{B}\rm{er}(\latent_t;s)$,  $\zeta$ as a Gaussian distribution $\mathcal{N}(\obs_t;v_1,v_2^2)$ where $\pxun(\lab_t, \latent_t) = \big[ v_1, v_2^2  \big]= \big[ a_{\lab_t,\latent_t}, \sigma_{\lab_t}\big]$ and $s = 0.5$. A transition matrix is used to parameterize the transitions between the extended hidden process $\pmb{v}=(\latent_{0:T},\lab_{0:T})$.
%In di-MTMCs, $p_{\theta}(\pmb{v}|\obs_{0:T})$ is exactly computable and we do not need to resort to a variational approximation. Indeed, it then suffices to apply the Algorithm~\ref{algo:algo_1} with $\pmb{v}$ as the hidden r.v.
%The original and auxiliary posterior distributions can then be recovered by marginalization $p_{\theta}(\lab_{0:T}|\obs_{0:T})=\sum_{\latent_{0:T}}p_{\theta}(\latent_{0:T},\lab_{0:T}|x)$ and $p_{\theta}(\latent_{0:T}|\obs_{0:T})=\sum_{\lab_{0:T}}p_{\theta}(\latent_{0:T},\lab_{0:T}|x)$. Therefore, the developments of Section~\ref{sec:deeptmc} does not concern the di-MTMC model.

%\begin{remark}
%\label{rq:discrete_tmc}
%Interestingly, there is no specific constraints needed to ensure the interpretability in the inference for di-MTMCs. The constraint of a discrete $\latent_{0:T}$ seems strong enough to ensure interpretable results.
%\end{remark}


\subsubsection{Experiments and results}
\label{sec:exp_res_tmc}
We now consider two scenarios
in which binary images are corrupted 
with non elementary noises. 
In the first scenario,
the hidden images $\lab_{0:T}$ are the \emph{camel}-type images of the Binary Shape Database and are corrupted
with the stationary multiplicative noise 
given in~\eqref{eq:noise_eq2} in Section~\ref{subsec:data_generation},
%\begin{scenario}[Stationary multiplicative noise] a noise
% \begin{equation}
% \label{scenario-1-tmc}
%     \obs_t|\lab_t,\latent_t \sim\mathcal{N}\left(a_{\lab_t};b_{\lab_t}^2\right) * \latent_t,
% \end{equation}
where $\latent_t\sim\mathcal{N}(0, 1)$, $a_{\omega_1}=0, a_{\omega_2}$ is a 
varying parameter and $b_{\omega_1}=b_{\omega_2}=0.2$.
Figure~\ref{fig:mult_noise_graph_a} displays
the results for the setting
$\beta_1=5,\beta_2=1$ in our variational approach. Scalar $\beta_1$ can be interpreted as enforcing the standardized Gaussian prior on the learnt latent variables, which is seemingly favorable on this example because of the way $\latent_{0:T}$ is generated. $\beta_2$ is also needed and seems to guide the optimization so that the estimated $\hat{{\lab}}_{0:T}$ corresponds to the desired segmentation. 
%\label{sce:mult_noise}
%\end{scenario}
%In the case of Scenario~\ref{sce:mult_noise},
%Importantly, no result could be obtained without modifying the ELBO with the $\beta$ scalars (and using the proposed variational gradient EM algorithm), hence the novelty of our training strategy compared to the existing literature on the studied Markov models. 
%Indeed, while discrete TMCs have already been introduced, to the best of our knowledge, such results have never been reported. 
\input{Figures/mult_noise_graph}
A particular classification 
is also displayed in Figure~\ref{fig:mult_noise_graph_b}.
As we see, our MTMC models improve the performance (up to a $7\%$-point
improvement) of the HMC-IN. This comparison illustrates the interest of the third
latent process $\latent_{0:T}$. A slight advantage goes to the models with continuous
$\latent_{0:T}$ (MTMC and DMTMC) over the di-MTMC which still performs better than the
HMC-IN model.
Note that in the case where we optimize directly the 
ELBO (\ie~$\beta_1=1$ and $\beta_2=0$), it 
has been observed that the classification
obtained is not interpretable. 
This observation validates
experimentally our strategy 
to adapt the objective function.
 
In the second scenario,
%\begin{scenario}[Non-stationary general noise] 
the hidden images $\lab_{0:T}$ are the \emph{dog}-type images
of the Binary Shape Database. They are corrupted by
a non-stationary general noise,
\begin{equation}
\label{scenario-2-tmc}
    \begin{cases}
    \begin{aligned}
    &\obs_t|\lab_t \sim\mathcal{N}\left(a_{\lab_t};\sigma^2\right), \text{ if } k\in\left\{1,\dots,\left\lfloor\frac{T}{2}\right\rfloor\right\},\\
    &\obs_t|\lab_t\sim a_{\lab_t} + \mathcal{E}\left(\vartheta\right), \text{ if } k\in\left\{\left\lfloor\frac{T}{2}\right\rfloor + 1, \dots, K\right\},\\
    \end{aligned}
    \end{cases}
\end{equation}
where $\mathcal{E}(\vartheta)$ is the exponential probability distribution of parameter $\vartheta$, $a_{\omega_1}=0, a_{\omega_2}$ is a varying parameter, $\sigma=0.2$ and 
$\vartheta=1.4$.
%\label{sce:non_statio}
%\end{scenario}
The main difficulty of this scenario is that the images are 
corrupted by two different noises with a relatively low level for 
both areas and have to be fitted in a unique model. 
For this scenario, we set $\beta_1=0.1$ and $\beta_2=0$. 
A small value of $\beta_1$ can be interpreted as a way to better fit the observations. 
Indeed, more flexibility seems to be needed to learn such a complex non-stationary noise. 
%$\forall a_{\omega_2}$; it seems that the presegmentation deduced from this model
%does not guide well the TMC models at all.

\input{Figures/nonstatio_gen_noise_graph}

\newpage
The reason why $\beta_2$ is set to $0$
is that the pre-classification obtained 
with the HMC-IN is poor and should not be used to learn the parameters in the MTMC. It has been observed
that other values deteriorate the 
final classification obtained with MTMC models.
The results are displayed in Figure~\ref{fig:nonstatio_noise_a} and
Figure~\ref{fig:nonstatio_noise_c} displays 
a particular classification.
It is clear that the TMC models with a continuous auxiliary latent r.v. (MTMC and DMTMC) offer a greater flexibility and are able to learn this complex multi-stationary noise. On the other hand the average classification provided by the di-MTMC or the HMC-IN models are irrelevant as soon as $a_{\omega_2}<2$.
This experiment illustrates the interest of a continuous auxiliary latent r.v. over discrete auxiliary latent r.v.; the latter being the only option that has been considered in the literature so far~\citep{gorynin2018assessing, li2019adaptive,chen2020modeling}.
These experiments show the interesting capabilities of the generalized models to provide results in presence of very general noises. Coupled to the deep parameterization,
a continuous third latent process
enables our models to bypass the need of an explicit expression of the conditional distribution of the noise.




% \section{Partially Pairwise Markov Chains}
% \label{sec-rnn}
% In this section, we propose a particular class of TMC which aims at extending the PMC model proposed in Section 
% \ref{sec-pmc}. The main motivation
% underlying this particular model is
% to introduce an explicit
% dependency on 
% the past observations $\obs_{t-1}$ of the 
% pair $(\lab_t,\obs_t)$, for all $t$. This
% dependency is introduced through
% the continuous latent process $\latent_{0:T}$
% and enables us to build an 
% explicit joint 
% distribution $\p(\lab_{0:T},\obs_{0:T})$
% which does 
% not satisfy the Markovian property of
% the PMC \eqref{eq:pmc_intro_uns}. The main difference
% with Section \ref{sec-tmc} is that 
% $\latent_{0:T}$ is now a conditional deterministic 
% latent process.
% The resulting model is called a Partially PMC (PPMC).
% As we will see, this particular
% construction enables us to use directly the Bayesian inference framework developed in Section~\ref{sec-pmc}.
% Finally, since PMCs appears
% as particular TMCs, the pretraining of
% deep parameterized PPMCs is a direct adaptation of 
% Section \ref{sec-deep-tmc}.
% %In this section we propose the Partially Pairwise Markov Chains (PPMC)
% % ~\citep{pieczynski2005restoring, lapuyade2010unsupervised} 
% % combined with Recurrent Neural Networks (RNN)~\citep{mikolov2014learning}. 
% % In the PPMC models, the Markovian property of $(\obs_{0:T}, \lab_{0:T})$ 
% % does not hold anymore and the conditioning depends on all the previous 
% % realizations of the observed r.v. However, $\lab_{0:T}$ given a realization 
% % of $\obs_{0:T}$ remains Markovian. On the other hand, a  
% % RNN is type of a neural networks which uses sequential data and depends on 
% % the previous elements within the sequence. 

% \subsection{Deterministic TMCs}
% \label{sec:def-ppmc}
% Let us focus on a particular case 
% of the TMC \eqref{tmc-trans}-\eqref{tmc-theta-3}. 
% From now on, 
% we consider that the conditional distribution $\eta$
% coincides with the Dirac distribution $\delta$, 
% and that function $\pzun$ only depends on $(\latent_{t-1},\obs_{t-1})$.
% Thus, $\latent_t$ becomes deterministic given $(\latent_{t-1},\obs_{t-1})$,
% \begin{equation}
% \label{z-ptmc}
% \latent_t=\pzun(\latent_{t-1},\obs_{t-1}) \text{.}
% \end{equation}
% Each variable $\latent_t$ can be interpreted
% as a summary of all the past observations $\obs_{t-1}$. Consequently, it is easy to see
% that 
% \eqref{tmc-theta-2} and \eqref{tmc-theta-3} now coincide with 
% $\p(\lab_t|\lab_{t-1},\obs_{t-1})$ and $\p(\obs_t|\lab_{t-1:t},\obs_{t-1})$, 
% respectively, and marginalizing \eqref{eq:tmc_intro} w.r.t.~$\latent_{0:T}$
% gives the explicit distribution of
% $(\lab_{0:T},\obs_{0:T})$,
% \begin{align}
%  \label{eq:ppmc_general}
%     \p(\lab_{0:T},\obs_{0:T})=\p(\lab_0,\obs_0)
%     \prod_{t=1}^T   & \underbrace{\vartheta(\lab_t;\pyun(\latent{t-1:t},\lab_{t-1},\obs_{t-1}))}_{\p(\lab_t|\lab_{t-1},\obs_{t-1})} \times \nonumber \\
%     & \quad \underbrace{\zeta(\obs_t;\pxun(\latent{t-1:t},\lab_{t-1:t},\obs_{t-1}))}
% _{p(\obs_t|\lab_{t-1:t},\obs_{t-1})} \text{,}
% \end{align}
% where $\latent_t$ satisfies \eqref{z-ptmc}.
% It can noted that $(\lab_{0:T},\obs_{0:T})$ is no longer
% Markovian. Remark that this property
% is also satisfied by the general TMC \eqref{eq:tmc_intro}. However,
% $\p(\lab_{0:T},\obs_{0:T})$ is now available in a closed-form 
% expression and the relationship between the 
% pair $(\lab_t,\obs_t)$ and the past observations is
% fully characterized by the function $\pzun$.


% This kind of parameterization has an advantage in 
% terms of Bayesian inference. Since $\latent_t$ is
% a deterministic function of $(\latent_{t-1},\obs_{t-1})$ (and so of $\obs_{t-1}$, by induction),
% the conditional
% posterior distribution $\p(\latent_t|\latent_{t-1},\obs_{0:T})$ reduces 
% to $\delta_{\pzun(\latent_{t-1},\obs_{t-1})}$.
% Consequently, Algorithm~\ref{algo:algo_theta_pmc} and
% Algorithm~\ref{algo:algo_hk_pmc} can be directly applied to estimate $\theta$ and $\lab_t$, for all $t$, by introducing the dependency in $\latent_{t-1:t}$ in functions $\pyun$
% and $\pxun$ of Section \ref{sec:inference_pmc}.
% An alternative point of view is that
% when $\latent_t$ is deterministic,
% Algorithm~\ref{algo:tmc_elbo_opt} can be seen as a particular instance of Algorithm~\ref{algo:algo_theta_pmc} 
% in which we have set $\q(\latent_{0:T}|\obs_{0:T})=\p(\latent_{0:T}|\obs_{0:T})$,
% $\beta_1=1$ and $\beta_2=0$. Indeed, for this particular setting the objective function
% \eqref{L-approx} coincides with the ELBO but also with the log-likelihood $\p(\obs_{0:T})$. 


% \subsection{Deep PPMCs} 
% \label{sec:dppmc}
% As previous models, we consider the case where
% PPMCs \eqref{eq:ppmc_general} are parameterized with DNNs. Such models will be
% referred to as DPPMCs. In the 
% particular case of PPMCs, 
% %$\pzun(\latent_{t-1},\obs_{t-1})$ is parameterized 
% %by a DNN. Since the objective is to model
% %long term dependency in the past observations, 
% $\pzun$ can be seen as a RNN, \ie~a neural network
% which admits the output of the network at previous
% time $t-1$ as input at time $t$~\citep{LSTM}.
% It is thus possible to directly combine our models
% with powerful RNN architectures such as 
% Long Short Term Memory (LSTM) RNNs or
% Gated Recurrent Unit (GRU) RNNs which have
% been developed to introduce
% emphasize sequential dependencies.
% Note that the gradient of $\pzun$ w.r.t.~$\theta$
% can also be computed with a version of the backpropagation algorithm adapted to
% RNNs~\citep{LSTM, GRU}.

% The pretraining of this deep architecture
% is direct. The constrained output layer step
% is an application of Paragraph \ref{constraint-tmc}
% with $\q(\latent_{0:T}|\obs_{0:T})=\p(\latent_{0:T}|\obs_{0:T})$, $\beta_1=1$ and $\beta_2=0$; so it can be seen as the step
% described for PMCs in Paragraph \ref{sec:constrained_archi} up
% to the additional input $\latent_{t-1:t}$.

% The second step of our pretraining procedure of Paragraph \ref{tmc-unfrozen} can also be simplified. Since in this particular case we have implicitly computed  the optimal conditional variational distribution $\q^{\rm opt}(\latent_t|\latent_{0:t-1},\obs_{0:T})=\delta_{\pzun(\latent_{t-1},\obs_{t-1})}(\latent_t)$, the reparameterized sample $\latent_{t-1:t}$ of Figure~\ref{fig:pretrain_dmtmc}  is now deterministic and coincides directly with the output of $\pzun$, as shown in  Figure~\ref{fig:pretrain_dppmc}. Note that the parameters of $\pzun$ are unfrozen.
% The training process is summarized in Algorithm~\ref{algo:algo_train_dppmc}.
% %Also, following Remark~\ref{rk:multiclass}, the prodecure can be extended to the multi-class and multi-channel case.


% %Now we  propose a new and practical way to take advantage of all the modeling possibilities that are introduced using a deep parameterization. More precisely, we propose to embed a RNN, denoted as $r_{\theta}$, in the model and we call it the Deep-PPMC (DPPMC). The auxiliary latent r.v. $\latent_{0:T}$ takes a new meaning within the DPPMC context: $\latent_t$ is now the deterministic hidden states of $r_{\theta}$, with values in $\mathbb{R}^{d_z}$. These new deterministic r.v. summarize the information contained in the previous observations.
% %In this case, the DPPMC model is defined by the equations:
% %\begin{eqnarray}
% %\label{eq:rnn}
% %\latent_t &=& r_{\theta}(\obs_{t-1},\latent_{t-1}) \text{, }\\
% %\label{ppmc-theta-1}
% %\p(\lab_t|\lab_{t-1},\obs_{1:t-1})&=&\vartheta(\lab_t;\pyun(\lab_{t-1},\latent_{t-1})) \text{, } \\
% %\label{ppmc-theta-2}
% %\p(\obs_t|\lab_t,\lab_{t-1},\obs_{1:t-1})&=&\zeta(\obs_t;\pxun(\lab_t,\lab_{t-1},\latent_{t-1})) \text{, }
% %\end{eqnarray}
% %where the two last equations can be related to Equations~\eqref{pmc-theta-1} and~\eqref{pmc-theta-2}.\\


% %On the other hand, we can similarly define the Deep-Partially Semi Pairwise Markov Chain (DPSPMC) where $\obs_t$ no longer depends on $\lab_{t-1}$  and the factorization reads
% %\begin{eqnarray}
% %\latent_t &=& r_{\theta}(\obs_{t-1},\latent_{t-1}) \text{, }\\
% %\label{sppmc-theta-1}
% %\p (\lab_t|\lab_{t-1},\obs_{1:t-1})&=&\vartheta(\lab_t;\pyun(\lab_{t-1},\latent_{t-1})) \text{, } \\
% %\label{sppmc-theta-2}
% %\p(\obs_t|\lab_t,\obs_{1:t-1})&=&\zeta(\obs_t;\pxun(\lab_t,\latent_{t-1})).
% %\end{eqnarray}
% %The DPSPMC model will be used in our experiments because of difficulties in the parameter estimation for the DPPMC model (also seen for the DPMC model) which requires further investigation out of the scope of this article. 
% %The graphical representations of the PPMC model and the PSPMC model are given in Figure~\ref{fig:ppmc_graphs}.


% %\begin{remark}
% %An intermediate linear layer is introduced in the model transforming the output of the RNN layer $\latent_t\in\mathbb{R}^{d_z}$ in a vector of $\mathbb{R}^{d_{\obs}}$ which can be fed into the subsequent parts of the network. This layer does not appear explicitly in the previous equations not to overload the expressions but will be mentionned in the technical details that follow.
% %Thanks to this linear layer, we ensure that we are able to work with arbitrary dimensions for the RNN internal states, while keeping conssitency with respect to the DPMC model to derive our pretraining procedure that is described next.
% %\end{remark}


% %\begin{figure}[h!]
% %\centering
% %\begin{subfigure}{0.3\textwidth}
% %\centering
% %\input{tikz/ppmc}
% %\caption{PPMC}
% %\label{fig:ppmc_graph}
% %\end{subfigure}
% %\begin{subfigure}{0.3\textwidth}
% %\centering
% %\input{tikz/pspmc}
% %\caption{PSPMC}
% %\label{fig:pspmc_graph}
% %\end{subfigure}
% %\caption{Graphical representations of the PPMC and the PSPMC. The light gray hexagons represent the deterministic variable $\latent_t$. The other graphical elements follow the convention from Figure~\ref{fig:pmc_graphs}.
% %}
% %\label{fig:ppmc_graphs}
% %\end{figure}



% \begin{figure}[htb]
%   \centering
%   \includegraphics[width=0.8\textwidth]{Figures/Graphical_models/d-ppmc.pdf}
%   \caption{Graphical and condensed representation of the parameterization of
%   $\pyun$ in the DPPMC model. 
%   %Note the residual connections between the inputs and the ouput of $\pzun$.
%   The dashed arrows represent the fact that some variables are copied. 
%   %For clarity, we do not represent the entries of $\pyun$ consisting of products of $\lab_{t-1}$, $\latent_{t-1:t}$ or $\obs_{t-1}$, due to the output layer constraint. %Residual connections between the $\pyun$ layer inputs and the last hidden layer of $\pyun$ are also omitted.
%   }
%   \label{fig:pretrain_dppmc}
% \end{figure}


% \begin{algorithm}[htbp!]
%   \caption{A general estimation algorithm for deep parameterizations of PPMC models.}
%   \label{algo:algo_train_dppmc}
%   \begin{algorithmic}[1]
%     \Require{$\obs_{0:T}$, the observation}
%     \Ensure{$\hat{{\lab}_{0:T}}$, the final classification}
%     \Statex{\textbf{Initialization of the output layer of $\pyun$ and $\pxun$}}
%     \State Estimate $\theta_{\fr}^*$ and $\hat{\lab}_{0:T}^{\pre}$ with Lines \eqref{line:nondeep1}-\eqref{line:nondeep3} of Algorithm~\ref{algo:algo_train_dpmc}
%     \Statex{\textbf{Pretraining of $\theta_{\ufr}$}}
%     \State   $\theta_{\ufr}^{(0)} \leftarrow$ ${\rm Backprop}(\hat{\lab}_{0:T}^{\pre},\obs_{0:T},\theta_{\fr}^*, \mathcal{C}_{\f}, \mathcal{C}_{\g})$
%     \Statex{\textbf{Fine-tuning of the complete model}}
%     \State Update all the models parameters (except $\theta_{\fr}$) with Algorithm~\ref{algo:algo_theta_pmc}
%     \State Compute $\hat{\lab}_{0:T}$ with Algorithm~\ref{algo:algo_hk_pmc} 
%   \end{algorithmic}
% \end{algorithm}



% \subsection{Simulations}
% We start again with the same experiments 
% as those in Section \ref{sec:pmc}, but we use 
% an alternative noise which aims
% at introducing longer dependencies on the observations. We now set
% \begin{equation}
%     \label{eq:longer_noise_eq1}
%     \obs_t| \lab_{t},\obs_{t-2:t-1} \sim \mathcal{N}\Big(\sin(a_{\lab_t}+0.2(\obs_{t-1}+\obs_{t-2}));
%     \sigma^2\Big).
% \end{equation}
% where $a_{\omega_1}=0$, $\sigma^2=0.25$ 
% and $a_{\omega_2}$ is a varying parameter.
% We compare the deep models of Section \ref{sec-pmc} (DSMPC and DPMC) with their natural extensions
% developed in this section (DPSPMC and DPPMC).
 
% Figure~\ref{fig:nonlin_corr_ppmc_sce1_a} illustrates the results involving the models we have just introduced. For $\pzun$ we use two independent standard RNNs with ReLU activation function, i.e. $\latent_t=[\latent_t^1,\latent_t^2]=[{\pzun}^1(\latent_{t-1}^1,\obs_{t-1}),
% {\pzun}^2(\latent_{t-1}^2,\obs_{t-1})]$; $\pyun$ (resp. $\pxun$) depends on $\latent_{t-1:t}^1$ (resp. $\latent_{t-1:t}^2$). %their output replaces the observation input of $\pyun$ and $\pxun$, respectively.
% In this setting, we found that the models worked the best when the dimensions of $\latent_t^1$ and of $\latent_t^2$  is $5$.
% %\ie~there are $5$ hidden neurons in each RNN. Graphical illustrations are provided in Figure~\ref{fig:nonlin_corr_ppmc_sce1_b}.
% We can see that the more general parameterizations embedded in   DPSPMC and DPPMC lead to an improvement of the DPMC models; each DPPMC model leading to a better accuracy than its DPMC counterpart. The ability to model long term dependencies proves to be important to better solve the correlated noise. This experiment illustrates a way to take advantage of a deterministic auxiliary process: by strengthening the sequential dependencies between the hidden random variables.

% \input{Figures/nonlin_corr_ppmc_sce1}

\begin{remark}
  We also propose an alternative use of the latent process, where our
  objective is to characterize explicitly the relationship
  between the pair $(\lab_t,\obs_t)$ and the past observations $\obs_{t-1}$ when
  $\latent_{0:T}$ is deterministic given the observations. Thus,  a closed-form
  expression of $\p(\lab_t,\obs_t|\lab_{t-1},\obs_{t-1})$ is available contrary to
  the general TMC introduced before.
  A direct advantage of the resulting TMC model is that
  it can be interpreted as the combination of a PMC model \eqref{eq:pmc_intro_uns}
  with an RNN~\citep{rumelhart1986learning,
  mikolov2014learning}, and that the distributions of interest can be computed
  exactly, without any approximation.
  This model is called a Partially Pairwise Markov Chain (PPMC), which is 
  detailed in the Appendix~\ref{chap:appendix4}.
\end{remark}


\section{Experiments on real datasets}
\label{sec:realworld}
We finally experiment our models on two 
real datasets. The first one is devoted to 
a medical images. The main challenge 
of this kind of data is that the noise associated to such images is unknown and non-usual; 
that is why we introduce our TMCs to measure the impact of the third latent process.
The next dataset is related to human activity recognition. 
For this problem, the dependencies between the r.v. (the class and the observed r.v.) 
are critical; that is why we focus on the impact of our PMCs.


\subsection{Unsupervised segmentation of biomedical images}
\label{sec:realworld_mct}
We first illustrate the potential of the generalized TMC models on real biomedical data.
The task consists in the segmentation of micro-computed tomography X-ray scans
of human arteries containing a metallic stent biomaterial\footnote{Data provided by Dr. Salomé Kuntz (GEPROMED, Strasbourg, France)}. 
These images are reminiscent of the synthetic experiment of 
Scenario~\eqref{scenario-2-tmc}: some regions exhibit a particular type of correlated noise (because of
 the beam hardening artifacts caused by the interactions
between X-rays and the metallic stent) and some regions do not. \textcolor{black}{However, the noise is unknown and has not been simulated contrary to Scenario \eqref{scenario-2-tmc}.}

Table~\ref{table:microct_scores} and Figure~\ref{fig:mct_illustrations} summarize the experiment. It can be seen that the classical models (HMC-IN and di-MTMC) are unable to handle the non-stationarity of the noise. The di-MTMC model even fail to provide any improvement over the HMC-IN model. On the other hand, major improvements can be seen when using the TMC models with a continuous auxiliary process, suggesting that the latter model offers more flexibility and that our parameter estimation algorithm enables to take advantage of it. These results on real-world data corroborates the results found in the synthetic experiment given in Section \ref{sec:exp_res_tmc}. Note that, in this case, we set $\beta_1=5$, $\beta_2=1$ and used the HMC-IN classification as a pre-segmentation. The network configurations are the same as in Section \ref{sec:exp_res_tmc}.


\subsection{Unsupervised clustering for human activity recognition}
\label{sec:realworld_har}

We now illustrate the performances of classical PMC models, deep PMC models and deep PPMC 
models on a real clustering task linked with human activity recognition. We use the Human 
Activity and Postural Transition (HAPT) dataset described 
in~\citep{reyes2016transition}\footnote{\url{http://archive.ics.uci.edu/ml/datasets/smartphone-based+recognition+of+human+activities+and+postural+transitions}}. 
It consists of three-dimensional time series that we wish to cluster into 
two classes: \emph{movement} and \emph{no movement}. \textcolor{black}{To solve this task, the models we used are the same as those introduced before, namely $\pyun$, $\pxun$, $\vartheta$ and $\zeta$ coincide with our first illustrative example in Section \ref{sec:generalParam} (\eqref{param-1}-\eqref{param-2}). 
In the case of the deep parameterizations, $\pyun$ and $\pxun$ have one (unfrozen) hidden layer with $100$ neurons and
the ReLU activation function. Moreover, in the case of the deep PPMC models, $\pzun$ is composed of two 
independent standard RNNs with ReLU activation function, \ie~
$\latent_t=[\latent_t^1,\latent_t^2]=[{\pzun}^1(\latent_{t-1}^1,\obs_{t-1}),
{\pzun}^2(\latent_{t-1}^2,\obs_{t-1})]$, with $10$ hidden neurons.}

The results are given in Table \ref{table:har_scores} for models sharing the same configurations with the models in Section \ref{sec:pmc} and \ref{sec:dppmc}. First of all, the modelization using the pairwise models seems very relevant in this application since we notice up to a $9\%$-point improvement over the HMC-IN model. In the case of the SPMCs, we clearly see the advantage of using deep parameterizations over the shallow models. The advantage of the deep parameterization is less significant in the PMC case. The contributions of the DPSPMC and DPPMC models are also less significant. The absence of gains in error rate when using the most complex models might be related to the limited length of the training sequences in this application (sequences of length between $15000$ and $20000$). 


\begin{table}
\centering
\setlength\tabcolsep{6pt}
\begin{tabular}{ccccc}
\toprule
Slice & HMC-IN & di-MTMC & MTMC & DMTMC \\\toprule
Average & $8.6$ & $8.6$ & $7.6$ & $\pmb{6.5}$\\
\bottomrule
\end{tabular}
\caption{Averaged error rates (\%) in unsupervised image 
segmentation with all the generalized TMCs assessed on ten micro-computed 
tomography slices. The detailed scores are given in Appendix~\ref{app:error_rates}.}
\label{table:microct_scores}
\end{table}

\input{Figures/mct_illustration}

\begin{table}
  \small
\centering
\setlength\tabcolsep{6pt}
\begin{tabular}{cccccccc}
\toprule
 Data & HMC-IN & SPMC & DSPMC & DPSPMC & PMC & DPMC & DPPMC \\\toprule
Average & $25.2$ &$21.3$ &$16.8$ &$\pmb{16.7}$ &$17.1$ &$16.8$ & $16.8$ \\
\bottomrule
\end{tabular}
\caption{Averaged error rates (\%) in the binary clustering of the first twenty raw entries of the HAPT dataset~\citep{reyes2016transition}. 
The detailed scores are given in Appendix~\ref{app:error_rates}.}
\label{table:har_scores}
\end{table}


  

\section{Conclusions}
In this chapter, we have proposed a general framework for PMC and TMC models which
fully exploits the modeling power offered by such models for unsupervised signal
processing. Contrary to previous work on \textcolor{black}{TMCs with a discrete
hidden data}, we have introduced a continuous latent process. For these models,
we have derived Bayesian inference algorithms for estimating their parameters
and the associated hidden r.v. and we have emphasized the case where the
parameterization relies on DNNs. Our algorithms rely on an objective function
deduced from the variational Bayesian inference but which has been modified to
include the interpretability of the discrete hidden r.v.

This contribution enables us to propose an efficient answer to three recurrent
questions linked with the practical applications of complex probabilistic
graphical models for sequential data: which probability distributions to choose,
how to parameterize them, and how to estimate their parameters in an
unsupervised way. For several applications, it has indeed been shown that our
global procedure leads to new models that consistently perform better than the
classical ones. Importantly, the ability of these models to tackle more complex
noises comes without no additional effort from the signal processing point of
view. Our experiments also suggest that it is possible to model complex noises
by using the universal approximating properties of DNNs and by training them in
an unsupervised way with the new algorithms that we propose.

On the other hand, while being invisible to a potential
practitioner, these new capabilities permitted by the embedded DNNs and by the
third auxiliary latent process come at the price of a more complex training
procedure. The latter is indeed cast in the context of variational inference
with inherent difficulties regarding the approximation of the lower bound, the
choice of the variational distribution or the choice of the penalizing
coefficients. However, since variational inference is a very popular research
topic, it could inspire many improvements for future works with the Generalized
Hidden Markov Models framework.
%  We also note that the DNN pretraining and the
% interpretability constraint require an available pre-segmentation. A future line
% of research involving self-supervised learning might prove itself as an
% efficient way to relax this requirement~\citep{zhu2020s3vae, gatopoulos2021self}.

\thispagestyle{empty}

% !TEX root = latex_avec_réduction_pour_impression_recto_verso_et_rognage_minimum.tex
\chapter{Medical Perspectives}
\markboth{CHAPTER 5. MEDICAL PERSPECTIVES}{Short right heading}
\label{chap:medical_perspectives}
\localtableofcontents
\pagebreak

% \begin{algorithm}[htbp!]
%     \caption{}
%     \label{}
%     \begin{algorithmic}[1]
%     \Require{}
%     \Ensure{}
%     \Statex{\textbf{}}
%     \\ 
%     \Statex{\textbf{}}
%     \\\end{algorithmic}
%     % \vspace*{0.2cm}
% \end{algorithm}
  



\section{Context and motivation}


Cardiovascular diseases (CVDs) represent a leading global cause of mortality, as
highlighted by data from the World Health Organization\footnote{https://www.who.int/}. 
CVDs include a wide range of conditions that affect the heart and blood vessels.
Among these, atherosclerosis is the most common cause of CVDs, which 
is characterized by the build-up of plaque inside the arteries.
This atheromatous plaque is made up of fat, cholesterol,
calcium, and other substances found in the bloodstream.
Over time, this plaque hardens, leading to the obstruction of the arteries
and can cause serious health problems~\citep{insull2009pathology}. 
For example, it can limit the flow of oxygen-rich blood to the organs 
and other parts of the
body~\citep{rafieian2014atherosclerosis}.
Figure~\ref{fig:data_arteria}
shows an example of a normal artery (A) and an artery
with atherosclerosis (B)~\footnote{http://vascularsurgeon.ie/peripheral-arterial-disease-pad/}.

\begin{figure}[htb!]
    \centering
    \includegraphics[width=0.45\textwidth]{Figures/Medical_images/peripheral_arterial.png}  
    \caption{Peripheral arterial disease results from narrowing or
    blockage of the arteries of the  legs.}
    \label{fig:data_arteria}
\end{figure}

In this context, the GEPROMED (European Research Group on Prostheses Applied to Vascular Surgery)
has been established to develop new biomaterials and surgical techniques for
vascular surgery. The group has access to a database of medical images, which
they have made available to us for the purpose of developing new methods for
medical images processing in vascular surgery.
The images in this database  are Computed Tomography (CT) or
micro-Computed Tomography (micro CT) images of the femoro-popliteal arterial
segment (SAFP) of different patients. 
The SAFP is one of the longest arteries in
the human body, subject to diverse mechanical forces 
(\eg~torsion, flexion, and extension) due to the movement of the
lower limbs. 

Atherosclerosis disease comprises 3 categories of plaque: calcified (calcium)
($\approx  70\%$), fibrous ($\approx 20\%$), and lipid 
($\approx 10\%$)~\citep{kuntz2021co}. 
To treat these, some endovascular techniques have been
developed, such as angioplasty and stenting.
There is no non-invasive method (imaging) that can accurately differentiate
lesions along the SAFP. The analysis is usually based on the preoperative CT scan 
(low resolution images), but
there are high-resolution scanners that allow a quasi-histological analysis of
the tissue. In other words, we have a micro CT scanner 
\textit{ex vivo} \footnote{The term \textit{ex vivo} refers to experimentation 
performed 
on tissue samples outside the living organism.}, 
and then correlate the images with the histology\footnote{
Histology is considered as a gold or criterion standard 
for the diagnosis of many diseases.}.
\cite{gangloff2020probabilistic} has already proposed a method to segment
micro CT images of the SAFP. 
However, a major limitation of this method is that it is not possible to directly
segment the CT images of the SAFP, which are of low resolution.
This problem is the motivation for the work presented in this chapter.

\section{Data and preprocessing}
\subsection{Data availability}
A protocol was developed to obtain the data for this study, which 
is described in~\citep{kuntz2021co} and is detailed in
Appendix~\ref{anex:protocol_database}.
Figure~\ref{fig:data_availability_summary} 
shows the available data.  We  have the histologic slices~\ref{fig:histo},
 the (2D) micro CT images~\ref{fig:mct_histo},
 and their corresponding ground 
truth~\ref{fig:mct_annotation}. 
The ground truth is composed of 6 classes, which are described in
Figure~\ref{fig:data_notation}.
\begin{figure}[htb!]
    \begin{subfigure}[b]{\textwidth}
            \centering
            \includegraphics[width=0.75\textwidth]{Figures/Medical_images/intro_histo.PNG}
            \caption{Three histologic slices. }
            \label{fig:histo}
    \end{subfigure}
    \begin{subfigure}[b]{\textwidth}
            \centering
            \includegraphics[width=0.75\textwidth]{Figures/Medical_images/intro_mCT.PNG}
            \caption{Three microCT images correlated with their
            histologic truth.}
            \label{fig:mct_histo}
    \end{subfigure}
    \begin{subfigure}[b]{\textwidth}
        \centering
        \includegraphics[width=0.75\textwidth]{Figures/Medical_images/intro_gt.PNG}
        \caption{Three expert annotated microCT images obtained.}
        \label{fig:mct_annotation}
    \end{subfigure}
    \caption{Illustration of part of the  available data for the study.
    Figure taken from~\citep{gangloff2020probabilistic}.}
    \label{fig:data_availability_summary}
\end{figure} 
\begin{figure}[htb!]
    \centering
    \includegraphics[width=0.7\textwidth]{Figures/Medical_images/intro_notation.PNG}
    \caption{Notation of the classes of the ground truth~\ref{fig:mct_annotation}.
    Figure taken from~\citep{gangloff2020probabilistic}.}
    \label{fig:data_notation}
\end{figure}
These annotations are only available for some slices
of the 3D scan as shown in 
Figure~\ref{fig:data_availability_2}.
Here, the red rectangles represent all the 2D slices of the 3D micro CT image.
However, the combined information (depicted as light gray and purple rectangles)
is not uniformly distributed across these slices; it is only present in certain
slices without following a specific pattern.
Figure~\ref{fig:data_availability}
shows the correlation between  the CT scanner and the micro CT scanner, 
that is predominantly available in segments of the artery
where specific lesions, particularly calcifications, are present.

\begin{figure}[H]
    \centering
    \includegraphics[width=0.40\textwidth]{Figures/Medical_images/data_2.pdf}
    \caption{Illustration of the available pair of information: 2D micro CT image 
    (light gray rectangle) and its corresponding ground truth (purple rectangle).
    The pairs of information are only available for some slices of the 3D micro CT image.
    The red rectangles represent all the 2D slices of the 3D micro CT image.
    Figure based on~\citep{kuntz2021co}}
    \label{fig:data_availability_2}
\end{figure}



The aim of this study is to assess the technical
feasibility of histological segmentation using the SAFP algorithm based on the
preoperative CT scan. The results of this study will provide initial data to
assess the value of a subsequent, larger-scale study to validate the diagnostic
capabilities of automated segmentation.
As far as we know,  there is no non-invasive method (imaging) that can accurately
differentiate lesions along the SAFP. Characterization of  AOMI 
plaques will enable a patient-centred treatment strategy to be devised, based on the 
type of plaque in the lesion. 
Automated segmentation will be a tool that will make it possible to dispense
with histopathological analysis and detect the type of plaque on the
preoperative CT scan. 
\begin{figure}[htb!]
    \centering
    \includegraphics[width=0.7\textwidth]{Figures/Medical_images/data_new.pdf}
    \caption{Illustration of the available correlation between CT scanner [B, B1, B2] and
    micro CT scanner [C, C1, C2] using standard references after Step~\ref{item:correlation}. 
    They represent different types of calcifications in SAFP plaques.
    Figure based on~\citep{kuntz2021co}.}
    \label{fig:data_availability}
\end{figure}
The expected long-term benefits for other patients are very significant. They
can be offered individualized treatment depending on the nature of their lesions
by adapting the medical device.
Previous work has shown that it is possible to segment the micro CT images of the
SAFP. However, a segmentation of the CT images is also necessary, since the
micro CT images are not available for all patients.
In this chapter, the objective is to perform image segmentation on CT
images of the SAFP. Our particular focus
lies on the most common type of plaques, the calcifications (sheet and nodular), 
which is a first step to segment other types of plaques in the future.

\subsection{Challenges}
\label{sec:challenges}
% In the previous sections, we have provided an overview of the available data and the
% foundational work upon which our research is built. 
While we have achieved
success in segmenting micro CT images, a notable limitation remains: the
segmentation of CT images. Extending our segmentation to CT
images is a challenging task for several reasons. First, the low resolution of CT
images introduces additional complexity into the segmentation process, in 
contrast to the higher-resolution micro CT images that~\cite{gangloff2020probabilistic} 
has been working with. As
depicted in Figure~\ref{fig:data_availability}, 
the discrepancy in image quality between the CT scanner (B1 and B2) and the micro CT 
scanner (C1 and C2) is evident. 
Moreover, the limited availability of data poses a significant
challenge. While we possess both 3D micro CT images and their corresponding 3D
CT images, establishing a clear and precise correspondence between the two 
sequences of images is far from straightforward. 
These images are not perfectly aligned, and
their correspondence is not as simple as a `mirror' image. Furthermore, the
availability of annotations for only some slices of the micro CT image is a
notable limitation when it comes to training a segmentation algorithm
for the CT images. 
% In essence, this chapter is dedicated to addressing the
% intricate complexities associated with CT image segmentation, which include low
% resolution, the intricacies of aligning micro CT and CT images for accurate
% segmentation, and the limited data.


\subsection{Pre-processing of the CT and micro CT images}
We have developed a workflow to segment the calcifications in the CT images
depicted in Figure~\ref{fig:medical_images_workflow}.
First, we select the region of interest within
the images, which corresponds to the artery segment containing the
calcifications. The selection process in the micro CT images is guided by
annotations and employs a box detection algorithm. In contrast, the
selection process in the CT images utilizes a centerline algorithm, that is 
provided by an expert. Normalization of the images is carried out
to facilitate the subsequent super resolution, and segmentation processes.\\

\begin{figure}[htb!]
    \centering
    \includegraphics[width=0.86\textwidth]{Figures/Medical_images/workflow.pdf}
    \caption{Our workflow for segmenting sheet and nodular calcifications in
    CT images of the SAFP is
    structured into five steps. First, we perform pre-processing of the CT and
    micro CT images, followed by a super resolution algorithm on the CT images,
    post-processing of the SR-CT images, supervised segmentation, and
    segmentation on the SR-CT images.}
    \label{fig:medical_images_workflow}
\end{figure}


Moreover, the original size of the CT images containing calcifications is often small
($5\times 5$ to $12\times 12$ pixels).
Thus, we apply a Super Resolution (SR) algorithm to increase the image  
resolution to facilitate the segmentation process.
In our case, a primary concern is the preservation of details in the CT images.
Focusing on methods that enhance resolution without losing crucial information
is key. 
Different SR techniques have
been suggested, encompassing optimization methods and deep learning approaches.
The latter have emerged as the most promising, with
exponential growth~\citep{li2021review}. 
We have considered the LapSRN algorithm proposed by~\cite{lai2017deep},
which  utilizes a Laplacian pyramid framework. 
This algorithm has been selected for its ability to accurately
 reconstruct high-frequency
details and reduce visual artifacts, which are crucial for medical images, especially CT scans.
We also studied SR algorithms via VAEs, from a point of view of 
the applicability to medical images, 
we won't be able to use those algorithms (more details in Appendix~\ref{sec:applicability_sr}).
This choice may not
be definitive, and we will continue to explore other SR algorithms in the future.\\
\begin{figure}[htb!]
    \centering
    \includegraphics[width=0.6\textwidth]{Figures/Medical_images/problem_1.pdf}
    \caption{Example of a CT image of the SAFP and its corresponding
    micro CT image. In addition, the corresponding Super Resolution CT image
    after applying the LapSRN algorithm with a factor of up-scaling of 8.}
    \label{fig:srct_example}
\end{figure}

Figure~\ref{fig:srct_example} 
shows an example of a SR-CT image obtained with the LapSRN algorithm.
The original CT image is of size $12\times 12$ pixels and the SR-CT image is of size
$96\times 96$ pixels, which is a factor of up-scaling of 8.
After applying the SR algorithm, a post-processing phase is undertaken to
eliminate any noise introduced by the SR algorithm. This
involves an analysis of the SR-CT images in comparison to the micro
CT images, enabling a medical interpretation of the results. The analysis
entails a histogram comparison and a visual inspection of the images to
determine the quality of the SR-CT images.


Figure~\ref{fig:srct_example_2} 
shows an example of a sequence of CT images where the calcifications are present.
These new sequences of SR-CT images will be used for the 
segmentation of the calcifications in the next steps.
\begin{figure}[htb!]
    \centering
    \includegraphics[width=1\textwidth]{Figures/Medical_images/ct_SRct.png}
    \caption{Example of a sequence of CT and SR-CT images.  From left to right, the 
    pairs of images (CT, SR-CT).
    The CT images correspond to a sequence of 2D slices of a 3D CT image, where
    the calcifications are present. The corresponding Super Resolution CT images
    are obtained with the LapSRN algorithm with a factor of up-scaling of 8.}
    \label{fig:srct_example_2}
\end{figure}

\newpage
\section{Medical image segmentation}
% (see  Figure~\ref{fig:medical_images_workflow}). 
% It is worth mentioning that this workflow remains subject to modifications, as we
% continue to refine the segmentation of SR-CT images. A primary limitation we
% have encountered is the challenge of interpreting the results, and we emphasize
% the importance of a medical interpretation to further enhance the workflow.
Semantic segmentation is a well-studied problem in the field of computer
vision. The objective is to assign a label to each pixel of an image. In the
context of medical imaging, this task is particularly challenging due to the
complexity of the images and the limited availability of annotated data.
In this section, we present the segmentation of the calcifications in the CT images
of the SAFP. This segmentation is performed in two steps. 
First, we perform a supervised segmentation using the pre-processed micro
CT (HR images), and their corresponding ground truth data. 
Once a final segmentation  model is obtained,  we perform a
segmentation on the SR-CT images. 
Our results are based on the U-Net model~\citep{ronneberger2015u}, 
and the Probabilistic U-Net~\citep{kohl2018probabilistic}. 

% \subsection{U-Net}
\label{sec:unet}
The U-Net architecture is a type of convolutional neural network (CNN) that was
specifically designed for biomedical image segmentation tasks
proposed by~\cite{ronneberger2015u}.
The U-Net architecture is a fully convolutional network that consists of a contracting path (encoder)
and an expansive path (decoder), which gives it the U-shape.
The contracting path follows the typical architecture of a convolutional network 
(Figure~\ref{fig:unet}).
% It consists of the repeated application of two $3\times 3$ convolutions (unpadded convolutions),
% each followed by a rectified linear unit (ReLU) and a $2\times 2$ max pooling
% operation with stride 2 for downsampling.
% At each downsampling step, the number of feature channels is doubled.
% Every step in the expansive path consists of an upsampling of the feature map
% followed by a $2\times 2$ convolution (up-convolution) that halves the number of
% feature channels, a concatenation with the correspondingly cropped feature map
% from the contracting path, and two $3\times 3$ convolutions, each followed by a
% ReLU. The cropping is necessary due to the loss of border pixels in every
% convolution. At the final layer, a $1\times 1$ convolution is used to map each
% 64-component feature vector to the desired number of classes. 
% % In total, the network has 23 convolutional layers.
This architecture has been remarkably successful due to its efficiency in
learning from a limited number of samples while accurately segmenting images.
The U-Net and its (non-stochastic) variants have been used in a variety of
medical image segmentation tasks such as the bone 
segmentation~\citep{caron2023segmentation, ganeshaaraj2022semantic}, and
the pancreas segmentation~\citep{sriram2020multilevel}.
\begin{figure}[htb!]
    \centering
    \includegraphics[width=0.88\textwidth]{Figures/unet_2.png}
    \caption{U-net architecture (example for $32\times32$ pixels in the lowest
    resolution). Each blue box corresponds to a multichannel feature map. The
    number of channels is denoted on top of the box. The x-y-size is provided at
    the lower left edge of the box. White boxes represent copied feature maps.
    The arrows denote the different operations.
    Figure taken from~\citep{ronneberger2015u}}
    \label{fig:unet}
\end{figure}
% \subsection{Probabilistic U-Net}

The Probabilistic U-Net was introduced by~\cite{kohl2018probabilistic},
and designed
to address the inherent ambiguities in real-world vision
problems, especially in medical imaging. 
With ambiguous problems, there is no single correct answer.
For example, the same image can be
segmented in different ways by different experts, leading  to different possible
segmentations. The overlap between structures in the image can also lead to
ambiguities.
The Probabilistic U-Net  incorporates a~\gls*{cvae}
 into the U-Net 
architecture (more details in Appendix~\ref{chap:appendix5}).
The latent space is a low-dimensional space where the segmentation variants are
represented as probability distributions.
A sample from the latent space 
is drawn and then injected into the U-Net to produce the
corresponding segmentation map. This model is trained using a variational
inference approach (see Subsection~\ref{subsec:vbi}),
 which allows the model to learn the distribution of
segmentations in the latent space.

% The central component of the Probabilistic U-Net is the latent space, which is
% the key to modeling the ambiguity of the segmentation problem.

\subsection{Results}
% A segmentation of the micro CT images of the SAFP has been presented with the U-Net
% model before in~\citep{gangloff2020probabilistic}. 
% Here,
We aim to specifically segment calcifications in SR-CT
images, which show areas of calcification in the artery. The
segmentation algorithms are trained on a dataset which contains
micro CT images, and their corresponding ground truth data representing the
calcification zones.
We present the results obtained with both, the U-Net, and Probabilistic
U-Net models. 
We evaluate their effectiveness by calculating the Dice 
score~\eqref{eq:dice_score}  on the test set.
This score provides an assessment of each model's performance in accurately
segmenting, and classifying each class within CT images, crucial for informed
doctor analysis


\paragraph*{Three class segmentation: }
Table~\ref{tab:dice_score_test} 
summarizes the Dice score on the test set
for the segmentation of the micro CT images.
Three classes are considered: background (Ba), nodular calcifications (NC), 
and sheet calcifications (SC). 
In terms of the Dice score, the Probabilistic U-Net model
outperforms the U-Net model for the calcifications classes.
Once the segmentation model is obtained, we perform a 
segmentation on the SR-CT images.\\
\begin{table}[htb!]
    \begin{center}
    % \small
    \begin{tabular}{|l|r|r|r|r|}
    \hline
    \multirow{2}{*}{Model}  &\multicolumn{3}{c|}{Dice score on the test set}\\
    \cline{2-4} 
      & \multicolumn{1}{c|}{Ba} & \multicolumn{1}{c|}{NC} & \multicolumn{1}{c|}{SC} \\ 
      \hline \hline
      \multicolumn{1}{|l|}{U-Net}      &0,7688   & 0,6032  & 0,5967  \\ \hline
      \multicolumn{1}{|l|}{Probabilistic U-Net}    & 0,7178 & 0,6214 & 0,6141\\ \hline
      \end{tabular}
      \vspace{-0.2cm}
      \caption{Dice score on the test set for the U-Net and Probabilistic U-Net models.
      Three classes are considered: background (Ba), nodular calcifications (NC), 
      and sheet calcifications (SC).}      
    \label{tab:dice_score_test}
    \end{center}
\end{table}

Figure~\ref{fig:proba_unet_sr_ct}
shows an example of segmentation of the micro CT image and its corresponding SR-CT image
with the U-Net and Probabilistic U-Net models.
We can see that both models are able to 
segment the calcifications in the SR-CT images, however, this analysis is
not possible with all the sequence of SR-CT images. 
The results need to be analyzed carefully by a medical expert.\\

\begin{figure}[htb!]
    \centering
    \includegraphics[width=1\textwidth]{Figures/Medical_images/ch5_seg_ct_mct.pdf}
    \caption{Example of a segmentation of the micro CT (first row) and 
    its corresponding SR-CT images (second row) 
    with the U-Net and Probabilistic U-Net models. }
    \label{fig:proba_unet_sr_ct}
\end{figure}


\paragraph*{Four class segmentation: }
Initially, our segmentation model for CT images was designed to differentiate
among three classes (background, nodular and sheet  calcifications). 
However, we observed that calcifications were not
consistently present across the entire sequence of CT images.
This led us to introduce an additional class into our model: the soft tissue (ST) class,
% (see Figures~\ref{fig:data_notation} ).
% This class has a consistent presence in all images and plays a crucial role in our analysis. 
that  encompasses both the arterial wall and the surrounding tissue, making its
segmentation vital for an accurate doctor's interpretation of the results.
Table~\ref{tab:dice_score_test_4} shows the Dice
scores obtained on the test set,
now configured to identify four classes.
\begin{table}[htb!]
    \begin{center}
    % \small
    \begin{tabular}{|l|r|r|r|r|r|}
    \hline
    \multirow{2}{*}{Model}  &\multicolumn{4}{c|}{Dice score on the test set}\\
    \cline{2-5} 
      & \multicolumn{1}{c|}{Ba}  & \multicolumn{1}{c|}{ST} & \multicolumn{1}{c|}{NC} & \multicolumn{1}{c|}{SC} \\ 
      \hline \hline
      \multicolumn{1}{|l|}{U-Net}      & 0,6027       & 0,6363  & 0,5807  & 0,6016   \\ \hline
      \multicolumn{1}{|l|}{Probabilistic U-Net}    & 0,6256     & 0,6439  & 0,5817  & 0,6751\\ \hline
      \end{tabular}
      \vspace{-0.2cm}
      \caption{Dice score on the test set for the U-Net and Probabilistic U-Net models, 
      with four classes: background (Ba), soft tissue (ST), 
      nodular calcifications (NC), and sheet calcifications (SC).}      
    \label{tab:dice_score_test_4}
    \end{center}
\end{table}

\begin{figure}[htb!]
    \centering
    \includegraphics[width=1\textwidth]{Figures/Medical_images/ch5_seg_seq_sfa006.pdf}
    \caption{CT slides, their corresponding SR-CT images, and their corresponding
    3 class and 4 class segmentations of  SR-CT images.}
    \label{fig:proba_unet_sr_ct_more_classes}
\end{figure}

Figure~\ref{fig:proba_unet_sr_ct_more_classes}
presents some slices of the SR-CT images and their corresponding segmentation
with Probabilistic U-Net models, with three and four classes.
When we examine these results with the doctors, it becomes apparent
that four-class segmentation offers greater interpretability compared to
three-class segmentation.

\section{Remaining challenges}
We have made significant progress in the segmentation of sheet and nodular
calcifications in CT images of the SAFP. However, several challenges remain to
be addressed. First,
% in the step improving image resolution, we face the problem of `domain
% switching', a common challenge in medical imaging, where models trained on one
% dataset may not perform optimally when applied to a different dataset~\citep{yan2019domain}. 
% In addition, 
the super-resolution algorithm itself does not take into
account the sequential nature of the images, as it is applied independently to
each slice. This can lead to inconsistencies across the image sequence. To
address this problem, we have explored a post-processing technique for SR-CT
images that aims to improve the consistency of results across the image sequence
prior to segmentation.

On the other hand, segmentation with the U-Net, and Probabilistic U-Net models
present the limitation related to their static nature. That is, when applied to
SR-CT images, the segmentation performed by these models
treats each slice independently, without taking into account the sequential
context that the images are part of a 3D image. This led to inconsistent
segmentation results, with  different results between two
consecutive slices. To address this problem, we adapted the input of each
model to include not only the target slice, but also the anterior and posterior
slices, creating a ``sliding window'' effect. This modification is intended to
incorporate some degree of sequential context into the segmentation process.
%  The
% results presented in this chapter reflect the improvements achieved with this
% adaptation.
In the future, we plan to explore other approaches to overcome these challenges,
with the goal of developing more sophisticated models that can more accurately
reflect the sequential and dynamic nature of the medical data, 
\eg~a sequential probabilistic U-Net model, a sequential SR VAE.




\section{Conclusions}
In this chapter, we have described  a structured workflow for the segmentation of sheet and
nodular calcifications in the CT images of the SAFP. This workflow encompasses
five steps: pre-processing of the CT and micro CT images, application of the
SR algorithm on the CT images, followed by post-processing of the SR-CT images, 
and finally the segmentation on the SR-CT images. 
First significant results from our research include those obtained using the LapSRN
algorithm for the SR task, and the U-Net and Probabilistic U-Net models for
segmentation. In particular, the Probabilistic U-Net model demonstrated superior
performance to the U-Net model in segmenting the calcification classes. In
addition, we observed that segmentation into four classes produces more detailed
results, allowing a clearer distinction between calcifications and soft tissue,
which is vital for a proper doctor's interpretation.
% We have placed special emphasis on two critical aspects of this workflow: the
% super resolution and the segmentation tasks.
% These components are crucial for addressing modeling challenges and
% are integral to our ongoing research efforts.
%  We have presented a brief
% literature review of the super resolution and segmentation steps. 
% This review not only helps in understanding the complexities we are addressing,
% but also guides us in identifying the most promising approaches and determining
% appropriate methods for future research.
\thispagestyle{empty}

% !TEX root = latex_avec_réduction_pour_impression_recto_verso_et_rognage_minimum.tex
\chapter*{Conclusions and Perspectives}
\markboth{CONCLUSIONS AND PERSPECTIVES}{CONCLUSIONS AND PERSPECTIVES}
\addcontentsline{toc}{chapter}{Conclusions and Perspectives}
% \epigraph{
% ``To boldly go where no one has gone before.'' }{Captain Kirk}

% %chapter abstract here


Throughout this thesis, our research has integrated traditional
probabilistic models with modern deep learning techniques to address 
various challenges in machine learning. We have focused on generative
sequential modeling, supervised, semi-supervised, 
unsupervised Bayesian classification, and a collaboration with the GEPROMED project
to address the segmentation of medical images.
\vspace{0.27cm}

First, we introduced a novel generative model based on Pairwise Markov Chains,
which effectively combines the strengths of Hidden Markov Models,
Recurrent Neural Networks, and the Stochastic RNNs.
This model considers  observed and latent variables as well as the
interactions between them, providing a more comprehensive representation of
sequential data. 
We developed a new parameter estimation method leveraging the variational inference
framework, which is both computationally efficient and straightforward to
implement. The integration of deep parameterizations within this PMC
model demonstrated superior performance on different 
datasets compared to traditional RNN and Stochastic RNN models.
We also highlighted the linear and stationary Gaussian PMC's ability to model complex
Gaussian distributions more effectively than previous models,  
by using the covariance function. 
\vspace{0.27cm}





Moreover, if we consider the latent variables as discrete, \ie~the labels
associated with each the observations,
we demonstrated the potential of the PMC for supervised, and unsupervised 
classification tasks. In a supervised setting, our first variational framework can 
be easily adapted. In an unsupervised setting, we can use the PMC with traditional
Bayesian parameter estimation methods, since the likelihood is tractable, 
\ie~VI is not necessary. However, the use of this Bayesian framework with neural 
networks in an unsupervised context is difficult due to the interpretation of the
latent variables as discrete labels. Thus, we proposed an alternative 
approach to address this issue by using a constrained output layer 
and a pretraining step to initialize the neural network.
\vspace{0.27cm}



Next, we extended our generative model to Triplet Markov Chains that 
incorporate an additional (continuous or discrete) process to model 
the interactions between the 
observed features and their corresponding labels. We illustrated the 
feasibility of creating diverse generative models based on
variational inference, which is particularly advantageous for datasets with
partially labeled observations or missing labels.
We proposed a new adapted parameter estimation methods for the TMC model,
that combines the variational inference framework, 
which is both computationally efficient, and interpretable in the context of 
sequential data classification.
Each context, semi-supervised and unsupervised classification, 
has its own challenges,
and we proposed different techniques to address them.
For the semi-supervised context, we proposed a relaxation of the discrete
variables using the Gumbel-Softmax trick, and 
for the unsupervised classification, we proposed a constrained output layer
and pre-classification.
\vspace{0.27cm}


In addition, our collaboration with the GEPROMED project allowed us to know the
challenges of medical image analysis. We proposed an adapted workflow to
address the segmentation of medical images, which is a helpful tool for
clinicians. We also applied classic super-resolution and segmentation
techniques to medical images, which are essential for improving the
interpretability of medical images. However, the results are not always
satisfactory or adapted to the available data. 
We applied a probabilistic segmentation model that incorporates the 
variational framework with conditional VAEs. 
\vspace{0.27cm}



In conclusion, the integration of traditional probabilistic models with
modern deep learning techniques has shown promising results in various
applications. The proposed models have demonstrated superior performance
compared to traditional models, and the variational inference framework has
proven to be a powerful tool for parameter estimation. 
Future work should continue to explore the integration of deep neural networks
with other probabilistic models to develop more robust and efficient generative models.
Research into various neural architectures (\eg~U-net),
and training paradigms could further
improve model performance and broaden their applicability.
While this thesis primarily focused on medical imaging, the proposed methods and
models have potential applications in other fields such as natural language
processing, bioinformatics, and finance. Future research could explore these
domains to validate the versatility and robustness of our models.
\vspace{0.27cm}

Moreover, the application of these models to medical image analysis
should be further explored to improve the interpretability of medical images
and enhance the workflow of clinicians. For example, the integration of
the TMCs with the U-NET architecture could provide a more comprehensive
representation of the sequential data,  where the labels become the segmented images,
and the observed features are the medical images.
\vspace{0.27cm}

In practice, semi-supervised classification tasks are
challenging due to the discrete nature of the latent variables.
The relaxation of the discrete variables using the Gumbel-Softmax trick
provides a workaround, but it introduces a trade-off between the optimization
of discrete variables and the quality of the approximation.
Researchers continue to explore ways to improve the optimization of models with
discrete variables, making them more tractable and effective for a wider range
of applications.
% \vspace{0.27cm}
In the unsupervised case, we also noted that the DNN pretraining and the
interpretability constraint require an available pre-classification. A future line
of research involving self-supervised learning might prove itself as an
efficient way to relax this requirement.

\vspace{0.27cm}

Finally, the stochastic realization theory can be also used to describe the
covariance series which can be produced by linear and stationary PMCs, similar
to the one used in the context of the GUM. The main difficulty is that they do
not admit a state-space model representation due to the new dependencies
introduced by the pairwise interactions, which makes the analysis more complex.
The trick is to interpret the PMC as a particular HMC in augmented dimension.
However, the theoretical analysis of the covariance series produced 
by general linear and stationary PMCs remains an open question.

% The development of a stochastic realization theory for these models would be a significant
% contribution to the field of probabilistic modeling.
% ~\citep{zhu2020s3vae, gatopoulos2021self}.

% \pagebreak
% \section*{Generative Sequential Modeling}
% \textbf{Family of the probabilistic model studied:}\\

% The probabilistic models studied in this thesis belong to the family of Markov
% Chain models. Specifically, we have focused on Hidden Markov Chains,
% Pairwise Markov Chains, and Triplet Markov Chains. These models are
% essential for understanding and modeling sequential data due to their ability to
% capture temporal dependencies and latent structures.

% \textbf{Model, innovation:}\\
% \textbf{Issue that has been addressed:}\\
% % \textbf{}\\


% \noindent\hfil\rule{0.5\textwidth}{.4pt}\hfil
% \pagebreak

     
% \section*{Semi-supervised Bayesian classification}
% \textbf{Family of the probabilistic model studied:}\\
% \textbf{Model, innovation:}\\
% \textbf{Issue that has been addressed:}\\
% % \textbf{}\\

% \noindent\hfil\rule{0.5\textwidth}{.4pt}\hfil

% While these techniques provide a workaround, 
% they are not without their limitations. 
% The relaxation introduces a trade-off between the 
% optimization of discrete variables and the quality of the approximation. 
% As the temperature parameter is annealed during training, 
% the model transitions from a more continuous relaxation to a more discrete one, 
% potentially affecting model performance.

% In practice, dealing with discrete variables often requires 
% careful consideration of the trade-offs and the application's 
% specific requirements. 
% Researchers continue to explore ways to improve 
% the optimization of models with discrete variables, 
% making them more tractable and effective for a wider range of applications.


% \pagebreak


% \section*{Unsupervised Bayesian classification}
% \textbf{Family of the probabilistic model studied:}\\
% \textbf{Model, innovation:}\\
% \textbf{Issue that has been addressed:}\\
% % \textbf{}\\
% \noindent\hfil\rule{0.5\textwidth}{.4pt}\hfil

% \begin{itemize}
%     \item How to link the model with the PMC in Chapter XXX
%     % ~\ref{chap:pmc}?
%     In the context, what we can modify the Variational PMC for unsupervised learning?
% \end{itemize}

% % \pagebreak
% % \section*{Perspectives on medical images}
% % \textbf{Family of the probabilistic model studied:}\\
% % \textbf{Model, innovation:}\\
% % \textbf{Issue that has been addressed:}\\
% % % \textbf{}\\

% % \noindent\hfil\rule{0.5\textwidth}{.4pt}\hfil

% % VAEs and the Super resolution VAEs. 
% % Conditional VAEs and the Super resolution CVAEs.
% % CVAEs do not require an assumption of the conditional variable which can be seen as 
% % a huge advantage or disadvantage depending on the application.

% % VAE-based models have often been criticised for their
% % feeble generative performance~\citep{chira2022image} but with new advancements
% % such as VDVAE , there is now strong evidence that deep VAEs have the potential
% % to outperform current state-of-the-art models for high-resolution image
% % generatio.\\

% % Pairwise Markov Model for super resolution tasks

% % In the further, we will try to use the PMC for super resolution tasks.???
% % In the future, we will
% % continue to explore other SR algorithms and segmentation models. We will also
% % continue to refine the segmentation of SR-CT images. A primary limitation we
% % have encountered is the challenge of interpreting the results, and we emphasize
% % the importance of a medical interpretation to further enhance the workflow.
% % % \pagebreak
% % % \section*{Possible questions and remarks}
% % % Link with difussion models

% % % Semi-supervised model and  the use of $\obs_{0:T}$ as a vector from left to right and
% % % not from right to left, any change in the model?
% % % (HMCs)For such models, choosing between a gradient ascent method or the EM-algorithm
% % % remains an open question that we do not address.
\thispagestyle{empty}


\begin{appendices}
% !TEX root = late\obs_avec_réduction_pour_impression_recto_verso_et_rognage_minimum.tex
\chapter{Additional material}
\label{chap:appendix}


\begin{algorithm}[htbp!]
    \caption{Expectation Maximization \citep{dempster1977maximum} }
    \label{algo:em_algorithm}
  \begin{algorithmic}[1]
    \Require{$\obs$, the observations.}
    \Ensure{$\hat{\theta}$ the set of estimated parameters.}
    \State Initialize the parameters $\theta^0$
    \State $j\leftarrow 0$
    \While{\text{convergence is not attained}}
        \Statex{\textbf{E-step:}}
        \State Define $\mathcal{Q}(\theta|\theta^j)$ by 
        \begin{equation}
        \label{eq:e-step}
        \mathcal{Q}(\theta|\theta^j)= \E_{p(\lab|\obs,\theta^j)}
        \left[\log p(\obs,\lab|\theta)\right] \text{.}
        \end{equation}
        \Statex{\textbf{M-step:}}
        \State Estimate the new set of parameters
        \begin{equation}
        \theta^{j+1} \leftarrow \argmax\obs_{\theta} \mathcal{Q}(\theta|\theta^j)
        \end{equation}
        \State  $j\leftarrow j+1$
    \EndWhile
    \State  $\hat{\theta} \leftarrow \theta^{{j}}$
  \end{algorithmic}
    % \vspace*{0.2cm}
  \end{algorithm}
% \subsection{Proof of proposition \ref{prop:cov}}
% \label{anex:proof_prop_1}

% \subsection{Proof of proposition \ref{prop:cov-pmc}}
% \label{anex:proof_prop_2}

% \subsection{Particle filter}
% \label{alg:particle_filter}

% Hereeee  we present the particle filter algorithm used in the experiments.
\newpage
\begin{lemma}
    \textbf{\citep{rao1973linear}}\\
    \label{prop:gaussian}
    Let $x \in \mathbb{R}^p$, $y \in \mathbb{R}^q$, 
    $F \in \mathbb{R}^{p \times q}$, $d \in \mathbb{R}^p$, $m \in \mathbb{R}^q$,
    $\Sigma_1$ and $\Sigma_2$ 
    be $p\times p$ and  $q\times q $ positive definite matrices, respectively. 
    Then  the following equality holds
    \begin{equation*}
        \int_{y \in  \mathbb{R}^q} 
        \mathcal{N}(x;\; F y  + d , \Sigma_1) \mathcal{N}(y;\; m , \Sigma_2) dy 
        = \mathcal{N}(x;\; Fm + d , \Sigma_1 + F \Sigma_2 F^T)\text{.\\}
    \end{equation*}  
\end{lemma}


\vspace{0.65cm}
\subsubsection*{Conditional Variational Autoencoder}
Let $\obs$, $\lab$, and $\latent$ be the input image, the corresponding ground truth,
and the latent representation, respectively.
The~\gls*{cvae} is an extension of VAE 
(see Example~\ref{ex:gaussian_case})
% ~\ref{sec:variational_autoencoder}) 
to conditional tasks such as image segmentation. 
Each component of the model is conditioned on some observed image $\obs$.\\
The ELBO objective function for the CVAE is defined as follows:
\begin{equation*}
    \mathcal{Q}_{\text{CVAE}}(\obs,\lab) = 
    \mathbb{E}_{q_{\phi}(\latent|\obs,\lab)}\left[\log p_{\theta}(\lab|\obs,\latent)\right] 
    - \text{KL}\left(q_{\phi}(\latent|\obs,\lab)||p(\latent|\obs)\right) \text{.}
\end{equation*}
\chapter{Generative Pairwise Markov Models}
\label{chap:appendix_22}

\tocless\section{Proof of Theorem \ref{prop:cov-pmc}}


Let's recall that in the stationary case, the function from $\mathbb{N}$ to
$\mathbb{R}$ that associates $r_k$ to any $k$ is a covariance function (or a
covariance sequence) if and only if, for any $T \geq 0$, 
the Toeplitz matrix
with the first row  $[r_0, r_1, \ldots,r_T]$ is a covariance matrix, \ie~it is
positive semi-definite. This set of constraints thus restricts the set of
possible sequences, and we aim to characterize this set.  
 $\{r_0, r_1, r_2, ...\}$ is a covariance function if and
only if $r_0 \geq 0$,  and  if
\[ C(z) = r_0 + 2 \sum_{k=1}^{+\infty} r_k z^k \]
is a function of the Carathéodory class, \ie~$C(z)$ has a positive real part
for $z$ in the open unit disk 
(Carathéodory-Toeplitz theorem~\citep{akhiezer1965classical})


% the Carathéodory-Toeplitz theorem~\citep{akhiezer1965classical} is a
% covariance function if

Thus, we look for values of $\tilde{A}$ and $\tilde{B}$ such that the 
covariance matrix $\Sigma^{\obs}_{T}$ with first row
$[1,\; \tilde{B},\; \tilde{A}^2,\; \tilde{A}^2\tilde{B},\; \tilde{A}^4,\; \tilde{A}^4\tilde{B},\; \dots ]$
 satisfies:
\begin{equation*}
  \forall T \in \mathbb{N}^*, \; \Sigma^{\obs}_{T} \geq 0 \iff \forall z \in 
  \{ u \in \mathbb{C}; \; |u|<1 \}, \Re\Big( 1+ 2(\tilde{B}z + \tilde{A}^2z^2) 
  \sum_{\tau=0}^{\infty}(\tilde{A}^2z^2)^\tau \Big) \geq 0 \text{,}
\end{equation*}
which is derived from:
\begin{align*}
  C(z) = & 1 + 2 ( \tilde{B}z +  \tilde{A}^2z^2 + \tilde{A}^2\tilde{B}z^3 +  \tilde{A}^4z^4 + \tilde{A}^4\tilde{B}z^5 + \dots) \\
  = & 1 + 2 \big[\tilde{B}z ((\tilde{A}^2z^2)^0  + (\tilde{A}^2z^2)^1 + (\tilde{A}^2z^2)^2 + (\tilde{A}^2z^2)^3 + \dots)\\
  & +  \tilde{A}^2z^2 ((\tilde{A}^2z^2)^0  + (\tilde{A}^2z^2)^1 + (\tilde{A}^2z^2)^2 + (\tilde{A}^2z^2)^3 + \dots) \big ]  \\
  = & 1 + 2(\tilde{B}z + \tilde{A}^2z^2) \sum_{\tau=0}^{\infty}(\tilde{A}^2z^2)^\tau
\end{align*}

% We have, for all \( z \in \{ u \in \mathbb{C}; \; |u|<1 \}, |\tilde{A}^2z^2|<1 \) 
% then \( \sum_{\tau=0}^{\infty}(\tilde{A}^2z^2)^\tau = \frac{1}{1- \tilde{A}^2z^2} \).\\

The positive real part condition is equivalent to:
\begin{align*}
&\Re\Big( 1+ 2(\tilde{B}z + \tilde{A}^2z^2) \sum_{\tau=0}^{\infty}(\tilde{A}^2z^2)^\tau \Big)  \geq 0\\
\overset{(i)}{\iff }& \; \Re\Big( 1+ 2\; \frac{\tilde{B}z + \tilde{A}^2z^2}{1-\tilde{A}^2z^2} \Big) \geq 0\\
\iff & \; \Re\Big( \frac{1 + 2\tilde{B}z + \tilde{A}^2z^2}{1-\tilde{A}^2z^2} \Big)  \geq 0\\
\overset{(ii)}{\iff}& \; \Re\Big( \frac{1 + 2\tilde{B}re^{i\theta} + \tilde{A}^2r^2e^{2i\theta}}{1-\tilde{A}^2r^2e^{2i\theta}} \Big) \geq 0\\
\iff & \; \Re\Big( \frac{(1 + 2\tilde{B}re^{i\theta} + \tilde{A}^2r^2e^{2i\theta})(1-\tilde{A}^2r^2e^{-2i\theta})}{|1-\tilde{A}^2r^2e^{2i\theta}|^2} \Big) \geq 0\\
\iff & \; \Re\Big( (1 + 2\tilde{B}re^{i\theta} + \tilde{A}^2r^2e^{2i\theta})(1-\tilde{A}^2r^2e^{-2i\theta}) \Big) \geq 0\\
  \iff & \; 1 + 2\tilde{B}r\cos(\theta) - 2\tilde{A}^2\tilde{B}r^3\cos(-\theta) - \tilde{A}^4r^4 \geq 0\\
  \overset{(iii)}{\iff} & \; 1 + 2\tilde{B}r\cos(\theta) - 2\tilde{A}^2\tilde{B}r^3\cos(\theta) - \tilde{A}^4r^4 \geq 0\\
  \iff & \; 1 + 2\tilde{B}r\cos(\theta)(1-\tilde{A}^2r^2) - \tilde{A}^4r^4 \geq 0  \text{, }
\end{align*}
where we used the following arguments:
\begin{enumerate}[label=(\roman*)]
\item $|\tilde{A}^2z^2|<1$ since $\tilde{A}\in [-1, 1]$ and $|z|<1$.
\item Writing $z = re^{i\theta}$, for all $r \in [0, 1)$ and $\theta \in [-\pi, \pi]$.
\item Cosine is an even function.
\end{enumerate}

Thus, we need to analyze the expression:
\begin{equation}
  \label{eq:condCarat}
    1 + 2\tilde{B}r\cos(\theta)(1-\tilde{A}^2r^2) - \tilde{A}^4r^4 \geq 0 \text{,}
\end{equation}
and we can distinguish four cases:
\begin{enumerate}
  \item Case $\tilde{A}=0$: Let us first consider the case where $\tilde{A}=0$. In this case, \eqref{eq:condCarat} simplifies to:
  \[
  1 + 2\tilde{B}r\cos(\theta) \geq 1 - 2|\tilde{B}| \geq 0,
  \]
  which implies $|\tilde{B}|\leq \frac{1}{2}$.
  
  \item Case $\tilde{B}=0$: We then have the condition $|\tilde{A}|\leq 1$, which is true.
  
  \item Case $\tilde{B}>0$: 
  \begin{align*}
        &1 + 2\tilde{B}r\cos(\theta)(1-\tilde{A}^2r^2) - \tilde{A}^4r^4 \\
        \geq & \; 1 - 2\tilde{B}(1-\tilde{A}^2) - \tilde{A}^4. 
  \end{align*}
  
  Note that $1-\tilde{A}^2r^2\geq 0$ and $\tilde{A}^4r^4 \geq 0$. Therefore,
  \begin{align*}
        &1 + 2\tilde{B}r\cos(\theta)(1-\tilde{A}^2r^2) - \tilde{A}^4r^4 \geq 0 \\
        \iff &\; \tilde{B} \leq \frac{\tilde{A}^2+1}{2}.
  \end{align*}
  
  \item Case $\tilde{B}<0$: 
  \begin{align*}
        &1 + 2\tilde{B}r\cos(\theta)(1-\tilde{A}^2r^2) - \tilde{A}^4r^4 \\
        \geq & \; 1 + 2\tilde{B}(1-\tilde{A}^2) - \tilde{A}^4. 
  \end{align*}
  
  Note that $1-\tilde{A}^2r^2 \geq 0$ and $\tilde{A}^4r^4 \geq 0$. Therefore,
  \begin{align*}
        &1 + 2\tilde{B}r\cos(\theta)(1-\tilde{A}^2r^2) - \tilde{A}^4r^4 \geq 0 \\
        \iff &\; \tilde{B} \geq - \frac{\tilde{A}^2+1}{2}.
  \end{align*}
\end{enumerate}
Then $\{r_k\}_{k \in \NN}$ is a covariance function if and only if
  \begin{equation*}
  -1 \leq \tilde{A} \leq 1 \quad \text{and} \quad -\frac{\tilde{A}^2 +1}{2} \leq \tilde{B} \leq \frac{\tilde{A}^2 +1}{2}.
  \end{equation*}


\vspace{0.65cm}  
Now, the objective is to determine if any such probability distribution function can be modeled by some PMC model. For this, we study the inverse mapping of:

\begin{align}
  \label{eq:phige}
      \phi : \theta \mapsto \big(\tilde{A} = \tilde{A}(\theta), \tilde{B} = \tilde{B}(\theta)\big),
\end{align}
where $\theta$ represents the set of parameters of the model.

We set $\gamma = b$, and \(f\) either as \(0\) or \(-a - bc\) (two particular
cases of the PMC), that coincide with \eqref{eq:cov-pmm-e}. 
The following expressions for \(\tilde{A}\) and \(\tilde{B}\) are obtained:
\begin{eqnarray*}
    \left\{
    \begin{matrix}
    \tilde{A} = \sqrt{ce} \quad \text{and} \quad \tilde{B} = b(c(1 - b^2\eta) + e\eta)  & \;  \text{if } f = 0, \\ 
    \tilde{A} = \sqrt{e^2\eta + a^2(1 - b^2\eta)} \quad \text{and} \quad \tilde{B} = be\eta - a(1 - b^2\eta) & \; \text{if } f = -a - bc.
    \end{matrix} \right.
\end{eqnarray*}

First, the case \(f = 0\), \(\gamma = b\) implies that \(a = -bc\), so the set of parameters is \(b, c, e, \eta\) since \(a, f\), and \(\gamma\) are functions of these parameters. Thus, \(\phi\) can be written as:

\begin{align}
\label{eq:inv1}
    \phi :(b, c, e, \eta) \mapsto \big(\tilde{A} = \sqrt{ce}, \tilde{B} = b(c(1 - b^2 \eta) + e \eta) \big).
\end{align}

The domain \((\tilde{A}, \tilde{B})\) has been characterized to obtain a
covariance matrix, \ie~\(\tilde{A} \in [-1, 1]\) and \(-\frac{\tilde{A}^2 +
1}{2} \leq \tilde{B} \leq \frac{\tilde{A}^2 + 1}{2}\), which  defines a surface
\(\mathcal{S}\). We obtain an inverse mapping
\(\phi^{-1}\) of Equation~\eqref{eq:inv1}, showing that for some \((\tilde{A},
\tilde{B}) \in \mathcal{S}\), there exists at least one PMC which yields an
observation probability distribution.
For simplicity, we do not show the detailed inverse mappings and their
calculations here, as they are lengthy and complex. The important result is that
such a mapping exists and is consistent with the conditions stated above.\\

Next, the case \(f = -a - bc\), \(\gamma = b\)  implies \(c = e\eta - ab\eta\). The set of parameters is then \(a, b, e, \eta\), and \(\phi\) can be written as:
\begin{align}
\label{eq:inv2}
    \phi :(a, b, e, \eta) \mapsto \big(\tilde{A} = \sqrt{e^2 \eta + a^2 (1 - b^2 \eta)}, \tilde{B} = be \eta - a (1 - b^2 \eta) \big).
\end{align}
Similarly, we can obtain the inverse mapping \(\phi^{-1}\) of Equation~\eqref{eq:inv2}, showing that for some \((\tilde{A}, \tilde{B}) \in \mathcal{S}\), there exists at least one PMC which yields an observation probability distribution.

% !TEX root = late\obs_avec_réduction_pour_impression_recto_verso_et_rognage_minimum.tex
\chapter{Supervised Bayesian classification}
\label{chap:appendix2}

% \yohan{Not here. I think that it should be discussed in the introduction of the
% next chapter and justify why supervised estimation is a direct consequence of
% your previous chapter. So you consider semi-supervised problems.}
PMCs can be adapted for the supervised classification task 
by considering an observed variable in an augmented dimension
$\obs_{t}  \leftarrow (\obs_{t},\lab_{t})$. 
We add a discrete variable $\lab_{t}$ label associated to 
$\obs_{t}$, for all $t \in \NN$.
% $\lab_t \in \Omega = \{\omega_1, \dots, \omega_C \}$ with $C$ the number of classes.
The parameter estimation is realized by maximizing the ELBO with 
respect to $\theta$ and $\phi$
where the general ELBO in~\eqref{eq:elbo_general} 
is still valid and  reads as
\begin{align*}
    % \label{eq:elbo-pmc_sup}
    \Qsup(\theta,\phi)= & - \int \log \left(\frac{\q(\latent_0|\obs_{0:T}, \lab_{0:T})}
    {p(\obs_0, \lab_0, \latent_0)}\right) \q(\latent_0|\obs_{0:T}, \lab_{0:T}) {\rm d} \latent_{0:T}
    \nonumber \\
    &- \sum_{t=1}^T \int  \log\! \left(\frac{\q(\latent_t|\latent_{0:
    t-1},\obs_{0:T}, \lab_{0:T})}{\p(\latent_t,\obs_t, \lab_t|\latent_{t-1},\obs_{t-1}, \lab_{t-1})}\right) 
    \q(\latent_{0:T}|\obs_{0:T}, \lab_{0:T}){\rm d}\latent_{0:T} \text{.}
\end{align*}
% \katyobs{the notation $Q_{sup}$ is not consistent with the previous notation???}
The transition distribution 
$\p((\obs_t, \lab_t), \latent_t| (\obs_{t-1}, \lab_{t-1}), \latent_{t-1})$
% given in~\eqref{pmc-transition} 
can be factorized in two terms as shown in~\eqref{eq:pmc_gen_transition}.
Without loss of generality, we can consider the following factorization,
\begin{align}
    \p(\latent_t,\obs_t, \lab_t|\latent_{t-1},\obs_{t-1}, \lab_{t-1}) = &    
    \p(\obs_{t},\lab_{t} | \latent_{t-1:t}, \obs_{t-1},\lab_{t-1}) 
    \p(\latent_t|\latent_{t-1}, \obs_{t-1},\lab_{t-1}) \nonumber \\
    = &  \p(\obs_{t}|\latent_{t-1:t}, \obs_{t-1},\lab_{t-1})
    \p(\lab_{t}|\latent_{t-1:t}, \obs_{t-1:t},\lab_{t-1})\times \nonumber \\
    \label{eq:pmc-superv}
    & \;  \p(\latent_t|\latent_{t-1}, \obs_{t-1},\lab_{t-1}) \text{,}
\end{align}
which is nothing more than a TMC with transition~\eqref{eq:pmc-superv}.
The ELBO now reads 
\begin{align}
    \label{eq:elbo_sup_pmc}
    \Qsup(\theta,\phi) =& \L_1(\theta,\phi) +  \L_2(\theta,\phi)
\end{align}
with 
\begin{align*}
    \L_1(\theta,\phi) =& \;  \E_{\q(\latent_0| \obs_{0:T}, \lab_{0:T} )} 
    \log \p(\obs_0| \latent_0) 
    + \E_{\q(\latent_0| \obs_{0:T}, \lab_{0:T} )} 
    \log \p(\lab_0 |\latent_0, \obs_0) 
    \\ &+  
    \sum_{t=1}^T \E_{\q(\latent_t|\latent_{0:t-1},\obs_{0:T}, \lab_{0:T})}
    \log \p(\obs_t|\latent_{t-1:t}, \obs_{t-1},\lab_{t-1} )
    \text{}\\
    &+  
    \sum_{t=1}^T \E_{\q(\latent_t|\latent_{0:t-1},\obs_{0:T}, \lab_{0:T})}
    \log \p(\lab_{t}|\latent_{t-1:t}, \obs_{t-1:t},\lab_{t-1}) \text{,}\\
    \L_2(\theta,\phi) =& - \dkl(\q(\latent_0| \obs_{0:T}, \lab_{0:T} ) | |  \p(\latent_0)) \\
    &- \sum_{t=1}^T  \dkl(\q(\latent_t|\latent_{0:t-1},\obs_{0:T}, \lab_{0:T}) 
    | | \p(\latent_t|\latent_{t-1}, \obs_{t-1},\lab_{t-1})) \text{.}
\end{align*}

The training procedure of the  PMC model 
presented in Algorithm~\ref{algo:algo_train_dpmc_gen}, 
can be adapted for the supervised classification task.
The only distinction with the previous algorithm is the set of parameters $\theta$,     
which now includes the parameters of the conditional distribution of the labels 
$\p(\lab_t|\latent_{t-1:t}, \obs_{t-1:t},\lab_{t-1}).$
% Since $\lab_{0:T}$ is assumed to be observed, 
% a variational distribution $\q$ for $\lab_{0:T}$ is not required.

% \begin{example}
%     We recall the example of binary image segmentation, %in particular 
%     where $\obs_{0:T}$ represents the noisy grayscale image 
%     and $\lab_{0:T}$ the original black and white image. 
%     The previous transition~\eqref{eq:pmc-superv} is simplified as follows
%     \begin{align*}
%         \p(\latent_t,\obs_t, \lab_t|\latent_{t-1},\obs_{t-1}, \lab_{t-1}) = &  
%         \p(\obs_{t}|\latent_{t-1:t}, \obs_{t-1})
%         \p(\lab_{t}|\obs_{t},\lab_{t-1})
%         \p(\latent_t|\latent_{t-1}, \obs_{t-1}) \text{,}
%     \end{align*}
%     and its graphical model is presented in Figure~\ref{fig:pmc_supervised_learning}. 
   
%     In this case, we can choose the Gaussian distribution for $\p(\obs_{t}|\latent_{t-1:t}, \obs_{t-1})$ and 
%     $\p(\latent_t|\latent_{t-1}, \obs_{t-1})$, and the Bernoulli distribution for $\p(\lab_{t}|\obs_{t},\lab_{t-1})$.
%     Moreover, the variational distribution (\eg  $\q(\latent_t | \obs_{t}, \lab_{t})$ )
%     can also be a Gaussian one.
%     Then the set of parameters $\theta$ includes the mean and covariance matrix of the Gaussian distributions
%     and the parameters of the Bernoulli ones; and the set of parameters $\phi$ includes 
%     the mean and covariance matrix of the Gaussian variational distributions.
%     Finally, the parameter estimation can be realized by maximizing the
%     associated ELBO with respect to $\theta$ and $\phi$ as in the previous section.
%     \begin{figure}[htb]
%         \centering
%         \includegraphics[width=0.4\textwidth]{Figures/Graphical_models/spmc_sup.pdf}
%         \caption{Graphical representation of a particular instance of the PMC with 
%         an observed variable in an augmented dimension  
%         $\obs_{t}  \leftarrow (\obs_{t},\lab_{t})$ for 
%         supervised classification.}
%         \label{fig:pmc_supervised_learning}
%     \end{figure}
    
% \end{example}

% % !TEX root = late\obs_avec_réduction_pour_impression_recto_verso_et_rognage_minimum.tex
\chapter{Semi-supervised Bayesian classification}
\label{chap:appendix3}
% \subsection{Modified variational sequential labeler}
The VSL~\citep{chen2019variational} is based on conditional 
VAEs~\citep{pagnoni2018conditional},
where at each time step $t$, the observation $\obs_t$ is generated
according its associated context $u_t$, which consists of the observations
other than $\obs_t$. 
The lower bound of the log-likelihood at each time step $t$ is given by
\begin{align*}
    \log \p(\obs_t | u_t) &\leq 
    \E_{\q(\latent_t|\obs_{0:T})} 
    \left[ \log \p(\obs_t | \latent_t, u_t)\p(\latent_t|u_t) \p(\lab_t|\latent_t, u_t)   \right]
    \text{, for all } t \in \LL \text{.}\\
    \log \p(\obs_t | u_t) &\leq 
    \E_{\q(\latent_t, \lab_t |\obs_{0:T})} 
    \left[ \log \p(\obs_t | \latent_t, u_t)\p(\latent_t|u_t) \p(\lab_t|\latent_t, u_t)  \right]
    \text{, for all } t \in \U  \text{.}
\end{align*}
The VSL model simplifies some dependencies by assuming that
$\p(\lab_t|\latent_t, u_t) = \p(\lab_t|\latent_t)$ and 
$\p(\obs_t | \latent_t, u_t) = \p(\obs_t | \latent_t)$.
While the associated variational distribution is given by
\begin{align*}  
\q(\latent_{0:T}, \labu| \obs_{0:T}, \labl) =& \prod_{t=0}^T 
\q(\latent_t| \obs_{0:T}) \prod_{t\in  \U}  \q(\lab_t| \latent_t) \text{,}
\end{align*}
 which satisfies  the factorization~\eqref{eq:fact-1}
with
\begin{align*}
 \q(\latent_t|\latent_{t-1},\lab_{t-1},\obs_{0:T},\lab_{t+1:T}^{\LL})
 &=\q(\latent_t|\obs_{0:T}) \text{,} \\
 \q(\lab_t|\lab_{t-1},\latent_t,\obs_{0:T},\lab_{t+1:T}^{\LL})
 &=\p(\lab_t|\latent_t) \text{, for all } t \in \U \text{.}
\end{align*}

Our proposed variation of this model considers
$u_t = (\obs_{t-1}, \latent_t)$, \ie~we assume the context $u_t$ depends on the previous observation $\obs_{t-1}$ 
and the current latent variable $\latent_t$.
The associated ELBO~\eqref{eq:elbo_seq2} 
for the VSL model is given by
\begin{align*}
\Qsemi & (\theta,\phi) \overset{\rm mVSL}{=}
    \sum_{t \in \LL} \E_{\q(\latent_{t}| \obs_{0:T})} \left(
     \log\p(\lab_{t}|\latent_{t}) \right) +\\
    % & \E_{\q(\latent_0|\obs_{0:T})} \log \p(\obs_0|\latent_0) -
    % \dkl(\q(\latent_0|\obs_{0:T})||\p(\latent_0)) + \nonumber \\
    & \! \sum_{t=0}^T\!\!
    \Bigg[ \E_{\q(\latent_{t}| \obs_{0:T})} \log \p(\obs_t|\latent_t) 
    -    \dkl(\q(\latent_t|\obs_{0:T})|| \p(\latent_t|\obs_{t-1}, \latent_{t-1} ))  \Bigg] 
    \text{.}
\end{align*}
% where $\obs_{-1} =\latent_{-1} = \emptyset$.
It consists of two terms and 
that the previous assumptions enable us to interpret it as an expectation 
according to $\q(\latent_{0:T}|\obs_{0:T})$. 
Thus, it is not necessary to sample discrete variables according to 
the G-S trick. Moreover, a regularization term $\beta$ can be introduced 
in the second part of the ELBO in 
order to encourage good performance on labeled data 
while leveraging the context of the noisy observations during reconstruction.
While this model simplifies the inference, 
it should be noted that in the generative process, 
the observation $\obs_t$ is conditionally independent of its associated label and may not
be adapted to some applications.
% The VSL simplifies the ELBO by setting 
% $\q(\lab_t| \latent_t) = \p(\lab_t| \latent_t)$, 
% which enables the use of classical variational inference with 
% only continuous latent variables. 
% To further improve performance, they also introduce a regularization term into 
% the loss function that encourages good performance on labeled data 
% while leveraging the context of the noisy image during reconstruction. 

% \subsection{Semi-supervised variational RNN}
% \label{sec:svrnn}
% The generative model used in the SVRNN~\citep{butepage2019predicting} 
% can also be seen as a particular version of the TMC model where the latent variable
% $\latent_t$ consists of the pair $\latent_t=(z'_t, h_t)$. The associated
% transition distribution reads:
% % \begin{align}
% % \label{eq:svrnn}
% %  \p(v_t|v_{t-1}) = \p(\lab_t|v_{t-1}) \p(\latent_t|\lab_t, v_{t-1}) \p(\obs_t|\lab_t,\latent_t,v_{t-1}) \text{,}
% % \end{align}
% \begin{align}
% \label{eq:svrnn}
%  \p(v_t|v_{t-1})  \overset{\rm \scriptscriptstyle SVRNN }{ = }\p(\lab_t|v_{t-1}) \p(\latent_t|\lab_t, v_{t-1}) \p(\obs_t|\lab_t,\latent_t,v_{t-1}) \text{,}
% \end{align}
% where 
% \begin{eqnarray*}
% \p(\lab_t|v_{t-1})&=& \p(\lab_t|h_{t-1})\text{,} \\
% \p(\latent_t|\lab_t,v_{t-1})&=&\delta_{f_{\theta}(z'_t,\lab_t,\obs_t,h_{t-1})}(h_t) \times \p(z'_t|\lab_t, h_{t-1}) \text{,} \\
% \p(\obs_t|\lab_t,\latent_t,v_{t-1})&=& \p(\obs_t|\lab_t, z'_t, h_{t-1}) \text{,}
% \end{eqnarray*}
% and where $f_{\theta}$ is a deterministic, 
% \ie  the variable $\latent'_t$ is a stochastic latent variable 
% and $h_t$ is deterministically given by 
% $h_t = f_{\theta}( z'_t, \obs_t, \lab_t, h_{t-1})$, 
% where $f_{\theta}$ is a function parameterized 
% by a Recurrent Neural Network (RNN), for example. 
% The variational distribution $\q(\latent_{0:T}, \labu|$ $ \obs_{0:T}, \labl)$ satisfies
% the factorization \eqref{eq:fact-2}
% with
% \begin{align*}
%  q(z'_t|\latent_{t-1},\lab_{t},\obs_{0:T},\lab_{t+1:T}^{\LL})=  \q(z'_t| \obs_t, \lab_t, h_{t-1})\text{,} \\
% q(\lab_t|\lab_{t-1},\latent'_{t-1},\obs_{0:T},\lab_{t+1:T}^{\LL})= 
% \q(\lab_t| \obs_t, h_{t-1}) \text{,}
% \end{align*}

% \begin{align}
%     \label{eq:elbo_svrnn}
%     \Qsemi(\theta,\phi) \overset{\rm \scriptscriptstyle SVRNN }{=}& \quad
%     \L^{\LL}(\theta,\phi) + \L^{\U}(\theta,\phi) + J^{\LL}(\theta,\phi) \text{,}
% \end{align}
% where 
% \begin{align}
%     \L^{\LL}(\theta,\varphi) = \sum_{t\in \LL}
%      & \E_{\q(\latent'_t| \obs_t, \lab_t, h_{t-1})} 
%        \log \p(\obs_t|\lab_t, \latent'_t, h_{t-1}) 
%        + \log(\p(\lab_t | h_{t-1} ))    \nonumber \\ 
%     &  - \dkl (\q(\latent'_t|\obs_t, \lab_t, h_{t-1})||
%     p(\latent'_t|\lab_t, h_{t-1} ))  \text{,} \\
%     \L^{\U}(\theta,\varphi) =&  \sum_{t\in \U}
%      \E_{\q(\latent'_t, \lab_t| \obs_t, h_{t-1})} 
%        \log \p(\obs_t|\lab_t, \latent'_t, h_{t-1})     \nonumber \\ 
%     & - \dkl (\q(\latent'_t|\obs_t, \lab_t, h_{t-1})) \nonumber  \\
%     & - \dkl (\q(\lab_t| \obs_t, h_{t-1})|| \p(\lab_t|h_{t-1} )) \text{,}\\
%     J^{\LL}(\theta,\phi) = & \sum_{t\in \LL} 
%     \E_{\tilde{p}(\lab_t, \obs_t)}
%     \log(\p(\lab_t | h_{t-1} ) \q(\lab_t| \obs_t, h_{t-1})) \text{,}
% \end{align}
% where $\tilde{p}(\lab_t, \obs_t)$, for $t\in \LL$, 
% denotes the empirical distribution of the data.
% Their final ELBO does not coincide with \eqref{eq:elbo_seq}. 
% The reason why is that they derive it 
% from the static case~\citep{jang2016categorical} and add 
% a penalization term $J^{\LL}(\theta,\phi)$  that encourages 
% $\p(\lab_t|h_{t-1})$ and $\q(\lab_t| \obs_t, h_{t-1})$ 
% to be close to the empirical distribution of the data.

% \begin{remark}
% Note that since $\Lat_t$ is deterministic given $(z'_t, \obs_t, \lab_t, h_{t-1})$,
% its posterior distribution becomes trivial, and thus 
% there is no need to consider a variational distribution for it.
% \end{remark}



% In this section, we present the results of the proposed models on
% semi-supervised binary image segmentation. Our goal is to recover the 
% segmentation of a binary image $(\Omega=\{\omega_1,\omega_2\})$
% from the noisy observations
% $\obs_{0:T}$ when a partially segmentation $\labl$ is available.
% In particular, $\vartheta(\lab_t; \cdot )$ (resp. $\varsigma(\lab_t;\cdot)$) 
% is set  as a Bernoulli distribution with parameters $\ropy$ (resp. $\roqy$). 
% As for the distribution  $\zeta(\obs_t; \cdot )$ 
% (resp. $\eta(\latent_t;\cdot)$ and  $\tau(\latent_t;\cdot)$), 
% we set it as a Gaussian distribution with parameters 
% $[ \muobs , \diag(\sigobs) ]$ (resp.  $[ \mulatentp , \diag(\siglatentp)]$
%  and $[ \mulatent , \diag(\siglatent) ]$),
% % \begin{align*}
% %     \p(v_t|v_{t-1}) &=  \p(\obs_t|\cdot) \p(\latent_t|\cdot)\p(\lab_t|\cdot) \nonumber\\
% %     \p(\obs_t | \cdot) &= \N(\obs_t; \mu_{px, t}, {\rm diag}(\sigma_{px,t}) ) \\
% %     \p(\latent_t | \cdot) &= \N(\obs_t; \mu_{pz, t}, {\rm diag}(\sigma_{pz,t}) ) \\
% %     \p(\lab_t | \cdot) &= \Ber(\lab_t;  \rho_{py,t}) \text{,}
% % \end{align*}
% where ${\rm diag(.)}$ denotes the diagonal matrix deduced from the values 
% of $\sigma_{\cdot,t}$.
% % and the parameters are $\theta = \{\mu_{pz,t}, \sigma_{pz,t}, \mu_{px,t}, \sigma_{px,t}, \rho_{py,t}\}$  
% % and $\phi = \{ \mu_{qz,t}, \sigma_{qz,t} , \rho_{qy,t}\}$.




% !TEX root = late\obs_avec_réduction_pour_impression_recto_verso_et_rognage_minimum.tex
\chapter{Unsupervised Bayesian classification}
\label{chap:appendix4}


\tocless\section{Partially Pairwise Markov Chains}
\label{sec-rnn}
In this section, we propose a particular class of TMC which aims at extending the PMC model 
proposed in Section 
\ref{sec:generalParam}. The main motivation
underlying this particular model is
to introduce an explicit
dependency on 
the past observations $\obs_{t-1}$ of the 
pair $(\lab_t,\obs_t)$, for all $t$. This
dependency is introduced through
the continuous latent process $\latent_{0:T}$
and enables us to build an 
explicit joint 
distribution $\p(\lab_{0:T},\obs_{0:T})$
which does 
not satisfy the Markovian property of
the PMC \eqref{eq:pmc_intro_uns}. The main difference
with Section \ref{sec-tmc} is that 
$\latent_{0:T}$ is now a conditional deterministic 
latent process.
The resulting model is called a~\gls*{ppmc}.
As we will see, this particular
construction enables us to use directly the Bayesian inference framework developed in Section~\ref{sec:generalParam}.
Finally, since PMCs appears
as particular TMCs, the pretraining of
deep parameterized PPMCs is a direct adaptation of 
Section \ref{sec-deep-tmc}.
%In this section we propose the Partially Pairwise Markov Chains (PPMC)
% ~\citep{pieczynski2005restoring, lapuyade2010unsupervised} 
% combined with Recurrent Neural Networks (RNN)~\citep{mikolov2014learning}. 
% In the PPMC models, the Markovian property of $(\obs_{0:T}, \lab_{0:T})$ 
% does not hold anymore and the conditioning depends on all the previous 
% realizations of the observed r.v. However, $\lab_{0:T}$ given a realization 
% of $\obs_{0:T}$ remains Markovian. On the other hand, a  
% RNN is type of a neural networks which uses sequential data and depends on 
% the previous elements within the sequence. 

\tocless\subsection{Deterministic TMCs}
\label{sec:def-ppmc}
Let us focus on a particular case 
of the TMC \eqref{tmc-trans}-\eqref{tmc-theta-3}. 
From now on, 
we consider that the conditional distribution $\eta$
coincides with the Dirac distribution $\delta$, 
and that function $\pzun$ only depends on $(\latent_{t-1},\obs_{t-1})$.
Thus, $\latent_t$ becomes deterministic given $(\latent_{t-1},\obs_{t-1})$,
\begin{equation}
\label{z-ptmc}
\latent_t=\pzun(\latent_{t-1},\obs_{t-1}) \text{.}
\end{equation}
Each variable $\latent_t$ can be interpreted
as a summary of all the past observations $\obs_{t-1}$. Consequently, it is easy to see
that 
\eqref{tmc-theta-2} and \eqref{tmc-theta-3} now coincide with 
$\p(\lab_t|\lab_{t-1},\obs_{t-1})$ and $\p(\obs_t|\lab_{t-1:t},\obs_{t-1})$, 
respectively, and marginalizing \eqref{eq:tmc_intro} w.r.t.~$\latent_{0:T}$
gives the explicit distribution of
$(\lab_{0:T},\obs_{0:T})$,
\begin{align}
 \label{eq:ppmc_general}
    \p(\lab_{0:T},\obs_{0:T})=\p(\lab_0,\obs_0)
    \prod_{t=1}^T   & \underbrace{\vartheta(\lab_t;\pyun(\latent{t-1:t},\lab_{t-1},\obs_{t-1}))}_{\p(\lab_t|\lab_{t-1},\obs_{t-1})} \times \nonumber \\
    & \quad \underbrace{\zeta(\obs_t;\pxun(\latent{t-1:t},\lab_{t-1:t},\obs_{t-1}))}
_{p(\obs_t|\lab_{t-1:t},\obs_{t-1})} \text{,}
\end{align}
where $\latent_t$ satisfies \eqref{z-ptmc}.
It can noted that $(\lab_{0:T},\obs_{0:T})$ is no longer
Markovian. Remark that this property
is also satisfied by the general TMC \eqref{eq:tmc_intro}. However,
$\p(\lab_{0:T},\obs_{0:T})$ is now available in a closed-form 
expression and the relationship between the 
pair $(\lab_t,\obs_t)$ and the past observations is
fully characterized by the function $\pzun$.


This kind of parameterization has an advantage in 
terms of Bayesian inference. Since $\latent_t$ is
a deterministic function of $(\latent_{t-1},\obs_{t-1})$ (and so of $\obs_{t-1}$, by induction),
the conditional
posterior distribution $\p(\latent_t|\latent_{t-1},\obs_{0:T})$ reduces 
to $\delta_{\pzun(\latent_{t-1},\obs_{t-1})}$.
Consequently, Algorithm~\ref{algo:algo_theta_pmc} and
Algorithm~\ref{algo:algo_hk_pmc} can be directly applied to estimate $\theta$ and $\lab_t$, for all $t$, by introducing the dependency in $\latent_{t-1:t}$ in functions $\pyun$
and $\pxun$ of Section \ref{sec:inference_pmc}.
An alternative point of view is that
when $\latent_t$ is deterministic,
Algorithm~\ref{algo:tmc_elbo_opt} can be seen as a particular instance of Algorithm~\ref{algo:algo_theta_pmc} 
in which we have set $\q(\latent_{0:T}|\obs_{0:T})=\p(\latent_{0:T}|\obs_{0:T})$,
$\beta_1=1$ and $\beta_2=0$. Indeed, for this particular setting the objective function
\eqref{L-approx} coincides with the ELBO but also with the log-likelihood $\p(\obs_{0:T})$. 


\tocless\subsection{Deep PPMCs} 
\label{sec:dppmc}
As previous models, we consider the case where
PPMCs \eqref{eq:ppmc_general} are parameterized with~\gls*{dnns}. 
Such models will be
referred to as~\gls*{dppmc}.  
In the 
particular case of PPMCs, 
%$\pzun(\latent_{t-1},\obs_{t-1})$ is parameterized 
%by a DNN. Since the objective is to model
%long term dependency in the past observations, 
$\pzun$ can be seen as a RNN, \ie~a neural network
which admits the output of the network at previous
time $t-1$ as input at time $t$~\citep{LSTM}.
It is thus possible to directly combine our models
with powerful RNN architectures such as 
Long Short Term Memory (LSTM) RNNs or
Gated Recurrent Unit (GRU) RNNs which have
been developed to introduce
emphasize sequential dependencies.
Note that the gradient of $\pzun$ w.r.t.~$\theta$
can also be computed with a version of the backpropagation algorithm adapted to
RNNs~\citep{LSTM, GRU}.

The pretraining of this deep architecture
is direct. The constrained output layer step
is an application of Paragraph \ref{constraint-tmc}
with $\q(\latent_{0:T}|\obs_{0:T})=\p(\latent_{0:T}|\obs_{0:T})$, $\beta_1=1$ and $\beta_2=0$; so it can be seen as the step
described for PMCs in Paragraph \ref{sec:constrained_archi} up
to the additional input $\latent_{t-1:t}$.

The second step of our pretraining procedure of Paragraph \ref{tmc-unfrozen} can also be simplified. Since in this particular case we have implicitly computed  the optimal conditional variational distribution $\q^{\rm opt}(\latent_t|\latent_{0:t-1},\obs_{0:T})=\delta_{\pzun(\latent_{t-1},\obs_{t-1})}(\latent_t)$, the reparameterized sample $\latent_{t-1:t}$ of Figure~\ref{fig:pretrain_dmtmc}  is now deterministic and coincides directly with the output of $\pzun$, as shown in  Figure~\ref{fig:pretrain_dppmc}. Note that the parameters of $\pzun$ are unfrozen.
The training process is summarized in Algorithm~\ref{algo:algo_train_dppmc}.
%Also, following Remark~\ref{rk:multiclass}, the prodecure can be extended to the multi-class and multi-channel case.


%Now we  propose a new and practical way to take advantage of all the modeling possibilities that are introduced using a deep parameterization. More precisely, we propose to embed a RNN, denoted as $r_{\theta}$, in the model and we call it the Deep-PPMC (DPPMC). The auxiliary latent r.v. $\latent_{0:T}$ takes a new meaning within the DPPMC context: $\latent_t$ is now the deterministic hidden states of $r_{\theta}$, with values in $\mathbb{R}^{d_z}$. These new deterministic r.v. summarize the information contained in the previous observations.
%In this case, the DPPMC model is defined by the equations:
%\begin{eqnarray}
%\label{eq:rnn}
%\latent_t &=& r_{\theta}(\obs_{t-1},\latent_{t-1}) \text{, }\\
%\label{ppmc-theta-1}
%\p(\lab_t|\lab_{t-1},\obs_{1:t-1})&=&\vartheta(\lab_t;\pyun(\lab_{t-1},\latent_{t-1})) \text{, } \\
%\label{ppmc-theta-2}
%\p(\obs_t|\lab_t,\lab_{t-1},\obs_{1:t-1})&=&\zeta(\obs_t;\pxun(\lab_t,\lab_{t-1},\latent_{t-1})) \text{, }
%\end{eqnarray}
%where the two last equations can be related to Equations~\eqref{pmc-theta-1} and~\eqref{pmc-theta-2}.\\


%On the other hand, we can similarly define the Deep-Partially Semi Pairwise Markov Chain (DPSPMC) where $\obs_t$ no longer depends on $\lab_{t-1}$  and the factorization reads
%\begin{eqnarray}
%\latent_t &=& r_{\theta}(\obs_{t-1},\latent_{t-1}) \text{, }\\
%\label{sppmc-theta-1}
%\p (\lab_t|\lab_{t-1},\obs_{1:t-1})&=&\vartheta(\lab_t;\pyun(\lab_{t-1},\latent_{t-1})) \text{, } \\
%\label{sppmc-theta-2}
%\p(\obs_t|\lab_t,\obs_{1:t-1})&=&\zeta(\obs_t;\pxun(\lab_t,\latent_{t-1})).
%\end{eqnarray}
%The DPSPMC model will be used in our experiments because of difficulties in the parameter estimation for the DPPMC model (also seen for the DPMC model) which requires further investigation out of the scope of this article. 
%The graphical representations of the PPMC model and the PSPMC model are given in Figure~\ref{fig:ppmc_graphs}.


%\begin{remark}
%An intermediate linear layer is introduced in the model transforming the output of the RNN layer $\latent_t\in\mathbb{R}^{d_z}$ in a vector of $\mathbb{R}^{d_{\obs}}$ which can be fed into the subsequent parts of the network. This layer does not appear explicitly in the previous equations not to overload the expressions but will be mentionned in the technical details that follow.
%Thanks to this linear layer, we ensure that we are able to work with arbitrary dimensions for the RNN internal states, while keeping conssitency with respect to the DPMC model to derive our pretraining procedure that is described next.
%\end{remark}


%\begin{figure}[h!]
%\centering
%\begin{subfigure}{0.3\textwidth}
%\centering
%\input{tikz/ppmc}
%\caption{PPMC}
%\label{fig:ppmc_graph}
%\end{subfigure}
%\begin{subfigure}{0.3\textwidth}
%\centering
%\input{tikz/pspmc}
%\caption{PSPMC}
%\label{fig:pspmc_graph}
%\end{subfigure}
%\caption{Graphical representations of the PPMC and the PSPMC. The light gray hexagons represent the deterministic variable $\latent_t$. The other graphical elements follow the convention from Figure~\ref{fig:pmc_graphs}.
%}
%\label{fig:ppmc_graphs}
%\end{figure}



\begin{figure}[htb]
  \centering
  \includegraphics[width=0.85\textwidth]{Figures/Graphical_models/d-ppmc.pdf}
  \caption{Graphical and condensed representation of the parameterization of
  $\pyun$ in the DPPMC model. 
  %Note the residual connections between the inputs and the ouput of $\pzun$.
  The dashed arrows represent the fact that some variables are copied. 
  %For clarity, we do not represent the entries of $\pyun$ consisting of products of $\lab_{t-1}$, $\latent_{t-1:t}$ or $\obs_{t-1}$, due to the output layer constraint. %Residual connections between the $\pyun$ layer inputs and the last hidden layer of $\pyun$ are also omitted.
  }
  \label{fig:pretrain_dppmc}
\end{figure}


\begin{algorithm}[htbp!]
  \caption{A general estimation algorithm for deep parameterizations of PPMC models.}
  \label{algo:algo_train_dppmc}
  \begin{algorithmic}[1]
    \Require{$\obs_{0:T}$, the observation}
    \Ensure{$\hat{{\lab}_{0:T}}$, the final classification}
    \Statex{\textbf{Initialization of the output layer of $\pyun$ and $\pxun$}}
    \State Estimate $\theta_{\fr}^*$ and $\hat{\lab}_{0:T}^{\pre}$ with Lines \eqref{line:nondeep1}-\eqref{line:nondeep3} of Algorithm~\ref{algo:algo_train_dpmc}
    \Statex{\textbf{Pretraining of $\theta_{\ufr}$}}
    \State   $\theta_{\ufr}^{(0)} \leftarrow$ ${\rm Backprop}(\hat{\lab}_{0:T}^{\pre},\obs_{0:T},\theta_{\fr}^*, \mathcal{C}_{\f}, \mathcal{C}_{\g})$
    \Statex{\textbf{Fine-tuning of the complete model}}
    \State Update all the models parameters (except $\theta_{\fr}$) with Algorithm~\ref{algo:algo_theta_pmc}
    \State Compute $\hat{\lab}_{0:T}$ with Algorithm~\ref{algo:algo_hk_pmc} 
  \end{algorithmic}
\end{algorithm}



\tocless\subsection{Simulations}
We start again with the same experiments 
as those in Section \ref{sec:pmc}, but we use 
an alternative noise which aims
at introducing longer dependencies on the observations. We now set
\begin{equation}
    \label{eq:longer_noise_eq1}
    \obs_t| \lab_{t},\obs_{t-2:t-1} \sim \mathcal{N}\Big(\sin(a_{\lab_t}+0.2(\obs_{t-1}+\obs_{t-2}));
    \sigma^2\Big).
\end{equation}
where $a_{\omega_1}=0$, $\sigma^2=0.25$ 
and $a_{\omega_2}$ is a varying parameter.
We compare the deep models of Section \ref{sec:generalParam} (DSMPC and DPMC) with their natural extensions
developed in this section (DPSPMC and DPPMC).
 
Figure~\ref{fig:nonlin_corr_ppmc_sce1_a} illustrates the results involving the models we have just introduced. For $\pzun$ we use two independent standard RNNs with ReLU activation function, i.e. $\latent_t=[\latent_t^1,\latent_t^2]=[{\pzun}^1(\latent_{t-1}^1,\obs_{t-1}),
{\pzun}^2(\latent_{t-1}^2,\obs_{t-1})]$; $\pyun$ (resp. $\pxun$) depends on $\latent_{t-1:t}^1$ (resp. $\latent_{t-1:t}^2$). %their output replaces the observation input of $\pyun$ and $\pxun$, respectively.
In this setting, we found that the models worked the best when the dimensions of $\latent_t^1$ and of $\latent_t^2$  is $5$.
%\ie there are $5$ hidden neurons in each RNN. Graphical illustrations are provided in Figure~\ref{fig:nonlin_corr_ppmc_sce1_b}.
We can see that the more general parameterizations embedded in   DPSPMC and DPPMC lead to an improvement of the DPMC models; each DPPMC model leading to a better accuracy than its DPMC counterpart. The ability to model long term dependencies proves to be important to better solve the correlated noise. This experiment illustrates a way to take advantage of a deterministic auxiliary process: by strengthening the sequential dependencies between the hidden random variables.

\input{Figures/nonlin_corr_ppmc_sce1}

\tocless\section{Additional material}
\tocless\subsection{Proof of Proposition \ref{prop:prop1}}
\label{app:prop1}
The ELBO 
\begin{equation}
\label{elbo-proof}
Q(\theta,\phi)= \sum_{\lab_{0:T}} \int \q(\lab_{0:T},\latent_{0:T}|\obs_{0:T}) \log\left(\frac{\p(\lab_{0:T},\latent_{0:T},\obs_{0:T})}{\q(\lab_{0:T},\latent_{0:T}|\obs_{0:T})}\right) {\rm d}  \latent_{0:T} 
\end{equation}
can be decomposed as 
\begin{align}
\label{F-decomposed}
Q(\theta,\phi)   &= \int  \overbrace{\left[\sum_{\lab_{0:T}}  \q(\lab_{0:T}|\latent_{0:T},\obs_{0:T})\right]}^{1} \q(\latent_{0:T}|\obs_{0:T})  \log\left(\frac{\p(\latent_{0:T},\obs_{0:T})}{\q(\latent_{0:T}|\obs_{0:T})}\right) {\rm d}  \latent_{0:T} \nonumber  \\ 
     %&\phantom{AAAAA} 
    & -\int \q(\latent_{0:T}|\obs_{0:T}) \dkl \left(\q(\lab_{0:T}|\latent_{0:T},\obs_{0:T})||\p(\lab_{0:T}|\latent_{0:T},\obs_{0:T})\right) {\rm d}  \latent_{0:T}, \\
    & \leq \int \q(\latent_{0:T}|\obs_{0:T})  \log\left(\frac{\p(\latent_{0:T},\obs_{0:T})}{\q(\latent_{0:T}|\obs_{0:T})}\right) {\rm d}  \latent_{0:T} = Q^{\rm opt} (\theta,\phi)\text{.}
\end{align}
We have $Q(\theta,\phi)= Q^{\rm opt} (\theta,\phi)$ when the KLD term in~\eqref{F-decomposed} is null, 
\ie~when $\q(\lab_{0:T}|\latent_{0:T},\obs_{0:T})=\p(\lab_{0:T}|\latent_{0:T},\obs_{0:T})$.
It remains to compute $Q^{\rm opt} (\theta,\phi)$.
Starting again from~\eqref{elbo-proof} where
we set 
$$\q(\lab_{0:T},\latent_{0:T}|\obs_{0:T})
=\q(\lab_{0:T}|\obs_{0:T})\p(\lab_{0:T}|\latent_{0:T},\obs_{0:T}),$$ 
the Markovian structure of $\p(\lab_{0:T},\latent_{0:T},\obs_{0:T})$
and the additive property of the logarithm function
give the decomposition~\eqref{elbo-1}-\eqref{elbo-3}.

Note that the computation of $Q^{\rm opt} (\theta,\phi)$
via~\eqref{elbo-1}-\eqref{elbo-3}
relies on the distribution~$\p(\lab_{t-1:t}|\latent_{0:T},\obs_{0:T})$.
It can be computed from a direct extension of the intermediate quantities $\alpha_{\theta,k}$ and
$\beta_{\theta,k}$ which are now defined as 
$\alpha_{\theta,k}(\lab_t)=\p(\lab_t,\latent_{0:t},\obs_{0:t})$
and 
$\beta_{\theta,k}(\lab_t)=\p(\latent_{t+1:K},\obs_{t+1:K}|\lab_t,\latent_t,\obs_t)$. Their computation is similar to 
\eqref{eq:alpha} and~\eqref{eq:beta}, except
that they now involve the transition 
$p(\lab_t,\latent_t,\obs_t|\lab_{t-1},\latent_{t-1},\obs_{t-1})$
rather than $p(\lab_t,\obs_t|\lab_{t-1},\obs_{t-1})$.

\vspace{0.7cm}
\tocless\subsection{Detailed error rates for experiments \ref{sec:realworld_mct} and \ref{sec:realworld_har} }
\label{app:error_rates}
This section provides the full results of the real world experiments described in 
Section~\ref{sec:realworld}.
Table~\ref{table:microct_scores_all} 
provides a comprehensive comparison of the error rates achieved by
different generalized Triplet Markov Chains in the context of
unsupervised image segmentation. The table presents detailed error rates for ten
micro-computed tomography slices, evaluated across four models: HMC-IN, di-MTMC,
MTMC, and DMTMC.

\newpage
\begin{table}[htb!]
\centering
\setlength\tabcolsep{6pt}
\begin{tabular}{ccccc}
\toprule
Slice & HMC-IN & di-MTMC & MTMC & DMTMC \\\toprule
\texttt{A}&	$8.5$&	$8.5$&	$6.5$&	$5.4$\\ \midrule
\texttt{B}&	$10.9$&	$10.9$&	$8.7$&	$6.5$\\ \midrule
\texttt{C}&	$6.9$&	$7.0$&	$6.0$&	$5.2$\\ \midrule
\texttt{D}&	$10.0$&	$10.1$&	$8.3$&	$6.1$\\ \midrule
\texttt{E}&	$6.5$&	$6.3$&	$6.2$&	$5.4$\\ \midrule
\texttt{F}&	$11.5$&	$11.5$&	$10.8$&	$9.3$\\ \midrule
\texttt{G}&	$4.6$&	$4.6$&	$3.9$&	$3.7$\\ \midrule
\texttt{H} & $8.6$ & $8.6$ & $8.5$ & $7.7$\\ \midrule
\texttt{I} & $11.5$ & $11.5$ & $10.1$ & $9.2$ \\ \midrule
\texttt{J}& $7.2$ & $7.2$ & $6.9$ & $6.5$\\ \midrule \midrule
Average & $8.6$ & $8.6$ & $7.6$ & $\pmb{6.5}$\\
\bottomrule
\end{tabular}
\caption{Detailed error rates (\%) in unsupervised image segmentation with all the generalized TMCs assessed on ten micro-computed tomography slices. See Section \ref{sec:realworld_mct}.}
\label{table:microct_scores_all}
\end{table}

\begin{table}
\centering
\scriptsize
\setlength\tabcolsep{6pt}
\begin{tabular}{cccccccc}
\toprule
 Data & HMC-IN & SPMC & DSPMC & DPSPMC & PMC & DPMC & DPPMC \\\toprule
\makecell{\texttt{acc\_exp01\_user01}}&$15.0$&$29.0$ &$20.9$ & $17.8$& $20.9$& $19.9$& $20.1$\\ \midrule
\makecell{\texttt{acc\_exp02\_user01}} &$16.0$&$20.3$ &$13.3$ &$12.4$ &$13.1$ & $18.2$& $14.6$\\ \midrule
\makecell{\texttt{acc\_exp03\_user02}} &$25.7$&$16.1$ &$11.7$ &$9.8$ &$11.7$ & $5.6$& $12.7$\\ \midrule
\makecell{\texttt{acc\_exp04\_user02}} &$24.3$&$15.2$&$10.9$ &$11.5$ &$10.9$ & $5.6$& $11.7$\\ \midrule
\makecell{\texttt{acc\_exp05\_user03}}&$21.1$ &$28.8$ &$23.2$ &$15.3$ &$22.4$ & $22.7$& $23.4$\\ \midrule
\makecell{\texttt{acc\_exp06\_user03}}&$26.3$ &$15.6$ &$12.9$ &$11.0$ &$12.3$ & $19.9$& $14.2$\\ \midrule
\makecell{\texttt{acc\_exp07\_user04}}&$23.3$ &$19.2$ &$14.4$ &$13.4$ &$23.3$ & $21.9$& $14.6$\\ \midrule
\makecell{\texttt{acc\_exp08\_user04}}&$26.3$ &$17.1$ &$13.1$ &$12.3$ &$12.9$ & $10.4$& $12.9$\\ \midrule
\makecell{\texttt{acc\_exp09\_user05}}&$24.3$ &$19.0$ &$14.9$ &$12.3$ &$14.7$ & $12.3$& $15.5$\\ \midrule
\makecell{\texttt{acc\_exp10\_user05}}&$25.8$ &$48.3$ &$24.5$ &$25.4$ &$24.3$ & $27.6$& $24.3$\\ \midrule
\makecell{\texttt{acc\_exp11\_user06}}&$27.7$ &$15.1$ &$12.7$ &$10.9$ &$12.7$ & $12.6$& $11.9$\\ \midrule
\makecell{\texttt{acc\_exp12\_user06}}&$36.9$ &$43.5$ &$42.8$ &$43.2$ &$42.8$ & $42.1$& $41.5$\\ \midrule
\makecell{\texttt{acc\_exp13\_user07}}&$26.1$ &$18.2$ &$14.6$ &$16.5$ &$14.4$ & $13.9$& $13.9$\\ \midrule
\makecell{\texttt{acc\_exp14\_user07}}&$26.0$ & $18.5$&$14.5$ &$21.9$ &$14.4$ & $18.9$& $13.6$\\ \midrule
\makecell{\texttt{acc\_exp15\_user08}}&$22.2$ &$16.7$ &$12.9$ &$9.0$ &$12.8$ & $10.0$& $13.0$\\ \midrule
\makecell{\texttt{acc\_exp16\_user08}}&$26.2$ &$19.4$ &$16.5$ &$14.7$ &$16.5$ & $15.8$& $14.3$\\ \midrule
\makecell{\texttt{acc\_exp17\_user09}}&$25.6$ &$17.0$ &$13.1$ &$17.9$ &$12.9$ & $14.0$& $11.0$\\ \midrule
\makecell{\texttt{acc\_exp18\_user09}}&$24.8$ & $13.8$&$10.9$ &$11.3$ &$10.8$ & $8.1$& $12.3$\\ \midrule
\makecell{\texttt{acc\_exp19\_user10}}&$26.1$ &$13.3$ &$10.4$ &$21.4$ &$10.3$ &$8.0$ & $15.2$\\ \midrule
\makecell{\texttt{acc\_exp20\_user10}}&$34.9$ &$22.1$ &$27.2$ &$26.8$ &$27.1$ & $29.1$& $25.9$\\ \midrule
\midrule
Average & $25.2$ &$21.3$ &$16.8$ & $\pmb{16.7}$&$17.1$ & $16.8$& $16.8$\\
\bottomrule
\end{tabular}
\caption{Detailed Error rates (\%) in the binary clustering of the first twenty raw entries of the HAPT dataset \citep{reyes2016transition}.}
\label{table:har_scores_all}
\end{table}


\color{black}
\tocless\subsection{Additional experiments}
In this section, we provide additional experiments. The first one consists 
in introducing experiments in the case where the number of classes is $C>2$.
In the second one, we study experimentally the impact of the variational
distribution for the TMC model of  Scenario~\eqref{scenario-2-tmc}.


\subsubsection{Multi-class extension}
In this section, we illustrate an extension of our models when $C>2$. For $C>2$, Eq.~\eqref{param-1} becomes a vector of softmax function,
\begin{equation}
    \f(\lab_{t-1},\obs_{t-1})=\left[\frac{e^{b_{\omega_1,\lab_{t-1}}}}{\sum_{j=1}^C e^{b_{\omega_j,\lab_{t-1}}}}, \cdots,\frac{e^{b_{\omega_C,\lab_{t-1}}}}{\sum_{j=1}^C e^{b_{\omega_j,\lab_{t-1}}}} \right] \text{,}
\end{equation}
while the distribution $\lambda(\lab_t,\f(\lab_{t-1},\obs_{t-1}))$ coincides
with the Categorical distribution whose parameters 
are described by $\f(\lab_{t-1},\obs_{t-1})$, \ie
\begin{equation}
  \p(\lab_t=\omega_i |\lab_{t-1},\obs_{t-1})\overset{\rm HMC}{=}\p(\lab_t=\omega_i |\lab_{t-1})= \frac{e^{b_{\omega_i,\lab_{t-1}}}}{\sum_{j=1}^C e^{b_{\omega_j,\lab_{t-1}}}}  \text{.}
\end{equation}

Figure~\ref{fig:mc_segmentation} displays an extension of Scenario~\eqref{eq:noise_eq1} to the multi-class case. One can note that the relative performances of the models remain similar to those established in the article.
 
 
\begin{figure}[H]
    \centering
\renewcommand{\arraystretch}{0.4}

\begin{tabular}{Z{0.08\columnwidth} Z{0.15\columnwidth} Z{0.15\columnwidth}
Z{0.15\columnwidth}Z{0.15\columnwidth}Z{0.15\columnwidth}}
 & $\lab_{0:T}$ & $\obs_{0:T}$ &{HMC-IN}&{SPMC}&DSPMC \\
$C=3$&
\includegraphics[width=0.15\columnwidth, cfbox=black 1pt 0pt]
{Figures/ress/multi_class/cattle_img.png}&
 \includegraphics[width=0.15\columnwidth, cfbox=black 1pt 0pt]
{Figures/ress/multi_class/cattle_X.png}
&
 \includegraphics[width=0.15\columnwidth, cfbox=black 1pt 0pt]
{Figures/ress/multi_class/cattle_hmcin.png}
&
 \includegraphics[width=0.15\columnwidth, cfbox=black 1pt 0pt]
{Figures/ress/multi_class/cattle_spmc.png}
&
 \includegraphics[width=0.15\columnwidth, cfbox=black 1pt 0pt]
{Figures/ress/multi_class/cattle_dspmc.png}\\
$C=3$&
\includegraphics[width=0.15\columnwidth, cfbox=black 1pt 0pt]
{Figures/ress/multi_class/cattle2_img.png}&
 \includegraphics[width=0.15\columnwidth, cfbox=black 1pt 0pt]
{Figures/ress/multi_class/cattle2_X.png}
&
 \includegraphics[width=0.15\columnwidth, cfbox=black 1pt 0pt]
{Figures/ress/multi_class/cattle2_hmcin.png}
&
 \includegraphics[width=0.15\columnwidth, cfbox=black 1pt 0pt]
{Figures/ress/multi_class/cattle2_spmc.png}
&
 \includegraphics[width=0.15\columnwidth, cfbox=black 1pt 0pt]
{Figures/ress/multi_class/cattle2_dspmc.png}\\
$C=5$ &
\includegraphics[width=0.15\columnwidth, cfbox=black 1pt 0pt]
{Figures/ress/multi_class/cattle3_img.png}&
 \includegraphics[width=0.15\columnwidth, cfbox=black 1pt 0pt]
{Figures/ress/multi_class/cattle3_X.png}
&
 \includegraphics[width=0.15\columnwidth, cfbox=black 1pt 0pt]
{Figures/ress/multi_class/cattle3_hmcin.png}
&
 \includegraphics[width=0.15\columnwidth, cfbox=black 1pt 0pt]
{Figures/ress/multi_class/cattle3_spmc.png}
&
 \includegraphics[width=0.15\columnwidth, cfbox=black 1pt 0pt]
{Figures/ress/multi_class/cattle3_dspmc.png}
\end{tabular}
    \caption{Multi-class segmentations with the HMC-IN, SPMC and DSPMC models. The noisy image is simulated according to Eq.~\eqref{eq:noise_eq1} from the paper with $a_{\omega_1}=0, a_{\omega_2}=1$ and $a_{\omega_3}=2$ for the top row, $a_{\omega_1}=0, a_{\omega_2}=0.5$ and $a_{\omega_3}=1$ for the middle row and $a_{\omega_1}=0, a_{\omega_2}=0.75, a_{\omega_3}=1.5, a_{\omega_4}=2.25$ and $a_{\omega_5}=3$ for the bottom row. Note that the segmentation can be affected by label switching, which is another different problem out of scope of the article.}
    \label{fig:mc_segmentation}
\end{figure}
 
\subsubsection{Influence of the variational distribution}
\label{app:var_distrib}
In this section, we performed new simulations in the case of the non-stationary noise experiment (Scenario~\eqref{scenario-2-tmc}) with $3$ different variational distributions for the DMTMC model, namely:
    \begin{equation*}
    q_{\phi}^1(\latent_{0:T}|\obs_{0:T})=\prod_{t=1}^T
    \mathcal{N}(\latent_t;\nu_{\phi}(\obs_t)),    
    \end{equation*}
    \begin{equation*}
        q_{\phi}^2(\latent_{0:T}|\obs_{0:T})=\prod_{t=1}^T
    \mathcal{N}(\latent_t;\nu_{\phi}(\latent_{t-1},\obs_t)), 
    \end{equation*}
    and
    \begin{equation*}
        q_{\phi}^3(\latent_{0:T}|\obs_{0:T})=\prod_{t=1}^T
    \mathcal{N}(\latent_t;\nu_{\phi}(\latent_{k-2}, \latent_{t-1},\obs_t)).
    \end{equation*}
Figure~\ref{fig:nonstatio_noise_additional} summarizes this additional experiment.

It can be observed that  the choice of the variational distribution does not lead to significant changes in the results as compared to, for example, the Mean-Field variational distribution with fully independent random variables. However, adding more dependencies led to worse results probably because of the complexity of the noise to estimate.
\begin{figure}
\centering
\begin{tikzpicture}[spy using outlines={rectangle, lens={scale=1.47},
    connect spies}]
\begin{axis}[
title=Scenario $(55)$,
grid=major,
axis x line=bottom, axis y line=left, width=0.5\textwidth, height=5.5cm,
compat=1.10,
ymin=0, ymax=0.5,
label style={font=\small}, %tick label style={font=\tiny},
legend style={font=\small},
%y tick label style={/pgf/number format/.cd, fixed, fixed %zerofill, precision=2,
%/tikz/.cd},
legend columns=1, ytick distance=0.05, legend
style={at={(1.55,0.8)},name=leg1,font=\footnotesize, draw=black},ytick distance=0.1, xtick distance=.2,
xmin=1.5,xmax=2.5, xshift=-0.cm, xlabel={$a_{\omega_2}$}, ylabel={Error rate (MPM)}] 


\addplot+[orange, thick, mark options={scale=0.7}] table[x index=0, y
index=1, col
sep=comma, skip first n=1]{Figures/data/nonstatio_noise_add/summary.csv};
\addlegendentry{with $q_{\phi}^1(\latent_{0:T}|\obs_{0:T})$}
\addplot+[teal, thick, mark options={scale=0.7}] table[x index=0, y
index=2, col
sep=comma,  skip first n=1]{Figures/data/nonstatio_noise_add/summary.csv};
\addlegendentry{with $q_{\phi}^2(\latent_{0:T}|\obs_{0:T})$}
\addplot+[brown, thick, mark options={scale=0.7}] table[x index=0, y
index=3, col
sep=comma,  skip first n=1]{Figures/data/nonstatio_noise_add/summary.csv};
\addlegendentry{with $q_{\phi}^3(\latent_{0:T}|\obs_{0:T})$}
\end{axis}
\end{tikzpicture}

\caption{Error rate from the unsupervised segmentations of Scenario~\eqref{scenario-2-tmc}. Results are averaged on all the \emph{dog}-type images from the database.}
\label{fig:nonstatio_noise_additional}
\end{figure}
\color{black}



% % \section{Super-resolution via Variational Auto-Encoders}
% \label{anex:perspectives_medical_images}
% % \subsection{Super-resolution via Variational Auto-Encoders}
% Super-resolution Variational Auto-Encoders architectures have been proposed
% by~\citep{gatopoulos2020super}, 
% which requires a dataset of high-resolution images and their corresponding
% low-resolution images. 
% \begin{figure}[htb]
%     \centering
%     \includegraphics[width=0.6\textwidth]{Figures/sr_vae_sup.PNG}
%     \caption{Stochastic dependencies of the proposed model. Our approach
%     takes advantage of a compressed representation y of the data in the
%     variational part, that is then utilized in the super-resolution in the
%     generative part.
%     Figure taken from~\citep{gatopoulos2020super}}
%     \label{fig:srvae_network_sup}
% \end{figure}

% On the one hand, an unsupervised real image denoising 
% and Super-Resolution approach via Variational
% AutoEncoder (dSRVAE) was proposed by~\citep{liu2020unsupervised}.
% The architecture of the proposed model is shown in Figure~\ref{fig:srvae_network}.

% \begin{figure}[htb]
%     \centering
%     \includegraphics[width=1\textwidth]{Figures/dSRVAE.png}
%     \caption{Complete structure of the proposed dSRVAE model. It includes
%     Denoising AutoEnocder (DAE) and Super-Resolution SubNetwork (SRSN). The
%     discriminator is attached for photo-realistic SR generation.
%     Figure taken from~\citep{liu2020unsupervised}}
%     \label{fig:srvae_network}
% \end{figure}
% !TEX root = late\obs_avec_réduction_pour_impression_recto_verso_et_rognage_minimum.tex
\chapter{Medical image segmentation}
\label{chap:appendix5}

\tocless\section{Protocol and Database Construction}
\label{anex:protocol_database}
We describe the protocol and the database construction as follows.
\begin{itemize}
    \item \textbf{Patients and explant recovery:}
    \begin{enumerate}
        \item Patients scheduled for transfemoral amputation in the 
        vascular surgery department are informed of the
        procedure and asked not to object during the pre-operative 
        consultation scheduled for the day before the operation.
        \item Management of the patient in
        accordance with current practice, with an injected preoperative CT scan.
        \item During routine transfemoral amputation surgery: recovery of the
        sample (portion of the amputation including the damaged artery) to be
        analysed after rinsing the artery lumen.
        \item Recovery of the explant with macroscopic analysis at GEPROMED and
        storage on their premises. The subjects' participation in the research
        ends after the surgery.
    \end{enumerate}
    \item \textbf{GEPROMED:} Ex-vivo microscanner imaging at GEPROMED.
    % , see Figure~\ref{fig:data_availability_all}.
    \begin{enumerate}
        % \item Placement of the explant on a special support.
        \item The microCT 3D images of the arteries are acquired at the CVPath Institute,
        Inc. (Gaithersburg, MD, USA).
        \item Histology are performed on the specimens 
    as described in~\cite{torii2019histopathologic}. \label{item:histo}
        \item Co-registration are subsequently performed manually
        between the microCT images and the histologic slices obtained 
        during the steps described above.
        The result of this step consists in pairs of data: the
        micro CT 2D image with its histologic ground truth. \label{item:co_registration}
        \item  An expert annotate the micro CT images using the histologic 
        ground truths in the GIMP software7. \label{item:annotation}
        They are 6 classes: 
        \begin{itemize}
            \item  soft tissue (ST): soft tissue, formaldehyde, thrombus, fibrous
            plaque.
        	\item fatty tissue (FT): fatty tissue, lipid pool.
        	\item sheet calcification (SC).
        	\item nodular calcification (NC).
        	\item specimen holder (SH).
        	\item background (Ba)
        \end{itemize}
        % gangloff2020probabilistic
        \item Collection and analysis of \textit{dicom} data (CT images): 
        Centerline information is available for the CT images.
        The centerline is given by an expert and is used to select the interest region
        of the images, since the lesion represents a small area of the artery,
        \ie~of the CT image. 
        \label{item:correlation_ct}
        \item Correlation between  CT scanner  and micro CT scanner  using standard
        references (collaterals, branches and specific lesions).\label{item:correlation}
    \end{enumerate}
\end{itemize}



\tocless\section{Previous Work}
% \paragraph{Segmentation of micro CT images:}  
\cite{gangloff2020probabilistic} has proposed different methods to segment the
micro CT images of the SAFP. They used pairs of micro CT images
histologically annotated micro-CT images, which constituted the training set. 
In other words, they mainly used the pairs of information obtained until 
Step~\ref{item:annotation} of the protocol described above.
The additional information
obtained after that step, which is related to the correlation between the CT scanner
and the micro CT scanner,  were not exploited from a segmentation point of view.
The authors have used a CNN based on the U-Net architecture~\citep{ronneberger2015u}
to segment the micro CT images into 6 classes.
We describe this technique with more details in 
Subsection~\ref{sec:unet}.
The number of classes was selected based on the histopathologists' advice.
% It has then been decided with the histopathologists that 6 classes were of
% interest to develop a first version of the algorithm
Notice that their work is a 2D supervised segmentation, 
since the pairs (micro CT image, ground truth) are only available for some slices
of the 3D micro CT image.
%  are only possible 
% for the slices where the histological ground truth is available.


The measure of performance is the Dice score, which is a measure of the similarity
between two sets of data. In the context of image segmentation,
for example, the Dice score can be used to evaluate the similarity between a
predicted segmentation mask and the ground truth segmentation mask.
The Dice score is defined as follows:
\begin{equation}
    \label{eq:dice_score}
    \text{Dice score} = \frac{2|A \cap B|}{|A| + |B|} \text{, }
\end{equation}
where $A$ and $B$ are the two sets of data.
This score is a number between 0 and 1, where 0 indicates no similarity and 1
indicates perfect similarity.

% The soft tissue, background, and specimen holder classes are well segmented 
% with Dice scores $> 0.86$.
% One remarkable fact is that the calcification classes 
% are segmented with Dice scores 
% $0.85$ and $0.64$ for the sheet calcification, and nodular calcification, respectively. 
% This is a good result considering the complexity of the images where
% the differentiation is almost impossible by the naked eye, even for an expert.

% \begin{remark}
%     The Dice score is a measure of the similarity between two sets of data. 
%     In the context of image segmentation,
%     for example, the Dice score can be used to evaluate the similarity between a
%     predicted segmentation mask and the ground truth segmentation mask.
% \end{remark}

% \paragraph{3D segmentation:} 
% \cite{gangloff2020probabilistic} also proposed a 3D 
% segmentation on the test SAFP, which is based on the 2D segmentation
% and probabilistic post-processing approaches. 
% A 3D segmentation of stents is also proposed.
% We will not present the details of the 3D segmentation, since it is not the 
% principal interest of this work.

% We take this work as a starting point for our study.
% In the next section, we describe the objective of our work and the challenges
% that we face. We also present the workflow that we have devised to address them.


% In summary, the main limitation of the previous work is that 
% it is still not possible to segment the CT images of the SAFP, 
% which are of low resolution and the annotations 
% (ground truth) are only available for the micro CT images.
% This problem is the motivation for the work presented in this chapter.





% \begin{figure}[htb!]
%     \centering
%     \includegraphics[width=1\textwidth]{Figures/Medical_images/all_data.jpg}
%     \caption{Atherosclerotic characterisation of fibrous (A), calcific (B),
%     and lipid (C) plaques of the popliteal artery with co-registration between
%     computed tomography (CT) angiography  and microCT 
%     with histology~\citep{kuntz2021co}.}
%     \label{fig:data_availability_all}
% \end{figure}




\tocless\section{Super resolution}
\tocless\subsection{Super resolution via VAEs}
Super-resolution (SR) techniques, while sharing a common objective of enhancing
image resolution, employ a variety of methods to achieve this goal. 
% In this
% section, we focus on a specific SR algorithm based on 
% VAEs~\citep{kingma2014} (see Chapter~\ref{chap:main_concepts})
% VAEs, a class of generative models rooted in the variational inference framework,
% offer to learning both a latent-variable model and its
% corresponding inference model.
% \subsection{Super-resolution via Variational Auto-Encoders}
Super-resolution VAEs architectures have been proposed
by~\citep{gatopoulos2020super}, 
which requires a dataset of high-resolution images and their corresponding
low-resolution images. 
\begin{figure}[htb!]
    \centering
    \includegraphics[width=0.6\textwidth]{Figures/sr_vae_sup.PNG}
    \caption{Stochastic dependencies of the proposed model. Our approach
    takes advantage of a compressed representation y of the data in the
    variational part, that is then utilized in the super-resolution in the
    generative part.
    Figure taken from~\citep{gatopoulos2020super}}
    \label{fig:srvae_network_sup}
\end{figure}
On the one hand, an unsupervised real image denoising 
and Super-Resolution approach via Variational
AutoEncoder (dSRVAE) was proposed by~\citep{liu2020unsupervised}.
The architecture of the proposed model is shown in Figure~\ref{fig:srvae_network}.

\begin{figure}[htb!]
    \centering
    \includegraphics[width=1\textwidth]{Figures/dSRVAE.png}
    \caption{Complete structure of the proposed dSRVAE model. It includes
    Denoising AutoEnocder (DAE) and Super-Resolution SubNetwork (SRSN). The
    discriminator is attached for photo-realistic SR generation.
    Figure taken from~\citep{liu2020unsupervised}}
    \label{fig:srvae_network}
\end{figure}

SR models based on VAEs are a branch of
image processing focused on generating high-resolution images from
low-resolution ones~\citep{appati2023deep, liu2020unsupervised,
gatopoulos2020super, hyun2020varsr}. 
These models have gained popularity due to their
effectiveness in modeling high-resolution images, traditionally dominated by
autoregressive models~\citep{li2001new,joshi2005learning}, 
and GANs~\citep{chira2022image}.
Images generated by VAEs present, in general, blurry details, which is a
limitation of these models. 
% \cite{gatopoulos2020super} proposed a 
% SR algorithm based on VAEs, which consists  of a DenseNet-based encoder, 
% a DenseNet-based decoder, and  a flow-based prior to generate crisp images.
% This model is still trained using the log-likelihood objective.
% They have introduced a downscaled representation of the image as a random variable,
% and utilize it in a SR manner. Then the model uses three latent variables, 
% where one is the downscaled version of the original image.

% Original ground truth images are commonly used to train SR algorithms. 
% \cite{liu2020unsupervised} proposed a SR algorithm based on VAEs that does not require
% ground truth images. This is a significant advantage since ground truth images
% are not always available. The proposed model consists of a joint image denoising 
% and SR model via Variational AutoEncoder in an unsupervised manner.
% Denoising is performed by a conditional VAE, where the encoder compresses 
% the clean image to learn the latent variables. Then the decoder learns to 
% extract the noise from the noisy image and with a latent variable.
% Once a clean image is obtained, the SR is performed by a
% SR subnetwork (SRSN). The basic structure of the SRSN is a set of
% hierarchical residual blocks. The denoised image is initially up-sampled 
% to the desired dimension by bicubic interpolation.
% In addition, the model is trained with a
% discriminator to improve the quality of the generated images.
% Since the model is trained in an unsupervised manner, the quality of the
% generated HR images is evaluated using a strategy based on the backprojection theory. 

\paragraph*{Applicability to medical images: }
\label{sec:applicability_sr}
Since the  LR-CT images we have are small ($5\times 5$ to $12\times 12$) pixels  
and the details are important for the segmentation, 
% (Step~\ref{item:super_resolution}).
the factor of up-scaling is important.
In addition, the images are noisy, which is a common problem in medical images.
The SR algorithms based on VAEs presented above are promising, however,
they are not suitable for our problem due to the up-scaling factor, and the
noise in the images. The SR algorithm based on VAEs presented 
in~\citep{gatopoulos2020super} proposes 
a factor of up-scaling of 2, which is not enough for our problem. Moreover, 
the input for training is an HR (micro CT) center line of the artery, 
which is not yet available. 
From a point of view of the applicability to medical images, 
we won't be able to use this algorithm. Although, it is a promising algorithm
for future researches in this field which can be related to the  work 
we have presented in previous chapters.
% On the other hand, the SR algorithm based on VAEs in~\citep{liu2020unsupervised}
% proposes a factor of up-scaling of 4, which is more suitable for our problem.
% \katy{Pending to check the factor of up-scaling and the reason why it is not suitable for our problem.}

%------------------------
% Podemos usar varias veces este algoritmo, es decir la salide meterla nuevamente
% en el algoritmo para obtener una imagen de mayor resolucion. 
% Esto es importante para nuestro problema, ya que las imagenes son pequenas, pero 
%  eso no aumenta la resolucion de los detalles, solo de la imagen.
%------------------------

% Different approaches are available for SR via VAEs such as 

% % Our interest in this section is to present the SR algorithm based on VAEs
% Very Deep Variational Autoencoder for SR (VDVAE-SR)~\citep{chira2022image}
% VDVAE-SR , a new model that aims to exploit the most recent deep VAE
% methodologies to improve upon the results of similar models. VDVAE-SR tackles
% image SR using transfer learning on pretrained VDVAEs. The
% presented model is competitive with other state-of-the-art models, having
% comparable results on image quality metrics.

% In particular, we can use a VAE to learn a mapping from the LR image to the HR image.
% The LR images act as inputs to the VAE, and the HR images act as the targets.
% The Encoder maps the LR images to a latent space, and the Decoder reconstructs
% the high-resolution images from the latent space representation.


% Advantages: probabilistic framewrok
% distribution of HR images


% Handling uncertaintyGeneration multiple outpuets 
% : flexibility

% Disavtanges: Complexity advantages 
% DeCarefull trainingReconstructuin details
% Detail preservation accurate diagnosis 

% reco
% nstructoiAccuracy interpretable 
% careful validation 

% Combination of both
\tocless\subsection{Laplacian pyramid SR network}
%  proposed a SR algorithm based on the Laplacian pyramid

The Laplacian Pyramid Super-Resolution Network (LapSRN), presented in
~\citep{lai2017deep}, is a 
method for single-image super-resolution using CNNs. 
It progressively reconstructs the sub-band residuals of high-resolution (HR) images
without requiring bicubic interpolation, which reduces computational complexity.
Figure~\ref{fig:lsrn_network} 
shows the LapSRN architecture where we can see the different layers of the network.
LapSRN directly extracts feature maps from low-resolution images and
progressively predicts sub-band residuals in a coarse-to-fine manner using
transposed convolutional layers for upsampling. It is trained end-to-end with
deep supervision using a robust Charbonnier loss function, which improves
accuracy and reduces visual artifacts. LapSRN stands out for its fast processing
speed, accuracy, and ability to generate multi-scale predictions in one
feed-forward pass, making it suitable for resource-aware applications.

\begin{remark}
    The Charbonnier loss function is a variant of the L1 loss function,
    commonly used in image processing and computer vision tasks, particularly
    for regression problems like image super-resolution.
    This loss function is defined as follows:
    \begin{equation*}
        \mathcal{L}_{\text{Charbonnier}}(x) = \sqrt{(y_{pred}-y_{true})^2 + \epsilon^2}
    \end{equation*}
    where $y_{pred}$ is the predicted value, $y_{true}$ is the ground truth value,
    and  $\epsilon$ is a small constant which ensures numerical stability and
    prevent division by zero.
    The inclusion of the $\epsilon$ term allows the Charbonnier loss to be less 
    sensitive to outliers than the $L_2$ loss, while being smoother and less abrupt 
    than the $L_1$ loss, which can be beneficial in training neural networks for 
    tasks like image super-resolution.
\end{remark}

% details multiple scales
% segmentation after fidelity anatomic structures
% interpretaibili

\begin{figure}[htb!]
    \centering
    \includegraphics[width=0.7\textwidth]{Figures/LSRN_network.jpg}
    \caption{Red arrows indicate convolutional layers. Blue arrows indicate
    transposed convolutions (upsampling). Green arrows denote element-wise
    addition operators, and the orange arrow indicates recurrent layers. 
    Figure taken from~\citep{lai2017deep}}
    \label{fig:lsrn_network}
\end{figure}

\cite{lai2017deep} have compared LapSRN with other classic SR algorithms
such as Super Resolution Convolutional Neural Network (SRCNN)~\citep{dong2015image},
Fast Super Resolution Convolutional Neural Network (FSRCNN)~\citep{dong2016accelerating},
Very Deep Convolutional Neural Network (VDSR)~\citep{kim2016accurate}, 
and some other state-of-the-art SR algorithms.
They have shown that LapSRN outperforms these
algorithms in terms of accuracy and visual quality.
% The authors have also proposed the Multi-scale LapSRN (MS-LapSRN) 
% in~\citep{lai2018fast}, which can reconstruct sharper and more accurate SR images.


\tocless\section{Probabilistic U-Net architecture}
% \paragraph{Probabilistic U-Net Architecture - }
The central component of the Probabilistic U-Net is the latent space, which is
the key to modeling the ambiguity of the segmentation problem.
The latent space is a low-dimensional space where the segmentation variants are
represented as probability distributions.
A sample from the latent space 
is drawn and then injected into the U-Net to produce the
corresponding segmentation map $S$, defined as follows:
\begin{equation*}
    S(\obs, \latent) = f_{comb}(f_{U-Net}(\obs),\latent) \text{.}
\end{equation*}
Here, $f_{U-Net}$ is the U-Net architecture and  $f_{comb}$
is the function that combines the information obtained from the latent space and
the output of the U-Net.


Figure~\ref{fig:proba_unet},
(a) represents the sampling process, where 
a sample is  drawn  from the prior distribution $p(\latent|\obs)$. 
Next,  the segmentation map $S$ is obtained.
%  by combining the sample into the U-Net.
Figure~\ref{fig:proba_unet}(b) 
represents the training process, where the model is trained with the
standard training procedure for conditional VAEs. The ELBO objective function
for the Probabilistic U-Net reads

\begin{equation*}
    \mathcal{Q}_{\text{P-U-Net}}(\obs,\lab) = 
    \mathbb{E}_{q_{\phi}(\latent|\obs, \lab)}
    \left[\log p_{\theta}(\lab|S(\obs, \latent))\right] 
    -\beta\; \text{KLD}\left(q_{\phi}(\latent|\obs,\lab)||p(\latent|\obs)\right) 
    \text{,}
\end{equation*}
where $\beta$ is a hyperparameter that controls the trade-off between the
reconstruction loss and the KLD term~\citep{higgins2017beta}.
The reconstruction loss is the cross-entropy between the segmentation map $S$ and
the ground truth $\lab$.
The KLD term is the Kullback-Leibler divergence between the
approximate posterior $q_{\phi}(\latent|\obs,\lab)$ and the prior
$p(\latent|\obs)$.

% A cross-entropy loss penalizes differences between S and Y (the cross-entropy
% loss arises from treating the output S as the parameterization of a pixel-wise
% categorical distribution Pc). Additionally there is a Kullback-Leibler
% divergence DKL(Q||P ) = Ez∼Q [log Q - log P ] which penalizes differences
% between the posterior distribution Q and the prior distribution P . Both losses
% are combined as a weighted sum with a weighting factor β, as done in [25]:

% A cross-entropy loss penalizes differences between S and Y (the cross-entropy
% loss arises from treating the output S as the parameterization of a pixel-wise
% categorical distribution Pc).
% The networks are trained with the standard training procedure for conditional
% VAEs (Fig. 1b), i.e.\ by minimizing the variational lower bound (Eq. 4). The
% main difference with respect to training a deterministic segmentation model, is
% that the training process additionally needs to find a useful embedding of the
% segmentation variants in the latent space.
% The encoder $q_{\phi}(\latent|\obs,\lab)$ maps the input image $\obs$ and the
% corresponding ground truth $\lab$ to a latent representation $\latent$.
% The decoder $p_{\theta}(\lab|\obs,\latent)$ maps the latent representation
% $\latent$ and the input image $\obs$ to the corresponding ground truth $\lab$.


\begin{figure}[htb!]
    \centering
    \includegraphics[width=1\textwidth]{Figures/proba_unet.png}
    \caption{The Probabilistic U-Net. (a) Sampling process. 
    The heatmap represents the
    probability distribution in the low-dimensional latent space. 
    (b) Training process illustrated for one training example.
    Figure taken from~\citep{kohl2018probabilistic}. }
    \label{fig:proba_unet}
\end{figure}





\end{appendices}

%BACK MATTER:
\backmatter
\listoffigures
\listoftables

%%%%%%%%%%%%%%%%%%%%%%%%%%%%%%%%%%%%%%%%%%%%%%
% bibliography
%%%%%%%%%%%%%%%%%%%%%%%%%%%%%%%%%%%%%%%%%%%%%%
% \bibliographystyle{apa-good}% the mandatory bibstyle
%%%%%%%%%%%%%%%%%%%%%%%%%%%%%%%%%%%%%%%%%%%%%%
\addcontentsline{toc}{chapter}{Bibliography}
\bibliography{sources}

\clearpage
% \thispagestyle{empty}
\newpage
\textcolor{white}{.}

\thispagestyle{empty}
\newpage
\textcolor{white}{.}

\thispagestyle{empty}
\newpage
\textcolor{white}{.}

\thispagestyle{empty}
\newpage


\thispagestyle{empty}



\vspace{1em}



\vspace{0.4cm}
\noindent {\bf \large Your title} \\ \vspace{0.4cm} 

\vspace*{1mm}

%backcover text here (abstract?)




\vspace*{20mm}



\includepdf[]{4eme.pdf}
\thispagestyle{empty}
\end{document}
