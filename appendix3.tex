% % !TEX root = late\obs_avec_réduction_pour_impression_recto_verso_et_rognage_minimum.tex
\chapter{Semi-supervised Bayesian classification}
\label{chap:appendix3}
% \subsection{Modified variational sequential labeler}
The VSL~\citep{chen2019variational} is based on conditional 
VAEs~\citep{pagnoni2018conditional},
where at each time step $t$, the observation $\obs_t$ is generated
according its associated context $u_t$, which consists of the observations
other than $\obs_t$. 
The lower bound of the log-likelihood at each time step $t$ is given by
\begin{align*}
    \log \p(\obs_t | u_t) &\leq 
    \E_{\q(\latent_t|\obs_{0:T})} 
    \left[ \log \p(\obs_t | \latent_t, u_t)\p(\latent_t|u_t) \p(\lab_t|\latent_t, u_t)   \right]
    \text{, for all } t \in \LL \text{.}\\
    \log \p(\obs_t | u_t) &\leq 
    \E_{\q(\latent_t, \lab_t |\obs_{0:T})} 
    \left[ \log \p(\obs_t | \latent_t, u_t)\p(\latent_t|u_t) \p(\lab_t|\latent_t, u_t)  \right]
    \text{, for all } t \in \U  \text{.}
\end{align*}
The VSL model simplifies some dependencies by assuming that
$\p(\lab_t|\latent_t, u_t) = \p(\lab_t|\latent_t)$ and 
$\p(\obs_t | \latent_t, u_t) = \p(\obs_t | \latent_t)$.
While the associated variational distribution is given by
\begin{align*}  
\q(\latent_{0:T}, \labu| \obs_{0:T}, \labl) =& \prod_{t=0}^T 
\q(\latent_t| \obs_{0:T}) \prod_{t\in  \U}  \q(\lab_t| \latent_t) \text{,}
\end{align*}
 which satisfies  the factorization~\eqref{eq:fact-1}
with
\begin{align*}
 \q(\latent_t|\latent_{t-1},\lab_{t-1},\obs_{0:T},\lab_{t+1:T}^{\LL})
 &=\q(\latent_t|\obs_{0:T}) \text{,} \\
 \q(\lab_t|\lab_{t-1},\latent_t,\obs_{0:T},\lab_{t+1:T}^{\LL})
 &=\p(\lab_t|\latent_t) \text{, for all } t \in \U \text{.}
\end{align*}

Our proposed variation of this model considers
$u_t = (\obs_{t-1}, \latent_t)$, \ie~we assume the context $u_t$ depends on the previous observation $\obs_{t-1}$ 
and the current latent variable $\latent_t$.
The associated ELBO~\eqref{eq:elbo_seq2} 
for the VSL model is given by
\begin{align*}
\Qsemi & (\theta,\phi) \overset{\rm mVSL}{=}
    \sum_{t \in \LL} \E_{\q(\latent_{t}| \obs_{0:T})} \left(
     \log\p(\lab_{t}|\latent_{t}) \right) +\\
    % & \E_{\q(\latent_0|\obs_{0:T})} \log \p(\obs_0|\latent_0) -
    % \dkl(\q(\latent_0|\obs_{0:T})||\p(\latent_0)) + \nonumber \\
    & \! \sum_{t=0}^T\!\!
    \Bigg[ \E_{\q(\latent_{t}| \obs_{0:T})} \log \p(\obs_t|\latent_t) 
    -    \dkl(\q(\latent_t|\obs_{0:T})|| \p(\latent_t|\obs_{t-1}, \latent_{t-1} ))  \Bigg] 
    \text{.}
\end{align*}
% where $\obs_{-1} =\latent_{-1} = \emptyset$.
It consists of two terms and 
that the previous assumptions enable us to interpret it as an expectation 
according to $\q(\latent_{0:T}|\obs_{0:T})$. 
Thus, it is not necessary to sample discrete variables according to 
the G-S trick. Moreover, a regularization term $\beta$ can be introduced 
in the second part of the ELBO in 
order to encourage good performance on labeled data 
while leveraging the context of the noisy observations during reconstruction.
While this model simplifies the inference, 
it should be noted that in the generative process, 
the observation $\obs_t$ is conditionally independent of its associated label and may not
be adapted to some applications.
% The VSL simplifies the ELBO by setting 
% $\q(\lab_t| \latent_t) = \p(\lab_t| \latent_t)$, 
% which enables the use of classical variational inference with 
% only continuous latent variables. 
% To further improve performance, they also introduce a regularization term into 
% the loss function that encourages good performance on labeled data 
% while leveraging the context of the noisy image during reconstruction. 

% \subsection{Semi-supervised variational RNN}
% \label{sec:svrnn}
% The generative model used in the SVRNN~\citep{butepage2019predicting} 
% can also be seen as a particular version of the TMC model where the latent variable
% $\latent_t$ consists of the pair $\latent_t=(z'_t, h_t)$. The associated
% transition distribution reads:
% % \begin{align}
% % \label{eq:svrnn}
% %  \p(v_t|v_{t-1}) = \p(\lab_t|v_{t-1}) \p(\latent_t|\lab_t, v_{t-1}) \p(\obs_t|\lab_t,\latent_t,v_{t-1}) \text{,}
% % \end{align}
% \begin{align}
% \label{eq:svrnn}
%  \p(v_t|v_{t-1})  \overset{\rm \scriptscriptstyle SVRNN }{ = }\p(\lab_t|v_{t-1}) \p(\latent_t|\lab_t, v_{t-1}) \p(\obs_t|\lab_t,\latent_t,v_{t-1}) \text{,}
% \end{align}
% where 
% \begin{eqnarray*}
% \p(\lab_t|v_{t-1})&=& \p(\lab_t|h_{t-1})\text{,} \\
% \p(\latent_t|\lab_t,v_{t-1})&=&\delta_{f_{\theta}(z'_t,\lab_t,\obs_t,h_{t-1})}(h_t) \times \p(z'_t|\lab_t, h_{t-1}) \text{,} \\
% \p(\obs_t|\lab_t,\latent_t,v_{t-1})&=& \p(\obs_t|\lab_t, z'_t, h_{t-1}) \text{,}
% \end{eqnarray*}
% and where $f_{\theta}$ is a deterministic, 
% \ie  the variable $\latent'_t$ is a stochastic latent variable 
% and $h_t$ is deterministically given by 
% $h_t = f_{\theta}( z'_t, \obs_t, \lab_t, h_{t-1})$, 
% where $f_{\theta}$ is a function parameterized 
% by a Recurrent Neural Network (RNN), for example. 
% The variational distribution $\q(\latent_{0:T}, \labu|$ $ \obs_{0:T}, \labl)$ satisfies
% the factorization \eqref{eq:fact-2}
% with
% \begin{align*}
%  q(z'_t|\latent_{t-1},\lab_{t},\obs_{0:T},\lab_{t+1:T}^{\LL})=  \q(z'_t| \obs_t, \lab_t, h_{t-1})\text{,} \\
% q(\lab_t|\lab_{t-1},\latent'_{t-1},\obs_{0:T},\lab_{t+1:T}^{\LL})= 
% \q(\lab_t| \obs_t, h_{t-1}) \text{,}
% \end{align*}

% \begin{align}
%     \label{eq:elbo_svrnn}
%     \Qsemi(\theta,\phi) \overset{\rm \scriptscriptstyle SVRNN }{=}& \quad
%     \L^{\LL}(\theta,\phi) + \L^{\U}(\theta,\phi) + J^{\LL}(\theta,\phi) \text{,}
% \end{align}
% where 
% \begin{align}
%     \L^{\LL}(\theta,\varphi) = \sum_{t\in \LL}
%      & \E_{\q(\latent'_t| \obs_t, \lab_t, h_{t-1})} 
%        \log \p(\obs_t|\lab_t, \latent'_t, h_{t-1}) 
%        + \log(\p(\lab_t | h_{t-1} ))    \nonumber \\ 
%     &  - \dkl (\q(\latent'_t|\obs_t, \lab_t, h_{t-1})||
%     p(\latent'_t|\lab_t, h_{t-1} ))  \text{,} \\
%     \L^{\U}(\theta,\varphi) =&  \sum_{t\in \U}
%      \E_{\q(\latent'_t, \lab_t| \obs_t, h_{t-1})} 
%        \log \p(\obs_t|\lab_t, \latent'_t, h_{t-1})     \nonumber \\ 
%     & - \dkl (\q(\latent'_t|\obs_t, \lab_t, h_{t-1})) \nonumber  \\
%     & - \dkl (\q(\lab_t| \obs_t, h_{t-1})|| \p(\lab_t|h_{t-1} )) \text{,}\\
%     J^{\LL}(\theta,\phi) = & \sum_{t\in \LL} 
%     \E_{\tilde{p}(\lab_t, \obs_t)}
%     \log(\p(\lab_t | h_{t-1} ) \q(\lab_t| \obs_t, h_{t-1})) \text{,}
% \end{align}
% where $\tilde{p}(\lab_t, \obs_t)$, for $t\in \LL$, 
% denotes the empirical distribution of the data.
% Their final ELBO does not coincide with \eqref{eq:elbo_seq}. 
% The reason why is that they derive it 
% from the static case~\citep{jang2016categorical} and add 
% a penalization term $J^{\LL}(\theta,\phi)$  that encourages 
% $\p(\lab_t|h_{t-1})$ and $\q(\lab_t| \obs_t, h_{t-1})$ 
% to be close to the empirical distribution of the data.

% \begin{remark}
% Note that since $\Lat_t$ is deterministic given $(z'_t, \obs_t, \lab_t, h_{t-1})$,
% its posterior distribution becomes trivial, and thus 
% there is no need to consider a variational distribution for it.
% \end{remark}



% In this section, we present the results of the proposed models on
% semi-supervised binary image segmentation. Our goal is to recover the 
% segmentation of a binary image $(\Omega=\{\omega_1,\omega_2\})$
% from the noisy observations
% $\obs_{0:T}$ when a partially segmentation $\labl$ is available.
% In particular, $\vartheta(\lab_t; \cdot )$ (resp. $\varsigma(\lab_t;\cdot)$) 
% is set  as a Bernoulli distribution with parameters $\ropy$ (resp. $\roqy$). 
% As for the distribution  $\zeta(\obs_t; \cdot )$ 
% (resp. $\eta(\latent_t;\cdot)$ and  $\tau(\latent_t;\cdot)$), 
% we set it as a Gaussian distribution with parameters 
% $[ \muobs , \diag(\sigobs) ]$ (resp.  $[ \mulatentp , \diag(\siglatentp)]$
%  and $[ \mulatent , \diag(\siglatent) ]$),
% % \begin{align*}
% %     \p(v_t|v_{t-1}) &=  \p(\obs_t|\cdot) \p(\latent_t|\cdot)\p(\lab_t|\cdot) \nonumber\\
% %     \p(\obs_t | \cdot) &= \N(\obs_t; \mu_{px, t}, {\rm diag}(\sigma_{px,t}) ) \\
% %     \p(\latent_t | \cdot) &= \N(\obs_t; \mu_{pz, t}, {\rm diag}(\sigma_{pz,t}) ) \\
% %     \p(\lab_t | \cdot) &= \Ber(\lab_t;  \rho_{py,t}) \text{,}
% % \end{align*}
% where ${\rm diag(.)}$ denotes the diagonal matrix deduced from the values 
% of $\sigma_{\cdot,t}$.
% % and the parameters are $\theta = \{\mu_{pz,t}, \sigma_{pz,t}, \mu_{px,t}, \sigma_{px,t}, \rho_{py,t}\}$  
% % and $\phi = \{ \mu_{qz,t}, \sigma_{qz,t} , \rho_{qy,t}\}$.



