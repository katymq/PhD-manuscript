% !TEX root = late\obs_avec_réduction_pour_impression_recto_verso_et_rognage_minimum.tex
\chapter{Supervised Bayesian classification}
\label{chap:appendix2}

% \yohan{Not here. I think that it should be discussed in the introduction of the
% next chapter and justify why supervised estimation is a direct consequence of
% your previous chapter. So you consider semi-supervised problems.}
PMCs can be adapted for the supervised classification task 
by considering an observed variable in an augmented dimension
$\obs_{t}  \leftarrow (\obs_{t},\lab_{t})$. 
We add a discrete variable $\lab_{t}$ label associated to 
$\obs_{t}$, for all $t \in \NN$.
% $\lab_t \in \Omega = \{\omega_1, \dots, \omega_C \}$ with $C$ the number of classes.
The parameter estimation is realized by maximizing the ELBO with 
respect to $\theta$ and $\phi$
where the general ELBO in~\eqref{eq:elbo_general} 
is still valid and  reads as
\begin{align*}
    % \label{eq:elbo-pmc_sup}
    \Qsup(\theta,\phi)= & - \int \log \left(\frac{\q(\latent_0|\obs_{0:T}, \lab_{0:T})}
    {p(\obs_0, \lab_0, \latent_0)}\right) \q(\latent_0|\obs_{0:T}, \lab_{0:T}) {\rm d} \latent_{0:T}
    \nonumber \\
    &- \sum_{t=1}^T \int  \log\! \left(\frac{\q(\latent_t|\latent_{0:
    t-1},\obs_{0:T}, \lab_{0:T})}{\p(\latent_t,\obs_t, \lab_t|\latent_{t-1},\obs_{t-1}, \lab_{t-1})}\right) 
    \q(\latent_{0:T}|\obs_{0:T}, \lab_{0:T}){\rm d}\latent_{0:T} \text{.}
\end{align*}
% \katyobs{the notation $Q_{sup}$ is not consistent with the previous notation???}
The transition distribution 
$\p((\obs_t, \lab_t), \latent_t| (\obs_{t-1}, \lab_{t-1}), \latent_{t-1})$
% given in~\eqref{pmc-transition} 
can be factorized in two terms as shown in~\eqref{eq:pmc_gen_transition}.
Without loss of generality, we can consider the following factorization,
\begin{align}
    \p(\latent_t,\obs_t, \lab_t|\latent_{t-1},\obs_{t-1}, \lab_{t-1}) = &    
    \p(\obs_{t},\lab_{t} | \latent_{t-1:t}, \obs_{t-1},\lab_{t-1}) 
    \p(\latent_t|\latent_{t-1}, \obs_{t-1},\lab_{t-1}) \nonumber \\
    = &  \p(\obs_{t}|\latent_{t-1:t}, \obs_{t-1},\lab_{t-1})
    \p(\lab_{t}|\latent_{t-1:t}, \obs_{t-1:t},\lab_{t-1})\times \nonumber \\
    \label{eq:pmc-superv}
    & \;  \p(\latent_t|\latent_{t-1}, \obs_{t-1},\lab_{t-1}) \text{,}
\end{align}
which is nothing more than a TMC with transition~\eqref{eq:pmc-superv}.
The ELBO now reads 
\begin{align}
    \label{eq:elbo_sup_pmc}
    \Qsup(\theta,\phi) =& \L_1(\theta,\phi) +  \L_2(\theta,\phi)
\end{align}
with 
\begin{align*}
    \L_1(\theta,\phi) =& \;  \E_{\q(\latent_0| \obs_{0:T}, \lab_{0:T} )} 
    \log \p(\obs_0| \latent_0) 
    + \E_{\q(\latent_0| \obs_{0:T}, \lab_{0:T} )} 
    \log \p(\lab_0 |\latent_0, \obs_0) 
    \\ &+  
    \sum_{t=1}^T \E_{\q(\latent_t|\latent_{0:t-1},\obs_{0:T}, \lab_{0:T})}
    \log \p(\obs_t|\latent_{t-1:t}, \obs_{t-1},\lab_{t-1} )
    \text{}\\
    &+  
    \sum_{t=1}^T \E_{\q(\latent_t|\latent_{0:t-1},\obs_{0:T}, \lab_{0:T})}
    \log \p(\lab_{t}|\latent_{t-1:t}, \obs_{t-1:t},\lab_{t-1}) \text{,}\\
    \L_2(\theta,\phi) =& - \dkl(\q(\latent_0| \obs_{0:T}, \lab_{0:T} ) | |  \p(\latent_0)) \\
    &- \sum_{t=1}^T  \dkl(\q(\latent_t|\latent_{0:t-1},\obs_{0:T}, \lab_{0:T}) 
    | | \p(\latent_t|\latent_{t-1}, \obs_{t-1},\lab_{t-1})) \text{.}
\end{align*}

The training procedure of the  PMC model 
presented in Algorithm~\ref{algo:algo_train_dpmc_gen}, 
can be adapted for the supervised classification task.
The only distinction with the previous algorithm is the set of parameters $\theta$,     
which now includes the parameters of the conditional distribution of the labels 
$\p(\lab_t|\latent_{t-1:t}, \obs_{t-1:t},\lab_{t-1}).$
% Since $\lab_{0:T}$ is assumed to be observed, 
% a variational distribution $\q$ for $\lab_{0:T}$ is not required.

% \begin{example}
%     We recall the example of binary image segmentation, %in particular 
%     where $\obs_{0:T}$ represents the noisy grayscale image 
%     and $\lab_{0:T}$ the original black and white image. 
%     The previous transition~\eqref{eq:pmc-superv} is simplified as follows
%     \begin{align*}
%         \p(\latent_t,\obs_t, \lab_t|\latent_{t-1},\obs_{t-1}, \lab_{t-1}) = &  
%         \p(\obs_{t}|\latent_{t-1:t}, \obs_{t-1})
%         \p(\lab_{t}|\obs_{t},\lab_{t-1})
%         \p(\latent_t|\latent_{t-1}, \obs_{t-1}) \text{,}
%     \end{align*}
%     and its graphical model is presented in Figure~\ref{fig:pmc_supervised_learning}. 
   
%     In this case, we can choose the Gaussian distribution for $\p(\obs_{t}|\latent_{t-1:t}, \obs_{t-1})$ and 
%     $\p(\latent_t|\latent_{t-1}, \obs_{t-1})$, and the Bernoulli distribution for $\p(\lab_{t}|\obs_{t},\lab_{t-1})$.
%     Moreover, the variational distribution (\eg  $\q(\latent_t | \obs_{t}, \lab_{t})$ )
%     can also be a Gaussian one.
%     Then the set of parameters $\theta$ includes the mean and covariance matrix of the Gaussian distributions
%     and the parameters of the Bernoulli ones; and the set of parameters $\phi$ includes 
%     the mean and covariance matrix of the Gaussian variational distributions.
%     Finally, the parameter estimation can be realized by maximizing the
%     associated ELBO with respect to $\theta$ and $\phi$ as in the previous section.
%     \begin{figure}[htb]
%         \centering
%         \includegraphics[width=0.4\textwidth]{Figures/Graphical_models/spmc_sup.pdf}
%         \caption{Graphical representation of a particular instance of the PMC with 
%         an observed variable in an augmented dimension  
%         $\obs_{t}  \leftarrow (\obs_{t},\lab_{t})$ for 
%         supervised classification.}
%         \label{fig:pmc_supervised_learning}
%     \end{figure}
    
% \end{example}
