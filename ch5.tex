% !TEX root = latex_avec_réduction_pour_impression_recto_verso_et_rognage_minimum.tex
\chapter{Medical Perspectives}
\markboth{CHAPTER 5. MEDICAL PERSPECTIVES}{Short right heading}
\label{chap:medical_perspectives}
\localtableofcontents
\pagebreak

% \begin{algorithm}[htbp!]
%     \caption{}
%     \label{}
%     \begin{algorithmic}[1]
%     \Require{}
%     \Ensure{}
%     \Statex{\textbf{}}
%     \\ 
%     \Statex{\textbf{}}
%     \\\end{algorithmic}
%     % \vspace*{0.2cm}
% \end{algorithm}
  



\section{Context and motivation}


Cardiovascular diseases (CVDs) represent a leading global cause of mortality, as
highlighted by data from the World Health Organization\footnote{https://www.who.int/}. 
CVDs include a wide range of conditions that affect the heart and blood vessels.
Among these, atherosclerosis is the most common cause of CVDs, which 
is characterized by the build-up of plaque inside the arteries.
This atheromatous plaque is made up of fat, cholesterol,
calcium, and other substances found in the bloodstream.
Over time, this plaque hardens, leading to the obstruction of the arteries
and can cause serious health problems~\citep{insull2009pathology}. 
For example, it can limit the flow of oxygen-rich blood to the organs 
and other parts of the
body~\citep{rafieian2014atherosclerosis}.
Figure~\ref{fig:data_arteria}
shows an example of a normal artery (A) and an artery
with atherosclerosis (B)~\footnote{http://vascularsurgeon.ie/peripheral-arterial-disease-pad/}.

\begin{figure}[htb!]
    \centering
    \includegraphics[width=0.45\textwidth]{Figures/Medical_images/peripheral_arterial.png}  
    \caption{Peripheral arterial disease results from narrowing or
    blockage of the arteries of the  legs.}
    \label{fig:data_arteria}
\end{figure}

In this context, the GEPROMED (European Research Group on Prostheses Applied to Vascular Surgery)
has been established to develop new biomaterials and surgical techniques for
vascular surgery. The group has access to a database of medical images, which
they have made available to us for the purpose of developing new methods for
medical images processing in vascular surgery.
The images in this database  are Computed Tomography (CT) or
micro-Computed Tomography (micro CT) images of the femoro-popliteal arterial
segment (SAFP) of different patients. 
The SAFP is one of the longest arteries in
the human body, subject to diverse mechanical forces 
(\eg~torsion, flexion, and extension) due to the movement of the
lower limbs. 

Atherosclerosis disease comprises 3 categories of plaque: calcified (calcium)
($\approx  70\%$), fibrous ($\approx 20\%$), and lipid 
($\approx 10\%$)~\citep{kuntz2021co}. 
To treat these, some endovascular techniques have been
developed, such as angioplasty and stenting.
There is no non-invasive method (imaging) that can accurately differentiate
lesions along the SAFP. The analysis is usually based on the preoperative CT scan 
(low resolution images), but
there are high-resolution scanners that allow a quasi-histological analysis of
the tissue. In other words, we have a micro CT scanner 
\textit{ex vivo} \footnote{The term \textit{ex vivo} refers to experimentation 
performed 
on tissue samples outside the living organism.}, 
and then correlate the images with the histology\footnote{
Histology is considered as a gold or criterion standard 
for the diagnosis of many diseases.}.
\cite{gangloff2020probabilistic} has already proposed a method to segment
micro CT images of the SAFP. 
However, a major limitation of this method is that it is not possible to directly
segment the CT images of the SAFP, which are of low resolution.
This problem is the motivation for the work presented in this chapter.

\section{Data and preprocessing}
\subsection{Data availability}
A protocol was developed to obtain the data for this study, which 
is described in~\citep{kuntz2021co} and is detailed in
Appendix~\ref{anex:protocol_database}.
Figure~\ref{fig:data_availability_summary} 
shows the available data.  We  have the histologic slices~\ref{fig:histo},
 the (2D) micro CT images~\ref{fig:mct_histo},
 and their corresponding ground 
truth~\ref{fig:mct_annotation}. 
The ground truth is composed of 6 classes, which are described in
Figure~\ref{fig:data_notation}.
\begin{figure}[htb!]
    \begin{subfigure}[b]{\textwidth}
            \centering
            \includegraphics[width=0.75\textwidth]{Figures/Medical_images/intro_histo.PNG}
            \caption{Three histologic slices. }
            \label{fig:histo}
    \end{subfigure}
    \begin{subfigure}[b]{\textwidth}
            \centering
            \includegraphics[width=0.75\textwidth]{Figures/Medical_images/intro_mCT.PNG}
            \caption{Three microCT images correlated with their
            histologic truth.}
            \label{fig:mct_histo}
    \end{subfigure}
    \begin{subfigure}[b]{\textwidth}
        \centering
        \includegraphics[width=0.75\textwidth]{Figures/Medical_images/intro_gt.PNG}
        \caption{Three expert annotated microCT images obtained.}
        \label{fig:mct_annotation}
    \end{subfigure}
    \caption{Illustration of part of the  available data for the study.
    Figure taken from~\citep{gangloff2020probabilistic}.}
    \label{fig:data_availability_summary}
\end{figure} 
\begin{figure}[htb!]
    \centering
    \includegraphics[width=0.7\textwidth]{Figures/Medical_images/intro_notation.PNG}
    \caption{Notation of the classes of the ground truth~\ref{fig:mct_annotation}.
    Figure taken from~\citep{gangloff2020probabilistic}.}
    \label{fig:data_notation}
\end{figure}
These annotations are only available for some slices
of the 3D scan as shown in 
Figure~\ref{fig:data_availability_2}.
Here, the red rectangles represent all the 2D slices of the 3D micro CT image.
However, the combined information (depicted as light gray and purple rectangles)
is not uniformly distributed across these slices; it is only present in certain
slices without following a specific pattern.
Figure~\ref{fig:data_availability}
shows the correlation between  the CT scanner and the micro CT scanner, 
that is predominantly available in segments of the artery
where specific lesions, particularly calcifications, are present.

\begin{figure}[H]
    \centering
    \includegraphics[width=0.40\textwidth]{Figures/Medical_images/data_2.pdf}
    \caption{Illustration of the available pair of information: 2D micro CT image 
    (light gray rectangle) and its corresponding ground truth (purple rectangle).
    The pairs of information are only available for some slices of the 3D micro CT image.
    The red rectangles represent all the 2D slices of the 3D micro CT image.
    Figure based on~\citep{kuntz2021co}}
    \label{fig:data_availability_2}
\end{figure}



The aim of this study is to assess the technical
feasibility of histological segmentation using the SAFP algorithm based on the
preoperative CT scan. The results of this study will provide initial data to
assess the value of a subsequent, larger-scale study to validate the diagnostic
capabilities of automated segmentation.
As far as we know,  there is no non-invasive method (imaging) that can accurately
differentiate lesions along the SAFP. Characterization of  AOMI 
plaques will enable a patient-centred treatment strategy to be devised, based on the 
type of plaque in the lesion. 
Automated segmentation will be a tool that will make it possible to dispense
with histopathological analysis and detect the type of plaque on the
preoperative CT scan. 
\begin{figure}[htb!]
    \centering
    \includegraphics[width=0.7\textwidth]{Figures/Medical_images/data_new.pdf}
    \caption{Illustration of the available correlation between CT scanner [B, B1, B2] and
    micro CT scanner [C, C1, C2] using standard references after Step~\ref{item:correlation}. 
    They represent different types of calcifications in SAFP plaques.
    Figure based on~\citep{kuntz2021co}.}
    \label{fig:data_availability}
\end{figure}
The expected long-term benefits for other patients are very significant. They
can be offered individualized treatment depending on the nature of their lesions
by adapting the medical device.
Previous work has shown that it is possible to segment the micro CT images of the
SAFP. However, a segmentation of the CT images is also necessary, since the
micro CT images are not available for all patients.
In this chapter, the objective is to perform image segmentation on CT
images of the SAFP. Our particular focus
lies on the most common type of plaques, the calcifications (sheet and nodular), 
which is a first step to segment other types of plaques in the future.

\subsection{Challenges}
\label{sec:challenges}
% In the previous sections, we have provided an overview of the available data and the
% foundational work upon which our research is built. 
While we have achieved
success in segmenting micro CT images, a notable limitation remains: the
segmentation of CT images. Extending our segmentation to CT
images is a challenging task for several reasons. First, the low resolution of CT
images introduces additional complexity into the segmentation process, in 
contrast to the higher-resolution micro CT images that~\cite{gangloff2020probabilistic} 
has been working with. As
depicted in Figure~\ref{fig:data_availability}, 
the discrepancy in image quality between the CT scanner (B1 and B2) and the micro CT 
scanner (C1 and C2) is evident. 
Moreover, the limited availability of data poses a significant
challenge. While we possess both 3D micro CT images and their corresponding 3D
CT images, establishing a clear and precise correspondence between the two 
sequences of images is far from straightforward. 
These images are not perfectly aligned, and
their correspondence is not as simple as a `mirror' image. Furthermore, the
availability of annotations for only some slices of the micro CT image is a
notable limitation when it comes to training a segmentation algorithm
for the CT images. 
% In essence, this chapter is dedicated to addressing the
% intricate complexities associated with CT image segmentation, which include low
% resolution, the intricacies of aligning micro CT and CT images for accurate
% segmentation, and the limited data.


\subsection{Pre-processing of the CT and micro CT images}
We have developed a workflow to segment the calcifications in the CT images
depicted in Figure~\ref{fig:medical_images_workflow}.
First, we select the region of interest within
the images, which corresponds to the artery segment containing the
calcifications. The selection process in the micro CT images is guided by
annotations and employs a box detection algorithm. In contrast, the
selection process in the CT images utilizes a centerline algorithm, that is 
provided by an expert. Normalization of the images is carried out
to facilitate the subsequent super resolution, and segmentation processes.\\

\begin{figure}[htb!]
    \centering
    \includegraphics[width=0.86\textwidth]{Figures/Medical_images/workflow.pdf}
    \caption{Our workflow for segmenting sheet and nodular calcifications in
    CT images of the SAFP is
    structured into five steps. First, we perform pre-processing of the CT and
    micro CT images, followed by a super resolution algorithm on the CT images,
    post-processing of the SR-CT images, supervised segmentation, and
    segmentation on the SR-CT images.}
    \label{fig:medical_images_workflow}
\end{figure}


Moreover, the original size of the CT images containing calcifications is often small
($5\times 5$ to $12\times 12$ pixels).
Thus, we apply a Super Resolution (SR) algorithm to increase the image  
resolution to facilitate the segmentation process.
In our case, a primary concern is the preservation of details in the CT images.
Focusing on methods that enhance resolution without losing crucial information
is key. 
Different SR techniques have
been suggested, encompassing optimization methods and deep learning approaches.
The latter have emerged as the most promising, with
exponential growth~\citep{li2021review}. 
We have considered the LapSRN algorithm proposed by~\cite{lai2017deep},
which  utilizes a Laplacian pyramid framework. 
This algorithm has been selected for its ability to accurately
 reconstruct high-frequency
details and reduce visual artifacts, which are crucial for medical images, especially CT scans.
We also studied SR algorithms via VAEs, from a point of view of 
the applicability to medical images, 
we won't be able to use those algorithms (more details in Appendix~\ref{sec:applicability_sr}).
This choice may not
be definitive, and we will continue to explore other SR algorithms in the future.\\
\begin{figure}[htb!]
    \centering
    \includegraphics[width=0.6\textwidth]{Figures/Medical_images/problem_1.pdf}
    \caption{Example of a CT image of the SAFP and its corresponding
    micro CT image. In addition, the corresponding Super Resolution CT image
    after applying the LapSRN algorithm with a factor of up-scaling of 8.}
    \label{fig:srct_example}
\end{figure}

Figure~\ref{fig:srct_example} 
shows an example of a SR-CT image obtained with the LapSRN algorithm.
The original CT image is of size $12\times 12$ pixels and the SR-CT image is of size
$96\times 96$ pixels, which is a factor of up-scaling of 8.
After applying the SR algorithm, a post-processing phase is undertaken to
eliminate any noise introduced by the SR algorithm. This
involves an analysis of the SR-CT images in comparison to the micro
CT images, enabling a medical interpretation of the results. The analysis
entails a histogram comparison and a visual inspection of the images to
determine the quality of the SR-CT images.


Figure~\ref{fig:srct_example_2} 
shows an example of a sequence of CT images where the calcifications are present.
These new sequences of SR-CT images will be used for the 
segmentation of the calcifications in the next steps.
\begin{figure}[htb!]
    \centering
    \includegraphics[width=1\textwidth]{Figures/Medical_images/ct_SRct.png}
    \caption{Example of a sequence of CT and SR-CT images.  From left to right, the 
    pairs of images (CT, SR-CT).
    The CT images correspond to a sequence of 2D slices of a 3D CT image, where
    the calcifications are present. The corresponding Super Resolution CT images
    are obtained with the LapSRN algorithm with a factor of up-scaling of 8.}
    \label{fig:srct_example_2}
\end{figure}

\newpage
\section{Medical image segmentation}
% (see  Figure~\ref{fig:medical_images_workflow}). 
% It is worth mentioning that this workflow remains subject to modifications, as we
% continue to refine the segmentation of SR-CT images. A primary limitation we
% have encountered is the challenge of interpreting the results, and we emphasize
% the importance of a medical interpretation to further enhance the workflow.
Semantic segmentation is a well-studied problem in the field of computer
vision. The objective is to assign a label to each pixel of an image. In the
context of medical imaging, this task is particularly challenging due to the
complexity of the images and the limited availability of annotated data.
In this section, we present the segmentation of the calcifications in the CT images
of the SAFP. This segmentation is performed in two steps. 
First, we perform a supervised segmentation using the pre-processed micro
CT (HR images), and their corresponding ground truth data. 
Once a final segmentation  model is obtained,  we perform a
segmentation on the SR-CT images. 
Our results are based on the U-Net model~\citep{ronneberger2015u}, 
and the Probabilistic U-Net~\citep{kohl2018probabilistic}. 

% \subsection{U-Net}
\label{sec:unet}
The U-Net architecture is a type of convolutional neural network (CNN) that was
specifically designed for biomedical image segmentation tasks
proposed by~\cite{ronneberger2015u}.
The U-Net architecture is a fully convolutional network that consists of a contracting path (encoder)
and an expansive path (decoder), which gives it the U-shape.
The contracting path follows the typical architecture of a convolutional network 
(Figure~\ref{fig:unet}).
% It consists of the repeated application of two $3\times 3$ convolutions (unpadded convolutions),
% each followed by a rectified linear unit (ReLU) and a $2\times 2$ max pooling
% operation with stride 2 for downsampling.
% At each downsampling step, the number of feature channels is doubled.
% Every step in the expansive path consists of an upsampling of the feature map
% followed by a $2\times 2$ convolution (up-convolution) that halves the number of
% feature channels, a concatenation with the correspondingly cropped feature map
% from the contracting path, and two $3\times 3$ convolutions, each followed by a
% ReLU. The cropping is necessary due to the loss of border pixels in every
% convolution. At the final layer, a $1\times 1$ convolution is used to map each
% 64-component feature vector to the desired number of classes. 
% % In total, the network has 23 convolutional layers.
This architecture has been remarkably successful due to its efficiency in
learning from a limited number of samples while accurately segmenting images.
The U-Net and its (non-stochastic) variants have been used in a variety of
medical image segmentation tasks such as the bone 
segmentation~\citep{caron2023segmentation, ganeshaaraj2022semantic}, and
the pancreas segmentation~\citep{sriram2020multilevel}.
\begin{figure}[htb!]
    \centering
    \includegraphics[width=0.88\textwidth]{Figures/unet_2.png}
    \caption{U-net architecture (example for $32\times32$ pixels in the lowest
    resolution). Each blue box corresponds to a multichannel feature map. The
    number of channels is denoted on top of the box. The x-y-size is provided at
    the lower left edge of the box. White boxes represent copied feature maps.
    The arrows denote the different operations.
    Figure taken from~\citep{ronneberger2015u}}
    \label{fig:unet}
\end{figure}
% \subsection{Probabilistic U-Net}

The Probabilistic U-Net was introduced by~\cite{kohl2018probabilistic},
and designed
to address the inherent ambiguities in real-world vision
problems, especially in medical imaging. 
With ambiguous problems, there is no single correct answer.
For example, the same image can be
segmented in different ways by different experts, leading  to different possible
segmentations. The overlap between structures in the image can also lead to
ambiguities.
The Probabilistic U-Net  incorporates a~\gls*{cvae}
 into the U-Net 
architecture (more details in Appendix~\ref{chap:appendix5}).
The latent space is a low-dimensional space where the segmentation variants are
represented as probability distributions.
A sample from the latent space 
is drawn and then injected into the U-Net to produce the
corresponding segmentation map. This model is trained using a variational
inference approach (see Subsection~\ref{subsec:vbi}),
 which allows the model to learn the distribution of
segmentations in the latent space.

% The central component of the Probabilistic U-Net is the latent space, which is
% the key to modeling the ambiguity of the segmentation problem.

\subsection{Results}
% A segmentation of the micro CT images of the SAFP has been presented with the U-Net
% model before in~\citep{gangloff2020probabilistic}. 
% Here,
We aim to specifically segment calcifications in SR-CT
images, which show areas of calcification in the artery. The
segmentation algorithms are trained on a dataset which contains
micro CT images, and their corresponding ground truth data representing the
calcification zones.
We present the results obtained with both, the U-Net, and Probabilistic
U-Net models. 
We evaluate their effectiveness by calculating the Dice 
score~\eqref{eq:dice_score}  on the test set.
This score provides an assessment of each model's performance in accurately
segmenting, and classifying each class within CT images, crucial for informed
doctor analysis


\paragraph*{Three class segmentation: }
Table~\ref{tab:dice_score_test} 
summarizes the Dice score on the test set
for the segmentation of the micro CT images.
Three classes are considered: background (Ba), nodular calcifications (NC), 
and sheet calcifications (SC). 
In terms of the Dice score, the Probabilistic U-Net model
outperforms the U-Net model for the calcifications classes.
Once the segmentation model is obtained, we perform a 
segmentation on the SR-CT images.\\
\begin{table}[htb!]
    \begin{center}
    % \small
    \begin{tabular}{|l|r|r|r|r|}
    \hline
    \multirow{2}{*}{Model}  &\multicolumn{3}{c|}{Dice score on the test set}\\
    \cline{2-4} 
      & \multicolumn{1}{c|}{Ba} & \multicolumn{1}{c|}{NC} & \multicolumn{1}{c|}{SC} \\ 
      \hline \hline
      \multicolumn{1}{|l|}{U-Net}      &0,7688   & 0,6032  & 0,5967  \\ \hline
      \multicolumn{1}{|l|}{Probabilistic U-Net}    & 0,7178 & 0,6214 & 0,6141\\ \hline
      \end{tabular}
      \vspace{-0.2cm}
      \caption{Dice score on the test set for the U-Net and Probabilistic U-Net models.
      Three classes are considered: background (Ba), nodular calcifications (NC), 
      and sheet calcifications (SC).}      
    \label{tab:dice_score_test}
    \end{center}
\end{table}

Figure~\ref{fig:proba_unet_sr_ct}
shows an example of segmentation of the micro CT image and its corresponding SR-CT image
with the U-Net and Probabilistic U-Net models.
We can see that both models are able to 
segment the calcifications in the SR-CT images, however, this analysis is
not possible with all the sequence of SR-CT images. 
The results need to be analyzed carefully by a medical expert.\\

\begin{figure}[htb!]
    \centering
    \includegraphics[width=1\textwidth]{Figures/Medical_images/ch5_seg_ct_mct.pdf}
    \caption{Example of a segmentation of the micro CT (first row) and 
    its corresponding SR-CT images (second row) 
    with the U-Net and Probabilistic U-Net models. }
    \label{fig:proba_unet_sr_ct}
\end{figure}


\paragraph*{Four class segmentation: }
Initially, our segmentation model for CT images was designed to differentiate
among three classes (background, nodular and sheet  calcifications). 
However, we observed that calcifications were not
consistently present across the entire sequence of CT images.
This led us to introduce an additional class into our model: the soft tissue (ST) class,
% (see Figures~\ref{fig:data_notation} ).
% This class has a consistent presence in all images and plays a crucial role in our analysis. 
that  encompasses both the arterial wall and the surrounding tissue, making its
segmentation vital for an accurate doctor's interpretation of the results.
Table~\ref{tab:dice_score_test_4} shows the Dice
scores obtained on the test set,
now configured to identify four classes.
\begin{table}[htb!]
    \begin{center}
    % \small
    \begin{tabular}{|l|r|r|r|r|r|}
    \hline
    \multirow{2}{*}{Model}  &\multicolumn{4}{c|}{Dice score on the test set}\\
    \cline{2-5} 
      & \multicolumn{1}{c|}{Ba}  & \multicolumn{1}{c|}{ST} & \multicolumn{1}{c|}{NC} & \multicolumn{1}{c|}{SC} \\ 
      \hline \hline
      \multicolumn{1}{|l|}{U-Net}      & 0,6027       & 0,6363  & 0,5807  & 0,6016   \\ \hline
      \multicolumn{1}{|l|}{Probabilistic U-Net}    & 0,6256     & 0,6439  & 0,5817  & 0,6751\\ \hline
      \end{tabular}
      \vspace{-0.2cm}
      \caption{Dice score on the test set for the U-Net and Probabilistic U-Net models, 
      with four classes: background (Ba), soft tissue (ST), 
      nodular calcifications (NC), and sheet calcifications (SC).}      
    \label{tab:dice_score_test_4}
    \end{center}
\end{table}

\begin{figure}[htb!]
    \centering
    \includegraphics[width=1\textwidth]{Figures/Medical_images/ch5_seg_seq_sfa006.pdf}
    \caption{CT slides, their corresponding SR-CT images, and their corresponding
    3 class and 4 class segmentations of  SR-CT images.}
    \label{fig:proba_unet_sr_ct_more_classes}
\end{figure}

Figure~\ref{fig:proba_unet_sr_ct_more_classes}
presents some slices of the SR-CT images and their corresponding segmentation
with Probabilistic U-Net models, with three and four classes.
When we examine these results with the doctors, it becomes apparent
that four-class segmentation offers greater interpretability compared to
three-class segmentation.

\section{Remaining challenges}
We have made significant progress in the segmentation of sheet and nodular
calcifications in CT images of the SAFP. However, several challenges remain to
be addressed. First,
% in the step improving image resolution, we face the problem of `domain
% switching', a common challenge in medical imaging, where models trained on one
% dataset may not perform optimally when applied to a different dataset~\citep{yan2019domain}. 
% In addition, 
the super-resolution algorithm itself does not take into
account the sequential nature of the images, as it is applied independently to
each slice. This can lead to inconsistencies across the image sequence. To
address this problem, we have explored a post-processing technique for SR-CT
images that aims to improve the consistency of results across the image sequence
prior to segmentation.

On the other hand, segmentation with the U-Net, and Probabilistic U-Net models
present the limitation related to their static nature. That is, when applied to
SR-CT images, the segmentation performed by these models
treats each slice independently, without taking into account the sequential
context that the images are part of a 3D image. This led to inconsistent
segmentation results, with  different results between two
consecutive slices. To address this problem, we adapted the input of each
model to include not only the target slice, but also the anterior and posterior
slices, creating a ``sliding window'' effect. This modification is intended to
incorporate some degree of sequential context into the segmentation process.
%  The
% results presented in this chapter reflect the improvements achieved with this
% adaptation.
In the future, we plan to explore other approaches to overcome these challenges,
with the goal of developing more sophisticated models that can more accurately
reflect the sequential and dynamic nature of the medical data, 
\eg~a sequential probabilistic U-Net model, a sequential SR VAE.




\section{Conclusions}
In this chapter, we have described  a structured workflow for the segmentation of sheet and
nodular calcifications in the CT images of the SAFP. This workflow encompasses
five steps: pre-processing of the CT and micro CT images, application of the
SR algorithm on the CT images, followed by post-processing of the SR-CT images, 
and finally the segmentation on the SR-CT images. 
First significant results from our research include those obtained using the LapSRN
algorithm for the SR task, and the U-Net and Probabilistic U-Net models for
segmentation. In particular, the Probabilistic U-Net model demonstrated superior
performance to the U-Net model in segmenting the calcification classes. In
addition, we observed that segmentation into four classes produces more detailed
results, allowing a clearer distinction between calcifications and soft tissue,
which is vital for a proper doctor's interpretation.
% We have placed special emphasis on two critical aspects of this workflow: the
% super resolution and the segmentation tasks.
% These components are crucial for addressing modeling challenges and
% are integral to our ongoing research efforts.
%  We have presented a brief
% literature review of the super resolution and segmentation steps. 
% This review not only helps in understanding the complexities we are addressing,
% but also guides us in identifying the most promising approaches and determining
% appropriate methods for future research.