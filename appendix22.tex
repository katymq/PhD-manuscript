\chapter{Generative Pairwise Markov Models}
\label{chap:appendix_22}

\tocless\section{Proof of Theorem \ref{prop:cov-pmc}}


Let's recall that in the stationary case, the function from $\mathbb{N}$ to
$\mathbb{R}$ that associates $r_k$ to any $k$ is a covariance function (or a
covariance sequence) if and only if, for any $T \geq 0$, 
the Toeplitz matrix
with the first row  $[r_0, r_1, \ldots,r_T]$ is a covariance matrix, \ie~it is
positive semi-definite. This set of constraints thus restricts the set of
possible sequences, and we aim to characterize this set.  
 $\{r_0, r_1, r_2, ...\}$ is a covariance function if and
only if $r_0 \geq 0$,  and  if
\[ C(z) = r_0 + 2 \sum_{k=1}^{+\infty} r_k z^k \]
is a function of the Carathéodory class, \ie~$C(z)$ has a positive real part
for $z$ in the open unit disk 
(Carathéodory-Toeplitz theorem~\citep{akhiezer1965classical})


% the Carathéodory-Toeplitz theorem~\citep{akhiezer1965classical} is a
% covariance function if

Thus, we look for values of $\tilde{A}$ and $\tilde{B}$ such that the 
covariance matrix $\Sigma^{\obs}_{T}$ with first row
$[1,\; \tilde{B},\; \tilde{A}^2,\; \tilde{A}^2\tilde{B},\; \tilde{A}^4,\; \tilde{A}^4\tilde{B},\; \dots ]$
 satisfies:
\begin{equation*}
  \forall T \in \mathbb{N}^*, \; \Sigma^{\obs}_{T} \geq 0 \iff \forall z \in 
  \{ u \in \mathbb{C}; \; |u|<1 \}, \Re\Big( 1+ 2(\tilde{B}z + \tilde{A}^2z^2) 
  \sum_{\tau=0}^{\infty}(\tilde{A}^2z^2)^\tau \Big) \geq 0 \text{,}
\end{equation*}
which is derived from:
\begin{align*}
  C(z) = & 1 + 2 ( \tilde{B}z +  \tilde{A}^2z^2 + \tilde{A}^2\tilde{B}z^3 +  \tilde{A}^4z^4 + \tilde{A}^4\tilde{B}z^5 + \dots) \\
  = & 1 + 2 \big[\tilde{B}z ((\tilde{A}^2z^2)^0  + (\tilde{A}^2z^2)^1 + (\tilde{A}^2z^2)^2 + (\tilde{A}^2z^2)^3 + \dots)\\
  & +  \tilde{A}^2z^2 ((\tilde{A}^2z^2)^0  + (\tilde{A}^2z^2)^1 + (\tilde{A}^2z^2)^2 + (\tilde{A}^2z^2)^3 + \dots) \big ]  \\
  = & 1 + 2(\tilde{B}z + \tilde{A}^2z^2) \sum_{\tau=0}^{\infty}(\tilde{A}^2z^2)^\tau
\end{align*}

% We have, for all \( z \in \{ u \in \mathbb{C}; \; |u|<1 \}, |\tilde{A}^2z^2|<1 \) 
% then \( \sum_{\tau=0}^{\infty}(\tilde{A}^2z^2)^\tau = \frac{1}{1- \tilde{A}^2z^2} \).\\

The positive real part condition is equivalent to:
\begin{align*}
&\Re\Big( 1+ 2(\tilde{B}z + \tilde{A}^2z^2) \sum_{\tau=0}^{\infty}(\tilde{A}^2z^2)^\tau \Big)  \geq 0\\
\overset{(i)}{\iff }& \; \Re\Big( 1+ 2\; \frac{\tilde{B}z + \tilde{A}^2z^2}{1-\tilde{A}^2z^2} \Big) \geq 0\\
\iff & \; \Re\Big( \frac{1 + 2\tilde{B}z + \tilde{A}^2z^2}{1-\tilde{A}^2z^2} \Big)  \geq 0\\
\overset{(ii)}{\iff}& \; \Re\Big( \frac{1 + 2\tilde{B}re^{i\theta} + \tilde{A}^2r^2e^{2i\theta}}{1-\tilde{A}^2r^2e^{2i\theta}} \Big) \geq 0\\
\iff & \; \Re\Big( \frac{(1 + 2\tilde{B}re^{i\theta} + \tilde{A}^2r^2e^{2i\theta})(1-\tilde{A}^2r^2e^{-2i\theta})}{|1-\tilde{A}^2r^2e^{2i\theta}|^2} \Big) \geq 0\\
\iff & \; \Re\Big( (1 + 2\tilde{B}re^{i\theta} + \tilde{A}^2r^2e^{2i\theta})(1-\tilde{A}^2r^2e^{-2i\theta}) \Big) \geq 0\\
  \iff & \; 1 + 2\tilde{B}r\cos(\theta) - 2\tilde{A}^2\tilde{B}r^3\cos(-\theta) - \tilde{A}^4r^4 \geq 0\\
  \overset{(iii)}{\iff} & \; 1 + 2\tilde{B}r\cos(\theta) - 2\tilde{A}^2\tilde{B}r^3\cos(\theta) - \tilde{A}^4r^4 \geq 0\\
  \iff & \; 1 + 2\tilde{B}r\cos(\theta)(1-\tilde{A}^2r^2) - \tilde{A}^4r^4 \geq 0  \text{, }
\end{align*}
where we used the following arguments:
\begin{enumerate}[label=(\roman*)]
\item $|\tilde{A}^2z^2|<1$ since $\tilde{A}\in [-1, 1]$ and $|z|<1$.
\item Writing $z = re^{i\theta}$, for all $r \in [0, 1)$ and $\theta \in [-\pi, \pi]$.
\item Cosine is an even function.
\end{enumerate}

Thus, we need to analyze the expression:
\begin{equation}
  \label{eq:condCarat}
    1 + 2\tilde{B}r\cos(\theta)(1-\tilde{A}^2r^2) - \tilde{A}^4r^4 \geq 0 \text{,}
\end{equation}
and we can distinguish four cases:
\begin{enumerate}
  \item Case $\tilde{A}=0$: Let us first consider the case where $\tilde{A}=0$. In this case, \eqref{eq:condCarat} simplifies to:
  \[
  1 + 2\tilde{B}r\cos(\theta) \geq 1 - 2|\tilde{B}| \geq 0,
  \]
  which implies $|\tilde{B}|\leq \frac{1}{2}$.
  
  \item Case $\tilde{B}=0$: We then have the condition $|\tilde{A}|\leq 1$, which is true.
  
  \item Case $\tilde{B}>0$: 
  \begin{align*}
        &1 + 2\tilde{B}r\cos(\theta)(1-\tilde{A}^2r^2) - \tilde{A}^4r^4 \\
        \geq & \; 1 - 2\tilde{B}(1-\tilde{A}^2) - \tilde{A}^4. 
  \end{align*}
  
  Note that $1-\tilde{A}^2r^2\geq 0$ and $\tilde{A}^4r^4 \geq 0$. Therefore,
  \begin{align*}
        &1 + 2\tilde{B}r\cos(\theta)(1-\tilde{A}^2r^2) - \tilde{A}^4r^4 \geq 0 \\
        \iff &\; \tilde{B} \leq \frac{\tilde{A}^2+1}{2}.
  \end{align*}
  
  \item Case $\tilde{B}<0$: 
  \begin{align*}
        &1 + 2\tilde{B}r\cos(\theta)(1-\tilde{A}^2r^2) - \tilde{A}^4r^4 \\
        \geq & \; 1 + 2\tilde{B}(1-\tilde{A}^2) - \tilde{A}^4. 
  \end{align*}
  
  Note that $1-\tilde{A}^2r^2 \geq 0$ and $\tilde{A}^4r^4 \geq 0$. Therefore,
  \begin{align*}
        &1 + 2\tilde{B}r\cos(\theta)(1-\tilde{A}^2r^2) - \tilde{A}^4r^4 \geq 0 \\
        \iff &\; \tilde{B} \geq - \frac{\tilde{A}^2+1}{2}.
  \end{align*}
\end{enumerate}
Then $\{r_k\}_{k \in \NN}$ is a covariance function if and only if
  \begin{equation*}
  -1 \leq \tilde{A} \leq 1 \quad \text{and} \quad -\frac{\tilde{A}^2 +1}{2} \leq \tilde{B} \leq \frac{\tilde{A}^2 +1}{2}.
  \end{equation*}


\vspace{0.65cm}  
Now, the objective is to determine if any such probability distribution function can be modeled by some PMC model. For this, we study the inverse mapping of:

\begin{align}
  \label{eq:phige}
      \phi : \theta \mapsto \big(\tilde{A} = \tilde{A}(\theta), \tilde{B} = \tilde{B}(\theta)\big),
\end{align}
where $\theta$ represents the set of parameters of the model.

We set $\gamma = b$, and \(f\) either as \(0\) or \(-a - bc\) (two particular
cases of the PMC), that coincide with \eqref{eq:cov-pmm-e}. 
The following expressions for \(\tilde{A}\) and \(\tilde{B}\) are obtained:
\begin{eqnarray*}
    \left\{
    \begin{matrix}
    \tilde{A} = \sqrt{ce} \quad \text{and} \quad \tilde{B} = b(c(1 - b^2\eta) + e\eta)  & \;  \text{if } f = 0, \\ 
    \tilde{A} = \sqrt{e^2\eta + a^2(1 - b^2\eta)} \quad \text{and} \quad \tilde{B} = be\eta - a(1 - b^2\eta) & \; \text{if } f = -a - bc.
    \end{matrix} \right.
\end{eqnarray*}

First, the case \(f = 0\), \(\gamma = b\) implies that \(a = -bc\), so the set of parameters is \(b, c, e, \eta\) since \(a, f\), and \(\gamma\) are functions of these parameters. Thus, \(\phi\) can be written as:

\begin{align}
\label{eq:inv1}
    \phi :(b, c, e, \eta) \mapsto \big(\tilde{A} = \sqrt{ce}, \tilde{B} = b(c(1 - b^2 \eta) + e \eta) \big).
\end{align}

The domain \((\tilde{A}, \tilde{B})\) has been characterized to obtain a
covariance matrix, \ie~\(\tilde{A} \in [-1, 1]\) and \(-\frac{\tilde{A}^2 +
1}{2} \leq \tilde{B} \leq \frac{\tilde{A}^2 + 1}{2}\), which  defines a surface
\(\mathcal{S}\). We obtain an inverse mapping
\(\phi^{-1}\) of Equation~\eqref{eq:inv1}, showing that for some \((\tilde{A},
\tilde{B}) \in \mathcal{S}\), there exists at least one PMC which yields an
observation probability distribution.
For simplicity, we do not show the detailed inverse mappings and their
calculations here, as they are lengthy and complex. The important result is that
such a mapping exists and is consistent with the conditions stated above.\\

Next, the case \(f = -a - bc\), \(\gamma = b\)  implies \(c = e\eta - ab\eta\). The set of parameters is then \(a, b, e, \eta\), and \(\phi\) can be written as:
\begin{align}
\label{eq:inv2}
    \phi :(a, b, e, \eta) \mapsto \big(\tilde{A} = \sqrt{e^2 \eta + a^2 (1 - b^2 \eta)}, \tilde{B} = be \eta - a (1 - b^2 \eta) \big).
\end{align}
Similarly, we can obtain the inverse mapping \(\phi^{-1}\) of Equation~\eqref{eq:inv2}, showing that for some \((\tilde{A}, \tilde{B}) \in \mathcal{S}\), there exists at least one PMC which yields an observation probability distribution.
