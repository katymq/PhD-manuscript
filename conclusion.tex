% !TEX root = latex_avec_réduction_pour_impression_recto_verso_et_rognage_minimum.tex
\chapter*{Conclusions and Perspectives}
\markboth{CONCLUSIONS AND PERSPECTIVES}{CONCLUSIONS AND PERSPECTIVES}
\addcontentsline{toc}{chapter}{Conclusions and Perspectives}
% \epigraph{
% ``To boldly go where no one has gone before.'' }{Captain Kirk}

% %chapter abstract here


Throughout this thesis, our research has integrated traditional
probabilistic models with modern deep learning techniques to address 
various challenges in machine learning. We have focused on generative
sequential modeling, supervised, semi-supervised, 
unsupervised Bayesian classification, and a collaboration with the GEPROMED project
to address the segmentation of medical images.
\vspace{0.27cm}

First, we introduced a novel generative model based on Pairwise Markov Chains,
which effectively combines the strengths of Hidden Markov Models,
Recurrent Neural Networks, and the Stochastic RNNs.
This model considers  observed and latent variables as well as the
interactions between them, providing a more comprehensive representation of
sequential data. 
We developed a new parameter estimation method leveraging the variational inference
framework, which is both computationally efficient and straightforward to
implement. The integration of deep parameterizations within this PMC
model demonstrated superior performance on different 
datasets compared to traditional RNN and Stochastic RNN models.
We also highlighted the linear and stationary Gaussian PMC's ability to model complex
Gaussian distributions more effectively than previous models,  
by using the covariance function. 
\vspace{0.27cm}





Moreover, if we consider the latent variables as discrete, \ie~the labels
associated with each the observations,
we demonstrated the potential of the PMC for supervised, and unsupervised 
classification tasks. In a supervised setting, our first variational framework can 
be easily adapted. In an unsupervised setting, we can use the PMC with traditional
Bayesian parameter estimation methods, since the likelihood is tractable, 
\ie~VI is not necessary. However, the use of this Bayesian framework with neural 
networks in an unsupervised context is difficult due to the interpretation of the
latent variables as discrete labels. Thus, we proposed an alternative 
approach to address this issue by using a constrained output layer 
and a pretraining step to initialize the neural network.
\vspace{0.27cm}



Next, we extended our generative model to Triplet Markov Chains that 
incorporate an additional (continuous or discrete) process to model 
the interactions between the 
observed features and their corresponding labels. We illustrated the 
feasibility of creating diverse generative models based on
variational inference, which is particularly advantageous for datasets with
partially labeled observations or missing labels.
We proposed a new adapted parameter estimation methods for the TMC model,
that combines the variational inference framework, 
which is both computationally efficient, and interpretable in the context of 
sequential data classification.
Each context, semi-supervised and unsupervised classification, 
has its own challenges,
and we proposed different techniques to address them.
For the semi-supervised context, we proposed a relaxation of the discrete
variables using the Gumbel-Softmax trick, and 
for the unsupervised classification, we proposed a constrained output layer
and pre-classification.
\vspace{0.27cm}


In addition, our collaboration with the GEPROMED project allowed us to know the
challenges of medical image analysis. We proposed an adapted workflow to
address the segmentation of medical images, which is a helpful tool for
clinicians. We also applied classic super-resolution and segmentation
techniques to medical images, which are essential for improving the
interpretability of medical images. However, the results are not always
satisfactory or adapted to the available data. 
We applied a probabilistic segmentation model that incorporates the 
variational framework with conditional VAEs. 
\vspace{0.27cm}



In conclusion, the integration of traditional probabilistic models with
modern deep learning techniques has shown promising results in various
applications. The proposed models have demonstrated superior performance
compared to traditional models, and the variational inference framework has
proven to be a powerful tool for parameter estimation. 
Future work should continue to explore the integration of deep neural networks
with other probabilistic models to develop more robust and efficient generative models.
Research into various neural architectures (\eg~U-net),
and training paradigms could further
improve model performance and broaden their applicability.
While this thesis primarily focused on medical imaging, the proposed methods and
models have potential applications in other fields such as natural language
processing, bioinformatics, and finance. Future research could explore these
domains to validate the versatility and robustness of our models.
\vspace{0.27cm}

Moreover, the application of these models to medical image analysis
should be further explored to improve the interpretability of medical images
and enhance the workflow of clinicians. For example, the integration of
the TMCs with the U-NET architecture could provide a more comprehensive
representation of the sequential data,  where the labels become the segmented images,
and the observed features are the medical images.
\vspace{0.27cm}

In practice, semi-supervised classification tasks are
challenging due to the discrete nature of the latent variables.
The relaxation of the discrete variables using the Gumbel-Softmax trick
provides a workaround, but it introduces a trade-off between the optimization
of discrete variables and the quality of the approximation.
Researchers continue to explore ways to improve the optimization of models with
discrete variables, making them more tractable and effective for a wider range
of applications.
% \vspace{0.27cm}
In the unsupervised case, we also noted that the DNN pretraining and the
interpretability constraint require an available pre-classification. A future line
of research involving self-supervised learning might prove itself as an
efficient way to relax this requirement.

\vspace{0.27cm}

Finally, the stochastic realization theory can be also used to describe the
covariance series which can be produced by linear and stationary PMCs, similar
to the one used in the context of the GUM. The main difficulty is that they do
not admit a state-space model representation due to the new dependencies
introduced by the pairwise interactions, which makes the analysis more complex.
The trick is to interpret the PMC as a particular HMC in augmented dimension.
However, the theoretical analysis of the covariance series produced 
by general linear and stationary PMCs remains an open question.

% The development of a stochastic realization theory for these models would be a significant
% contribution to the field of probabilistic modeling.
% ~\citep{zhu2020s3vae, gatopoulos2021self}.

% \pagebreak
% \section*{Generative Sequential Modeling}
% \textbf{Family of the probabilistic model studied:}\\

% The probabilistic models studied in this thesis belong to the family of Markov
% Chain models. Specifically, we have focused on Hidden Markov Chains,
% Pairwise Markov Chains, and Triplet Markov Chains. These models are
% essential for understanding and modeling sequential data due to their ability to
% capture temporal dependencies and latent structures.

% \textbf{Model, innovation:}\\
% \textbf{Issue that has been addressed:}\\
% % \textbf{}\\


% \noindent\hfil\rule{0.5\textwidth}{.4pt}\hfil
% \pagebreak

     
% \section*{Semi-supervised Bayesian classification}
% \textbf{Family of the probabilistic model studied:}\\
% \textbf{Model, innovation:}\\
% \textbf{Issue that has been addressed:}\\
% % \textbf{}\\

% \noindent\hfil\rule{0.5\textwidth}{.4pt}\hfil

% While these techniques provide a workaround, 
% they are not without their limitations. 
% The relaxation introduces a trade-off between the 
% optimization of discrete variables and the quality of the approximation. 
% As the temperature parameter is annealed during training, 
% the model transitions from a more continuous relaxation to a more discrete one, 
% potentially affecting model performance.

% In practice, dealing with discrete variables often requires 
% careful consideration of the trade-offs and the application's 
% specific requirements. 
% Researchers continue to explore ways to improve 
% the optimization of models with discrete variables, 
% making them more tractable and effective for a wider range of applications.


% \pagebreak


% \section*{Unsupervised Bayesian classification}
% \textbf{Family of the probabilistic model studied:}\\
% \textbf{Model, innovation:}\\
% \textbf{Issue that has been addressed:}\\
% % \textbf{}\\
% \noindent\hfil\rule{0.5\textwidth}{.4pt}\hfil

% \begin{itemize}
%     \item How to link the model with the PMC in Chapter XXX
%     % ~\ref{chap:pmc}?
%     In the context, what we can modify the Variational PMC for unsupervised learning?
% \end{itemize}

% % \pagebreak
% % \section*{Perspectives on medical images}
% % \textbf{Family of the probabilistic model studied:}\\
% % \textbf{Model, innovation:}\\
% % \textbf{Issue that has been addressed:}\\
% % % \textbf{}\\

% % \noindent\hfil\rule{0.5\textwidth}{.4pt}\hfil

% % VAEs and the Super resolution VAEs. 
% % Conditional VAEs and the Super resolution CVAEs.
% % CVAEs do not require an assumption of the conditional variable which can be seen as 
% % a huge advantage or disadvantage depending on the application.

% % VAE-based models have often been criticised for their
% % feeble generative performance~\citep{chira2022image} but with new advancements
% % such as VDVAE , there is now strong evidence that deep VAEs have the potential
% % to outperform current state-of-the-art models for high-resolution image
% % generatio.\\

% % Pairwise Markov Model for super resolution tasks

% % In the further, we will try to use the PMC for super resolution tasks.???
% % In the future, we will
% % continue to explore other SR algorithms and segmentation models. We will also
% % continue to refine the segmentation of SR-CT images. A primary limitation we
% % have encountered is the challenge of interpreting the results, and we emphasize
% % the importance of a medical interpretation to further enhance the workflow.
% % % \pagebreak
% % % \section*{Possible questions and remarks}
% % % Link with difussion models

% % % Semi-supervised model and  the use of $\obs_{0:T}$ as a vector from left to right and
% % % not from right to left, any change in the model?
% % % (HMCs)For such models, choosing between a gradient ascent method or the EM-algorithm
% % % remains an open question that we do not address.