\chapter*{Introduction générale}
\markboth{INTRODUCTION}{INTRODUCTION} 
\addcontentsline{toc}{chapter}{Introduction générale}


% \section*{Context} 
% This thesis explores and models sequential data through the application of
% various probabilistic models with latent variables, complemented by deep neural
% networks. The motivation for this research is the development of dynamic models
% that adeptly capture the complex temporal dynamics inherent in sequential data.
% Designed to be versatile and adaptable, these models aim to be applicable across
% domains including classification, prediction, and data generation, and adaptable
% to diverse data types. 
% The research focuses on several key areas, each detailed
% in its respective chapter. Initially, the fundamental principles of deep
% learning, and Bayesian estimation are introduced. Sequential data modeling is then
% explored, emphasizing the Markov chain models, which set the stage for the
% generative models discussed in subsequent chapters. 
% In particular, the research delves into the sequential Bayesian classification 
% of data in supervised,
% semi-supervised, and unsupervised contexts. The integration of deep neural
% networks with well-established probabilistic models is a key strategic aspect of
% this research, leveraging the strengths of both approaches to address complex
% sequential data problems more effectively. This integration leverages the
% capabilities of deep neural networks to capture complex nonlinear relationships,
% significantly improving the applicability and performance of the models.


% In addition to our contributions, this thesis also
% proposes novel approaches to address specific challenges posed by the
% \gls*{gepro}. These proposed solutions reflect the practical and a possible
% impactful application of this research, demonstrating its potential
% contribution to the field of vascular surgery.
\vspace{.65cm}
\section*{Contexte}
Cette thèse vise à modéliser des données séquentielles à travers l'utilisation
de modèles probabilistes à variables latentes et paramétrés par des
architectures de type réseaux de neurones profonds. Notre objectif est de
développer des modèles dynamiques capables de capturer des dynamiques
temporelles complexes inhérentes aux données séquentielles tout en étant
applicables dans des domaines variés tels que la classification, la prédiction
et la génération de données pour n'importe quel type de données séquentielles.\\


Notre approche se concentre sur plusieurs problématiques liés à la modélisation
de ce type de données, chacune étant détaillé dans un chapitre de ce manuscrit.
Dans un premier temps, nous balayons les principes fondamentaux de
l'apprentissage profond et de l'estimation bayésienne. Par la suite, nous nous
focalisations sur la modélisation de données séquentielles par des modèles de
Markov cachés qui constitueront le socle commun des modèles génératifs
développés par la suite. Plus précisément, notre travail s'intéresse au problème
de la classification (bayésienne) séquentielle de séries temporelles dans
différents contextes : supervisé (les données observées sont étiquetées) ;
semi-supervisé (les données sont partiellement étiquetées) ; et enfin non
supervisés (aucune étiquette n'est disponible). Pour cela, la combinaison de
réseaux de neurones profonds avec des modèles probabilistes markoviens vise à
améliorer le pouvoir génératif des modélisations plus classiques mais pose de
nombreux défis du point de vue de l'inférence bayésienne : estimation d'un grand
nombre de paramètres, estimation de lois à postériori et interprétabilité de
certaines variables cachées (les labels). En plus de proposer une solution pour
chacun de ces problèmes, nous nous intéressons également à des approches
novatrices pour relever des défis spécifiques en imagerie médicale posés par le
Groupe Européen de Recherche sur les Prothèses Appliquées à la Chirurgie
Vasculaire (GEPROMED).\\


\vspace{.20cm}
\section*{Plan}
Notre manuscrit est organisé en 5 chapitres. 

Le chapitre~\ref{chap:main_concepts} 
consiste en une introduction technique dans lequel nous discutons du principe de
l'apprentissage profond et de l'estimation bayésienne. Nous y introduisons
également des modèles Markoviens pour le traitement des données temporelles.

Le chapitre~\ref{chap:pmc} 
propose s’intéresse aux chaînes de Markov génératives,
en se concentrant spécifiquement sur les chaînes de Markov couples (PMCs). 
Nous montrons que ce modèle propose un cadre unificateur pour les modèles de
Markov cachés ainsi que les récentes architectures de type « réseaux de neurones
récurrents stochastiques ». Nous proposons une paramétrisation de ces modèles
basée sur des réseaux de neurones profonds et nous détaillons des méthodes
d'estimation paramétriques basées sur l'adaptation de l'inférence variationnelle
au cas séquentiel. Nous mettons en évidence le pouvoir génératif de ces nouveaux
modèles, tant d'un point de vue expérimental que théorique.

Le chapitre~\ref{chap:semi_supervised_pmc_tmc}  
vise à utiliser les modèles précédemment développés pour le problème de la
classification séquentielle de données. Dans la mesure où le cas supervisé ne
présente pas de difficultés supplémentaires par rapport aux techniques mises en
place dans le Chapitre 2, nous nous intéressons au cas où les étiquettes/labels
associés aux données ne sont que partiellement accessibles. Cette contrainte
nous amène à revoir les méthodes d'inférence variationnelle précédemment
discutées et à étendre nos modèles de manière à pouvoir prendre en compte deux
types de variables cachées : les variables latentes du modèles et les labels non
accessibles que l'on cherche à retrouver. Pour cela, nous faisons appel aux
modèles de Markov triplet. Notre approche est validée par des simulations
numériques portant sur le problème de segmentation d'images binaires en contexte
semi-supervisé.




Le chapitre~\ref{chap:unsp_pmc_tmc} 
étend le problème au cas non supervisé. L'application directe des méthodes
précédentes peut conduire à l'apprentissage de modèles probabilistes dans
lesquels la variable étiquette/label n'est pas interprétable physiquement (ex :
classe blanc/noir associée à un pixel en niveau de gris), en particulier dans
des modèles reposant sur un grand nombre de paramètres. Pour résoudre ce
problème, nous proposons des méthodes d'estimation ad-hoc visant à prendre en
compte cette contrainte d'interprétabilité. Pour ce faire, nous commençons avec
des modèles de Markov couple visant à modéliser le couple observation/label,
puis nous réintroduisons dans un second temps une troisième variable latente
continue visant à complexifier la loi du couple précédent. Les apports de nos
modèles couple/triplet, paramétrisés par des architectures profondes, ainsi que
de nos algorithmes d'estimation paramétrique sont évalués sur différentes tâches
telles que la segmentation d'images biomédicales ou la reconnaissance
d'activités humaines.

Enfin, le chapitre~\ref{chap:medical_perspectives}  
 donne quelques perspectives sur les outils développés précédemment pour des
problématiques relatives aux données manipulées par le GEPROMED. Nous y
décrivons quelques problématiques liées aux images médicales acquises dans un
cadre préopératoire, présentons des résultats préliminaires et proposons une
feuille de route pour s'attaquer aux différents défis restant. 


Finalement, le manuscrit s'achève par un résumé des résultats ainsi qu'une discussion
sur les orientations futures pour l'exploration et l'application des résultats
obtenus.

