% !TEX root = late\obs_avec_réduction_pour_impression_recto_verso_et_rognage_minimum.tex
\chapter{Medical image segmentation}
\label{chap:appendix5}

\tocless\section{Protocol and Database Construction}
\label{anex:protocol_database}
We describe the protocol and the database construction as follows.
\begin{itemize}
    \item \textbf{Patients and explant recovery:}
    \begin{enumerate}
        \item Patients scheduled for transfemoral amputation in the 
        vascular surgery department are informed of the
        procedure and asked not to object during the pre-operative 
        consultation scheduled for the day before the operation.
        \item Management of the patient in
        accordance with current practice, with an injected preoperative CT scan.
        \item During routine transfemoral amputation surgery: recovery of the
        sample (portion of the amputation including the damaged artery) to be
        analysed after rinsing the artery lumen.
        \item Recovery of the explant with macroscopic analysis at GEPROMED and
        storage on their premises. The subjects' participation in the research
        ends after the surgery.
    \end{enumerate}
    \item \textbf{GEPROMED:} Ex-vivo microscanner imaging at GEPROMED.
    % , see Figure~\ref{fig:data_availability_all}.
    \begin{enumerate}
        % \item Placement of the explant on a special support.
        \item The microCT 3D images of the arteries are acquired at the CVPath Institute,
        Inc. (Gaithersburg, MD, USA).
        \item Histology are performed on the specimens 
    as described in~\cite{torii2019histopathologic}. \label{item:histo}
        \item Co-registration are subsequently performed manually
        between the microCT images and the histologic slices obtained 
        during the steps described above.
        The result of this step consists in pairs of data: the
        micro CT 2D image with its histologic ground truth. \label{item:co_registration}
        \item  An expert annotate the micro CT images using the histologic 
        ground truths in the GIMP software7. \label{item:annotation}
        They are 6 classes: 
        \begin{itemize}
            \item  soft tissue (ST): soft tissue, formaldehyde, thrombus, fibrous
            plaque.
        	\item fatty tissue (FT): fatty tissue, lipid pool.
        	\item sheet calcification (SC).
        	\item nodular calcification (NC).
        	\item specimen holder (SH).
        	\item background (Ba)
        \end{itemize}
        % gangloff2020probabilistic
        \item Collection and analysis of \textit{dicom} data (CT images): 
        Centerline information is available for the CT images.
        The centerline is given by an expert and is used to select the interest region
        of the images, since the lesion represents a small area of the artery,
        \ie~of the CT image. 
        \label{item:correlation_ct}
        \item Correlation between  CT scanner  and micro CT scanner  using standard
        references (collaterals, branches and specific lesions).\label{item:correlation}
    \end{enumerate}
\end{itemize}



\tocless\section{Previous Work}
% \paragraph{Segmentation of micro CT images:}  
\cite{gangloff2020probabilistic} has proposed different methods to segment the
micro CT images of the SAFP. They used pairs of micro CT images
histologically annotated micro-CT images, which constituted the training set. 
In other words, they mainly used the pairs of information obtained until 
Step~\ref{item:annotation} of the protocol described above.
The additional information
obtained after that step, which is related to the correlation between the CT scanner
and the micro CT scanner,  were not exploited from a segmentation point of view.
The authors have used a CNN based on the U-Net architecture~\citep{ronneberger2015u}
to segment the micro CT images into 6 classes.
We describe this technique with more details in 
Subsection~\ref{sec:unet}.
The number of classes was selected based on the histopathologists' advice.
% It has then been decided with the histopathologists that 6 classes were of
% interest to develop a first version of the algorithm
Notice that their work is a 2D supervised segmentation, 
since the pairs (micro CT image, ground truth) are only available for some slices
of the 3D micro CT image.
%  are only possible 
% for the slices where the histological ground truth is available.


The measure of performance is the Dice score, which is a measure of the similarity
between two sets of data. In the context of image segmentation,
for example, the Dice score can be used to evaluate the similarity between a
predicted segmentation mask and the ground truth segmentation mask.
The Dice score is defined as follows:
\begin{equation}
    \label{eq:dice_score}
    \text{Dice score} = \frac{2|A \cap B|}{|A| + |B|} \text{, }
\end{equation}
where $A$ and $B$ are the two sets of data.
This score is a number between 0 and 1, where 0 indicates no similarity and 1
indicates perfect similarity.

% The soft tissue, background, and specimen holder classes are well segmented 
% with Dice scores $> 0.86$.
% One remarkable fact is that the calcification classes 
% are segmented with Dice scores 
% $0.85$ and $0.64$ for the sheet calcification, and nodular calcification, respectively. 
% This is a good result considering the complexity of the images where
% the differentiation is almost impossible by the naked eye, even for an expert.

% \begin{remark}
%     The Dice score is a measure of the similarity between two sets of data. 
%     In the context of image segmentation,
%     for example, the Dice score can be used to evaluate the similarity between a
%     predicted segmentation mask and the ground truth segmentation mask.
% \end{remark}

% \paragraph{3D segmentation:} 
% \cite{gangloff2020probabilistic} also proposed a 3D 
% segmentation on the test SAFP, which is based on the 2D segmentation
% and probabilistic post-processing approaches. 
% A 3D segmentation of stents is also proposed.
% We will not present the details of the 3D segmentation, since it is not the 
% principal interest of this work.

% We take this work as a starting point for our study.
% In the next section, we describe the objective of our work and the challenges
% that we face. We also present the workflow that we have devised to address them.


% In summary, the main limitation of the previous work is that 
% it is still not possible to segment the CT images of the SAFP, 
% which are of low resolution and the annotations 
% (ground truth) are only available for the micro CT images.
% This problem is the motivation for the work presented in this chapter.





% \begin{figure}[htb!]
%     \centering
%     \includegraphics[width=1\textwidth]{Figures/Medical_images/all_data.jpg}
%     \caption{Atherosclerotic characterisation of fibrous (A), calcific (B),
%     and lipid (C) plaques of the popliteal artery with co-registration between
%     computed tomography (CT) angiography  and microCT 
%     with histology~\citep{kuntz2021co}.}
%     \label{fig:data_availability_all}
% \end{figure}




\tocless\section{Super resolution}
\tocless\subsection{Super resolution via VAEs}
Super-resolution (SR) techniques, while sharing a common objective of enhancing
image resolution, employ a variety of methods to achieve this goal. 
% In this
% section, we focus on a specific SR algorithm based on 
% VAEs~\citep{kingma2014} (see Chapter~\ref{chap:main_concepts})
% VAEs, a class of generative models rooted in the variational inference framework,
% offer to learning both a latent-variable model and its
% corresponding inference model.
% \subsection{Super-resolution via Variational Auto-Encoders}
Super-resolution VAEs architectures have been proposed
by~\citep{gatopoulos2020super}, 
which requires a dataset of high-resolution images and their corresponding
low-resolution images. 
\begin{figure}[htb!]
    \centering
    \includegraphics[width=0.6\textwidth]{Figures/sr_vae_sup.PNG}
    \caption{Stochastic dependencies of the proposed model. Our approach
    takes advantage of a compressed representation y of the data in the
    variational part, that is then utilized in the super-resolution in the
    generative part.
    Figure taken from~\citep{gatopoulos2020super}}
    \label{fig:srvae_network_sup}
\end{figure}
On the one hand, an unsupervised real image denoising 
and Super-Resolution approach via Variational
AutoEncoder (dSRVAE) was proposed by~\citep{liu2020unsupervised}.
The architecture of the proposed model is shown in Figure~\ref{fig:srvae_network}.

\begin{figure}[htb!]
    \centering
    \includegraphics[width=1\textwidth]{Figures/dSRVAE.png}
    \caption{Complete structure of the proposed dSRVAE model. It includes
    Denoising AutoEnocder (DAE) and Super-Resolution SubNetwork (SRSN). The
    discriminator is attached for photo-realistic SR generation.
    Figure taken from~\citep{liu2020unsupervised}}
    \label{fig:srvae_network}
\end{figure}

SR models based on VAEs are a branch of
image processing focused on generating high-resolution images from
low-resolution ones~\citep{appati2023deep, liu2020unsupervised,
gatopoulos2020super, hyun2020varsr}. 
These models have gained popularity due to their
effectiveness in modeling high-resolution images, traditionally dominated by
autoregressive models~\citep{li2001new,joshi2005learning}, 
and GANs~\citep{chira2022image}.
Images generated by VAEs present, in general, blurry details, which is a
limitation of these models. 
% \cite{gatopoulos2020super} proposed a 
% SR algorithm based on VAEs, which consists  of a DenseNet-based encoder, 
% a DenseNet-based decoder, and  a flow-based prior to generate crisp images.
% This model is still trained using the log-likelihood objective.
% They have introduced a downscaled representation of the image as a random variable,
% and utilize it in a SR manner. Then the model uses three latent variables, 
% where one is the downscaled version of the original image.

% Original ground truth images are commonly used to train SR algorithms. 
% \cite{liu2020unsupervised} proposed a SR algorithm based on VAEs that does not require
% ground truth images. This is a significant advantage since ground truth images
% are not always available. The proposed model consists of a joint image denoising 
% and SR model via Variational AutoEncoder in an unsupervised manner.
% Denoising is performed by a conditional VAE, where the encoder compresses 
% the clean image to learn the latent variables. Then the decoder learns to 
% extract the noise from the noisy image and with a latent variable.
% Once a clean image is obtained, the SR is performed by a
% SR subnetwork (SRSN). The basic structure of the SRSN is a set of
% hierarchical residual blocks. The denoised image is initially up-sampled 
% to the desired dimension by bicubic interpolation.
% In addition, the model is trained with a
% discriminator to improve the quality of the generated images.
% Since the model is trained in an unsupervised manner, the quality of the
% generated HR images is evaluated using a strategy based on the backprojection theory. 

\paragraph*{Applicability to medical images: }
\label{sec:applicability_sr}
Since the  LR-CT images we have are small ($5\times 5$ to $12\times 12$) pixels  
and the details are important for the segmentation, 
% (Step~\ref{item:super_resolution}).
the factor of up-scaling is important.
In addition, the images are noisy, which is a common problem in medical images.
The SR algorithms based on VAEs presented above are promising, however,
they are not suitable for our problem due to the up-scaling factor, and the
noise in the images. The SR algorithm based on VAEs presented 
in~\citep{gatopoulos2020super} proposes 
a factor of up-scaling of 2, which is not enough for our problem. Moreover, 
the input for training is an HR (micro CT) center line of the artery, 
which is not yet available. 
From a point of view of the applicability to medical images, 
we won't be able to use this algorithm. Although, it is a promising algorithm
for future researches in this field which can be related to the  work 
we have presented in previous chapters.
% On the other hand, the SR algorithm based on VAEs in~\citep{liu2020unsupervised}
% proposes a factor of up-scaling of 4, which is more suitable for our problem.
% \katy{Pending to check the factor of up-scaling and the reason why it is not suitable for our problem.}

%------------------------
% Podemos usar varias veces este algoritmo, es decir la salide meterla nuevamente
% en el algoritmo para obtener una imagen de mayor resolucion. 
% Esto es importante para nuestro problema, ya que las imagenes son pequenas, pero 
%  eso no aumenta la resolucion de los detalles, solo de la imagen.
%------------------------

% Different approaches are available for SR via VAEs such as 

% % Our interest in this section is to present the SR algorithm based on VAEs
% Very Deep Variational Autoencoder for SR (VDVAE-SR)~\citep{chira2022image}
% VDVAE-SR , a new model that aims to exploit the most recent deep VAE
% methodologies to improve upon the results of similar models. VDVAE-SR tackles
% image SR using transfer learning on pretrained VDVAEs. The
% presented model is competitive with other state-of-the-art models, having
% comparable results on image quality metrics.

% In particular, we can use a VAE to learn a mapping from the LR image to the HR image.
% The LR images act as inputs to the VAE, and the HR images act as the targets.
% The Encoder maps the LR images to a latent space, and the Decoder reconstructs
% the high-resolution images from the latent space representation.


% Advantages: probabilistic framewrok
% distribution of HR images


% Handling uncertaintyGeneration multiple outpuets 
% : flexibility

% Disavtanges: Complexity advantages 
% DeCarefull trainingReconstructuin details
% Detail preservation accurate diagnosis 

% reco
% nstructoiAccuracy interpretable 
% careful validation 

% Combination of both
\tocless\subsection{Laplacian pyramid SR network}
%  proposed a SR algorithm based on the Laplacian pyramid

The Laplacian Pyramid Super-Resolution Network (LapSRN), presented in
~\citep{lai2017deep}, is a 
method for single-image super-resolution using CNNs. 
It progressively reconstructs the sub-band residuals of high-resolution (HR) images
without requiring bicubic interpolation, which reduces computational complexity.
Figure~\ref{fig:lsrn_network} 
shows the LapSRN architecture where we can see the different layers of the network.
LapSRN directly extracts feature maps from low-resolution images and
progressively predicts sub-band residuals in a coarse-to-fine manner using
transposed convolutional layers for upsampling. It is trained end-to-end with
deep supervision using a robust Charbonnier loss function, which improves
accuracy and reduces visual artifacts. LapSRN stands out for its fast processing
speed, accuracy, and ability to generate multi-scale predictions in one
feed-forward pass, making it suitable for resource-aware applications.

\begin{remark}
    The Charbonnier loss function is a variant of the L1 loss function,
    commonly used in image processing and computer vision tasks, particularly
    for regression problems like image super-resolution.
    This loss function is defined as follows:
    \begin{equation*}
        \mathcal{L}_{\text{Charbonnier}}(x) = \sqrt{(y_{pred}-y_{true})^2 + \epsilon^2}
    \end{equation*}
    where $y_{pred}$ is the predicted value, $y_{true}$ is the ground truth value,
    and  $\epsilon$ is a small constant which ensures numerical stability and
    prevent division by zero.
    The inclusion of the $\epsilon$ term allows the Charbonnier loss to be less 
    sensitive to outliers than the $L_2$ loss, while being smoother and less abrupt 
    than the $L_1$ loss, which can be beneficial in training neural networks for 
    tasks like image super-resolution.
\end{remark}

% details multiple scales
% segmentation after fidelity anatomic structures
% interpretaibili

\begin{figure}[htb!]
    \centering
    \includegraphics[width=0.7\textwidth]{Figures/LSRN_network.jpg}
    \caption{Red arrows indicate convolutional layers. Blue arrows indicate
    transposed convolutions (upsampling). Green arrows denote element-wise
    addition operators, and the orange arrow indicates recurrent layers. 
    Figure taken from~\citep{lai2017deep}}
    \label{fig:lsrn_network}
\end{figure}

\cite{lai2017deep} have compared LapSRN with other classic SR algorithms
such as Super Resolution Convolutional Neural Network (SRCNN)~\citep{dong2015image},
Fast Super Resolution Convolutional Neural Network (FSRCNN)~\citep{dong2016accelerating},
Very Deep Convolutional Neural Network (VDSR)~\citep{kim2016accurate}, 
and some other state-of-the-art SR algorithms.
They have shown that LapSRN outperforms these
algorithms in terms of accuracy and visual quality.
% The authors have also proposed the Multi-scale LapSRN (MS-LapSRN) 
% in~\citep{lai2018fast}, which can reconstruct sharper and more accurate SR images.


\tocless\section{Probabilistic U-Net architecture}
% \paragraph{Probabilistic U-Net Architecture - }
The central component of the Probabilistic U-Net is the latent space, which is
the key to modeling the ambiguity of the segmentation problem.
The latent space is a low-dimensional space where the segmentation variants are
represented as probability distributions.
A sample from the latent space 
is drawn and then injected into the U-Net to produce the
corresponding segmentation map $S$, defined as follows:
\begin{equation*}
    S(\obs, \latent) = f_{comb}(f_{U-Net}(\obs),\latent) \text{.}
\end{equation*}
Here, $f_{U-Net}$ is the U-Net architecture and  $f_{comb}$
is the function that combines the information obtained from the latent space and
the output of the U-Net.


Figure~\ref{fig:proba_unet},
(a) represents the sampling process, where 
a sample is  drawn  from the prior distribution $p(\latent|\obs)$. 
Next,  the segmentation map $S$ is obtained.
%  by combining the sample into the U-Net.
Figure~\ref{fig:proba_unet}(b) 
represents the training process, where the model is trained with the
standard training procedure for conditional VAEs. The ELBO objective function
for the Probabilistic U-Net reads

\begin{equation*}
    \mathcal{Q}_{\text{P-U-Net}}(\obs,\lab) = 
    \mathbb{E}_{q_{\phi}(\latent|\obs, \lab)}
    \left[\log p_{\theta}(\lab|S(\obs, \latent))\right] 
    -\beta\; \text{KLD}\left(q_{\phi}(\latent|\obs,\lab)||p(\latent|\obs)\right) 
    \text{,}
\end{equation*}
where $\beta$ is a hyperparameter that controls the trade-off between the
reconstruction loss and the KLD term~\citep{higgins2017beta}.
The reconstruction loss is the cross-entropy between the segmentation map $S$ and
the ground truth $\lab$.
The KLD term is the Kullback-Leibler divergence between the
approximate posterior $q_{\phi}(\latent|\obs,\lab)$ and the prior
$p(\latent|\obs)$.

% A cross-entropy loss penalizes differences between S and Y (the cross-entropy
% loss arises from treating the output S as the parameterization of a pixel-wise
% categorical distribution Pc). Additionally there is a Kullback-Leibler
% divergence DKL(Q||P ) = Ez∼Q [log Q - log P ] which penalizes differences
% between the posterior distribution Q and the prior distribution P . Both losses
% are combined as a weighted sum with a weighting factor β, as done in [25]:

% A cross-entropy loss penalizes differences between S and Y (the cross-entropy
% loss arises from treating the output S as the parameterization of a pixel-wise
% categorical distribution Pc).
% The networks are trained with the standard training procedure for conditional
% VAEs (Fig. 1b), i.e.\ by minimizing the variational lower bound (Eq. 4). The
% main difference with respect to training a deterministic segmentation model, is
% that the training process additionally needs to find a useful embedding of the
% segmentation variants in the latent space.
% The encoder $q_{\phi}(\latent|\obs,\lab)$ maps the input image $\obs$ and the
% corresponding ground truth $\lab$ to a latent representation $\latent$.
% The decoder $p_{\theta}(\lab|\obs,\latent)$ maps the latent representation
% $\latent$ and the input image $\obs$ to the corresponding ground truth $\lab$.


\begin{figure}[htb!]
    \centering
    \includegraphics[width=1\textwidth]{Figures/proba_unet.png}
    \caption{The Probabilistic U-Net. (a) Sampling process. 
    The heatmap represents the
    probability distribution in the low-dimensional latent space. 
    (b) Training process illustrated for one training example.
    Figure taken from~\citep{kohl2018probabilistic}. }
    \label{fig:proba_unet}
\end{figure}




